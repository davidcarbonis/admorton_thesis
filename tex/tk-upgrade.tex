\section{The CMS Tracker Upgrade}\label{sec:tk-upgrade}
 
\subsection{The High-Luminosity Large Hadron Collider} 
During Long Shutdown 3 (2023-2025), the LHC is planned to be upgraded to provide an instantaneous luminosity up to $5-7.5 \times {10}^{34}$\percms, allowing a total integrated luminosity of 3000\fbinv to be delivered to the ATLAS and CMS experiments.
The increase in increasing the instantaneous luminosity of the LHC

In order to fully exploit the scientific potential of the LHC, the machine is planned to operate at a luminosity up to one order of magnitude higher than obtained with the nominal design, increasing to to $5-7.5 \times {10}^{34}$\percms, corresponding to an average number of proton-proton interactions per 40\MHz bunch crossing of 140 to 200.
It is intended that the High-Luminosity LHC (HL-LHC) upgrade~\cite{hl-lhc-prelim-design-report} will deliver a total integrated luminosity of 3000\fbinv to the CMS experiment, enabling precision SM and Higgs measurements, searches for rare processes and their potential deviations from the SM, and the discovery reach for multi-\TeV massive particles. 

\subsection{The Phase-II Tracker Upgrade}
\editComment{Rephrase the following}

During the LS3, the CMS silicon tracker, will need to be completely replaced due to both approximately 15 years of radiation damage during operation and to maintain a high track reconstruction efficiency and a low misidentification rate under increased pileup conditions, requiring an increase in sensor granularity~\cite{P2TrackerTDR}.
The radiation hardness of the tracker must also be improved in order to withstand the increased fluence of the HL-LHC.

In order to keep the L1 acceptance rate below the 750\kHz maximum in the high pileup HL-LHC environment without raising the transverse energy (\ET) or transverse momentum (\pT) thresholds (losing potentially interesting physics events), the L1 trigger will also make use of charged particle tracks from the outer tracking detector.
 
A novel design of two closely spaced silicon sensors, capable of rejecting hits generated by low \pT particles, has been proposed~\cite{jjonespixel,markthesis}. Correlated pairs of clusters between the two sensors which are compatible with a high \pT track (greater than 2-3 GeV), called \textit{stubs} (see Fig.~\ref{stubs}), are transferred to the L-1 trigger and are expected to provide an effective rate reduction of approximately 10~\cite{mpessimperf,2dptmoduleconcept}. Further details on the two types of \pT modules intended to be used for the outer tracker and their placement can be found in~\cite{CMS_Upgrade_TP,P2TrackerTDR}.

Using stubs as an input, the L1 trigger requires input data formatting, track reconstruction and track fitting to be undertaken within an overall latency of 4\mus.  

\begin{figure}[!h]
\centering
\includegraphics[width=5in]{CMS-bw-logo.pdf}
% where an .eps filename suffix will be assumed under latex,
% and a .pdf suffix will be assumed for pdflatex; or what has been declared
% via \DeclareGraphicsExtensions.
\caption{Cluster matching in $p_\mathrm{T}$-modules~\cite{P2TrackerTDR}. (a) Correlating closely spaced clusters between two sensor layers, separated by a few mm, allows discrimination of transverse momentum based on the particle bend in the CMS magnetic field, assuming that the particle originated at the beam-line. (b) The same transverse momentum corresponds to a larger distance between signals for a given sensor spacing. (c) A larger spacing is needed in the endcap disks to achieve the same discrimination. Only tracks with \pT $> 2-3$\GeVc are transferred off-detector.
}
\label{stubs}
\end{figure}

\subsection{An FPGA Based Track Finding Architecture}

A scalable, configurable and redundant system architecture based on a fully time-multiplexed design, using current FPGA technology, has been proposed.

\subsubsection{Linear $\chi^{2}$ Track Fitter}

\subsubsection{2 GeV Tracking}
