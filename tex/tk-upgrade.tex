\chapter{Development of a Level-1 Track Trigger for the CMS Phase 2 Upgrade}\label{chapter:tk-upgrade}
%As the statistical gains for an experiment that is operated at a constant luminosity increasingly diminish over time, it is planned to preserve and extend the LHC's physics discovery potential by operating the LHC with an increased instantaneous luminosity.
Before the start of these higher luminosity operations, the then life-expired CMS tracker will need replacing.
The new tracker will not only need to have increased radiation hardness to withstand the increased \PU environment, but also the capability to provide limited tracking information to the L-1 trigger in order to keep the L-1 acceptance rate below 750\kHz.

This chapter introduces the motivations behind the high luminosity upgrade of the LHC, the planned upgrade of the CMS tracker and the studies undertaken for one of the proposed track finding systems for the upgrade tracker.

\section{The High-Luminosity Large Hadron Collider} \label{sec:hl-lhc}
In order to fully exploit the physics discovery potential of the LHC, it is planned to increase the instantaneous luminosity the accelerator can deliver by up to an order of magnitude greater than the nominal design.

The High-Luminosity Large Hadron Collider (HL-LHC) upgrade is intended to increase the instantaneous luminosity of the LHC up  to $7.5 \times {10}^{34}$\percms.
This corresponds to an average number of proton-proton interactions (\PU) per 40\MHz bunch crossing of between 140 and 200 and a total integrated luminosity of up to of 3000\fbinv being provided to both the ATLAS and CMS experiments during the 10 year planned lifetime of the HL-LHC.

The installation of the HL-LHC upgrade is planned to take occur during Long Shutdown 3 (LS3), which is currently expected to start during 2024~\cite{ApollinariG.:2017ojx}. 
The timing of LS3 is motivated in part by the need to replace the inner triplet quadrupole magnets that focus the beams at the ATLAS and CMS collision regions are expected to be near life-expired due to radiation exposure~\cite{hl-lhc-prelim-design-report,CMS_Upgrade_TP}.

The instantaneous luminosity, $L$, of an accelerator and its beam parameters are related by~\cite{ApollinariG.:2017ojx}: 

\begin{equation}
L \propto \frac{n_{b}N^{2}_{p}}{\beta^{*}} R  \;
\label{eq:machineLumi}
\end{equation}

where $n_{b}$ is the number of bunches, $N^{2}_{p}$ is the number of protons per bunch, $\beta^{*}$ is the focal length (beam $\beta$ value) at the collision point, and $R$ is a crossing-angle-dependent luminosity geometrical reduction factor.

As it is not practical to increase the number of proton bunches due to the resultant heat loads induced by electron clouds, the increase in the machine's luminosity will be achieved by increasing the number of protons per bunch and by reducing $\beta^{*}$~\cite{ApollinariG.:2017ojx}.
Replacing Linac2 with the new Linear accelerator 4 (Linac4)~\cite{linac4} during the Long Shutdown 2 (2019-2020) will allow for the number of protons per bunch to be increased by a factor of two compared to the nominal LHC design (and to increase the injection energy by a factor of three).
The new, more radiation tolerant, quadrupole magnets to be installed during LS3 will provide the higher magnetic field strength and the aperture needed to provide the lower $\beta^{*}$ required to increase the instantaneous luminosity. 

\section{The Phase-II Outer Tracker Upgrade}\label{sec:tk-upgrade}
To meet the significant challenges of, and exploit, the increased instantaneous luminosity delivered by the HL-LHC, the CMS detector will be substantially upgraded.
This upgrade will take place during LS3 and will not only deliver the improved radiation hardness to handle the increase in radiation from the increased \PU but also greater detector granularity to reduce occupancy and enhanced bandwidth and triggering capabilities to avoid compromising physics potential~\cite{P2TrackerTDR,CMS_Upgrade_TP}.

The Phase-II upgrade will see the entire silicon tracking detector being replaced with one comprised of a pixel Inner Tracker and pixel and strip Outer Tracker that have the following properties:
\begin{itemize}
\item \textbf{Improved radiation hardness} is required so that the tracker is able to withstand the increased fluence of the HL-LHC (up to $2.3\times10^{16} n_{eq}/cm^{2}$ for the innermost layers) and operate efficiently up to the target luminosity. A margin of about $50\%$ will be required to accommodate the target luminosity being exceeded and the uncertainties in the anticipated radiation exposure.
\item \textbf{Increased sensor granularity} is required to ensure that the channel occupancy is kept at or below the per cent (per mille) level for the Outer (Inner) Tracker to ensure that a high track reconstruction efficiency and a low misidentification rate is maintained under the increased \PU conditions. This will also enable improved track separation in dense environments, such as high \pT jets, compared to the current pixel detector.
\item \textbf{Reduced material in the tracking volume} will significantly enhance the performance of the detector.
%\item \textbf{robust pattern recognition} - enabling fast and efficient track finding, which is especially important for the HLT, in the high \PU environment.
\item \textbf{Level-1 trigger contributions} are required in order to maintain L-1 trigger performance. It has been shown that the performance of the L-1 trigger will deteriorate in the high luminosity environment from both the rate increase and the reduced efficiencies of the L-1 selection algorithms~\cite{CMS_Upgrade_TP}.
Raising the upgraded calorimeters' and muon chambers' trigger thresholds would have minimal impact on the rate, and would negatively impact sensitivity to BSM physics that predicts new low mass particles~\cite{CMS_Upgrade_TP}.
Therefore the L-1 bandwidth and latency will be increased (from 100\kHz to 750\kHz and from $3.2\mus$ to $12.5\mus$ respectively) and tracking information will be included in the L-1 decision process to preserve and improve trigger performance.
\item \textbf{An extended tracking acceptance} of  up to $|\eta| = 4$ in the forward region will greatly improve the overall physics capabilities of the CMS experiment as the density of jets associated with vector boson increases with pseudorapidity~\cite{CMS_Upgrade_TP}. By extension, measurements of missing transverse energy, total energy and jet b-tagging acceptance will also be improved.
\end{itemize}

Therefore, the Inner Tracker is designed to cover the range up to $|\eta| = 4$ using $100-150\mum$ thick planar silicon pixel sensors, measuring either $25\times100\mum^{2}$ or $50\times50\mum^{2}$.
These sensors provide the low (per mille) occupancy and track separation with the negligible inefficiencies required.

As with the previous pixel detectors, the Inner Tracker is also designed for easy installation and removal to facilitate repairs and replacement of degraded parts.
Further discussion of the Inner Tracker can be found in the Phase-II Technical Design Report~\cite{P2TrackerTDR}.

As tracking information is required to make L-1 decisions at the HL-LHC, the design of the Outer Tracker has been driven by the need to provide tracking information to the L-1 trigger.
Given that it will not be possible to read out the entire Outer Tracker for the L-1 trigger for every bunch crossing, a novel design of a pair of closely-spaced silicon sensor layers, separated by a few mm, that are capable of rejecting low transverse momentum tracks has been proposed~\cite{jjonespixel,markthesis}.
These sensors, known as the \emph{$\pT$-modules}, are able to discriminate against low transverse momentum charged particle tracks.

As the bend angle of a charged particle in a magnetic field depends on its transverse momentum, a $\pT$-module is able to  reject tracks below a configurable \pT threshold by comparing the distance between clusters of hits between its two sensor layers, as demonstrated in Figure~\ref{fig:stubs}(a).
The \pT threshold is designed to be configurable as the separation between the clusters increases with a module's distance from the beam if the sensor spacing remain unchanged, as illustrated in Figure~\ref{fig:stubs}(b).
The sensor spacing however, is increased for the endcap disks, where the $\pT$-modules are orientated perpendicular to the beam line, in order to maintain comparable discrimination due to projective effects, as shown in Figure~\ref{fig:stubs}(c).

\begin{figure}[!h]
\centering
\includegraphics[width=5in]{figs/tk-upgrade/pTsketches.png}
% where an .eps filename suffix will be assumed under latex,
% and a .pdf suffix will be assumed for pdflatex; or what has been declared
% via \DeclareGraphicsExtensions.
\caption{Cluster matching in the $\pT$-modules proposed for the Outer Tracker~\cite{P2TrackerTDR} as described in the text; (a) demonstrates how correlating pairs of closely-spaced clusters between the two sensor layers allows for the discrimination of a track candidate's transverse momentum; (b) shows that if the sensor spacing remains unchanged, that the separation between the two clusters increases the further a module is away from the beam line; and (c) illustrates that the sensor spacing of modules in the endcap disks, which are perpendicular to the beam line, is required to be larger because of projective effects.
}
\label{fig:stubs}
\end{figure}

By correlating pairs of clusters on-detector that are consistent with a track with a transverse momentum of about 2\GeV or greater, an effective data rate reduction of approximately a factor of 10 is achieved before the resultant \emph{stubs} are transferred to the L-1 trigger~\cite{mpessimperf,2dptmoduleconcept}.

Two \pT-modules are being developed for the Outer Tracker upgrade: 2S \emph{strip-strip} modules and PS \emph{pixel-strip} modules.
The 2S~modules, are designed to be used at radii $r>60$\cm from the beam line, where the hit occupancies are lower and each sensor has an active area of 0.05\cm~$\times$~9.14\cm.
Both 2S~module strip layers have a pitch of 90\mum in the transverse plane ($r$-$\varphi$) and a strip length of 5.03\cm along the direction of the beam axis, $z$.
Each PS~module sensor layer has an active area of 4.69\cm~$\times$~9.60\cm and will be used at radii in the range  $20<r<60$\cm where the occupancies are highest.
The upper PS~module layers consist of a silicon strip sensor and a silicon pixel sensor, both with a pitch of 100\mum in $r$-$\varphi$, and a strip length in $z$ of 2.35\cm for the strips and 1.47\mm for the pixels.
The finer granularity provided by the pixel layer affords better resolution along the $z$ axis, which is crucial for vertex identification in the high \PU environment of the HL-LHC.
Further details on the two \pT-modules can be found in~\cite{P2TrackerTDR,CMS_Upgrade_TP}.

The current proposed layout of the Phase-II Outer Tracker, referred to as the \emph{tilted barrel} geometry, is depicted in the upper diagram in Figure~\ref{fig:trackerlayout}, and a previous proposal, referred to as the \emph{flat barrel} geometry, is shown in the lower diagram~\cite{CMS_Upgrade_TP}.
Both plots illustrate the PS and 2S module positions in the six barrel layers and the five endcap disks on either side of the barrel, with only modules located at $|\eta| < 2.4$ being configured to send stub data off-detector.
The geometries are so named as they were inspired by whether or not the modules in the three innermost barrel layers are tilted so that their normals point towards the interaction region.
The advantages of the tilted geometry over the original flat barrel are that it not only improves stub-finding efficiency for tracks with large incident angles but also reduces the overall cost of the system~\cite{P2TrackerTDR}.
Due to the maturity of the preparations for the review between the three competing proposed track finder systems, discussed later in Section~\ref{sec:TMTT}, at the time the tilted barrel geometry was adopted for the Phase-II Outer Tracker TDR it was decided to use the flat barrel geometry for results produced for the review.

\begin{figure}[h]
\centering
\includegraphics[width=0.8\textwidth,trim={1.1truecm 0truecm 1truecm 12truecm},clip]{figs/tk-upgrade/tiltedbarrelmap.pdf}
\includegraphics[width=0.8\textwidth,trim={0.7truecm 0truecm 1truecm 0truecm},clip]{figs/tk-upgrade/mersilayout.pdf}
\caption{One quadrant of the Phase-II Outer Tracker layout, showing the placement of the the PS (blue) and 2S (red) modules. The upper diagram shows the currently proposed \emph{tilted barrel} geometry~\cite{P2TrackerTDR,tiltedGeometry}, and the lower diagram shows an older proposal for the layout, known as the \emph{flat barrel} geometry \cite{CMS_Upgrade_TP}.}
\label{fig:trackerlayout}
\end{figure}

Figure~\ref{fig:dataFlow} illustrates the data flow and latency requirements from the \pt-modules to the off-detector electronics for the upgraded tracker.
Out of the total L-1 latency of 12.5\mus, about $1\mus$ is required for generation, packaging and transmission of stubs from the tracker front-end (FE) electronics to the Data, Trigger and Control (DTC) system. 
Approximately $4\mus$ is available for the reconstruction of tracks from data arriving at the DTC.
The rest of the available latency is allocated for the correlation of tracks with trigger primitives from the calorimeters and muon systems ($3.5\mus$), the propagation of the L-1 decision to the hardware/firmware  buffers ($1\mus$) and an additional safety margin ($3\mus$)~\cite{TMTT_JINST}.

\begin{figure}[h]
\centering
\includegraphics[width=\textwidth]{figs/tk-upgrade/dataflow.pdf}
% where an .eps filename suffix will be assumed under latex,
% and a .pdf suffix will be assumed for pdflatex; or what has been declared
% via \DeclareGraphicsExtensions.
\caption{Illustration of data flow and latency requirements starting from the \pt-modules and front-end (FE) electronics and running through to the off-detector electronics dedicated to forming the L-1 trigger decision~\cite{TMTT_JINST}.}
\label{fig:dataFlow}
\end{figure}

The architecture of any Track Finder system proposed, which will take the pre-processed stubs as input and output fully reconstructed tracks for the L-1, will be constrained by the system's latency budget and how the detector is cabled to the DTC system.
The $4\mus$ latency constraint will limit the amount of processing that can be done for the finding and fitting of tracks and the choice of cabling scheme for the detector will determine how data is distributed and processed throughout the Track Finder system.

\section{A Time-Multiplexed Track Finder}\label{sec:TMTT}
Three different L-1 track finders have been explored by the CMS Collaboration.
One of the proposals uses Associative Memory (\emph{AM}) ASICs for track finding and FPGAs for track fitting~\cite{P2TrackerTDR,AM}.
The other two proposasls are all-FPGA approaches, one using a fully Time-Multiplexed Track (\emph{TMTT}) finder which uses the Hough Transform to identify track candidates~\cite{TMTT_JINST} and one using a ``road search'' (\emph{tracklet}) algorithm to reconstruct tracks~\cite{P2TrackerTDR,tracklet}.
Hardware demonstrators for each of these proposed L-1 track finders were constructed to prove the feasibility of each approach, which were reviewed in 2016.

In rest of this section the architecture and components of the \emph{TMTT} Track Finding Processor are discussed. 

\subsection{The Track Finding Architecture}\label{subsec:TFA}
The proposed FPGA-based Hough Transform Track Finder is a scalable, flexible and redundant design based on a fully time-multiplexed architecture for implementation on commercially available FPGAs, as previously demonstrated by the Phase-I Calorimeter Trigger Upgrade~\cite{phase1L1TDR} discussed in Section~\ref{paragraph:L1}.
As discussed in Section~\ref{paragraph:L1}, a time-multiplexed design has a number of advantages, including that only a single Track Finding Processor (TFP) is required to demonstrate the full system as each processor is identical in every respect.

Unlike the Phase-I Calorimeter Trigger, it is not feasible to process the entire output of the Phase-II Outer Tracker in a single processor for a given time slice.
This is because of the limits imposed on the system by the total data and latency bandwidth a single FPGA-based processor can handle.

The number of independent track finding processors was determined by how the DTC system was connected to the tracker.
At the time of the 2016 review it was assumed that the detector would be cabled to the DTC system such that each DTC board would process all data from a \ie 45 degree $\varphi$-sector, known as a \emph{detector octants}, in the tracker.
Consequently, the proposed track finding system was divided into \emph{processor octants} that were offset from the detector octants by about $22.5$ degrees in $\phi$, as shown in Figure~\ref{fig:tmttarch}, in order to handle data duplication across hardware boundaries.
The detector octants are not uniform as the geometry of the tracker does not have an exact eight-fold symmetry.

This baseline track finding system architecture, illustrated in Figure~\ref{fig:tmttarch}, uses two neighbouring DTC boards to time-multiplex and duplicate stub data across processing octant boundaries before each DTC transmits 50\% of its data to one TFP and 50\% to the neighbouring TFP.
Based on current electronics and the high speed links available, the data requires 18 TFPs per processing octant (one for each time slice, resulting in a full system requiring 144 TFPs).

\begin{figure}[h]
\centering
\includegraphics[width=1.00\textwidth]{figs/tk-upgrade/tmttarch.pdf}
\caption{An illustration of the baseline system architecture described in the text, demonstrating how two neighbouring DTCs time-multiplex and duplicate stub data across processing octants and how it transmits the processed data to two neighbouring TFPs~\cite{TMTT_JINST}}
\label{fig:tmttarch}
\end{figure}

A hardware demonstrator of the baseline system consisting of five Master Processor Virtex-7 (MP7) cards~\cite{mp7ref}, capable of processing one phi-octant of the tracker with a time-multiplexing factor of 36, was used to validate the feasibility of the proposed full system using hardware available at the time of the 2016 review.
All of the results achieved, and a complete description of the system, are given in~\cite{TMTT_JINST}.

\subsection{The Track Finding Processor}\label{subsec:TFP}
The Track Finding Processor shown in Figure~\ref{fig:TFP} consists of four self-contained components:
\begin{itemize}
\item {\bf Geometric Processor (GP):} Responsible for pre-processing the stubs from the DTC.
\item {\bf Hough Transform (HT):} A highly parallelised initial coarse track finding that identifies track candidates that are consistent with a track in the \rphi plane, greatly reducing the data volume and combinatorics that have to be considered by the subsequent stages.
\item {\bf Kalman Filter (KF):} A track filtering and fitting stage which removes incorrectly reconstructed tracks, stubs that are incorrectly associated to a track and precisely fits helix parameters.
\item {\bf Duplicate Removal (DR):} A final pass filter that uses the precise fit information to remove duplicate tracks generated by the \HT.
\end{itemize}

Each of these components is described in more detail below.

\begin{figure}[h]
\centering
\includegraphics[width=0.78\textwidth]{figs/tk-upgrade/demoslice1.pdf}
% where an .eps filename suffix will be assumed under latex,
% and a .pdf suffix will be assumed for pdflatex; or what has been declared
% via \DeclareGraphicsExtensions.
\caption{The four self-contained logical components of the Track Finding Processor, where each box (block) in the diagram represents a single FPGA. The two FPGAs for the two detector octant sources and the sink FPGA and the optical links between all components are also shown.}
\label{fig:TFP}
\end{figure}

\subsubsection{Geometric Processor}\label{subsubsec:GP}
Each GP performs two tasks: the conversion of the 48-bit DTC stubs into a 64-bit format extended format that is used to reduce the HT processing load and assignment of the stubs in each sector into a subsector. 
Each sector is composed of 2 sub-sectors in $\phi$ and 18 in $\eta$.

This division of the processing octants simplifies the task of the downstream logic, allowing the track finding to be carried out independently and in parallel within each sub-sector. 
The chosen $\eta$ binning is sufficiently fine to ensure that any track found by the \rphi HT is consistent with a straight line in the \rz plane, despite the fact the \HT itself only searches for tracks in the \rphi plane, thus rejecting incompatible track candidates.

Stubs that are compatible with more than one sub-sector, usually due to track curvature in $\phi$, are duplicated. 

Stubs are assigned to sub-sectors occurs in a three stage process:
\begin{itemize}
\item A rough $\eta$ sorting into six bins;
\item A subsequent fine $\eta$ sorting into three bins and;
\item A $\phi$ sorting into two bins. 
\end{itemize}

Each of the TFP's logic blocks shown in Figure~\ref{fig:TFP} has been designed to be highly reconfigurable and can easily be adapted to any alternative sub-sector definition.

\subsubsection{Hough Transform}
The Hough Transform algorithm is a widely used method of detecting geometric features in digital image processing~\cite{HT}.
While the \HT can be used to find any shape that can be parametrised, its simplest form of detecting straight lines is the most relevant for the proposed track finder.
In this case, the \HT describes a point $(x,y)$ in real space as a straight line with a gradient and intercept ($m,c$) in the parameter space known as \emph{Hough}-space.
Conversely, a point in Hough-space corresponds to a straight line in real space. 
Therefore, a straight line corresponding to a set of points in real-space is given by the intersect of a set of lines corresponding to the real-space points in Hough-space.

The \HT is used by the TFP to find the tracks of charged particles with $\pT > 3\GeV$ in the \rphi plane, which has a better resolution than the \rz plane, independently for each $\eta$-$\phi$ sub-sector within each processing octant.

The radius of curvature, $R$ (cm), for a charged particle's trajectory can be described as a function of the particle's \pT, charge $q$ and of the homogeneous magnetic field, $B$, in which it is travelling:

\begin{equation}
R = \frac{\pt}{0.003\,qB} \;
\label{eq:R}
\end{equation}

A stub associated with such a charged particles' trajectory and which has coordinates ($r$,$\varphi$) is related to $R$ by:

\begin{equation}
\frac {r}{2\,R} = \sin\left(\varphi-\phi\right)\;
\label{eq:stub_R}
\end{equation}

where $\phi$ is the angle of the track in the transverse plane at the origin~\cite{markthesis}. 

From Equation~(\ref{eq:R}), it can be seen that tracks with large \pT ($> 2-3\GeV$) have a large $R$.
This allows for the use of the small angle approximation to simplify the right-hand side of Equation~(\ref{eq:stub_R}).
If energy losses are neglected from processes such as multiple scattering and Bremsstrahlung, the position of the stubs will be compatible with the trajectory described by Equation~(\ref{eq:stub_R}) (\ie $R$ can be assumed to be constant).

Therefore, Equations~(\ref{eq:R}) and~(\ref{eq:stub_R}) can be combined to describe how a stub's position in ($r, \varphi$) can be transformed into a line in (\qpt,$\phi$) Hough-space where:

\begin{equation}
\phi = \varphi - \frac{0.0015\,qB}{\pt}\cdot r \;
\label{eq:localHT}
\end{equation}

If a particle produces multiple stubs, such as those represented by the six dots in the left-hand side of Figure~\ref{fig:HT}, they can be used to identify a track candidate through their intersection point, which is also shown in the figure (right-hand side).
The coordinates of the intersection point in Hough-space also provides an initial estimate of the track's $p_{T}$ and $\phi$ helix parameters, as defined in Section~\ref{subsec:definitions}.

\begin{figure}[!h]
\centering
\includegraphics[width=0.80\textwidth]{figs/tk-upgrade/HT.pdf}
% where an .eps filename suffix will be assumed under latex,
% and a .pdf suffix will be assumed for pdflatex; or what has been declared
% via \DeclareGraphicsExtensions.
\caption{Illustration of the Hough Transform. The left-hand side shows the trajectory of a single charged particle in one quarter of the tracker barrel in the $x-y$ plane. The right-side illustrates the six lines in Hough-Space which correspond to the six dots in real space.}
\label{fig:HT}
\end{figure}

As the radii of the stubs gives the line gradients in Hough-space, they will always be positive.
Therefore, it is preferable to measure the stub coordinates relative to the point where the corresponding track crosses a cylinder in the $x-y$ plane of radius $T$.
By choosing an appropriate value of $T$, the transformation of $r \rightarrow r_{T}$ and $\phi \rightarrow \phi_{T}$ increases the size of Hough Transform phase space used.
This allows more precise measurements of the intersection point to be made and consequently fewer fake tracks and duplicates are found.
The optimal value of $T$ was determined to be 58\cm~\cite{TMTT_JINST}.

As the $R$ for the lowest \pT track (3\GeV) to be considered is greater than the outer radius of the tracking detector ($r$ = 1.2~m), all relevant particles are expected to traverse at least six barrel layers or endcap disks. 
The threshold for the identification of a track candidate however, is set at a minimum of five detector layers or disks in order to allow for detector or readout inefficiencies. 
This threshold can be further reduced to four layers to account for the reduced geometric coverage between $0.89 < \eta < 1.16$ or for dead detector layers or disks without significantly increasing the volume of data considered.

A detailed description of the firmware implementation of the \HT for the demonstrator system is given in~\cite{TMTT_JINST,IEEE}.

\subsubsection{Kalman Filter}\label{subsubsec:KF}
While the \HT is highly efficient at finding genuine tracks, it was found in simulation that over half of the genuine tracks found contain at least one incorrectly associated stub.
If ignored, the presence of such incorrectly associated stubs would degrade the resolution of the helix parameters fitted to reconstructed tracks that are associated with a particle.
In addition, simulation studies indicated that approximately half of the track candidates created by the Hough Transform did not have stubs associated to the same particle in at least four tracker layers/disks (\ie were fake).
Therefore, a \KF was developed to precisely fit the track parameters given its ability to simultaneously remove these incorrectly associated stubs and ``fake'' tracks while obtaining the best possible estimate of the reconstructed track’s helix parameters.

While the \KF is the optimal filter for linear systems and, the optimal linear filter for non-linear systems, it also has several aspects that make it suitable for FPGA implementation compared to global track fitting methods~\cite{Fruhwirth:1987fm}, namely:

\begin{itemize}
\item {The matrices involved are small and their size is independent of the number of measurements, minimising the logic required to implement them;}
\item {The only matrix inversion involved is for a small matrix.}
\end{itemize}


Figure~\ref{fig:KF} illustrates the \KF filtering and fitting process for a track candidate in the \rz plane of the barrel, where each line segment represents the predicted track trajectory at a given stage in the fitting process.
The \KF begins with an estimate of the track parameters and their covariance matrix (containing the measurement uncertainties) from the \HT array, which along with the $\eta$-$\phi$ segment assignment, is known as the \emph{state}.
Stubs are iteratively added to the predicted state in order to produce an updated state estimate formed of the weighted combination of the predicted state and the measurement.
This weighting, known as the \emph{Kalman gain}, is derived from the relative uncertainties of the predicted state and measurement and is used to control how the state's track parameters are updated.
Therefore, with every additional measurement added to the state, the uncertainty of the state's estimate decreases.
This is illustrated in Figure~\ref{fig:KF} by the size of the shaded area around the line segments decreasing with every stub added~\cite{TMTT_JINST}.

\begin{figure}[!h]
\centering
\includegraphics[width=\textwidth]{figs/tk-upgrade/kf_states.pdf}
% where an .eps filename suffix will be assumed under latex,
% and a .pdf suffix will be assumed for pdflatex; or what has been declared
% via \DeclareGraphicsExtensions.
\caption{An illustrated example of the \KF filtering procedure for a track candidate in the \rz plane of the barrel as described in the text~\cite{TMTT_JINST}.
}
\label{fig:KF}
\end{figure}

The filtering of state following each update makes use number of configurable criteria, including \pT, $\chi^2$, and the minimum number of stubs from PS modules.
It can also be configured to take into account and skip missing layers when the expected stub is either missing or deemed to be incorrectly associated with the track.

In the event multiple stubs are found on the same layer, each can be propagated with up to the four best states being kept and presented to a final state selector.
Preference is given to states with the fewest missing layers and the smallest $\chi^2$.
An example of this is shown in Figure~\ref{fig:KF}, where the two dashed line segments correspond to the projected trajectories of the track from the stubs labelled 2a and 2b.
As each stub is compatible with the expected track trajectory at that stage, both are propagated.
The track associated to stub 2b however, is rejected after stub 4 has been added to the track propagated from stub 2a, due to failing a $\chi^2$ cut in two consecutive layers.

%The implementation of 	
%The final fit is always performed after a fixed period of time.
%Consequently there is no truncation in the traditional sense given that all tracks candidates are read out.
%Therefore, while events which a large number of track candidates, such as dense jets, may only be partially filtered, they will always be read out.
%
A full description of the Kalman formalism is given in~\cite{Fruhwirth:1987fm}.
The details of the \emph{TMTT} project's implementation of the Kalman Filter using FPGAs for online track reconstruction is given in~\cite{TMTT_JINST,SSummers}.

\subsubsection{Duplicate Removal}
Over half of the track candidates at input to the DR are duplicate tracks created by the HT.
Instead of comparing pairs of tracks to see if they are the same, a more elegant and subtle DR algorithm is used which takes into account how the \HT produces these duplicate tracks.
This approach is illustrated in Figure~\ref{fig:DR}.
In this example, the \HT has produced candidates in the yellow and two green HT cells from the five stubs, which correspond to the blue lines in Hough Space, that have been produced by a single particle.
As all three candidates however, contain the same stubs, they will be fitted by the \KF with identical helix parameters in the same (yellow) cell regardless of the original HT cell.
By comparing a track's fitted parameters with the \HT cell in which they were initially found, any track whose fitted parameters does not correspond to the same HT cell in which the \HT found the track in is rejected.

\begin{figure}[!h]
\centering
\includegraphics[width=0.80\textwidth]{figs/tk-upgrade/A50_algo.pdf}
% where an .eps filename suffix will be assumed under latex,
% and a .pdf suffix will be assumed for pdflatex; or what has been declared
% via \DeclareGraphicsExtensions.
\caption{Illustration of how duplicates are formed by the \rphi \HT, as discussed in the text~\cite{TMTT_JINST}.}
\label{fig:DR}
\end{figure}

There is however, a small subtlety, as the DR algorithm rejects a small number of non-duplicated tracks due to resolution effects resulting from the discretised implementation of Hough Transform arrays, , the algorithm performs second pass through the rejected tracks.
This second pass looks for tracks which have fitted parameters that do not correspond to the HT cell of a track from the first pass.
As such tracks are probably not duplicates, they are recovered.

A more detailed description of the firmware implementation of the \DR for the demonstrator system is discussed in~\cite{TMTT_JINST}.

\section{Simulation Studies}\label{sec:TmttSimStudies}
This section presents a number of simulation studies that were undertaken as part of the development of the \emph{TMTT} finder demonstrator system both before and following the 2016 review.
All of the results discussed were obtained using simulated \ttbar events with an average PU ($<\textrm{PU}>$) of 200 interactions and use digitised output from the \HT.
The set of metrics used to evaluate the performance of the proposed track finder systems in the 2016 review are used for the results presented.
These are defined below in Section~\ref{subsec:definitions}.

As mentioned in Section~\ref{sec:tk-upgrade}, at the time of the review the flat barrel geometry described earlier was used for all the studies undertaken, as depicted in the lower diagram in Figure~\ref{fig:trackerlayout}.
As such, unless stated otherwise, the results discussed below use the flat barrel geometry instead of the current tilted geometry.

The results presented in Section~\ref{subsec:chi2} for the linearised $\chi^{2}$ track fitting algorithm involve the use of a \emph{Seed Filter} (SF) stage that was run after the \HT stage.
The SF removes stubs in a \HT cell that are inconsistent with a straight line in the \emph{$r-z$} plane.
This process filters out incorrectly reconstructed tracks and stubs that were incorrectly assigned to tracks.
It was found that using a SF stage before the \KF stage did not improve the overall performance of the system due to the effectiveness of the \KF's filtering.

\subsection{Definitions}\label{subsec:definitions}
A common set of parameters and metrics are used throughout the next sections of this chapter to describe tracks and how well the track fitters have reconstructed them.
They are defined as below in Sections~\ref{subsec:helixParameter}-~\ref{subsec:recoTracks}.

\subsubsection{Helix Parameters}\label{subsec:helixParameter}
The helical trajectory of a charged particle at the impact point is described by five helix parameters.
In the CMS these parameters are defined as:

\begin{itemize}
\item $\bm{p_{T}}$ - the transverse momentum of the track;
\item $\bm{\phi_{0}}$ - the track angle in the transverse plane;
\item $\bm{z_{0}}$ - the \emph{longitudinal} impact parameter, \ie the distance in z from the point of closest approach to the interaction point;
\item $\bm{\cot(\theta)}$ - the cotangent of the \emph{dip} (polar) angle, related to $\eta$ by $\cot(\theta) = \sinh (\eta)^{-1}$;
\item $\bm{d_{0}}$ - the \emph{transverse} impact parameter, \ie the distance of the track vertex from the interaction point in the $x-y$ plane. 
\end{itemize}

As the \HT and track fitting algorithms discussed all assume that all tracks originate at the interaction point, $d_{0}$ is not given in the results below as it is fixed to zero.

\subsubsection{Reconstructed Tracks}\label{subsec:recoTracks}
The common definitions of track reconstruction efficiency~\cite{TMTT_JINST} used for the three proposed L-1 Track Finder systems are used for the results presented in this chapter:

\begin{itemize}
\item The reconstruction efficiency ($\epsilon$) is measured relative to all generated charged particles from the primary interaction that produce a track that satisfies the following definition: produces stubs in at least four layers/disks of the tracker, $\pT > 3\GeV$, $|\eta| < 2.4$, $|z_{0}| < 30\cm$ and $d_{xy} < 1\cm$, where $d_{xy}$ is the distance in the $x-y$ plane from the point of closest approach to the interaction point.
\item A track is defined as being correctly reconstructed or \emph{matched} if the reconstructed track has stubs associated to the particle in at least four tracker layers/disks. Tracks which fail this matching criteria are known either as \emph{unmatched} or \emph{fake} tracks.
\item If the reconstruction of a charged particle produces more than one track, these additional tracks are considered to be \emph{duplicates}.
\item If all a reconstructed track's stubs originated from the same particle, the track is defined as being \emph{perfectly} reconstructed ($\epsilon_{P}$). 
\end{itemize}

This stricter definition of \emph{perfect} track reconstruction efficiency is typically used in quoting results from the entire chain (\ie all four components of the TFP discussed in Section~\ref{subsec:TFP}).
Otherwise, the nominal definition of track reconstruction efficiency is used as the presence of stubs incorrectly associated with a track is to be expected if only part of the TFP chain has been run.
Where appropriate, the results for both definitions are given.

Similarly, the common definitions for the resolution of the track parameters used for both the three proposed L-1 Track Finder systems and in the Phase-II Upgrade of the CMS tracker Technical Design Report~\cite{P2TrackerTDR} are used for the results discussed below.
For all of the track parameters quoted, the resolution was defined as the difference between the reconstructed track's helix parameter and the matched simulated track's helix parameter.


\subsection{Linearised $\chi^{2}$ Track Fitting Studies}\label{subsec:chi2}
Three different track fitting algorithms were explored for the track fitter component of the TFP used in the 2016 hardware demonstrator review: a Kalman Filter, a Linear Regression (LR) algorithm and a Linearised $\chi^{2}$ Fit algorithm.

The development of a Kalman Filter was motivated by its ability to filter incorrectly assigned stubs from tracks and the remove fake track candidates at the same time as precisely fitting track parameters.
As the \KF provided the best perfect track reconstruction efficiency and fake track candidate rejection rate out of the three fitting algorithms explored, it was selected as the baseline fitter for the TFP in the 2016 review.

A Linear Regression (LR) track fitting algorithm was developed as an alternative to the KF~\cite{TMTT_FLP}.
As high \pT tracks should form a straight line in the \emph{\rphi} and \emph{\rz} planes, the linear nature of the LR fit's mathematics makes it well suited to perform independent fits in each plane using minimal latency and resources.
This algorithm also required the use of the SF in order for it to deliver optimal performance.

A linearised $\chi^{2}$ track fit was the first fitting algorithm to be studied by the \emph{TMTT} project.
Given the limited time and resource constraints at the start of the \emph{TMTT} project, there was the urgent need to quickly evaluate potential track fitting algorithms so that the optimal one could be implemented in a complete track finder system for the 2016 review.
Following discussions with both the \emph{tracklet} and \emph{AM} projects, it was decided a linearised $\chi^{2}$ fit based on the algorithm proposed by the \emph{tracklet} project would be investigated.
A linearised $\chi^{2}$ fit determines improved helix parameters for the track candidate by calculating the residuals between the stubs and the seeded track that minimise the $\chi^{2}$ of the fit.
The general form of the $\chi^{2}$ fit and the derivation of the track derivatives used by the algorithm were provided in a private communication~\cite{CMS_DN-14-043} and were used to produce a \emph{TMTT} implementation of it.


In Section~\ref{subsubsec:chi2maths}, the general form of the $\chi^{2}$ fit is described, detailing how these hit residuals were used to obtain a fit of a track's helix parameters.
Following this, a discussion of the development and outcomes for the $\chi^{2}$ fit algorithm is given in Section~\ref{subsubsec:chi2software}.

%The general form of the $\chi^{2}$ fit describing how these hit residuals are used to obtain a fit of a track's helix parameters is detailed in Section~\ref{subsubsec:chi2maths}.
%A discussion of the development and outcomes of the software implementation of the fitting algorithm are given in .
%The calculation of the track derivatives for the barrel layer hits and endcap disk hits used by the algorithm, included a correction factor for $\phi$ in the outer disks to account for the fact that these modules do not point directly towards the interaction point, which is described in Appendix~\ref{app:chi2}.

\subsubsection{General Form of a $\chi^{2}$ Fit}\label{subsubsec:chi2maths}
For the general form of a $\chi^{2}$ fit for a track, $f$, described by its helix parameters, $\overrightarrow{h}$, and the position of its hits (\ie stubs), $s_{i}$, where $i$ labels the different measurements, the expected trajectory of the track, $f_{i}$, is initially linearly expanded around the estimate of the helix parameters $\overline{h}$:

\begin{equation}
f_{i}(\overrightarrow{h} ) = f_{i}(\overrightarrow{h} + \delta \overrightarrow{h}) \;
                           = f_{i}(\overline{h}) + \delta \overrightarrow{h} \frac{\partial f_{i}}{\partial \overrightarrow{h}} + \mathcal{O}(\delta \overrightarrow{h}^{2}) \;
\label{eq:chi1}
\end{equation}

The $\chi^{2}$ of such a track is expressed as:

\begin{equation}
\begin{split}
\chi^{2} &= \sum_{ij} \big(f_{i}(\overrightarrow{h}) - s_{i} \big) V^{-1}_{ij}  \big(f_{j}(\overrightarrow{h}) - s_{j} \big)  \\
         &= \sum_{ij} \big( f_{i}(\overline{h})  - s_{i} + \delta \overrightarrow{h} \frac{\partial f_{i}}{\partial \overrightarrow{h}} \big) V^{-1}_{ij}  \big( f_{j}(\overline{h})  - s_{j} + \delta \overrightarrow{h} \frac{\partial f_{j}}{\partial \overrightarrow{h}} \big)  \\
         &= \sum_{ij} \big( \delta f_{i} + \delta \overrightarrow{h} \frac{\partial f_{i}}{\partial \overrightarrow{h}} \big) V^{-1}_{ij}  \big( \delta f_{j} + \delta \overrightarrow{h} \frac{\partial f_{j}}{\partial \overrightarrow{h}} \big)
\end{split}
\label{eq:chi2}
\end{equation}

where $\delta f_{i} \equiv f_{i}(\overline{h}) - s_{i}$ are the residuals between the expected position of the track (given by the seed helix parameters) and the position of the track given by the stub, and $V^{-1}_{ij} = diag(\sigma^{2}_{ii})$ is the variance matrix that describes the uncertainty associated with the measurement of the stubs.

By minimising the $\chi^{2}$, $\delta h$ can be determined:

\begin{equation}
0 = \frac{\partial \chi^{2}}{\partial \delta \overrightarrow{h_{k}}} = \sum_{ij} \frac{\partial f_{i}}{\partial \delta \overrightarrow{h_{k}}} V^{-1}_{ij} ( \delta f_{j} + \delta \overrightarrow{h} \frac{\partial f_{j}}{\partial \overrightarrow{h}} ) + \sum_{ij}	( \delta f_{i} + \delta \overrightarrow{h} \frac{\partial f_{i}}{\partial \overrightarrow{h}} ) V^{-1}_{ij} \frac{\partial f_{j}}{\partial \delta \overrightarrow{h_{k}}}  \;
\label{eq:chi3}
\end{equation}

By defining the matrices $D_{ij} = \frac{\partial f_{i}}{\partial h_{k}}$ and $M = D^{T} V^{-1} D$, Equation~(\ref{eq:chi3}) can be rewritten and solved for $\delta h$:

\begin{equation}
0 = D^{T} V^{-1} \delta f + M \delta h \Rightarrow \delta h = - M^{-1} D^{T} \delta f \;
\label{eq:chi4}
\end{equation}

Therefore Equation~(\ref{eq:chi4}) provides a simple linear form for how the track helix parameters should be updated for a set of residuals with respect to the seed track candidate.

Similarly the $\chi^{2}$ of the fit can also be expressed in a linear form:

\begin{equation}
\begin{split}
\chi^{2} &= (\delta f + D \delta h)^{T}(\delta f + D \delta h) \\
         &= (\delta f - DM^{-1}D^{T}\delta f)^{T} (\delta f - DM^{-1}D^{T}\delta f) \\
         &= \delta f^{T} (1- DM^{-1}D^{T}) (1- DM^{-1}D^{T}) \delta f \\
         &= \delta f^{T} (1- DM^{-1}D^{T}) \delta f \\
         &= \delta f^{T} \delta f - \delta f^{T} DM^{-1}D^{T} \delta f \\
         &= \chi^{2}_{seed} + \delta f^{T} D \delta \overrightarrow{h} 
\end{split}
\label{eq:chi5}
\end{equation}

As the linear forms of Equations~(\ref{eq:chi4}) and~(\ref{eq:chi5}) consist of repeated addition and multiplication operations of the matrices involved, they are naturally suitable for implementation on an FPGA.

This is because while FPGAs can easily perform such operations, potential complications arise when considering the calculation of the track derivatives that form the elements of $D$.
Determining these elements would not be trivial given the presence of a large number of division operations and trigonometric functions for the endcaps' derivatives.
Therefore, any implementation in firmware for an FPGA will require the use of lookup tables containing the precomputed values of the derivatives in order to quickly update a track's helix parameters without exceeding latency requirements.

\subsubsection{$\chi^{2}$ Track Fitter Software Implementation}\label{subsubsec:chi2software}
%From equations~(\ref{eq:chi4}) and~(\ref{eq:chi5}) and the track derivatives derived in Appendix~\ref{app:chi2}, a software implementation of the linearised $\chi^{2}$ track fit algorithm was developed.
From Equations~(\ref{eq:chi4}) and~(\ref{eq:chi5}) and the track derivatives derived in~\cite{CMS_DN-14-043}, a software implementation of the linearised $\chi^{2}$ track fit algorithm was developed.

Initially the algorithm was implemented using floating point calculations in order to both debug and optimise its performance.
This process involved confirming that the track derivatives and stub residuals were correctly calculated and the track finding efficiencies and parameter resolution of the $\chi^{2}$ were comparable to the \emph{tracklet} project's software implementation at the time.

Using the exact expressions for the track derivatives with floating point precision and calculating each derivative as required however, would not be feasible to implement in any future FPGA based firmware given the system's resources and constraints.
Therefore, the lowest order approximations of the track derivatives (containing the fewest possible number of free parameters) were determined in order to develop a version of the algorithm that used tabulated track derivatives and digitised variables and calculations that gave performance comparable to the original derivatives with floating point precision
The performance of both of the versions of the algorithm is discussed below in terms of their track reconstruction efficiencies and the resolution of the fitted track parameters.

Following the development of the ``discretised'' mathematics version of the algorithm, an iterative filtering process based on the residuals of the stubs was explored.
Stubs that were incorrectly associated with a track or were associated with an incorrectly reconstructed track were expected to have larger residuals than those which don't.
Multiple iterations of the fitting algorithm were run, where stubs that have a residual above a threshold value being discarded after each iteration prior to the track being refit.
The performance of differing numbers of fitting iterations were compared against each other and the \KF in terms of the purity of the matched tracks, the fraction of fake tracks found, and the impact on the track parameter resolutions.


\begin{table}[htbp]
\topcaption {Track finding performance on simulated \ttbar events with a $<\text{PU}>$ of 200 events, after the \HT and the full chain for both the exact floating point and discretised calculations of the track derivatives used by the $\chi^{2}$ track fit.
}

\label{tab:chi2-exactVsApprox}
 \centering
% \resizebox{\textwidth}{!}{
% This right-aligns numbers in column, but centers them under column title.
 \begin{tabular}{cccccc}
   \hline
   \bf{Stage} & \bf{$\bm{\epsilon}$ [\%]} & \bf{$\bm{\epsilon_{P}}$ [\%]} & $\bm{<N_{tracks}>}$ & \bf{Fakes [\%]} & \bf{Duplicates [\%]}  \\
        \hline
   HT &  97.0 & 43.1 & 351.2 & 43.9 & 37.0 \\  
   \hline
   $\chi^{2}$+DR & 95.0 & 85.8 & 86.4 & 15.7 & 9.5 \\
   (floating point) & & & & & \\
   \hline
   $\chi^{2}$+DR & 94.9 & 85.6 & 87.4 & 15.5 & 10.9 \\  
   (discretised) & & & & & \\   
%   \hline
%   KF+DR & 94.1 & 94.1 & 82.1 & 21.1 & 4.5 \\
   \hline
   
 \end{tabular}%}
\end{table}

Table~\ref{tab:chi2-exactVsApprox} compares the tracking performance between the floating point and ``discretised'' mathematics implementations of the $\chi^{2}$ track fitting algorithm and the raw track finding output from the \HT.
It can be seen that whilst the \HT finds tracks with high efficiency, over half have at least one incorrectly associated stub and a significant number of the tracks found were fake or duplicated tracks.
Both floating point and discretised mathematics implementations give comparable results, indicating that the approximations made were acceptable.
The $\chi^{2}$ track fit increases the purity of the reconstructed tracks by a factor of two and eliminates the majority of the fake tracks, while the \DR algorithm removes the majority of the duplicates.


\begin{figure}[htb]
\centering
\includegraphics[width=0.495\textwidth]{figs/tk-upgrade/results-chi2fitter/ptRelResVsEta_It_1_ApproxVsExact.pdf}
\includegraphics[width=0.495\textwidth]{figs/tk-upgrade/results-chi2fitter/phi0ResVsEta_It_1_ApproxVsExact.pdf}
\\
\includegraphics[width=0.495\textwidth]{figs/tk-upgrade/results-chi2fitter/z0ResVsEta_It_1_ApproxVsExact.pdf}
\includegraphics[width=0.495\textwidth]{figs/tk-upgrade/results-chi2fitter/cotThetaResVsEta_It_1_ApproxVsExact.pdf}
\caption{
Relative \pt resolution, $\phi_{0}$ resolution, $z_{0}$ resolution and $cot(\theta)$ resolution measured for primary reconstructed tracks in simulated \ttbar events with a $<\textrm{PU}>$ of 200 events for the floating point (red) and discretised mathematics (blue) implementations of the linearised $\chi^{2}$ fit algorithm for a single fitting iteration.
}
\label{fig:chi2HelixParametersResVsEtaApproxVsExact}
\end{figure}

%Running multiple iterations of the fitting algorithm had been previously been considered in the context of improving upon the resolution of a fitted track's helix parameters, but had been shown to produce only marginal improvements.

Figure~\ref{fig:chi2HelixParametersResVsEtaApproxVsExact} shows that resolutions of the four track parameters as a function of $\eta$ for primary tracks in \ttbar events with a $<\textrm{PU}>$ of 200 events for both the floating point and discretised implementations of the algorithm.
The resolution of each of the helix parameters compares well not just between the two implementations of the algorithm's mathematics, but also with those obtained for the barrel region by the offline track reconstruction which is able to use all information from the detector with more sophisticated reconstruction techniques~\cite{P2TrackerTDR}, guaranteeing their usefulness to the L-1 trigger.
The increasing degradation of the track parameters with increasing pseudorapidity results from combined effects of the reduced hit precision in the endcap disks, shorter effective lever arm available and the increasing amount of material that particles pass through.

In order to filter out these incorrectly assigned stubs from matched tracks, and also remove fake tracks, the residuals calculated for each stub following the fit were considered.
These ``fake'' stubs were expected have large residuals compared to stubs correctly associated to genuine tracks or stubs belonging to a fake track.

Therefore, in order to decrease the fake rate and increase the matched track purity the stub with the worst/largest residual was compared against a configurable threshold.
If the residual failed to meet the threshold criteria, it would be removed from the track and the track would be refitted using only its remaining stubs.
This process was repeated until the latency budget was exceeded/no further stubs were removed, with no further consideration of the remaining stubs' residuals following the final threshold.

During the optimisation of this stub quality check it was found that a track could end up having fewer than the minimum of four stubs required to be considered track candidate - thus potentially discarding a matched track by mistake.
To avoid this while still retaining the improved matched track purity and reduced fake rate, a looser residual threshold was applied for tracks only containing four stubs.

\begin{table}[htbp]
\topcaption {Track finding performance on simulated \ttbar events with a $<\textrm{PU}>$ of 200 events, for one to four fitting iterations, $N_{It}$, of the $\chi^{2}$ track fit and for the \KF. 
Further fitting iterations are not shown as further improvement was observed.
}
\label{tab:chi2_iterations}
  \centering
% This increases column spacing.
%  \resizebox{\textwidth}{!}{
% This right-aligns numbers in column, but centers them under column title.
 \begin{tabular}{ccccccc}
   \hline
   \bf{Track Fitter} & $N_{It}$ & \bf{$\bm{\epsilon}$ [\%]} & \bf{$\bm{\epsilon_{P}}$ [\%]} & $\bm{<N_{tracks}>}$ & \bf{Fakes [\%]} & \bf{Duplicates [\%]}  \\
   \hline
   $\chi^{2}$+DR & 1 & 94.9 & 85.6 & 87.4 & 15.5 & 10.9 \\  
   & 2 & 93.8 & 91.0 & 73.8 & 6.6 & 7.7 \\
   & 3 & 93.1 & 91.0 & 71.4 & 5.3 & 6.9 \\
   & 4 & 93.0 & 91.0 & 71.1 & 5.2 & 6.8 \\
   \hline
   KF+DR & - & 94.1 & 94.1 & 82.1 & 21.1 & 4.5 \\
   \hline
   
 \end{tabular}%}
\end{table}

Table~\ref{tab:chi2_iterations} illustrates how the tracking performance of the linearised $\chi^{2}$ fitting algorithm improves over successive fitting iterations and how it compares to the \KF.
Performing one additional fitting iteration considerably improved the performance of the $\chi^{2}$ track fit, as the removal of incompatible stubs reduced the fraction of fake tracks by over a half and considerably increased the purity of the fitted matched tracks.
Further successive fitting iterations however, yield diminishing returns, with no further improvements seen following four iterations of the $\chi^{2}$ fitting algorithm.
This was not unexpected, as with the mean number of hits associated to a track being seven, only up to three stubs (across four fitting iterations) can be removed for the majority of tracks.
In contrast, the \KF achieves a tracking efficiency comparable to the $\chi^{2}$ fitter after two fitting iterations but with none of the matched tracks containing any incorrect stubs. 
This suggests that if a more sophisticated method of removing bad quality stubs were used in the linearised $\chi^{2}$ fitter, very few matched tracks would be discarded with improved purity.

Figure~\ref{fig:chi2HelixParametersResIterationsComparison} compares the the helix parameter resolutions obtained by the $\chi^{2}$ track fit after one and four fitting iterations with those achieved by the \KF.
It can be seen that the additional fitting iterations performed by the $\chi^{2}$ fitting algorithm considerably improved each of the track parameter's resolution obtained in the forward regions.
While the $\chi^{2}$ fitter's \pt relative and $\phi_{0}$ resolutions were comparable to those achieved by the \KF, its $z_{0}$ and $cot(\theta)$ resolutions at high pseudorapidity were considerably less precise than those obtained with the \KF.
It was found that the \KF fitter's $z_{0}$ and $cot(\theta)$ resolutions in the forward regions were improved with the inclusion of higher order terms in the relevant track derivatives.
These additional terms were not considered in the``discretised'' mathematics implementation of the $\chi^{2}$ track fitting algorithm as they introduced additional free parameters that would have resulted in their associated lookup tables being too large to implement on existed FPGAs.

\begin{figure}[htb]
\centering
\includegraphics[width=0.495\textwidth]{figs/tk-upgrade/results-chi2fitter/ptRelResVsEta_IterationComparison.pdf}
\includegraphics[width=0.495\textwidth]{figs/tk-upgrade/results-chi2fitter/phi0ResVsEta_IterationComparison.pdf}
\\
\includegraphics[width=0.495\textwidth]{figs/tk-upgrade/results-chi2fitter/z0ResVsEta_IterationComparison.pdf}
\includegraphics[width=0.495\textwidth]{figs/tk-upgrade/results-chi2fitter/cotThetaResVsEta_IterationComparison.pdf}
\caption{
Relative \pt resolution, $\phi_{0}$ resolution, $z_{0}$ resolution and $cot(\theta)$ resolution measured for primary reconstructed tracks in simulated \ttbar events with a $<\textrm{PU}>$ of 200 events for the discretised mathematics implementation of the linearised $\chi^{2}$ fit algorithm for one (black) and four (blue) fitting iterations. The \KF (red) is also included for comparison.
}
\label{fig:chi2HelixParametersResIterationsComparison}
\end{figure}

Following the parallel development of both the linearised $\chi^{2}$ track fit and the \KF algorithms, it was decided that development of the former would be discontinued.
This decision was made as the \KF was capable of achieving both a higher track finding efficiency with 100.0\% purity for matched tracks and superior $z_{0}$ and $cot(\theta)$ resolutions in the forward regions.
In addition, considerable progress had been made with a firmware implementation of the \KF.

In contrast, the linearised $\chi^{2}$ track was not competitive in terms of track resolution and reconstruction ability, especially with respect to the stricter tracking efficiency definition.
There were also concerns over the potential feasibility of tabulating all (or the most frequently used) track derivatives in the endcap disks for FPGAs that were commercially available at the time.

\subsection{Tracking at low transverse momenta studies}\label{subsec:Tmtt2GeV}
The flexibility to reconstruct tracks down to a lower \pT threshold of 2\GeV is potentially desirable and so the impact of this potential requirement on the performance of the proposed track-finder system was studied.

These studies were initially undertaken as part of the robustness studies required for the 2016 demonstrator review, which investigated the impact that lowering the track reconstruction \pT threshold had on the \HT.
Therefore, the results for these studies were produced using the flat barrel geometry.

Following the 2016 review, the studies into tracking at lower transverse momenta were further developed by optimising the \KF algorithm .
These results were produced with the preferred tilted barrel geometry.

As the number and width of \qpt \HT columns used varies from the standard fixed number and size of columns typically used, thus varying the \pT resolution available, results in the following subsections are expressed as a function of 1/\pT instead of \pT.

\subsubsection{Hough Transform Optimisation and Results}\label{subsubsec:lowPtOptHT}
Lowering the \pT threshold from 2\GeVc required modifying the GP and HT configuration parameters to ensure adequate duplication in $\phi$ and increasing the number of \qpt columns by 50\% to take into account the increased \pt range whilst maintaining the same precision.
The increased number of \qpt columns increases the required FPGA resources by 50\% and the output data rate from the \HT by a factor of 2.2.

Without applying further modifications, there was a considerable degradation in the track reconstruction efficiency in the range $2\GeVc < \pt \leq 2.7\GeVc$, due to these low momenta tracks being dominated by \MS.
This results in a significant fraction of stubs not intersecting within a single \HT cell and thus failing to exceed the threshold criteria and generate track candidates.
To mitigate against these efficiency losses, the precision of the \HT cells along \qpt and $\phi_{T}$ for the range $2\GeVc < \pt \leq 2.7\GeVc$ was reduced by a factor of two (\ie $2 \times 2$ cells were merged)
The concept of a variable precision \HT had been implemented in firmware as part of another series of studies.

In addition, the \KF state $\chi^2$ cuts for tracks with $\pt \leq 2.7\GeVc$ were optimised to reflect the increased hit position uncertainty from the decreased precision of the hits in these \HT cells.
The optimisation of the \KF state $\chi^2$ cut was done to reduce the number of duplicate and fake tracks as far as possible without negatively impacting on the \HT track reconstruction efficiency.
Figure~\ref{fig:2GeVFlatEff} shows how the tracking efficiency improves following the use of the variable precision \HT with and without optimised \KF state cuts after both the \HT and the full demonstrator chain. 

\begin{figure}[htb]
\centering
\includegraphics[width=0.495\textwidth]{figs/tk-upgrade/results-lowPtTracking/htTrackingEffVsInvPtFlatGeometry_5000.pdf}
\includegraphics[width=0.495\textwidth]{figs/tk-upgrade/results-lowPtTracking/kfTrackingEffVsInvPtFlatGeometry_5000.pdf}
\caption{The post-\HT (left) and post-\KF (right) tracking efficiency for tracks with $\pt > 2\GeVc$ for \ttbar events with a $<\textrm{PU}>$ of 200 events. The default configuration where only the number of \qpt columns were increased are shown in red and the configuration with the increased number of columns, HT cell merging and \KF state cuts optimisation shown in blue).
}
\label{fig:2GeVFlatEff}	
\end{figure}

%%% Disucss table here
Table~\ref{tab:trackFindingPerformance2GeVHT} shows the impact that the decreased precision \HT cells and optimised \KF state cuts have on tracking performance both following the \HT and after the full chain has been run.
It was clear that whilst the merging of adjacent \HT cells recovers tracks that did not previously intersect within a single \HT cell, the tracking efficiency following running the full chain was significantly less than that post-\HT.
As can be seen in Figure~\ref{fig:2GeVFlatEff}, these losses occur for tracks where the particle's \pT ($\frac{1}{\pT}$) is less (greater) than 3\GeV ($0.33/\GeV$), as the \KF does not take the effects of \MS into account.
This shortcoming of the \KF also accounts for it not being as efficient at removing fake tracks, with an observed increase of 5\% in the fraction of fakes reconstructed.
The duplicate removal algorithm however, remains effective at removing almost all the duplicates.

\begin{table}[htbp]
\topcaption {Track finding performance on simulated \ttbar events with a $<\textrm{PU}>$ of 200 events, after the \HT and the full chain have been considered for the configurations of only increasing just the number of \qpt columns (\emph{Default}),and also applying \HT cell merging and the optimised \KF state cuts (\emph{Optimised}).
}
\label{tab:trackFindingPerformance2GeVHT}
  \centering
% This increases column spacing.
%  \resizebox{\textwidth}{!}{
% This right-aligns numbers in column, but centers them under column title.
\begin{tabular}{cccccc}
   \hline
   \bf{Configuration} & \bf{Stage} & \bf{$\bm{\epsilon}$ [\%]} & $\bm{<N_{tracks}>}$ & \bf{Fakes [\%]} & \bf{Duplicates [\%]}  \\
        \hline
    Default & \bf{HT}     & 93.6 & 713.2 & 34.0 & 44.5 \\  
    & \bf{Full chain}     & 89.2 & 193.9 & 21.1 & 5.1 \\      
%   \hline
%    Merge & \bf{HT}     & 94.6 & 799.2 & 40.5 & 39.7 \\  
%    & \bf{Full chain}     & 89.8 & 206.5 & 25.4 & 3.9 \\      
    \hline
    Optimised & \bf{HT}     & 94.6 & 799.2 & 40.5 & 39.7 \\  
    & \bf{Full chain}     & 90.0 & 210.4 & 26.3 & 3.8 \\      
   \hline
   
 \end{tabular}%}
\end{table}

The impact of the \KF not considering the impact of \MS increasing the uncertainty in a hit's position at lower lower transverse momenta is further illustrated by Figure~\ref{fig:2GeVFlatChi2Ndf}, which shows the distributions of $\chi^{2}$ per number of degree of freedom ($\frac{\chi^{2}}{ndf}$) as a function of $\frac{1}{\pT}$ for genuine tracks produced by the \KF.
If all sources of uncertainties were accounted for, the ideal distribution of $\frac{\chi^{2}}{ndf}$ would be unity for all values of \pT ($\frac{1}{\pT}$), in contrast to the observed dramatic increase above approximately 3\GeV ($0.33/\GeV$).

\begin{figure}[htb]
\centering
\includegraphics[width=0.60\textwidth]{figs/tk-upgrade/results-lowPtTracking/kfChi2NdfVsInvPtFlatGeometry_5000.pdf}
\caption{$\frac{\chi^{2}}{ndf}$ as a function of $\frac{1}{\pT}$ for genuine tracks produced by the \KF.}
\label{fig:2GeVFlatChi2Ndf}
\end{figure}

%% Discuss the resolution plots here
Figure~\ref{fig:htHelixParametersResVsInvPt} shows that the resolutions of the track parameters fitted by the \KF were comparable for $\pT (frac{1}{\pT})  < 3\GeV (0.33/\GeV)$ both before and after the \HT and associated \KF optimisation for tracks originating from the primary interaction.
The slight degradation in the $\phi_{0}$ resolution results from the decreased precision coordinates from the \HT and the small improvement observed in the $z_{0}$ resolution is due to the \KF being able to consider genuine stubs that were not previously found by the \HT.

\begin{figure}[htb]
\centering
\includegraphics[width=0.495\textwidth]{figs/tk-upgrade/results-lowPtTracking/qOverPtResVsInvPtFlatGeometry_5000.pdf}
\includegraphics[width=0.495\textwidth]{figs/tk-upgrade/results-lowPtTracking/phi0ResVsInvPtFlatGeometry_5000.pdf}
\\
\includegraphics[width=0.495\textwidth]{figs/tk-upgrade/results-lowPtTracking/z0ResVsInvPtFlatGeometry_5000.pdf}
\includegraphics[width=0.495\textwidth]{figs/tk-upgrade/results-lowPtTracking/cotThetaResVsInvPtFlatGeometry_5000.pdf}
\caption{
$\frac{q}{\pT}$ resolution, $\phi_{0}$ resolution, $z_{0}$ resolution and cot$\theta$ resolution measured for both the default configuration where only the number of \HT \qpt columns have been increased (red) and after the \HT and \KF optimisations have been applied (blue) for primary reconstructed tracks in simulated \ttbar events with a $<\textrm{PU}>$ of 200 events for tracks with $\pt > 2\GeVc$.
}
\label{fig:htHelixParametersResVsInvPt}
\end{figure}

\subsubsection{Kalman Filter Optimisation and Results}\label{subsubsec:lowPtOptKF}
Incorporation of \MS into the \KF involved including a \emph{process noise} term, namely, the variance of the multiple scattering angles, to the \emph{measurement noise} (\ie measurement error) term already present in the \KF covariance matrix.
In this updated form, the \KF now can consider stubs that are compatible with those that have undergone \MS, allowing tracks with previously discarded stubs to be reconstructed and resulting in more accurate $\chi^{2}$ values which can be used to better discriminate against fake tracks.

For small deflection angles and relativistic particles, the standard deviation of the distribution of deflection angles, $\sigma_{\theta}$, for any layer is given by~\cite{Lynch:1990sq}~:

\begin{equation}
\sigma_{\theta} = \frac{13.6\MeV}{\beta c p} q \sqrt{\frac{x}{X_{0}}} [1 + 0.088 \log_{10}{\frac{x}{X_{0}}}]  \;
\label{eq:scatter1}
\end{equation}

where the momentum, velocity, electrical charge of the incident particle and thickness of the scattering medium in radiation lengths are given by $p$, $\beta c$, $q$ and $\frac{x}{X_{0}}$, respectively.
The result from this equation has been found to have an accuracy no greater than 11\%~\cite{Lynch:1990sq}.

With the particles involved having relativistic velocities (\ie $\beta c \cong 1$) and scattering in the \rz plane ignored as the impact of multiple scattering is considerably smaller hit position resolution in \rz, the multiple scattering contribution in the \rphi plane can be expressed as:

\begin{equation}
\sigma_{\theta} = \frac{k}{\pT}
\label{eq:scatter2}
\end{equation}

where $k$ is the coefficient of proportionality that describes the constant terms of Equation~(\ref{eq:scatter2}) in the relativistic limit.

From the simplified form of Equation~(\ref{eq:scatter2}), two alternative forms of the coefficient $k$, which should require minimal resources and latency, were investigated:

\begin{itemize}
\item \textbf{constant coefficient:} a constant coefficient of the order of the average anticipated scattering angle is used. The typical scattering angle for $2-3\GeV$ tracks is of the order of a milliradian.
\item \textbf{layer-dependent coefficient:} the coefficient used depends on the layer ID (\ie the layer/disk of the stub used to update the Kalman state) in order to take into account the impact of repeated scattering from passing through multiple layers increasing the uncertainty associated with the hit position.
\end{itemize}

The initial layer-dependent coefficients were obtained through experimentally determining, using simulation, the \MS contribution to the observed variance in $\phi$.
Both these initial-layer dependent coefficients and the initial constant coefficient of a milliradian were subsequently further optimised in order to recover as much tracking efficiency as  possible.
Similarly, the \KF state $\chi^{2}$ cuts for both approaches were also tuned in order to reject the optimal number of fake and duplicate tracks without compromising on tracking efficiency.

\begin{figure}[htb]
\centering
\includegraphics[width=0.495\textwidth]{figs/tk-upgrade/results-lowPtTracking/htAvgNumDuplicatesVsInvPtTiltedGeometry_5000.pdf}
\includegraphics[width=0.495\textwidth]{figs/tk-upgrade/results-lowPtTracking/kfAvgNumDuplicatesVsInvPtTiltedGeometry_5000.pdf}
\caption{The average number duplicate tracks per matched track as a function of $\frac{1}{\pT}$ following reconstruction by the \HT (left) and fitting and filtering by the \KF and \DR (right) for where the \HT cell merging \pT threshold is set to 2.7\GeV (red) and 3.5\GeV (blue). 
The constant coefficient for the \MS contribution was used for these \KF results.
}
\label{fig:2GeVfracDups}
\end{figure}

During the comparative studies of the two \MS coefficients, it was found that the average number of duplicate tracks per matched track did not remain constant as a function of the simulated track's $\pT (\frac{1}{\pT} )$.
As shown in Figure~\ref{fig:2GeVfracDups}, there was an increase in the number of duplicates produced near the $2 \times 2$ merging of \HT cells threshold of  $\pT = 3\GeV (\frac{1}{\pT} = 0.33 1/\GeV$ following both the \HT and \DR stages, implying that the \HT produces more duplicates at low \pT in the full precision cells.
Given that the number of duplicate tracks produced in the decreased precision \HT cells were well controlled, the \pT ($\frac{1}{\pT}$) threshold for the $2 \times 2$ merging of \HT cells was increased from $2.7\GeV$ ($0.37/\GeV$) to $3.5\GeV$ ($0.29/\GeV$).
Despite this change decreasing the number of duplicates below the \pt ($\frac{1}{\pT}$) threshold of $3.5\GeV$ ($0.29/\GeV$), an increase in the number of duplicates produced per matched track near the \HT cell merging threshold was still present.
While these increases in the duplicate rate have yet to be fully understood, they are suspected to have arisen from resolution effects at the boundaries between the differently sized \HT cells given their proximity to the \pT threshold for merging \HT cells.
As the \pT threshold of $3.5\GeV$ recovered a further 0.2\% of the tracks that were previously lost to \MS and reduced the overall duplicate rate by 2.8\%, this change was adopted by the project and all the results presented below use this increased threshold. 

\begin{figure}[htb]
\centering
\includegraphics[width=0.65\textwidth]{figs/tk-upgrade/results-lowPtTracking/kfTrackingEffVsInvPtTiltedGeometry_5000.pdf}
%\includegraphics[width=\textwidth]{figs/tk-upgrade/results-lowPtTracking/kfTrackingEffVsInvPtFlatGeometry_5000.pdf}
\caption{Tracking efficiency as a function of $\frac{1}{\pT}$ for \ttbar events with a $<\textrm{PU}>$ of 200 events after the full chain has been run, where the \KF has not been modified to take \MS into account (red), a constant coefficient for \MS is used (black) and a layer dependent coefficient for \MS is used (blue).
}
\label{fig:2GeVTiltEff}	
\end{figure}

\begin{table}[htbp]
\topcaption {Track finding performance on simulated \ttbar events with a $<\textrm{PU}>$ of 200 events, after the full demonstrator chain for the three differing \KF configurations where \MS is not considered ($k = 0$), a constant \MS coefficient (const $k$) is used and a layer dependent \MS coefficient ($k(layer)$)is used. 
The track finding efficiencies, $\epsilon$, following each stage are given along with the mean number of tracks, $<N_{tracks}>$, and the fraction of those tracks which are either fake or duplicate tracks.
}
\label{tab:trackFindingPerformance2GeVKF}
\centering
% \resizebox{\textwidth}{!}{
\begin{tabular}{ccccc}
   \hline
   \bf{Scattering coefficient} & \bf{$\bm{\epsilon}$ [\%]} & $\bm{<N_{tracks}>}$ & \bf{Fakes [\%]} & \bf{Duplicates [\%]}  \\
   \hline
%     & \bf{HT}     &  96.2 & 752.0 & 28.2 & 47.3 \\  
   $k = 0$  & 93.6 & 216.0 & 13.3 & 9.4 \\      
   \hline
   const $k$ & 94.2 & 216.3 & 10.3 & 11.2 \\      
   \hline
    $k(layer)$ & 94.2 & 222.1 & 10.8 & 12.3 \\  
   \hline
   
\end{tabular}%}
\end{table}

Figure~\ref{fig:2GeVTiltEff} and Table~\ref{tab:trackFindingPerformance2GeVKF} illustrate that tracking efficiency improves when the \KF's covariance matrix accounts for \MS.
At high \pT (low $\frac{1}{\pT}$) however, the tracking efficiencies of both \KF configurations incorporating \MS were up to 1\% worse than the configuration where the effects of \MS are not considered.
This degraded performance results from the process noise term not considering the impact of the density effect, which becomes more important at increasing energies, reducing the effect stopping power of the material being traversed.
In addition, Table~\ref{tab:trackFindingPerformance2GeVKF} shows that compared to just the \HT optimisations alone, for both \MS coefficients, the \KF was more effective at rejecting incorrectly reconstructed tracks by up to an additional 3-4\%.
In contrast, the fraction of duplicates increases for both coefficients by 3-5\% for the full chain.
As illustrated in Figure~\ref{fig:2GeVfracDups}, this increase occurs between $3.2\GeV$ and $5\GeV$ - near the $\pT = 3.5\GeV (\frac{1}{\pT} = 0.29 1/\GeV)$ threshold for merging adjacent \HT cells.
Currently it is not understood how incorporating \MS into the \KF causes this, but it is suspected to be related to the use of the reduced precision of \HT cells.
%Tracks with \pt approaching the \pt threshold at which reduced precision of \HT cells are used are have the potential to be built out of both stubs from both the full and reduced precision \HT cells.

\begin{figure}[htb]
\centering
\includegraphics[width=0.495\textwidth]{figs/tk-upgrade/results-lowPtTracking/qOverPtResVsInvPtTiltedGeometry_5000.pdf}
\includegraphics[width=0.495\textwidth]{figs/tk-upgrade/results-lowPtTracking/phi0ResVsInvPtTiltedGeometry_5000.pdf}
\\
\includegraphics[width=0.495\textwidth]{figs/tk-upgrade/results-lowPtTracking/z0ResVsInvPtTiltedGeometry_5000.pdf}
\includegraphics[width=0.495\textwidth]{figs/tk-upgrade/results-lowPtTracking/cotThetaResVsInvPtTiltedGeometry_5000.pdf}
\caption{$\frac{q}{\pT}$ resolution, $\phi_{0}$ resolution, $z_{0}$ resolution and cot$\theta$ resolution determined for primary reconstructed tracks in simulated \ttbar events with a $<\textrm{PU}>$ of 200 events. The distribution for when the \KF has not been modified to take \MS into account is given in red, for a constant \MS coefficient in black and a layer-dependent \MS coefficient in blue.
}
\label{fig:kfHelixParametersResVsInvPt}
\end{figure}

Figure~\ref{fig:kfHelixParametersResVsInvPt} shows the resolutions of the helix parameters for primary reconstructed tracks as a function of track $\frac{1}{\pT}$ in simulation for both of the \MS coefficients considered and for when \MS was not accounted for at all in the \KF covariance matrix.
As effects of \MS only dominate at low $\pT$, there was no improvement observed in the resolutions of the helix parameters for high \pT tracks.
The $\frac{q}{\pT}$ resolution however, was noticeably worse in the range $0.18 1/\GeV < frac{1}{\pT}< 0.33 1/\GeVc$ for both of the \MS noise terms considered.
While the cause of this degradation in the $\frac{q}{\pT}$ resolution is as yet to be determined, it is suspected, like the increased duplicate rate in the same \pT range, to be 
related to the reduced precision of \HT cells.
The $\phi_{0}$ precision at low \pT was found to be improved as the increased uncertainty in the position each successive $\phi$ measurement resulted in the \KF giving greater weight to the innermost and more precise measurements. 
Similarly, the $\frac{q}{\pT}$ resolution was more precise at low \pT for $frac{1}{\pT} > 0.4$. 
Only minor differences were observed for the $z_{0}$ and cot$\theta$ resolutions as the \MS terms did not degrade the $z$ measurement error.
%This was expected as in the \rz plane, $z_{0}$ and cot$\theta$ are defined as/This was expected as the \MS terms do not impact measurements made in the \rz plane.

Incorporating a noise term to account for the effects \MS is further justified by considering the reconstructed tracks' $\frac{\chi^{2}}{ndf}$ as a function of $\frac{1}{\pT}$ in Figure~\ref{fig:2GeVTiltChi2Ndf}.
In contrast to when the \KF lacks a noise term accounting for \MS, the distributions of the \KF's performance for both \MS coefficients considered implies that \MS was the dominant source of uncertainty in the track measurements and that it has been well accounted for.

\begin{figure}[htb]
\centering
\includegraphics[width=0.6\textwidth]{figs/tk-upgrade/results-lowPtTracking/kfChi2NdfVsInvPtTiltedGeometry_5000.pdf}
\caption{ $\frac{\chi^{2}}{ndf}$ as a function of $\frac{1}{\pT}$ for \ttbar events with a $<\textrm{PU}>$ of 200 events after the full chain has been run. The distribution for when the \KF has not been modified to take \MS into account is given in red, for a constant \MS coefficient in black and a layer-dependent \MS coefficient in blue.
}
\label{fig:2GeVTiltChi2Ndf}	
\end{figure}

As shown in Figures~\ref{fig:2GeVTiltEff} and~\ref{fig:kfHelixParametersResVsInvPt} and Table~\ref{tab:trackFindingPerformance2GeVKF}, there were only small differences in the performance of the two \MS coefficients.
This is due the amount of material traversed by a track not being constant for a single layer given that the amount of material contributions in the Inner and Outer Trackers, between the Inner and Outer Trackers and services varies as a function of pseudorapidity~\cite{P2TrackerTDR}.

Following the improved performance considering the impact of \MS on the \HT and \KF discussed in these studies, these optimisations have been incorporated into the firmware for both TFP components.
The constant \MS term was chosen for inclusion into the \KF as it demonstrated the best performance and required the least resources to implement.

\subsection{Summary}\label{subsec:Tmtt2GeV}
Software studies were undertaken as part of the development of the \emph{TMTT} collaboration's proposed track finding system for the CMS Phase-II Outer Tracker in order to evaluate the performance of different aspects of the system.
These studies included the development and evaluation of a linearised $\chi^{2}$ track fitting algorithm and determining the robustness of the system if the minimum track reconstruction \pT threshold was reduced from 3\GeV to 2\GeV.

The development of the linearised $\chi^{2}$ track fitting algorithm initially involved validating the algorithm using floating point calculations prior to producing a version that produced comparable results using tabulated track derivatives and digitised variables and calculations. 
It was demonstrated that by performing multiple iterations, the algorithm was capable of removing both incorrectly associated stubs from genuine tracks and unmatched tracks, resulting in 100\% track purity for more than 97\% of all matched tracks and helix parameter resolutions comparable to the expected offline resolution in the barrel region.

Following the parallel development of the linearised $\chi^{2}$ track fit and the \KF algorithms, it was decided to discontinue development of the former, as the latter was capable of reconstructing all genuine tracks with no incorrectly associated stubs and fitting $z_{0}$ and $cot(\theta)$ more precisely at high $/eta$.
In addition, whilst substantial progress had been made with a firmware implementation of the \KF, there were concerns over the feasibility of tabulating the linearised $\chi^{2}$ track fit's most frequently used track derivatives for the endcap disks for commercially available FPGAs.

%%%
Reducing the minimum track reconstruction \pT threshold from 3\GeV to 2\GeV required modifying the TFP's configuration parameters to ensure adequate duplication in $\phi$ and increasing the number of \qpt columns by 50\% to maintain the same \HT cell precision over the larger \qpt range considered.
As tracks with $\pT <  3\GeV$ are increasingly dominated by \MS, a significant fraction of stubs did not intersect within a single \HT cell which resulted in a reduced track reconstruction efficiency below $2.7\GeV$.
These reconstruction efficiency losses were mitigated against by reducing the precision of the \HT cells along \qpt and $\phi_{T}$ by a factor of two for $2 < \pT \leq 2.7\GeV$.
A process noise term was added to the \KF to describe the effect of \MS increasing the uncertainty in the hit position in the \rphi plane.
Two noise terms were evaluated, one that described the effect as a function of the track's \pT and the other as a function of the track's \pT and the layer/disk of the stub added to the Kalman state.
Both noise terms produced comparable results and were shown to not only improve the track reconstruction efficiency of the system, but also to account for the dominant source of uncertainty in the track measurements.
This performance could be further improved by constructing a process noise term that better accounts for the amount of material traversed and by establishing the cause behind the increased production of duplicate tracks observed near the boundary between normal and reduced precision \HT cells.
