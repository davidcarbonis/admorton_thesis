\section{The CMS Tracker Upgrade}\label{sec:tk-upgrade}

The Compact Muon Solenoid (CMS) is a large, general purpose particle detector at the LHC, designed to investigate a wide range of physics phenomena. A detailed description of the CMS detector, together with a definition of the coordinate system used and the relevant kinematic variables, can be found in \cite{oldcms}.
 
The CMS silicon tracker, the innermost sub-detector, will need to be replaced in its entirety during the shut-down preceding HL-LHC operations, due to both approximately 15 years of radiation damage during operation and to maintain a high track reconstruction efficiency and a low misidentification rate under increased pileup conditions, requiring an increase in sensor granularity \cite{P2TrackerTDR}. The radiation hardness of the tracker must also be improved in order to withstand the increased fluence.

In order to keep the L1 acceptance rate below the 750\kHz maximum in the high pileup HL-LHC environment without raising the transverse energy (\ET) or transverse momentum (\pT) thresholds (losing potentially interesting physics events), the L1 trigger will also make use of charged particle tracks from the outer tracking detector.
 
A novel design of two closely spaced silicon sensors, capable of rejecting hits generated by low \pT particles, has been proposed \cite{jjonespixel,markthesis}. Correlated pairs of clusters between the two sensors which are compatible with a high \pT track (greater than 2-3 GeV), called \textit{stubs} (see Fig.~\ref{stubs}), are transferred to the L-1 trigger and are expected to provide an effective rate reduction of approximately 10 \cite{mpessimperf,2dptmoduleconcept}. Further details on the two types of \pT modules intended to be used for the outer tracker and their placement can be found in \cite{CMS_Upgrade_TP} and \cite{P2TrackerTDR}.

Using stubs as an input, the L1 trigger requires input data formatting, track reconstruction and track fitting to be undertaken within an overall latency of 4 $\mu$s. In this paper, a scalable, configurable and redundant system architecture based on a fully time-multiplexed design, using current FPGA technology, is proposed. 

\begin{figure}[!h]
\centering
\includegraphics[width=5in]{CMS-bw-logo.pdf}
% where an .eps filename suffix will be assumed under latex,
% and a .pdf suffix will be assumed for pdflatex; or what has been declared
% via \DeclareGraphicsExtensions.
\caption{Cluster matching in $p_\mathrm{T}$-modules~\cite{P2TrackerTDR}. (a) Correlating closely spaced clusters between two sensor layers, separated by a few mm, allows discrimination of transverse momentum based on the particle bend in the CMS magnetic field, assuming that the particle originated at the beam-line. (b) The same transverse momentum corresponds to a larger distance between signals for a given sensor spacing. (c) A larger spacing is needed in the endcap disks to achieve the same discrimination. Only tracks with $p_{\mathrm{T}}>2-3$\GeVc are transferred off-detector.
}
\label{stubs}
\end{figure}
