\chapter{The CMS Tracker Upgrade}\label{chapter:tk-upgrade}
 
\section{The High-Luminosity Large Hadron Collider} \label{sec:hl-lhc}
During Long Shutdown 3 (2023-2025), the High-Luminosity Large Hadron Collider (HL-HLC) upgrade is expected to be installed, with the instantaneous luminosity of the LHC increasing up to $5-7.5 \times {10}^{34}$\percms, corresponding to an average number of proton-proton interactions per 40\MHz bunch crossing of 140 to 200, and a total integrated luminosity of 3000\fbinv to the ATLAS and CMS experiments.

\editComment{LHC upgrade timeline image?} 

Increasing the LHC's instantaneous luminosity is motivated by the need to replace the inner triplet quadrupole magnets which focus the beams at the ATLAS and CMS collision regions, that are expected to be near life expired due to radiation exposure by 2023~\cite{hl-lhc-prelim-design-report,CMSCollaboration:2015zni}.
This increase in instantaneous luminosity will provide the experiments the ability to overcome the diminishing statistical gains that occur the longer an experiment is operated for at constant luminosity, and so enable greater precision SM and Higgs measurements, searches for rare processes and their potential deviations from the SM, and the discovery reach for multi-\TeV massive particles.

The instantaneous luminosity of the machine and the beam parameters are related by: the number of bunches $n_{b}$, the number of protons per bunch $N^{2}_{p}$, the beam beta value (focal length) at the collision point $\beta^{*}$, and a crossing angle dependent luminosity geometrical reduction factor $R$,

\begin{equation}
L \propto \frac{n_{b}N^{2}_{p}}{\beta^{*}} R \\
\label{eq:machineLumi}
\end{equation}

As it is not practical to increase the number of proton bunches due to the resultant heat loads induced by electron clouds, the increase in the machine's luminosity will be achieved by increasing the number of protons per bunch and by  reducing $\beta^{*}$.
Replacing Linac2 with the new Linear accelerator 4 (Linac4) during the Long Shutdown 2 (2019-2020) will allow for the number of protons per bunch to be increased by a factor of two compared to the nominal LHC design (and to increase the injection energy by a factor of three)~\cite{linac4}.
The new more radiation tolerant quadrupole magnets to be installed during LS3 will provide the larger magnetic field strength and aperture required to provide the lower $\beta^{*}$ required for increasing the instantaneous luminosity. 

\section{The Phase-II Outer Tracker Upgrade}\label{sec:tk-upgrade}

To meet the significant challenges of, and exploit, the increased instantaneous luminosity environment of the HL-LHC, the CMS detector's ``Phase-II Upgrade'' during the LS3 will deliver the required improved radiation hardness for the increase in radiation and to manage the high \PU HL-LHC environment with greater detector granularity to reduce occupancy, and enhanced bandwidth and triggering capabilities to avoid compromising physics potential~\cite{CMSCollaboration:2015zni,P2TrackerTDR}.

The Phase-II upgrade will see both the entire silicon tracking detector being replaced with one comprised of a pixel Inner Tracker and pixel and strip Outer Tracker which have:
\begin{itemize}
\item \bf{improved radiation hardness} - being able to withstand the increased fluence of the HL-LHC (up to $2.3\times10^{16} n_{eq}/cm^{2}$ for the innermost layers)and operate efficiently up to the targeted luminosity of 3000\fbinv, with a margin of $\approx50\%$ to accommodate the target being exceeded and the uncertainties in the anticipated radiation exposure.
\item \bf{increased sensor granularity} - so that the channel occupancy is kept at or below the per cent (per mille) level for the Outer (Inner) Tracker, allowing for a high track reconstruction efficiency and a low misidentification rate under the increased \PU conditions. This will also enable improved track separation in dense environments, such as high \pT jets, compared to the current pixel detector and fully exploit the vast volume of data produced.
\item \bf{reduced material in the tracking volume} - the current tracker's performance is significantly impacted by the amount of material present, as are the calorimeters and overall performance of CMS.
Significant reducing the tracker's material budget will greatly enhance CMS' performance at the HL-LHC.
\item \bf{robust pattern recognition} - enabling fast and efficient track finding, which is especially important for the HLT, in the high \PU environment.
\item \bf{level-1 trigger contributions} - it has been shown that the L1 trigger performance will deteriorate in the high luminosity environment from both the rate increase and the reduced efficiencies of the L1 selection algorithms.
To preserve and improve trigger performance, the bandwidth and latency will need increasing and tracking information will be included in the L1 decision process for the first time as additional data from the upgraded calorimeters and muon chambers will be insufficient.
\item \bf{extended tracking acceptance} - the overall physics capabilities of the CMS experiment would greatly benefit from extended coverage of the tracker and calorimeters up to $|\eta| = 4$ in the forward region.
\end{itemize}

With the above requirements in mind, the pixel Inner Tracker is designed to cover up to $|\eta| = 4$ with $100-150\mum$ thick planar silicon pixel sensors, measuring either $25\times100\mum^{2}$ or $50\times50\mum^{2}$\footnote{This is a reduction of a factor of $\approx 6$ compared to the Phase-0 and Phase-I pixel detectors}, which provide the low (per mille) occupancy and track separation with the negligible inefficiencies required in the harsh radiation environment.
Akin to the previous pixel detectors, the Phase-II pixel is also designed for easy installation and removal to facilitate the replacement of degraded parts.
Further discussion of the proposed Phase-II Inner Tracker is detailed in the Phase-II Technical Design Report~\cite{P2TrackerTDR}.


The enhanced triggering capabilities required of the Outer Tracker 


These enhanced triggering capabilities will require the Phase-II Outer Tracker to provide 	 to the L1 trigger in order to keep the L1 acceptance rate below the 750\kHz maximum in the high \PU HL-LHC environment without raising the transverse energy (\ET) or transverse momentum (\pT) thresholds, and thus losing potentially interesting physics events.



A novel design of two closely spaced silicon sensors, capable of rejecting hits generated by low \pT particles, has been proposed~\cite{jjonespixel,markthesis}. Correlated pairs of clusters between the two sensors which are compatible with a high \pT track (greater than 2-3 GeV), called \textit{stubs} (see Fig.~\ref{stubs}), are transferred to the L-1 trigger and are expected to provide an effective rate reduction of approximately 10~\cite{mpessimperf,2dptmoduleconcept}. Further details on the two types of \pT modules intended to be used for the outer tracker and their placement can be found in~\cite{CMS_Upgrade_TP,P2TrackerTDR}.

Using stubs as an input, the L1 trigger requires input data formatting, track reconstruction and track fitting to be undertaken within an overall latency of 4\mus.  

\begin{figure}[!h]
\centering
\includegraphics[width=5in]{CMS-bw-logo.pdf}
% where an .eps filename suffix will be assumed under latex,
% and a .pdf suffix will be assumed for pdflatex; or what has been declared
% via \DeclareGraphicsExtensions.
\caption{Cluster matching in $p_\mathrm{T}$-modules~\cite{P2TrackerTDR}. (a) Correlating closely spaced clusters between two sensor layers, separated by a few mm, allows discrimination of transverse momentum based on the particle bend in the CMS magnetic field, assuming that the particle originated at the beam-line. (b) The same transverse momentum corresponds to a larger distance between signals for a given sensor spacing. (c) A larger spacing is needed in the endcap disks to achieve the same discrimination. Only tracks with \pT $> 2-3$\GeVc are transferred off-detector.
}
\label{stubs}
\end{figure}
 
Three different L1 track finders have been explored by the CMS Collaboration, one using Associative Memory (AM) ASICs for track finding and FPGAs for track fitting, and two all-FPGA approaches, one using a Hough Transform (\HT) and the other a road search (``tracklet'') algorithm to reconstruct tracks respectively.



\section{An FPGA Based Track Finding Architecture}

A scalable, configurable and redundant system architecture based on a fully time-multiplexed design, using current FPGA technology, has been proposed.

\subsection{Linear $\chi^{2}$ Track Fitter}

\subsection{2 GeV Tracking}
