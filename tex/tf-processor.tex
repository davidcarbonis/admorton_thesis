\section{Track Finding Processor}\label{sec:tf-processor}
The Track Finding Processor (TFP) described is based on a time-multiplexed approach. The outer tracking detector is split into eight $\phi$ octants, referred to as \textit{detector octants}, where $\phi$ is the azimuthal angle of the track. Each \textit{processing octant}, offset from the detector octant by 22.5 degrees in $\phi$ in order to handle data duplication across hardware boundaries, receives data from the Data, Trigger and Control (DTC) system. The DTC reads out each detector octant, unpacks and converts the stubs to a global coordinate system, and transmission to one or two processing octants, or if consistent with both, duplicates the stub into both octants. 

Data consistent with each processing octant is sent out on separate links and processed by \textit{N} identical TFPs, negating the need for any data sharing downstream within or after the track finding process. This has the advantages of demonstrating a final system with one TFP which is easily scalable, allows for spare TFPs for online recovery in case of failure or online testing of new algorithms and allows each TFP to operate independently, thus reducing the requirements on system wide synchronisation.

As the system is based on a time-multiplexed approach, where parallel nodes each process a single event from multiple sources, each TFP processes only one event in \textit{N}. For our demonstrator system N was chosen to be thirty six, based on the currently available electronics and input/output links. 