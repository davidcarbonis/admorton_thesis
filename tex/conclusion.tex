\chapter{Conclusion}\label{chapter:conclusion}
In this thesis, a search for the production of a singly produced top quark in association with a Z boson using $35.9$\fbinv of proton-proton collision data at $\sqrt{s} = 13\TeV$ collected during 2016 by the CMS detector at the LHC was presented.
This thesis also presents the outcomes of several software studies for a proposed track finder system for the future CMS silicon tracking detector at the HL-LHC.

\section{Summary of Results}
A search for the tZq dilepton final state using a shape based analysis was performed using $35.9\fbinv$ of proton-proton collision data at $\sqrt{s} = 13\TeV$ collected by CMS during 2016.
The analysis focussed on understanding the contributions from processes that produce non-prompt leptons and suppressing the Z+jets and \ttbar processes.

While a \ttbar enriched control region confirmed that \ttbar processes were accurately modelled, it was observed in a Z+jets enriched control region that although the NLO Z+jets MC sample recommended by CMS provided the best description of data, it was incorrectly normalised.
Therefore, a Z+jets enriched control region was used to derive a scale factor to correctly normalise the NLO Z+jets sample, with a systematic uncertainty associated with this scale factor derived using the LO sample which correctly described the normalisation of the process.
Subsequently good agreement between simulation and data was observe after of all event corrections and the event selection criteria were applied. 
Boosted decision trees were used in order to further enhance the separation between the tZq signal from backgrounds using a set of variables which were determined to have the greater discriminating power.

A binned maximum likelihood fit was used to determine a limit on the tZq cross section for the dilepton final state of $xxx^{yyy}_{zzz}$ fb.
This corresponds to an observed significance of $aaa \pm bbb \sigma$ and an expected significance from simulation of $ccc \pm ddd$. 
This is the first limit made for the production cross section of the tZq dilepton final state at the LHC and is consistent within errors with the SM predictions.

%%%
The studies into the suitability 

In order to take full advantage of the high luminosity environment of the HL-LHC, it is planned replace the current CMS tracking detector with one which is capable of contributing information to the CMS Level-1 trigger.
One of the proposed track finder systems for the upgraded CMS tracker identifies track candidates using time-multiplexed Hough Transforms in the r-$\phi$ plane, followed by Kalman Filter to precisely fit track parameters to the candidates and remove fake tracks and a duplicate removal process.
A linearised $\chi^{2}$ fit was explored as an alternative track fitting algorithm

\section{Future CMS tZq measurements}
With searches for tZq being statistically limited to varying degrees due to its small production cross section, the discovery of the process should be achievable through the inclusion of additional data collected by the CMS experiment.
Given that evidence has already found for the trilepton final state, it is expected that 

The accuracy of the measurement

Recently the uncertainties associated with the JER have been updated during the reprocessing of the 2016 dataset to include the systematic uncertainties in addition to the statistical uncertainties.
At the time of writing this thesis, these reprocessed samples and the impact of the revised total JER uncertainties has not been propagated through the analysis.

\section{Future development of an FPGA Based Track Finder for the CMS Tracker Upgrade}
If the linearised $chi^{2}$ track fitting algorithm is to be 
%%%Future work
For the linearised $chi^{2}$ track fitting algorithm to be considered in the future, the following would to be addressed:
\begin{itemize}
\item An improved track fitting efficiency which obtains a high, if not 100\%, tracks purity, in order to be competitive with both the \KF and \LR which are currently able achieve 100\% purity.
\item Can the algorithm resources be reduced and/or are there sufficient resources on the board be implement the algorithm with.
\item Whether or not the inclusion of the fifth helix parameter, the vertex impact parameter in the x-y plane, $d_{0}$, would provide comparable or improved performance to other fitting algorithms which produce it.
\item The cause of the increased number of duplicate tracks compared to the \KF, and whether or not they can be reduced to a similar level.
\end{itemize}

There are a number of potential improvements for tracking down to 2\GeV which merit further investigation which include:
\begin{itemize}
\item understanding why the duplicate rate increases near the boundary between normal and reduced precision \HT cells increase when \MS is accounted for.
\item determining suitable coefficients as functions of both \pT and $\eta$ experimentally through simulation in order to more accurately account for the amount of material traversed and thus a more accurate description of the uncertainty in the hit position caused by \MS.
\item whether separate \KF $\chi^{2}$ cuts for the \rphi and r-z planes could enhance performance, given that the dominant uncertainty contribution for the former varies depending on $\pT$.
\item further studies into optimising the \pT threshold used for reducing the precision of \HT cells.
\item \pt dependent threshold criteria for the \HT.
\item further optimisation of the \KF in relation to any such changes.
\end{itemize}
