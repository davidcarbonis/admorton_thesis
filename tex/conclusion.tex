\chapter{Conclusion}\label{chapter:conclusion}
Following the restart of the LHC in 2015, the LHC's increased centre-of-mass collision energies and instantaneous luminosities have made it possible to undertake measurements of rare processes involving top quark and electroweak interactions.
In this thesis a search was presented for the production of a top quark in association with a Z boson using the dilepton final state using a shape based analysis.
This analysis focussed on understanding and constraining processes that involve the production of two promptly produced leptons that are consistent with a Z boson decay and those that involve at least one non-promptly produced lepton.
A Boosted Decision Tree was used to further enhance the separation between the signal from tZq production and background, using a set of variables that were identified as having the greatest discriminating power.
Using a Maximum Likelihood Fit, cross sections of $\sigma (\textrm{tZq}, Z \rightarrow e^{+} e^{-}) = 505.5^{147.4}_{-154.3}$ fb and $\sigma (\textrm{tZq}, Z \rightarrow \mu^{+} \mu^{-}) = 505.5^{147.4}_{-154.3}$ fb were measured for the signal process.
These measurements correspond to an observed excess over the background-only hypothesis of $2.72\sigma$ and $2.50\sigma$ for the $ee$ and $\mu\mu$ channels, respectively.
Using simulation, the expected significance was determined to be $0.46\sigma$, $0.54\sigma$ and $0.70\sigma$ for the $ee$ and $\mu\mu$ channels and their combination, respectively.

These results constitute the first measurement of tZq that has been made using the dilepton final state thesis.
Given that the results presented have not been fully reviewed by the CMS collaboration, further work is required in order to both understand the higher than expected significance of the measurement and to achieve the standard required for journal publication on behalf of the CMS collaboration.

As part of 

As the search for tZq is statistically limited, it is expected that discovery of this process using the dilepton final state should be achievable by extending the analysis to incorporate additional data collected by the CMS experiment.
The sensitivity of this combined measurement 

This result is based on the February 2017 reprocessing of the 2016 data and September 2016 reprocessing of the corresponding simulation samples.
These datasets have subsequently been reprocessed to incorporate updated jet energy corrections and improved alignments and calibrations of the CMS detector.
Combining these reprocessed datasets with the 2017 dataset should improve the significance of the result to around $999\sigma$.

Addition

Z+jets and ttbar regions to constrain 

Alternative lepton selection and b-tagging algorithms have improved the event selection efficiency of other single top searches and their potential for improving the sensitivity of this analysis deserves investigation.

Once an accurate measurement of the tZq cross section can be made, it should be possible to probe  the strength of the WZ and tZ couplings in this process and compare them to SM predictions.
In addition, limits on FCNC

%For the combination of both final states, at 95\% CL, the observed signal strength for tZq production was determined to be $5.366_{-1.638}^{+1.565}$, corresponding to an expected significance of $3.547 \sigma$. %$4.74^{+1.69}_{-1.51$}

%Using the reference NLO cross section of $\sigma (tZq, Z \rightarrow l^{+} l^{-}$) = 94.2 fb~\cite{Campbell:2013yla}, this signal strength corresponds to a cross section of $505.5^{147.4}_{-154.3}$ fb. %$94.2^{+95.3}_{-98.4475}$.

%As this analysis is currently blinded, only the expected upper limit on the cross section for tZq using the dilepton state and its corresponding significance are presented.
%
%The expected upper limit on the tZq cross section was determined to be 189.4 fb, corresponding to an expected significance of $0.95 \sigma$.
%
%This expected result is consistent with the SM prediction and with the measurements of the tZq cross section made by the ATLAS and CMS collaborations at $\sqrt{s} = 13 \TeV$ using the trilepton final state~\cite{Aaboud:2017ylb,Sirunyan:2017nbr}.  
%Once unblinded, this will be the first limit that has been set on tZq production using the dilepton final state.

%%%

%%%%%%
%TMTT%
%%%%%%
The \emph{TMTT} collaboration has proposed a track finder system for the CMS tracker at the HL-LHC that is capable of contributing information to the CMS Level-1 trigger.
The \emph{TMTT} track finding system identifies track candidates using time-multiplexed Hough Transforms in the \rphi plane, a Kalman Filter to filter these candidates and and precisely fit track parameters to them and a duplicate removal process.
In this thesis a number of studies were presented that were undertaken as part of the development of the this track finding system.

Prior to the hardware demonstrator review in 2016 of the three propsoed track finding systems, a linearised $\chi^{2}$ track fitting algorithm was explored as an alternative to the \KF.
The linearised $\chi^{2}$ track fitting algorithm was shown to be capable of both fitting precise helix parameters to the tracks found by the \HT and filtering out hits incorrectly assigned to tracks and incorrectly reconstructed tracks.
Following the evaluation and comparison of both the $\chi^{2}$ track fit algorithm and the \KF, it was decided not to continue development of the former algorithm.
This decision was made as it was determined that the \KF filtering and fitting performance was superior than that of the $\chi^{2}$ track fit, particularly in the forward regions.

If the the linearised $\chi^{2}$ track fitting algorithm is to be considered a viable alternative to the other track fitters developed for the \emph{TMTT} project.
While only a small number of track derivatives are required for the calculations in the barrel region, it is uncertain whether there are sufficient resources to tabulate the endcap derivatives required on current hardware.
If it can be demonstrated that current FPGAs can implement this algorithm, it will need to be demonstrated that the $\chi^{2}$ track fitter's performance is competitive with the \KF and \LR.
While it may not be possible to make the $\chi^{2}$ track fitter filter tracks as effectively as the \KF or \LR , other improvements, such including the fitting of the transverse impact parameter, may improve its competitiveness.


The flexibility for the upgraded tracker to be able to reconstruct tracks down to a lower \pT threshold of 2\GeV is potentially desirable and was initially studied as part of the 2016 demonstrator review.
The ability of the proposed track finding system to reconstruct such low transverse momenta tracks ($2\GeV < \pT < 3\GeV$) was shown to be considerably improved by accounting for the effects of \MS. 
For the \HT, this involved using decreased precision \HT cells to mitigate against scattering causing stubs to be found in adjacent cells.
The \KF's covariance matrix was modified to incorporate the uncertainty in the hit position cased by the effects of \MS by including a term that described the average scattering angle as a function of \pT.

Despite these improvements, there are still a number of key areas that require investigating in order to understand the current limitations of the work done so far and how it may be improved upon.
Currently it is not understood why the duplicate rate increases near the boundary between normal and reduced precision \HT cells.
This effect needs to be understood before an optimal value for the cell merging threshold can be determined.
The \KF's performance is likely to be further improved by using a scattering constant term that accurately takes the volume of material a track has passed through into account.
Implementing separate \KF $\chi^{2}$ cuts for the \rphi and \rz planes is another potential improvement given that the dominant uncertainty contribution for the former varies depending on $\pT$.

The \emph{TMTT} collaboration demonstrated in the 2016 review that a complete track finding system for the upgraded CMS tracker that met the baseline system requirements could be built using currently available technology.
Since 2016 the development and optimisation of \emph{TMTT} track finding system has continued using the so-called tilted barrel geometry for anticipated hardware for the final track finding system.
By the end of 2018 it is anticipated that the proposed systems of the \emph{TMTT} and \emph{tracklet} projects will begin to converge in order to begin development of all-FPGA track finding system.
The prototype for the final system is anticipated to developed by 202X and installed during LS3 prior to the start of HL-LHC operations in 2026.

%Given the limited experimental evidence of BSM physics, a large number of BSM physics models, driven by theoretical and ascetic arguments, have been proposed to account for the shortcomings of the SM.
%While the analysis presented in this thesis concerns the search for a SM process, the tZq cross section is sensitive to modifications of the tZ coupling posited by a number of BSM theories.
%
%As eluded to in Section~\ref{subsec:weakForce}, any flavour changing process involving a neutral weak current in the SM cannot occur at the tree level and requires a loop processes involving a virtual W exchange.
%A number of BSM theories however, introduce top quark FCNC decay contributions at the tree level, such as Supersymmetry (SUSY) models and those proposing additional Higgs doublets and/or quark singlets~\cite{AguilarSaavedra:2004wm}.
%The presence of such new tZ couplings would enhance the production rate of both \ttZ and tZq by several orders of magnitude and should be observable at the LHC. 
%As of to date however, no evidence for BSM FCNCs have been observed for the tZ coupling for both single top and \ttbar processes~\cite{Sirunyan:2017kkr}.
