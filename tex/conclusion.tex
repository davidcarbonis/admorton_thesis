\chapter{Conclusion}\label{chapter:conclusion}

In this thesis, a search for the production of a singly produced top quark in association with a Z boson using proton-proton collision data recorded by the CMS detector at the LHC at $\sqrt{13}$ during 2016 and a number of software studies for a proposed track finder for a future CMS tracker upgrade were presented.

\section{Summary of Results}
Using the complete 2016 dataset of 35.9\fbinv, a shape based analysis was performed to search for tZq in the dilepton final state, using \editComment{a MVA technique} to aid in separating the signal process from the backgrounds, especially the dominant Z+jets background.
A cross section of $xxx^{yyy}_{zzz}$ fb was measured for the signal process, with an observed significance of $aaa \pm bbb \sigma$ compared to the expected significance from simulation of $ccc \pm ddd$. 
\editComment{Some comment about the significance of this result.}


Building on the successful development and demonstration of a track finding architecture for the CMS detector at the HL-LHC on currently available technology for both track finding and fitting charged particles with transverse momentum greater than 3\GeV at 40\MHz the latency constraints and operational conditions of the HL-LHC, 


where track candidates were identified using time-multiplexed Hough Transforms in the r-$\phi$ plane before being cleaned and fitted by a Kalman Filter and duplicates removed by exploiting how duplicates form in the Hough Transform. 
Whilst the demonstrator system had 

\section{Future CMS tZq measurements}
The tZq analysis

\section{Future development of an FPGA Based Track Finder for the CMS Tracker Upgrade}
