\chapter{Conclusion}\label{chapter:conclusion}

A search for the production of a singly produced top quark in association with a Z boson using proton-proton collision data recorded by the CMS detector at the LHC at $\sqrt{13}$ during 2016 and a number of software studies for a proposed track finder for a future CMS tracker upgrade were presented.

\section{Summary of Results}
Using the complete 2016 dataset of 35.9\fbinv, a shape based analysis was performed to search for tZq in the dilepton final state.

A cross section of $xxx^{yyy}_{zzz}$ fb was measured for the signal process, with an observed significance of $aaa \pm bbb \sigma$ compared to the expected significance from simulation of $ccc \pm ddd$. 

A track finding architecture for the CMS detector at the HL-LHC has been developed and demonstrated on currently available technology (MP7 processing boards) with hardware results validated directly against simulation. Track candidates are identified using time-multiplexed Hough Transforms in the r-$\phi$ plane before being cleaned and fitted by a Kalman Filter and duplicates removed with a duplicate removal algorithm which is based off how duplicates form in the Hough Transform. The demonstrator system has shown that track finding and fitting charged particles with transverse momentum greater than 3\GeV at 40\MHz is possible within the latency constraints and operational conditions of the HL-LHC. It is expected that further improvements to the system can be accomplished as algorithms are refined and new technology becomes available.

\section{Future CMS tZq measurements}
\section{Future development of an FPGA Based Track Finder for the CMS Tracker Upgrade}
