\chapter{Conclusion}\label{chapter:conclusion}
In this thesis, a search for the production of a singly produced top quark in association with a Z boson using proton-proton collision data recorded by the CMS detector at the LHC at $\sqrt{13}$ during 2016 was presented.
In addition to this physics search, a number of software studies for a proposed track finder for a future CMS tracker upgrade were also presented.

\section{Summary of Results}
Using the complete 2016 dataset of 35.9\fbinv, a shape based analysis was performed to search for tZq in the dilepton final state, using \editComment{a MVA technique} to aid in separating the signal process from the backgrounds, especially the dominant Z+jets background.
A cross section of $xxx^{yyy}_{zzz}$ fb was measured for the signal process, with an observed significance of $aaa \pm bbb \sigma$ compared to the expected significance from simulation of $ccc \pm ddd$. 
\editComment{Some comment about the significance of this result.}



In order to take advantage of the high luminosity environment of the HL-LHC, a track finding architecture for the CMS tracking detector consisting of currently avaliable hardware and building ... has been successfully demonstrated.


Building on the successful development and demonstration of a track finding architecture for the CMS detector at the HL-LHC on currently available technology for both track finding and fitting charged particles with transverse momentum greater than 3\GeV at 40\MHz the latency constraints and operational conditions of the HL-LHC, 


where track candidates were identified using time-multiplexed Hough Transforms in the r-$\phi$ plane before being cleaned and fitted by a Kalman Filter and duplicates removed by exploiting how duplicates form in the Hough Transform. 
Whilst the demonstrator system had 

\section{Future CMS tZq measurements}
As the tZq analysis presented in this thesis is 

\section{Future development of an FPGA Based Track Finder for the CMS Tracker Upgrade}
While the 
%%% Future Improvements?
\subsubsection{Chi2}\label{subsubsec:chi2future}
%%%Future work
For the linearised $chi^{2}$ track fitting algorithm to be considered in the future, the following would to be addressed:
\begin{itemize}
\item An improved track fitting efficiency which obtains a high, if not 100\%, tracks purity, in order to be competitive with both the \KF and \LR which are currently able achieve 100\% purity.
\item Can the algorithm resources be reduced and/or are there sufficient resources on the board be implement the algorithm with.
\item Whether or not the inclusion of the fifth helix parameter, the vertex impact parameter in the x-y plane, $d_{0}$, would provide comparable or improved performance to other fitting algorithms which produce it.
\item The cause of the increased number of duplicate tracks compared to the \KF, and whether or not they can be reduced to a similar level.
\end{itemize}

\subsubsection{2 GeV}\label{subsubsec:2GeVoutlook}
There are a number of potential improvements for tracking down to 2\GeV which merit further investigation which include:
\begin{itemize}
\item understanding why the duplicate rate increases near the boundary between normal and reduced precision \HT cells increase when \MS is accounted for.
\item determining suitable coefficients as functions of both \pT and $\eta$ experimentally through simulation in order to more accurately account for the amount of material traversed and thus a more accurate description of the uncertainty in the hit position caused by \MS.
\item whether separate \KF $\chi^{2}$ cuts for the \rphi and r-z planes could enhance performance, given that the dominant uncertainty contribution for the former varies depending on $\pT$.
\item further studies into optimising the \pT threshold used for reducing the precision of \HT cells.
\item \pt dependent threshold criteria for the \HT.
\item further optimisation of the \KF in relation to any such changes.
\end{itemize}
