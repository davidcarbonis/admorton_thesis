\chapter{Conclusion}\label{chapter:conclusion}
In this thesis a search was presented for the production of a top quark in association with a Z boson using the dilepton final state using a shape based analysis.
A Boosted Decision Tree was used to further enhance the separation between the signal from tZq production and background, using a set of variables that were identified as having the greatest discriminating power.

As this analysis is currently blinded, only the expected upper limit on the cross section for tZq using the dilepton state and its corresponding significance are presented.
The expected upper limit on the tZq cross section was determined to be 189.4 fb, corresponding to an expected significance of $0.95 \sigma$.
This expected result is consistent with the SM prediction and with the measurements of the tZq cross section made by the ATLAS and CMS collaborations at $\sqrt{s} = 13 \TeV$ using the trilepton final state~\cite{Aaboud:2017ylb,Sirunyan:2017nbr}.  
Once unblinded, this will be the first limit that has been set on tZq production using the dilepton final state.

%This analysis focussed on understanding the contributions from processes that produce non-prompt leptons and suppressing the Z+jets and \ttbar processes.
%
%While a \ttbar enriched control region confirmed that \ttbar processes were accurately modelled, it was observed in a Z+jets enriched control region that although the NLO Z+jets MC sample recommended by CMS provided the best description of data, it was incorrectly normalised.
%
%Therefore, a Z+jets enriched control region was used to derive a scale factor to correctly normalise the NLO Z+jets sample, with a systematic uncertainty associated with this scale factor derived using the LO sample which correctly described the normalisation of the process.
%Subsequently good agreement between simulation and data was observe after of all event corrections and the event selection criteria were applied. 

%%%

As the search for tZq is statistically limited, it is expected that this process should be observable for the dilepton final state by extending the analysis to incorporate additional data collected by the CMS experiment.
The sensitivity of this measurement will be improved by using the proton-proton collision datasets for and their associated simulated samples which have been centrally reprocessed to provide an optimal event reconstruction by CMS.
The incorporation of the updated uncertainties relating to these reprocessed datasets is  expected to improve the accuracy of the measurement.
Alternative lepton selection and b-tagging algorithms have improved the event selection efficiency of other single top searches and their potential for improving the sensitivity of this analysis deserves investigation.
Once an accurate measurement of the tZq cross section can be made, it should be possible to probe  the strength of the WZ and tZ couplings in this process and compare them to SM predictions.

%Given the limited experimental evidence of BSM physics, a large number of BSM physics models, driven by theoretical and ascetic arguments, have been proposed to account for the shortcomings of the SM.
%While the analysis presented in this thesis concerns the search for a SM process, the tZq cross section is sensitive to modifications of the tZ coupling posited by a number of BSM theories.
%
%As eluded to in Section~\ref{subsec:weakForce}, any flavour changing process involving a neutral weak current in the SM cannot occur at the tree level and requires a loop processes involving a virtual W exchange.
%A number of BSM theories however, introduce top quark FCNC decay contributions at the tree level, such as Supersymmetry (SUSY) models and those proposing additional Higgs doublets and/or quark singlets~\cite{AguilarSaavedra:2004wm}.
%The presence of such new tZ couplings would enhance the production rate of both \ttZ and tZq by several orders of magnitude and should be observable at the LHC. 
%As of to date however, no evidence for BSM FCNCs have been observed for the tZ coupling for both single top and \ttbar processes~\cite{Sirunyan:2017kkr}.

In order to take full advantage of the high luminosity environment of the HL-LHC, it is planned replace the current CMS tracking detector with one which is capable of contributing information to the CMS Level-1 trigger.
One of the proposed track finder systems for the upgraded CMS tracker identifies track candidates using time-multiplexed Hough Transforms in the \rphi plane, followed by Kalman Filter to precisely fit track parameters to the candidates and remove fake tracks and a duplicate removal process.

A linearised $\chi^{2}$ fit was explored as an alternative track fitting algorithm

Studies  
 for a proposed track finder system for the future CMS tracking detector at the HL-LHC were also presented.

A linearised $\chi^{2}$ track fit algorithm was developed and was shown to be capable of filtering the track candidates produced by the \HT and fitting precise helix parameters to them.
Following evaluating this fitting algorithm and comparing it to the Kalman filter that was developed in parallel it was decided not to continue development of the former algorithm as it was determined that the \KF performance was superior, particularly in the forward regions, and should therefore be pursued in preference to the linearised $\chi^{2}$ fitter.

The reconstruction efficiency and fitting of the parameters of low transverse momenta tracks ($2\GeV < \pT < 3\GeV$) was shown to be considerably improved by accounting for the effects of \MS.
This involved using decreased precision cells in the \HT to mitigate against scattering causing stubs to be found in adjacent \HT cells.
The incorporation of an uncertainty term into the \KF's covariance matrix, which described the average scattering angle as a function of \pT, was shown to be as effective as one that also depended on the layer that a hit was found in.

Currently the development of the linearised $\chi^{2}$ track fitting algorithm for the \emph{TMTT} project's track finder is not being pursued.
For this track fitting algorithm to be considered as a viable alternative to the other track fitters used by the proposed \HT-based track finding system, it will need to address the following:

\begin{itemize}
\item Whether or not the stub filtering can be improved to the extend that all of the genuine tracks fitted contain no incorrectly associated stubs. Both the \KF and \LR can be used to remove all incorrectly associated stubs from the genuine tracks they fit. 
\item Determine whether or not there are sufficient resources to implement the algorithm in firmware.
\item Whether or not the inclusion of the fifth helix parameter, the vertex impact parameter in the x-y plane, would provide comparable or improved performance to other fitting algorithms fit this parameter.
\end{itemize}

There are a number of areas of investigation that could potentially improve both the \HT's and \KF's performance at low  transverse momenta.
Currently it is not understood why the duplicate rate increases near the boundary between normal and reduced precision \HT cells.
This effect needs to be understood before an optimal value for the cell merging threshold can be determined.
The ability of the \KF to account for \MS could be improved by using a scattering constant term that is parameterised in such a way that it accurately took into account the volume of material a track has passed through.
The performance of the \KF could also be further improved by using separate \KF $\chi^{2}$ cuts for the \rphi and \rz planes given that the dominant uncertainty contribution for the former varies depending on $\pT$.