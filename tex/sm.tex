\chapter{Theory}\label{chapter:theory}
\section{The Standard Model}\label{sec:sm}
The Standard Model (SM) of particle physics is a theory which describes all the known elementary particles and their interactions with the weak, strong, and electromagnetic forces.
While the SM is not a complete description of nature, the theory has made incredibly accurate predictions with only a handful of small deviations from it being observed.

This chapter introduces and briefly describes the theoretical framework of the SM and the physics of top quark.
Throughout this thesis \emph{natural units} where the fundamental constants $c$, $\hbar$ and $k_{B}$ are equal to unity, and Einstein's summation convention are used.

\subsection{Fundamental Particles}\label{subsec:particles}
The SM describes all matter as spin-$\frac{1}{2}$ particles known as fermions and the fundamental forces are mediated by spin-$1$ gauge bosons.
The spin-$0$ Higgs boson arises as a consequence of the electroweak symmetry breaking as a means to imbue the fermions and weak force gauge bosons with mass.

Matter consists of six quarks, particles which interact through the strong, electromagnetic and weak forces, and six leptons, particles which experience the electromagnetic and weak forces~\cite{LagrangiansSM}.
Each fermion has an associated anti-matter equivalent which has an identical mass but opposite charge.
Both types of fermion are subdivided into three ``generations'' of particles where each subsequent generation of particles is identical, except for their quantum number and mass~\cite{ElectroweakStrong}.
Table~\ref{tab:fermions} lists the charges, weak isospins and masses of the quarks and leptons for each of the three generations.

The ``up-type'' and ``down-type'' quarks of each generation have both an electrical charge of $\frac{+2}{3}$ and $\frac{-1}{3}$  respectively and a \emph{colour charge} (or anti-colour charge) of red, blue or green.
As the phenomena of \emph{colour confinement} only allows for colourless states, quarks form composite particles called hadrons.
Typically hadrons are composed of a quark anti-quark pair, collectively termed mesons, or of three quarks , referred to as baryons.
Exotic hadrons formed of larger groupings of quarks can be also formed, with both tetraquark and pentaquark states being observed and discovered, respectively, by the LHCb detector~\cite{Aaij:2014jqa,Aaij:2015tga}.

Each generation of leptons consist of a charged lepton which interacts through the electromagnetic and weak forces, and a corresponding neutral near massless lepton, known as a neutrino, which interacts solely through the weak force.
As with the quarks, the charged lepton of each subsequent generation is more massive than the last.
Initially the SM assumed that neutrinos were massless, but the discovery of neutrino flavour oscillation implies that they have a non-zero mass. 
It is currently unknown whether the hierarchy of the neutrino mass eigenstates is analogous to that of the charged lepton or otherwise~\cite{Nath:2018rqn}.

\begin{table}[htbp]
\topcaption {
The Standard Model fermions and their properties.
}
\label{tab:fermions}
  \centering
  \resizebox{\textwidth}{!}{
% This right-aligns numbers in column, but centers them under column title.
 \begin{tabular}{llcllccc}
   \hline
   & \textbf{Generation} & \textbf{Particle} & & \textbf{Mass \MeV} & \textbf{Electric Charge} & \textbf{Colour Charge} & \textbf{Weak Isospin}\\
   \hline
   \multirow{3}{*}{Quarks}  & \multirow{2}{*}{I} & up \textit{$u$}  & 2.3 & $+ \frac{2}{3}$ & 0 & $+ \frac{1}{2}$ \\
   & & down & \textit{$d$} & $4.8$ & $- \frac{1}{3}$ & $0$ & $- \frac{1}{2}$ \\
   & \multirow{2}{*}{II} & charm & \textit{$c$}  & $2.3$ & $+ \frac{2}{3}$ & 0 & $+ \frac{1}{2}$ \\
   & & strange & \textit{$s$}  & $4.8$ & $- \frac{1}{3}$ & $0$ & $- \frac{1}{2}$ \\
   & \multirow{2}{*}{II} & top & \textit{$t$}  & $2.3$ & $+ \frac{2}{3}$ & $0$ & $+ \frac{1}{2}$ \\
   & & bottom &\textit{$b$}  & $4.8$ & $- \frac{1}{3}$ & $0$ & $- \frac{1}{2}$ \\
   \hline
   \multirow{3}{*}{Leptons}  & \multirow{2}{*}{I} & electron \textit{$e$}  & $0.511$ & $-1$ & $0$ & $- \frac{1}{2}$ \\
   & & electron neutrino & \textit{$\nu_{e}$}  & $< 2 \times 10^{-6}$ & $0$ & $0$ & $+ \frac{1}{2}$ \\
   & \multirow{2}{*}{II} & muon & \textit{$\mu$}  & $106$ & $-1$ & $0$ & $-\frac{1}{2}$ \\
   & & muon neutrino & \textit{$\nu_{\mu}$}  & $< 0.19$ & $0$ & $0$ & $+ \frac{1}{2}$ \\
   & \multirow{2}{*}{II} & tau & \textit{$\tau$}  & $1777$ & $0$ & $0$ & $- \frac{1}{2}$ \\
   & & tau neutrino & \textit{$\nu_{\tau}$}  & $<18.2$ & $0$ & $0$ & $+ \frac{1}{2}$ \\   
   \hline   
 \end{tabular}}
\end{table}

The SM contains five integer spin gauge bosons, shown in table~\ref{tab:bosons} along with their corresponding masses, charge, and weak isospins.
The four spin-$1$ vector bosons mediate the electromagnetic, weak and strong forces.
The massless photon, $\gamma$, mediates the electromagnetic force and the massive neutral $Z^0$ and charged  $W^\pm$ bosons mediate the weak force.
Massless gluons mediate the strong force and have one of eight colour states~\cite{LagrangiansSM}. 
The spin-$0$ Higgs boson arises from the breaking of the electroweak symmetry and accounts for fundamental particles acquire mass.

\begin{table}[htbp]
\topcaption {
The fundamental forces of nature and the SM gauge bosons which mediate them.
}
\label{tab:bosons}
  \centering
  \resizebox{\textwidth}{!}{
% This right-aligns numbers in column, but centers them under column title.
 \begin{tabular}{lccccc}
   \hline
   Gauge Bosons & & Mass (\GeV) & Electrical Charge & Colour Charge & Weak Isospin \\
   \hline
   Photon & $\gamma$ & $0$ & $0$ & $0$ & $0$ \\
   \hline
   W & $\text{W}^{\pm}$ & $80.385 \pm 0.015$ & $\pm 1$ & $0$ & $\pm 1$ \\
   Z & $\text{Z}^0$ & $91.1876 \pm 0.0021$ & $0$ & $0$ & $0$ \\
   Higgs & $\text{h}^{0}$ & $125 \pm 0.24$ & $0$ & $0$ & $- \frac{1}{2}$ \\
   \hline
   Gluon & $g$ & $0$ & $0$ & $r \overline{g}$, $r \overline{b}, g \overline{r}, g \overline{b}, b \overline{r}, b \overline{g} , \frac{1}{\sqrt{2}}(r \overline{r} - g \overline{g}), \frac{1}{\sqrt{6}}(r \overline{r} + g \overline{g} - 2 b \overline{b})$ & $0$ \\
   \hline   
 \end{tabular}}
\end{table}	

\subsection{Gauge Symmetries}\label{subsec:gaugeSymmetries}
The SM is a Quantum Field Theory (QFT) 
that is based on the $SU(3)_{C} \times SU(2)_{L} \times U(1)_{Y}$ symmetry group.



 where the interactions of the fermonic fields 
Lagrangian formalism


Symmetries are a fundamental component of any physical theory.

The SM Lagrangian density is constructed of four terms:

\begin{equation}
\mathcal{L}_{SM} = \mathcal{L}_{Gauge} + \mathcal{Fermion} + \mathcal{Higgs} + \mathcal{Yukawa}
\end{equation}

where $\mathcal{L}_{Gauge}$ describes the spin-$1$ gauge boson fields that arise from imposing 

required to keep the Lagrangian invariant


, $\mathcal{Fermion}$ describes the fermion fields, $\mathcal{Higgs}$ describes the scalar Higgs field and $\mathcal{Yukawa}$ describes the interactions between the Higgs field and fermions and gauge bosons.

NOETHER

Impact of QM

The SM is a renormalisable Quantum Field Theory (QFT) based on the $SU(3)\times SU(2)_{L} \times U(1)$ symmetry group~\cite{LagrangiansSM}.


 which permeate the Universe. 
The Lagrangian formalisation, used in QFTs to describe the dynamics of a system, has the Lagrangian ``$L$'' described as the difference between the kinetic and potential energy of the system\cite{LagrangiansSM}. 
QFTs usually make use of the Lagrangian Density ``$\mathcal{L}$'', defined as\cite{QFT}:

\begin{equation}
L = \int \mathrm{d^{3}}x \mathcal{L}
\end{equation}

With the general form of the Lagrangian Density being defined as:

\begin{equation}
\mathcal{L} = \mathcal{L} ( \varphi_{i}, \partial _{\mu} \varphi_{i} )
\end{equation}


where 


Where $\partial _{\mu}\varphi_{i} \equiv \partial \varphi / (\partial x^{\mu} )$ is the four-gradient of the field $\phi$ and where the i's are implicitly summed according to Einstein's summation convention\cite{ElectroweakStrong}.

The Lagrangian acts upon a system, with all information pertaining to the system's quantum state being described by a wave function. 
The amplitude of the wave function can be interpreted as the probability amplitude from which a measurement of an observable physical quantity can be obtained\cite{Isham}. 

An important feature of modern physical theories is that the laws of physics pertaining to a system do not vary under observation –- they are `invariant''. 
Examples of such invariant or conserved quantities include electrical charge from the $U(1)$ group’s symmetry in electromagnetism, energy-momentum from space-time symmetry and angular momentum from rotational symmetry\cite{Haywood}. 
These equivalent descriptions of the same system are related by groups of transformations, which if invariant when applied to the wave function, relate to observable properties\cite{QFT}. 
If the transformations on the system have no space-time dependence, the transformation is said to be ``global'', and if the transformations do have a space-time dependence then the transformation is said to be ''local''. 
A Lagrangian which has continuous local symmetry is said to be gauge invariant\cite{Haywood}. 

As defined above however, the Lagrangian Density $\mathcal{L}$ is not gauge invariant due to its dependence on the derivative $\partial _{\mu}$. To illustrate this, the Lagrangian which describes free-field fermions\cite{QFT}, 

\begin{equation}
\mathcal{L}_{0} = \bar{\psi}(i\gamma^{\mu}\partial_{\mu} - m)\psi
\end{equation}

Which when undergoing a local phase transformation,

\begin{equation}
\psi(x) \rightarrow \psi'(x) = \psi(x)e^{-ixf(x)}
\bar{\psi(x)} \rightarrow \bar{\psi'}(x) = \psi(x)e^{+ixf(x)}
\end{equation}

Transforms as:

\begin{equation}
\mathcal{L}_{0} \rightarrow \mathcal{L}_{0}' = \mathcal{L}_{0} + q \bar{psi}(x)\gamma^{\mu}\psi(x)\partial_{\mu}f(x)
\end{equation}

This transformation is clearly not invariant. Invariance can be restored by introducing a gauge field $A_{\mu}(x)$, associated with the $\psi(x)$ field, which transforms according to the gauge transformation. 
The minimal substitution in $\mathcal{L}_{0}$ which achieves this is the replacement of the derivative $\partial_{\mu}$  with the so-called ``covariant derivative''\cite{QFT}:

\begin{equation}
\partial_{\mu} \rightarrow D_{\mu} = [ \partial_{\mu} - icA_{\mu}(x) ]
\end{equation}

Which transforms as in the same way as the $\psi(x)$ field:

\begin{equation}
D_{\mu}\psi(x) \rightarrow e^{-icf(x)}D_{\mu}\psi(x)
\end{equation}

Thus the Lagrangian which describes the system in the QFT, remains invariant. 
The interaction between the vector gauge field $A_{\mu}(x)$ and the $\psi(x)$ can be interpreted as excitations in the vector field interacting with the particles described by $\psi(x)$, such as photons interacting with electrons in Quantum Electrodynamics. 
The constant c is the coupling constant for the vector field, which differs between different gauge fields. 
In the case of Quantum Electrodynamics, $c = q$, where q is the charge of an electron\cite{QFT}.

The differences between bosonic and fermionic particles can now be considered in the context of how they are affected by considering the individual particles within a system and how they are ordered. 
As particles with integer-spin must be quantised according to Bose-Einstein statistics and half-integer spin particles by Fermi-Dirac statistics, their wave functions must be symmetrical and anti-symmetrical respectively\cite{QM}:

\begin{equation}
\psi_{symmetric}(x_{a},x_{b}) = \psi_{symmetric}(x_{b},x_{a})
\psi_{anti-symmetric}(x_{a},x_{b}) = -\psi_{anti-symmetric}(x_{b},x_{a})
\end{equation}

As such, the Fermi Exclusion Principle, where two fermions are unable to exist in the same quantum state, can be formalised as\cite{QM}:

\begin{equation}
\psi_{anti-symmetric}(x_{a},x_{a}) = 0 \qquad \forall \quad a
\end{equation}

\subsection{Quantum Electrodynamics}\label{subsec:QED}
Quantum Electrodynamics (QED) is the theory which describes the Electromagnetic force between all electrically charged particles within with the SM. 
It has a single gauge boson, the neutrally charged photon. 
Because of the photon's lack of mass, the Electromagnetic force has an infinite range. 
As mentioned above, the Electromagnetic interaction conserves electrical charge, which is described by the $U(1)_{hypercharge}$ group symmetry in QFT\cite{QFT}. 

\subsection{Weak Interactions}\label{subsec:weakForce}
The Weak force is responsible for weak isospin processes. 
It has three massive gauge bosons, the electrically charged $W^{\pm}$ and neutral $Z^{0}$ bosons. 
The projection of the weak isospin along the z-axis is the conserved quantity of the Weak interaction, which in QFT is described by the $SU(2)_{weak isospin}$ group symmetry. 
The range and strength of the Weak force is considerably less than that of the Electromagnetic force due to the short lifespan of the massive gauge bosons\cite{ElectroweakStrong}. 


CP VIOLATION!	

CKM

\subsection{Electroweak Unification}\label{subsec:electroweak}

Both the Electromagnetic and Weak interactions can be described as a single interaction: the Electroweak interaction. 
The conserved quantity of this force, the weak hypercharge, is related to the conserved quantities of electrical charge and the z-projection of weak isospin, of its two constituent interactions. 
At sufficiently high energies, the two separate manifestations of the electroweak force unify into a single force. 
However, the $SU(2)_{weak isospin} \times U(1)_{hypercharge}$ symmetry of the electroweak interaction is not exact as whilst local invariance requires that the gauge boson fields be massless in order for the QFT to be renormalisable, the $W^{\pm}$ and $Z^{0}$ bosons are relatively massive. 
In order to retain a renormalisable theory, an additional mechanism, which introduces the masses of the weak bosons, must be introduced. 
The Higgs mechanism is the simplest solution to the breaking of the symmetry of the electroweak interaction\cite{LagrangiansSM}. 

One of the most pressing problems was that while the $SU(3)_{colour}$ symmetry is exact, the $SU(2)_{isospin} \times U(1)_{hypercharge}$ ``electroweak'' symmetry is said to be ``broken''. 
QFTs require massless vector fields in order to be locally invariant but the $W^{\pm}$ and $Z^{0}$ bosons are observed to be massive in contrast to the massless photon. 
The Higgs mechanism is the simplest solution to this paradox, with the scalar Higgs field being responsible for the massive bosons~\cite{oldcms}. 
Both the CMS and ATLAS experiments at CERN have independently confirmed the existence of an unknown boson at $\approx$ 125\GeV, which was later confirmed to be consistent with the Higgs Boson, the smallest possible excitation of its associated namesake field~\cite{HiggsCMS,HiggsATLAS}. 
Searches to determine whether this is the SM Higgs or not (several theories including SUSY propose multiple Higgs~\cite{Khalil:2003vd,Gianotti:2002xx}) will take place after the phase-0 upgrades of the LHC. 

%%%Spontaneous symmetry breaking

Originally, in order to ensure that the SM Lagrangian remained invariant under weak isospin rotations, the weak force's gauge bosons were required to be massless.
The lack of evidence for such massless gauge bosons and the subsequent discovery of massive weak gauge bosons however, 

\subsection{Quantum Chromodynamics}\label{subsec:QED}
The strong force and its interactions with quarks is described by the gauge theory known as Quantum Chromodynamics (QCD).
QCD describes these interactions through eight massless spin-$1$ gluons 


As QCD is based on the $SU(3)_{colour}$ group, it is non-Abelian, allowing
Quantum Chromodynamics (QCD) is the gauge theory which describes the strong force interactions between quarks and the eight massless gluons.
These interactions are 

 with the non-Abelian 

The Strong Force is mediated by eight massless spin-1 gluons, which acts upon the conserved Strong Force charge: colour\cite{ElectroweakStrong}. 
The conservation of colour is described by the $SU(3)_{colour}$ group and the symmetry is exact. 
Colour charge is unrelated to the visual perception of colour, but stems from the fact that unlike the electroweak interaction which has positive and negative charges, there are three colour charges. 
An important difference between the gauge bosons of the electromagnetic and Strong Force is that whilst the photon is chargeless, gluons are not. 
Gluons carry both a colour and anti-colour charge and only interact with coloured particles (quarks and other gluons)~\cite{ElectroweakStrong}. 


A phenomena unique to the strong interaction is that the effective strong interaction coupling constant $\alpha_{s}$ tends to zero as the energy scales increase, despite the gluon being massless. 
In other words, the constant $\alpha_{s}$ increases as the separation between colour charged particles increases. 
This ``asymptotic freedom'' is caused by the virtual quark-anti quark sea containing virtual gluons which increases the force between quarks for greater separations, in contrast to the screening effect of virtual electrically neutral photons in electromagnetic interactions. 
The increase in $\alpha_{s}$ with the separation between colour charged particles means all particles must be ``colourless'', with the consequence that quarks are confined to exist in bound states\cite{ElectroweakStrong}. 




If enough energy is put into a bound state, with the intent of breaking ``colour confinement'', the energy of the colour fields between the coloured particles will increase until it is more energetically favourable for quark pair production to occur than to increase the separation between the original two quarks\cite{Griffiths}. 


As hadrons are composite particles, whose constituent partons cannot be directly measured in isolation,  the


\section{Top Physics}\label{sec:top-physics}
The existence of a third generation of quarks was first hypothesised in 1973 by Makoto Kobayashi and Toshihide Maskawa in order to account for the CP violation observed in kaon decays~\cite{Kobayashi:1973fv}.
While the discovery of the tau lepton in 1975 and the bottom quark in 1977~\cite{Herb:1977ek} reinforced the motivation for a weak isospin partner to the bottom quark, as the top quark's mass was larger than intially assumed, it would remain unobserved until a sufficiently powerful collider was built.
In 1995 the top quark was observed at the Tevatron at the Fermi National Accelerator Laboratory by the CDF and D\O experiments~\cite{Abe:1995hr,D0:1995jca}.

The top quark's mass, $m_{top}$, of $173.44 \pm 0.51 \pm 0.71 \GeV$ makes it the most massive known fundamental particle and is responsible for imbuing it with properties which have no equivalent for the other five quarks~\cite{Patrignani:2016xqp}.
$m_{top}$ is greater than the mass of a W boson, making it capable of decaying into an on-shell W boson and a down-type quark (almost exclusive a bottom quark) and giving it a much short lifetime than the other quarks.
The top quark's lifespan of $5 \times 10^{-25}$ seconds is several orders of magnitude smaller than the characteristic timescale of the strong interaction.
This results in the top quark being the only quark which decays before it can hadronise. 
As such, the top quark provides a unique probe into the nature of a ``bare'' quark, such as its spin and polarisation, through studying the angular distributions of its decay products~\cite{Khachatryan:2015dzz}.
This also makes it possible to determine the helicity of the W boson involved in the decay.

Given all these properties, the top quark makes an excellent probe of the Wtb vertex and sensitive to any anomalous couplings that would impact it.
Additionally, with the top mass being greater than that of any other fundamental particle, it has the strongest Yukawa coupling to the Higgs field.
Consequently, many have conjectured that the top quark has a special role to play in electroweak symmetry breaking~\cite{Giammanco:2017xyn}
 
%The top quark has not been studied to the same extent as the other quarks due to its later discovery and the relatively low top quark production rate at the Tevatron limiting statistics.
%The greater operational energy and integrated luminosity provided by the LHC however, has produced greater statistics will be available to probe the nature of the top quark~\cite{Shibata:2008sy}. 

\subsection{Top quark pair production}\label{subsec:ttbarTheory}
Top quarks are predominantly produced by pair-production (\ttbar) through strong interactions at hadron colliders.
As illustrated in the Feynman diagrams in figure~\ref{fig:feyn_ttbar}, at Leading Order (LO) \ttbar events are produced by either gluon fusion or quark anti-quark annihilation. 
While approximately 90\% of \ttbar events produced at the Tevatron occured via quark fusion, 80-90\% of \ttbar events at the LHC are produced by gluon fusion for $\sqrt{s} = 8-14\TeV$~\cite{Deliot:2011np,Patrignani:2016xqp}.
These differences in production rates result from the nature of the particles being collided at both machines, \ie quark fusion requiring a sea quark at the LHC, and the higher centre-of-mass energies at the LHC resulting in gluons carrying a larger fraction of the proton's energy.

\begin{figure}[htbp]
\begin{center}
\includegraphics[width=0.97\textwidth]{figs/top-physics/ttbar_feyn.jpg}
\caption{The three Leading Order Feynman diagrams for top quark pair production at hadron colliders. Quark-anti quark annihilation is illustrated on the top row and gluon fusion on the bottom. Gluon fusion is the main production mode at the LHC and quark fusion at the Tevatron.}
\label{fig:feyn_ttbar}
\end{center}
\end{figure}

\editComment{Cross sections table} 

Top quark pair production can also occur in association with a vector boson, albeit at relatively small cross sections compared to both \ttbar and singly produced top quarks.
Despite their cross sections, these channels are important backgrounds which require detailed understanding in order to be able to probe for new physics with similar or smaller cross sections, such as \ttH or tZq~\cite{Khachatryan:2014ewa}.

\subsection{Single top quark production}\label{subsec:singleTopTheory}
Top quarks can also be produced singly through weak interactions, albeit with smaller cross sections than \ttbar given the relative weakness of the electroweak coupling compared to the strong coupling.

Such processes are an powerful probe between the top and electroweak sector of the SM.
The Wtb vertex allows for the direct measurement of the $\abs{V_{tb}}$ element of the Cabibbo-Kobayashi-Maskawa (CKM) matrix and thus test whether the CKM matrix is unitary as presumed or otherwise~\cite{Shibata:2008sy}.
Additionally, these processes can be used to determine the top quark's spin by considering the angular distribution of its decay products.

Understanding single top quark processes is also important from an experimental viewpoint as:
\begin{itemize}
\item these processes form backgrounds for not only SM processes such as \ttbar, but for for Higgs and BSM physics searches, such as anomalous couplings.
\item precision measurements of these processes can be used to constrain Parton Distribution Functions~\cite{Guffanti:2010yu}.
\end{itemize}

The main SM single top production mechanisms are categorised by the virtuality of the W boson involved in the interaction.
As each of these production channels have differing initial and final states they are typically considered separately.

\begin{figure}[!h]
\centering
\includegraphics[width=1.00\textwidth]{figs/top-physics/singletop_feyn.jpg}
\caption{Leading order diagrams for single top quark production channels via weak processes: (a) s-channel, (b) t-channel and (c) tW.}
\label{fig:singleTopDiagrams}
\end{figure}

\editComment{Cross sections table} 

s-channel production, as shown in figure~\ref{fig:singleTopDiagrams}(a), is the quark anti-quark annihilation into an off-shell W boson which decays into a top and anti-bottom quark.
This process has the lowest single top production cross section at the LHC given the virtuality of the W boson required to produce the top quark and need for the anti-quark to originate from a sea quark.
Given its low cross section and a final state topology similar to larger background processes, the s-channel has yet to be observed at the LHC~\cite{Khachatryan:2016ewo}.

The t-channel, as shown in figure~\ref{fig:singleTopDiagrams}(b), is the dominant single top prodution mechanism at the LHC.
The process involves the scattering of the W boson off a bottom quark originating either from a sea quark or from gluon splitting.
Initially observed at the Tevatron~\cite{Aaltonen:2009jj,Abazov:2009ii}, the t-channel has since been studied at higher energies at the LHC, with all results to date remaining consistent with the SM~\cite{Berta:2017ghf,Morton:2018wkb}.	

The tW-channel, as shown in figure~\ref{fig:singleTopDiagrams}(c), is where a top quark is produced in association with an on-shell W boson.
In contrast to being negligible at the Tevatron, the tW-channel is accessible at LHC and was discovered in 2014~\cite{Chatrchyan:2014tua}.
The discovery of the tW-channel was particularly anticipated as the presence of the on-shell W boson in the final state makes it an excellent probe of the Wtb vertex and the theoretical challenges that were involved in describing its interference with \ttbar production at next-to-leading order.

\subsection{Single top production in association with a Z boson}\label{subsec:tZqTheory}
With the high centre-of-mass energies and integrated luminosities available at the LHC, it has become possible to not only undertake precision studies of singly and pair produced top quarks, but also to probe the top quark and electroweak sector.
Such measurements provide not only the ability to perform precision tests of SM predictions, but are also sensitive to new physics such as new electroweak bosons, new fermions, and Flavour Changing Neutral Currents (FCNC).

One may initially assume that given the larger production cross section for \ttbar compared to single top processes, that it would be considerably more difficult to probe the top-electroweak sector through single top quark production processes.
The top quark and W boson coupling however, can only be probed through the single top tW-channel given that the W boson couples to the initial state quarks for \ttW, as illustrated in figure~\ref{fig:feyn_ttV}.
tH has yet to be observed~\cite{CMS:2018jsz} as it is much more difficult to access than \ttH due to the destructive interference between the tH and HW vertices~\cite{Maltoni:2001hu}.
In contrast, \ttZ has a lower production cross section than the combined tZ and $\overline{\text{t}}$Z cross section~\cite{Campbell:2013yla}.
This is the result of the large number of particles present in the \ttZ final state suppressing \ttZ production despite the relative strength of the strong coupling.
CMS has made measurements of \ttH, \ttW, and \ttZ, all in excess of five standard deviations and consistent with their SM predictions~\cite{Sirunyan:2017uzs,Sirunyan:2018hoz}.

\begin{figure}[p]
\centering
\includegraphics[width=\textwidth]{figs/top-physics/CMS-TOP-17-005_Figure_001.pdf}
\caption{Leading order \ttW (left) and \ttZ (right) production diagrams~\cite{Sirunyan:2017uzs}. Unlike for \ttZ and \ttH, the gauge boson in \ttW is not radiated from the top quark, but from the initial state quarks.}
\label{fig:deyn_ttV}
\end{figure}

The analysis presented in this thesis is the search for a single top quark produced in association with a $\text{Z}^0$ boson with an additional jet, known as \emph{tZq}, in the dilepton final state.

tZq is the production of a t-channel single top quark, where a $\text{Z}^0$ boson is radiated off one of the quark legs, as shown in figures~\ref{fig:feyn_tZq}(a)-(e), or from the exchanged W boson, as shown in figure~\ref{fig:feyn_tZq}(f).

\begin{figure}[p]
\centering
\includegraphics[width=0.47\textwidth]{figs/top-physics/tZq_feyn1.jpg}
\includegraphics[width=0.47\textwidth]{figs/top-physics/tZq_feyn2.jpg}
\includegraphics[width=0.47\textwidth]{figs/top-physics/tZq_feyn3.jpg}
\includegraphics[width=0.47\textwidth]{figs/top-physics/tZq_feyn4.jpg}
\includegraphics[width=0.47\textwidth]{figs/top-physics/tZq_feyn5.jpg}
\includegraphics[width=0.47\textwidth]{figs/top-physics/tZq_feyn6.jpg}
\caption{Leading order tZq production diagrams, where the Z boson is radiated off one of the quark lines in (a)-(e) and from the the non-resonant contribution to the tqZ process from the W boson exchange in (f).}
\label{fig:feyn_tZq}
\end{figure}

The final states

The presence of both the top quark coupling to the Z boson and the WWZ coupling makes this process a unique precision probe of the top quark and the electroweak sector.
Additionally, tZq is not only a background other rare SM processes (such as tH), but also one to BSM physics with similar signatures such as Flavour Changing Neutral Currents (FCNC) where the top decays directly into a Z boson~\cite{AguilarSaavedra:2004wm}.	

%%%%
As CMS was capable of observing \ttZ was using data from Run-I of the LHC~\cite{Khachatryan:2014ewa}, a tZ search was also undertaken by CMS.
While this measurement had a siginifiance of 2.9 standard deviations~\cite{Sirunyan:2017kkr}, given that the tZ and $\overline{\text{t}}$Z cross sections scale 


\begin{figure}[h]
\centering
\includegraphics{figs/top-physics/TplusZtot.png}
\caption{The next-to-leading order cross section for single top and top pair prodcution with and without an associated Z boson as a function of the centre-of-energy for proton-proton collisions at the LHC~\cite{Campbell:2013yla}.
}
\label{fig:topCrossSections}
\end{figure}

%%%%%

Previously the only final state searched fully leptonic final state of tZq have been searched for at $\sqrt{s} = 8\TeV$ by CMS~\cite{Sirunyan:2017kkr} and $\sqrt{s} = 13\TeV$ by ATLAS and CMS~\cite{Aaboud:2017ylb,Sirunyan:2017nbr}.

Decay modes:
The initial searches for tZq focussed on the fully leptonic final state where the on-shell Z boson and W boson from the top decay into leptons.

Given that the tZq final states' topologies are similar to its numerous backgrounds which have much larger cross sections, initially searches for tZq 

With the increase in the tZq cross section 
of the LHC centre of mass energy from 8\TeV to 13\TeV
 	
The final state where only the Z boson decays leptonically and the W boson hadronically


\section{Shortcomings of the Standard Model Physics}\label{sec:bsm}
The SM has been incredibly successful at accurately predicting the majority of the properties of the known fundamental particles up to the electroweak scale.
Despite this however, given the inability of the SM to incorporate gravity and address a number of experimental observations, it is apparent that there must be physics Beyond the Standard Model (BSM).

One of the major and most apparent shortcomings of the SM is its inability to explain why there is an asymmetry between matter and anti-matter in the universe.
While CP violation does occur within the SM, it is insufficient to account for the degree of the matter excess observed.

Gravity currently is described by the extremely successful classical theory of General Relativity (GR).
GR however, is fundamentally incompatible with the SM and  has produced contradictory results such as the predicted cosmological constant and Higgs field's vacuum energy density.
While attempts have been made to reconcile the two theories, no successful quantum theory of gravity has been produced to date~\cite{Sola:2013gha}.	

One of the other serious theoretical issues with the SM is the so-called \emph{hierarchy problem} concerning the vast discrepancies between aspects of the weak force and gravity~\cite{Burdman:2007ck}.
As the vacuum expectation value of the Higgs field determines the mass of the weak bosons, for the observed masses of these bosons, one would expect a vacuum expectation value of approximately 246\GeV.
Given that the radiative corrections to the observable mass of the Higgs boson are proportional to the energy scale of new physics, this would imply that the Higgs vacuum expectation value would be either zero or of the order of the BSM physics energy scale.
In order to obtain the observed Higgs mass the cancellations required from the radiative corrections must be extremely ``fine tuned''.
While there is nothing fundamentally wrong with this, many scientists find such fine tuning to be \emph{unnatural}.

Other astronomical and cosmological inconsistencies include the presence of \emph{dark matter} (DM) and \emph{dark energy} in the universe.
The observations of the rotation of galaxies, gravitational lensing, structure of the universe and the Cosmic Microwave background, indicates that there must be a form of ``dark'' matter present~\cite{Aghanim:2018eyx}.
The accelerating expansion of the universe is also unaccounted for and implies the existence of a ``dark energy'' to drive this~\cite{Peebles:2002gy,Aghanim:2018eyx}.

Perhaps the greatest inconsistency experimentally observed with the SM is that of massive neutrinos.
The first indication of massive neutrinos was made by the ``Homestake'' experiment which found that the fraction of electron neutrinos arriving from the Sun was at the most half (if not less) what was expected~\cite{PhysRevLett.20.1205}.
While this observation could be explained by neutrinos experiencing flavour oscillations, this would require neutrinos to have mass in contrast to the SM in order for their flavour eigenstates to mix with their mass eigenstates.
Further experiments have confirmed however, that neutrinos do experience flavour oscillations as they propagate through space and thus must have mass~\cite{Fukuda:1998mi,Ahmad:2001an,PhysRevD.88.032002}.
