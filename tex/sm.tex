\chapter{Theory}\label{chapter:theory}
\section{The Standard Model}\label{sec:sm}
The Standard Model (SM) of particle physics is a theory which describes all the known elementary particles and their interactions with the weak, strong, and electromagnetic forces through a renormalisable Quantum Field Theory (QFT).
While the SM is not a complete description of nature, the theory has made incredibly accurate predictions with only a handful of small deviations from it being observed.

This chapter introduces and briefly describes the theoretical framework of the SM and the physics of top quark.
Throughout this thesis \emph{natural units} typically used in High Energy Physics, where the fundamental constants $c$, $\hbar$ and $k_{B}}$ are equal to unity, and Einstein's summation convention are used.

\subsection{Fundamental Particles}\label{subsec:particles}

All matter is described as spin-$\frac{1}{2}$ particles known as fermions and the fundamental forces are mediated by spin-$1$ gauge bosons.
The spin-$0$ Higgs boson arises as a consequence of the electroweak symmetry breaking as a means to imbue the fermions and Weak gauge bosons with mass.
Matter consists of three \emph{generations} of fermion: 
Matter consists of two types of fermion, leptons and quarks, based on 

Matter consists of three ``generations'' of two types of fermion, leptons and quarks~\cite{ElectroweakStrong}.
Quarks experience all of the fundamental forces of nature, while leptons experience all but the Strong Force~\cite{LagrangiansSM}. 
In each generation, for fermions and quarks alike, there are two different fundamental particles. 
Each subsequent generation of particles is identical, except for their quantum number and mass. 

Quark particles in each generation either have an electrical charge of $\frac{+2}{3}$ or $\frac{-1}{3}$ and fermions have either electrical charge -1 or 0 (neutral)\cite{ElectroweakStrong}. 

Quarks are the fundamental particles of which hadrons, composite particles formed of quarks, are formed. 
Hadrons are either mesons which are formed of two quarks or baryons which are formed of three quarks. 
Exotic hadrons formed of larger groupings (four or more) of quarks have been hypothesised, but only one resonance, namely a tetraquark candidate whose quark content still has to be confirmed, has been observed\cite{PhysRevLett.112.222002}. 
The first generation of quarks comprises of the up and down quarks, which form the protons and neutrons that are found in conventional atomic matter. 
The second and third generations are each subsequently more massive than the first generation and comprise of the strange and charm quarks and top and bottom quarks respectively. 

Each charged lepton has an associated neutral, near massless, lepton known as a neutrino. 
As neutrinos have no associated electrical charge, their only interaction with other particles in the SM is through the weak force. 
As with the quarks, each subsequent generation's particles are more massive than the last. 
Whilst the SM assumes that neutrinos are massless, the ``Homestake'' experiment's measurements showed that the fraction of electron neutrinos arriving from the Sun was at the most half (if not less) what was expected\cite{PhysRevLett.20.1205}. 
Neutrino flavour oscillation would explain the observed solar neutrino flux, but would require neutrinos to have a non-zero mass. 
In 2013, the T2K collaboration presented results which confirmed the existence of neutrino oscillation\cite{PhysRevD.88.032002}. 
Whilst there are upper bounds on the neutrino masses from cosmological constraints, no experiment to date has been sensitive enough to determine the masses\cite{1475-7516-2006-06-019}. 


\begin{table}[htbp]
\topcaption {
The Standard Model fermions and their properties.
}
\label{tab:fermions}
  \centering
  \resizebox{\textwidth}{!}{
% This right-aligns numbers in column, but centers them under column title.
 \begin{tabular}{llcllccc}
   \hline
   & \textbf{Generation} & \textbf{Particle} & & \textbf{Mass \MeV} & \textbf{Electric Charge} & \textbf{Colour Charge} & \textbf{Weak Isospin}\\
   \hline
   \multirow{3}{*}{Quarks}  & \multirow{2}{*}{I} & up \textit{$u$}  & 2.3 & $+ \frac{2}{3}$ & 0 & $+ \frac{1}{2}$ \\
   & & down & \textit{$d$} & $4.8$ & $- \frac{1}{3}$ & $0$ & $- \frac{1}{2}$ \\
   & \multirow{2}{*}{II} & charm & \textit{$c$}  & $2.3$ & $+ \frac{2}{3}$ & 0 & $+ \frac{1}{2}$ \\
   & & strange & \textit{$s$}  & $4.8$ & $- \frac{1}{3}$ & $0$ & $- \frac{1}{2}$ \\
   & \multirow{2}{*}{II} & top & \textit{$t$}  & $2.3$ & $+ \frac{2}{3}$ & $0$ & $+ \frac{1}{2}$ \\
   & & bottom &\textit{$b$}  & $4.8$ & $- \frac{1}{3}$ & $0$ & $- \frac{1}{2}$ \\
   \hline
   \multirow{3}{*}{Leptons}  & \multirow{2}{*}{I} & electron \textit{$e$}  & $0.511$ & $-1$ & $0$ & $- \frac{1}{2}$ \\
   & & electron neutrino & \textit{$\nu_{e}$}  & $< 2 \times 10^{-6}$ & $0$ & $0$ & $+ \frac{1}{2}$ \\
   & \multirow{2}{*}{II} & muon & \textit{$\mu$}  & $106$ & $-1$ & $0$ & $-\frac{1}{2}$ \\
   & & muon neutrino & \textit{$\nu_{\mu}$}  & $< 0.19$ & $0$ & $0$ & $+ \frac{1}{2}$ \\
   & \multirow{2}{*}{II} & tau & \textit{$\tau$}  & $1777$ & $0$ & $0$ & $- \frac{1}{2}$ \\
   & & tau neutrino & \textit{$\nu_{\tau}$}  & $<18.2$ & $0$ & $0$ & $+ \frac{1}{2}$ \\   
   \hline   
 \end{tabular}}
\end{table}

In the SM there are five gauge bosons, each of which is an integer spin particle that mediate the weak, strong and electromagnetic forces. 
The massless photon,$\gamma$) and the massive neutral $Z^0$ boson mediate the neutral electroweak 

 mediates the electromagnetic force, the charged $W^\pm$ and neutral $Z^0$ boson mediates the weak force and eight types of gluon mediate the Strong Force\cite{LagrangiansSM}. 



\begin{table}[htbp]
\topcaption {
The fundamental forces of nature and the SM gauge bosons which mediate them.
}
\label{tab:bosons}
  \centering
  \resizebox{\textwidth}{!}{
% This right-aligns numbers in column, but centers them under column title.
 \begin{tabular}{lccccc}
   \hline
   \textbf{Gauge Boson} & & \textbf{Mass (\GeV)} & \textbf{Electrical Charge} & \textbf{Colour Charge} & \textbf{Weak Isospin}\\
   \hline
   photon & ($\gamma$) & $0$ & $0$ & $0$ & $0$ \\
   \hline
   W & $\text{W}^{\pm}$ & $80.385 \pm 0.015$ & $\pm 1$ & $0$ & $\pm 1$ \\
   Z & \Z0 & $91.1876 \pm 0.0021$ & $0$ & $0$ & 0 \\
   Higgs & $\text{h}^{0}$ & $125 \pm 0.24$ & $0$ & $0$ & $- \frac{1}{2}$ \\
   \hline
   Strong & $g$ & $0$ & $0$ & $r \overline{g}, r \overline{b}, g \overline{r}, g \overline{b}, b \overline{r}, b \overline{g} , \frac{1}{\sqrt{2}(r \overline{r} - g \overline{g}), \frac{1}{\sqrt{6}(r \overline{r} + g \overline{g} - 2 b \overline{b})} $ & $0$ \\
   \hline   
 \end{tabular}}
\end{table}





\subsection{Gauge Symmetries}\label{subsec:gaugeSymmetries}
The SM is a renormalisable Quantum Field Theory (QFT) based on the $SU(3)\times SU(2) \times U(1)$ symmetry group~\cite{LagrangiansSM}.

QFTs treat matter as the excitation of fermionic fields

 which permeate the Universe. 
The Lagrangian formalisation, used in QFTs to describe the dynamics of a system, has the Lagrangian ``$L$'' described as the difference between the kinetic and potential energy of the system\cite{LagrangiansSM}. 
QFTs usually make use of the Lagrangian Density ``$\mathcal{L}$'', defined as\cite{QFT}:

\begin{equation}
L = \int \mathrm{d^{3}}x \mathcal{L}
\end{equation}

With the general form of the Lagrangian Density being defined as:

\begin{equation}
\mathcal{L} = \mathcal{L} ( \varphi_{i}, \partial _{\mu} \varphi_{i} )
\end{equation}

Where $\partial _{\mu}\varphi_{i} \equiv \partial \varphi / (\partial x^{\mu} )$ is the four-gradient of the field $\phi$ and where the i's are implicitly summed according to Einstein's summation convention\cite{ElectroweakStrong}.

The Lagrangian acts upon a system, with all information pertaining to the system's quantum state being described by a wave function. 
The amplitude of the wave function can be interpreted as the probability amplitude from which a measurement of an observable physical quantity can be obtained\cite{Isham}. 

An important feature of modern physical theories is that the laws of physics pertaining to a system do not vary under observation –- they are `invariant''. 
Examples of such invariant or conserved quantities include electrical charge from the $U(1)$ group’s symmetry in electromagnetism, energy-momentum from space-time symmetry and angular momentum from rotational symmetry\cite{Haywood}. 
These equivalent descriptions of the same system are related by groups of transformations, which if invariant when applied to the wave function, relate to observable properties\cite{QFT}. 
If the transformations on the system have no space-time dependence, the transformation is said to be ``global'', and if the transformations do have a space-time dependence then the transformation is said to be ''local''. 
A Lagrangian which has continuous local symmetry is said to be gauge invariant\cite{Haywood}. 

As defined above however, the Lagrangian Density $\mathcal{L}$ is not gauge invariant due to its dependence on the derivative $\partial _{\mu}$. To illustrate this, the Lagrangian which describes free-field fermions\cite{QFT}, 

\begin{equation}
\mathcal{L}_{0} = \bar{\psi}(i\gamma^{\mu}\partial_{\mu} - m)\psi
\end{equation}

Which when undergoing a local phase transformation,

\begin{equation}
\psi(x) \rightarrow \psi'(x) = \psi(x)e^{-ixf(x)}
\bar{\psi(x)} \rightarrow \bar{\psi'}(x) = \psi(x)e^{+ixf(x)}
\end{equation}

Transforms as:

\begin{equation}
\mathcal{L}_{0} \rightarrow \mathcal{L}_{0}' = \mathcal{L}_{0} + q \bar{psi}(x)\gamma^{\mu}\psi(x)\partial_{\mu}f(x)
\end{equation}

This transformation is clearly not invariant. Invariance can be restored by introducing a gauge field $A_{\mu}(x)$, associated with the $\psi(x)$ field, which transforms according to the gauge transformation. 
The minimal substitution in $\mathcal{L}_{0}$ which achieves this is the replacement of the derivative $\partial_{\mu}$  with the so-called ``covariant derivative''\cite{QFT}:

\begin{equation}
\partial_{\mu} \rightarrow D_{\mu} = [ \partial_{\mu} - icA_{\mu}(x) ]
\end{equation}

Which transforms as in the same way as the $\psi(x)$ field:

\begin{equation}
D_{\mu}\psi(x) \rightarrow e^{-icf(x)}D_{\mu}\psi(x)
\end{equation}

Thus the Lagrangian which describes the system in the QFT, remains invariant. 
The interaction between the vector gauge field $A_{\mu}(x)$ and the $\psi(x)$ can be interpreted as excitations in the vector field interacting with the particles described by $\psi(x)$, such as photons interacting with electrons in Quantum Electrodynamics. 
The constant c is the coupling constant for the vector field, which differs between different gauge fields. 
In the case of Quantum Electrodynamics, $c = q$, where q is the charge of an electron\cite{QFT}.

The differences between bosonic and fermionic particles can now be considered in the context of how they are affected by considering the individual particles within a system and how they are ordered. 
As particles with integer-spin must be quantised according to Bose-Einstein statistics and half-integer spin particles by Fermi-Dirac statistics, their wave functions must be symmetrical and anti-symmetrical respectively\cite{QM}:

\begin{equation}
\psi_{symmetric}(x_{a},x_{b}) = \psi_{symmetric}(x_{b},x_{a})
\psi_{anti-symmetric}(x_{a},x_{b}) = -\psi_{anti-symmetric}(x_{b},x_{a})
\end{equation}

As such, the Fermi Exclusion Principle, where two fermions are unable to exist in the same quantum state, can be formalised as\cite{QM}:

\begin{equation}
\psi_{anti-symmetric}(x_{a},x_{a}) = 0 \qquad \forall \quad a
\end{equation}

\subsection{Quantum Electrodynamics}\label{subsec:QED}
Quantum Electrodynamics (QED) is the theory which describes the Electromagnetic force between all electrically charged particles within with the SM. 
It has a single gauge boson, the neutrally charged photon. 
Because of the photon's lack of mass, the Electromagnetic force has an infinite range. 
As mentioned above, the Electromagnetic interaction conserves electrical charge, which is described by the $U(1)_{hypercharge}$ group symmetry in QFT\cite{QFT}. 

\subsection{Weak Interactions}\label{subsec:weakForce}
The Weak force is responsible for weak isospin processes. 
It has three massive gauge bosons, the electrically charged $W^{\pm}$ and neutral $Z^{0}$ bosons. 
The projection of the weak isospin along the z-axis is the conserved quantity of the Weak interaction, which in QFT is described by the $SU(2)_{weak isospin}$ group symmetry. 
The range and strength of the Weak force is considerably less than that of the Electromagnetic force due to the short lifespan of the massive gauge bosons\cite{ElectroweakStrong}. 


CP VIOLATION!	

\subsection{Electroweak Unification}\label{subsec:electroweak}
Both the Electromagnetic and Weak interactions can be described as a single interaction: the Electroweak interaction. 
The conserved quantity of this force, the weak hypercharge, is related to the conserved quantities of electrical charge and the z-projection of weak isospin, of its two constituent interactions. 
At sufficiently high energies, the two separate manifestations of the electroweak force unify into a single force. 
However, the $SU(2)_{weak isospin} \times U(1)_{hypercharge}$ symmetry of the electroweak interaction is not exact as whilst local invariance requires that the gauge boson fields be massless in order for the QFT to be renormalisable, the $W^{\pm}$ and $Z^{0}$ bosons are relatively massive. 
In order to retain a renormalisable theory, an additional mechanism, which introduces the masses of the weak bosons, must be introduced. 
The Higgs mechanism is the simplest solution to the breaking of the symmetry of the electroweak interaction\cite{LagrangiansSM}. 

One of the most pressing problems was that while the $SU(3)_{colour}$ symmetry is exact, the $SU(2)_{isospin} \times U(1)_{hypercharge}$ ``electroweak'' symmetry is said to be ``broken''. 
QFTs require massless vector fields in order to be locally invariant but the $W^{\pm}$ and $Z^{0}$ bosons are observed to be massive in contrast to the massless photon. 
The Higgs mechanism is the simplest solution to this paradox, with the scalar Higgs field being responsible for the massive bosons~\cite{oldcms}. 
Both the CMS and ATLAS experiments at CERN have independently confirmed the existence of an unknown boson at $\approx$ 125\GeV, which was later confirmed to be consistent with the Higgs Boson, the smallest possible excitation of its associated namesake field~\cite{HiggsCMS,HiggsATLAS}. 
Searches to determine whether this is the SM Higgs or not (several theories including SUSY propose multiple Higgs~\cite{Khalil:2003vd,Gianotti:2002xx}) will take place after the phase-0 upgrades of the LHC. 


\subsection{Quantum Chromodynamics}\label{subsec:QED}
Quantum Chromodynamics (QCD) is the theory which describes the strong force interactions between quarks and the eight massless gluons.

 with the non-Abelian 

The Strong Force is mediated by eight massless spin-1 gluons, which acts upon the conserved Strong Force charge: colour\cite{ElectroweakStrong}. 
The conservation of colour is described by the $SU(3)_{colour}$ group and the symmetry is exact. 
Colour charge is unrelated to the visual perception of colour, but stems from the fact that unlike the electroweak interaction which has positive and negative charges, there are three colour charges. 
An important difference between the gauge bosons of the electromagnetic and Strong Force is that whilst the photon is chargeless, gluons are not. 
Gluons carry both a colour and anti-colour charge and only interact with coloured particles (quarks and other gluons)~\cite{ElectroweakStrong}. 


A phenomena unique to the strong interaction is that the effective strong interaction coupling constant $\alpha_{s}$ tends to zero as the energy scales increase, despite the gluon being massless. 
In other words, the constant $\alpha_{s}$ increases as the separation between colour charged particles increases. 
This ``asymptotic freedom'' is caused by the virtual quark-anti quark sea containing virtual gluons which increases the force between quarks for greater separations, in contrast to the screening effect of virtual electrically neutral photons in electromagnetic interactions. 
The increase in $\alpha_{s}$ with the separation between colour charged particles means all particles must be ``colourless'', with the consequence that quarks are confined to exist in bound states\cite{ElectroweakStrong}. 
If enough energy is put into a bound state, with the intent of breaking ``colour confinement'', the energy of the colour fields between the coloured particles will increase until it is more energetically favourable for quark pair production to occur than to increase the separation between the original two quarks\cite{Griffiths}. 



\section{Top Physics}\label{sec:top-physics}
The existence of a third generation of quarks was first hypothesised in 1973 by Makoto Kobayashi and Toshihide Maskawa in order to account for the CP violation observed in kaon decays~\cite{Kobayashi:1973fv}.
While the discovery of the tau lepton in 1975 and the bottom quark in 1977~\cite{Herb:1977ek} reinforced the motivation for a weak isospin partner to the bottom quark, as the top quark's mass was larger than intially assumed, it would remain unobserved until a sufficiently powerful collider was built.
In 1995 the top quark was observed at the Tevatron at the Fermi National Accelerator Laboratory by the CDF and D\O experiments~\cite{Abe:1995hr,D0:1995jca}.

The top quark's mass, $m_{top}$, of $173.44 \pm 0.51 \pm 0.71 \GeV$ makes it the most massive known fundamental particle and imbues it with properties which have no equivalent for the other five quarks~\cite{LHC:2014combination}.


As the top quark has the largest mass of any fundamental particle, which results in it having the strongest coupling to the Higgs field, many have conjectured that the top quark has a special role to play in  electroweak symmetry breaking~\cite{}.


One of the most interesting of these properties is the top quark's lifetime of $5 \times 10^{-25}$ seconds which is less than the characteristic time of the strong interaction

Comment on lifespan - add diagram for decay modes. Comment on what it decays into.
Decays almost exclusively to Wb

With the mass of the top quark being close to the electroweak symmetry breaking scale, measuring top quark decay and single top quark production provides an excellent probe of the weak interaction and a means to indirectly search for evidence of new physics beyond the SM. 


Given the relationship between a fermion's mass and its decay width, in contrast to the light quarks, the the top quark's incomparable mass results in it having a lifetime of ... which is shorter than the characteristic time of the strong force.
 
As such, it is the only known ``bare'' quark in nature, even if only for a brief amount of time. 

 Top quarks will also be a significant component of the background for new signal searches in the TeV range, requiring an understanding of this signal in order to find new physics beyond the SM~\cite{Quadt}.

%%% UPDATE
The top quark has not been studied to the same extent as the other quarks due to its later discovery and  the relatively low top quark production rate at the Tevatron limiting statistics.
With the LHC's design centre-of-mass energy and instantaneous luminosity being approximately two orders of magnitude greater than the Tevatron's, study of  


due to the relatively low production rate, and thus statistics, the data which was used to analyse the top quark's properties was limited.
Because of the LHC’s greater operational energy and integrated luminosity, greater statistics will be available to probe the nature of the top quark~\cite{Shibata:2008sy}. 

\subsection{Top quark pair production}\label{subsec:ttbarTheory}
Top quarks are predominantly produced by pair-production (\ttbar) through strong interactions at hadron colliders.
As illustrated in the Feynman diagrams in figure~\ref{fig:feyn_ttbar}, at Leading Order (LO) \ttbar events are produced by either gluon fusion or quark anti-quark annihilation. 
While approximately 90\% of \ttbar events produced at the Tevatron occured via quark fusion, 80-90\% of \ttbar events at the LHC are produced by gluon fusion for $\sqrt{s} = 8-14\TeV$~\cite{}.
These differences result from 
of both the higher centre-of-mass energy of the LHC resulting 

\begin{figure}[htbp]
\begin{center}
\includegraphics[width=0.97\textwidth]{figs/top-physics/ttbar_feyn.jpg}
\caption{The three Leading Order Feynman diagrams for top quark pair production at hadron colliders. Quark-anti quark annihilation is illustrated on the top row and gluon fusion on the bottom. Gluon fusion is the main production mode at the LHC and quark fusion at the Tevatron.}
\label{fig:feyn_ttbar}
\end{center}
\end{figure}

Top quark pair production can also occur in association with a vector boson, albeit at relatively small cross sections compared to both \ttbar and singly produced top quarks.
Despite their cross sections, these channels are important backgrounds which require detailed understanding in order to be able to probe for new physics with similar or smaller cross sections, such as $\ttH	$ or tZq~\cite{Khachatryan:2014ewa}.

\subsection{Single top quark production}\label{subsec:singleTopTheory}
Top quarks can also be produced singly through weak interactions, albeit with smaller cross sections than \ttbar given the relative weakness of the electroweak coupling compared to the strong coupling.

Such processes are an powerful probe between the top and electroweak sector of the SM as they allow the direct measurement of the $\abs{V_{tb}}$ element of the Cabibbo-Kobayashi-Maskawa (CKM) matrix and thus test whether the CKM matrix is unitary as presumed or otherwise~\cite{Shibata:2008sy}.

Measuring single top quark processes is also important as such processes are backgrounds not only for 

for Higgs and BSM physics searches, such as anomalous couplings, and can be used to constrain Parton Distribution Functions (PDFs).	


The main SM single top production mechanisms are categorised by the virtuality of the W boson involved in the interaction.
As each of these production channels have differing initial and final states they are typically considered separately.

s-channel production, as shown in figure~\ref{fig:singleTopDiagrams}(a), is the quark anti-quark annihilation into an off-shell W boson which decays into a top and anti-bottom quark.
This process has the lowest single top production cross section at the LHC given the virtuality of the W boson required to produce the top quark and need for the anti-quark to originate from a sea quark.
Given its low cross section and a final state topology similar to larger background processes, the s-channel has yet to be observed at the LHC~\cite{Khachatryan:2016ewo}.

The t-channel, as shown in figure~\ref{fig:singleTopDiagrams}(b), is the dominant single top prodution mechanism at the LHC.
The process involves the scattering of the W boson off a bottom quark originating either from a sea quark or from gluon splitting.
Initially observed at the Tevatron~\cite{Aaltonen:2009jj,Abazov:2009ii}, the t-channel has since been studied at higher energies at the LHC, with all results to date remaining consistent with the SM~\cite{Berta:2017ghf,Morton:2018wkb}.	

The tW-channel, as shown in figure~\ref{fig:singleTopDiagrams}(c), is where a top quark is produced in association with an on-shell W boson.
In contrast to being negligible at the Tevatron, the tW-channel is accessible at LHC and was discovered in 2014~\cite{Chatrchyan:2014tua}.
The discovery of the tW-channel was particularly anticipated as at the presence of the on-shell W boson in the final state makes it an excellent probe of the Wtb vertex and 

\begin{figure}[!h]
\centering
\includegraphics[width=1.00\textwidth]{figs/top-physics/singletop_feyn.jpg}
\caption{Leading order diagrams for single top quark production channels via weak processes: (a) s-channel, (b) t-channel and (c) tW.}
\label{fig:singleTopDiagrams}
\end{figure}

\subsection{Single top production in association with a Z boson}\label{subsec:tZqTheory}
The production of a single top quark in association with a \Z0 boson with an additional jet, referred to as \emph{tZq}, is a rare SM process that is based on single top production in the t-channel.
As illustrated in figure~\ref{fig:feyn_tZq}, the \Z0 boson is radiated off

This is a rare SM  The $Z^{0}$ boson is radiated off one of the quark legs, or from an exchanged W boson (F).
A greater understanding of the behaviour of this background is paramount to searches for BSM physics as this process is an irreducible background for many such searches of new physics (such as for Flavour Changing Neutral Current processes)~\cite{Quadt}. 

\begin{figure}[p]
\centering
\includegraphics[width=0.47\textwidth]{figs/top-physics/tZq_feyn1.jpg}
\includegraphics[width=0.47\textwidth]{figs/top-physics/tZq_feyn2.jpg}
\includegraphics[width=0.47\textwidth]{figs/top-physics/tZq_feyn3.jpg}
\includegraphics[width=0.47\textwidth]{figs/top-physics/tZq_feyn4.jpg}
\includegraphics[width=0.47\textwidth]{figs/top-physics/tZq_feyn5.jpg}
\includegraphics[width=0.47\textwidth]{figs/top-physics/tZq_feyn6.jpg}
\caption{Leading order tZq production diagrams. Diagram (f) represents the non-resonant contribution to the tqZ process.}
\label{fig:feyn_tZq}
\end{figure}

Prior to the first run of the LHC, proton-antiproton data for the top quark was collected at the Tevatron during both Run-I (at a centre-of-mass energy of 1.8 TeV) and Run-II (at a centre-of-mass energy of 1.96 TeV), corresponding to an integrated luminosity of 8.7 \fbinv.
The \ttbar production (di-lepton, lepton+jets and all-jets) cross section was within 9\% of the expected SM result~\cite{Lister:2008it} and the measurement of the top quark’s mass had a relative precision of 0.75\%~\cite{Group:2009ad}.
The Tevatron also found both the first evidence for the production of single top quarks~\cite{Abazov:2006gd} as well as discovering the production channel and from the measured cross-section, both collaborations were able to directly determine the CKM matrix element which describes the Wtb coupling and determine that it had a 95\% confidence level of being consistent with the SM~\cite{Aaltonen:2009jj}.

During 2012, $19.7\pm0.5 \fbinv$ at a centre-of-mass energy of 8 TeV (roughly four times larger than the 7 TeV data) was collected by the CMS Collaboration at the LHC. 
Analysis of the top quark pair production cross-section is generally in good agreement with the SM predictions at next-to-next-to-leading order (NNLO). 
The pT spectrum for data for leptons, jets and top quarks however, is softer than the predictions and similarly to CMS measurements at 7\TeV, the NLO+NNLO calculations fails to describe the data for all $\pT^{\ttbar}$ values~\cite{Khachatryan:2015oqa}. 
Measurements of the W boson helicity fractions from the decay of a single top quark, at centre-of-mass energies of 8 TeV, are in agreement with the SM predictions at NNL0  and have a similar precision to the W boson helicity fractions measurements from top quark pair production at 7 TeV at the LHC~\cite{Khachatryan:2014vma}.

At the LHC, for proton-proton collisions at 8 TeV the predicted cross-section at next-to-leading order (NLO) for SM tZ production is~\cite{Campbell:2013yla}:

\begin{equation}
\sigma(tZ)= 160_{-2}^{+7} (scale)_{-11}^{+11} (PDF) fb \;.
\sigma(\bar{t}Z)= 76_{-1}^{+4} (scale)_{-5}^{+5} (PDF) fb \;.
\end{equation}

This leads to an overall cross-section of $236_{-16}^{+19}$ fb. 
These cross-sections were determined using the CTEQ6M set of Parton Distribution Functions (PDF)~\cite{Pumplin:2002vw}. 
The $\bar{t}$Z cross-section is roughly half that of the tZ cross-section due to the ratio of the up quark PDF to the down quark PDF is circa 0.5 in the x range typical for these processes\cite{Campbell:2013yla}. 
The CMS collaboration have measured the $\ttbar$Z cross-section to be\cite{Khachatryan:2014ewa}

\begin{equation}
\sigma (\ttbar Z)= 200_{-70}^{+80} (statistics)_{-30}^{+40} (systematics) fb \;.
\end{equation}

The cross-section between $\ttbar$Z and the sum of the tZ and $\bar{t}$Z cross-sections is comparable as whilst single top + Z processes are electroweak interactions (in contrast to the QCD-induced top quark pair production) they have fewer daughters in the final state and the Z boson condition for $\ttbar$Z considerably reduces the rate\cite{Campbell:2013yla}.
As such, a search for tZ processes should be achievable during CMS Run-1 data. This is supported by CMS’ Run-I results for $\ttbar$Z~\cite{Khachatryan:2014ewa}. 
A CMS Analysis Note and a journal paper supporting a three sigma probability of evidence of single top production in association with a Z boson and a jet are currently being produced by the HEP Group at Brunel University London~\cite{Sirunyan:2017kkr}.

Until now, the 2012 data taken at 8 TeV, has been used by the research group to analyse these processes.
Following the restart of the LHC after the phase-0 upgrades, new data are anticipated to be available with the increase to a higher centre-of-mass energy and with anticipated luminosities of up to 3000~\fbinv\cite{ECFA}. 
Such statistics will be used to gain more precise measurements of top quark pair production and the production of single top quarks and as a result, a better understanding of the underlying processes involved.


\section{Beyond The Standard Model}\label{sec:bsm}
The SM has been incredibly successful at accurately predicting the majority of the properties of the known fundamental particles up to the electroweak scale.
Despite this however, given the inability of the SM to incorporate gravity and address a number of experimental observations, it is apparent that there must be physics Beyond the Standard Model (BSM).

Gravity currently is described by the extremely successful classical theory of General Relativity (GR).
Despite attempts to reconcile GR with the SM, including contradictory results such as the predicted cosmological constant and Higgs field's vacuum energy density, as to date no successful quantum theory of gravity has been produced~\cite{Sola:2013gha}.

The SM is also incapable of explaining other astronomical and cosmological inconsistencies that have been observed.
A form of \emph{Dark Matter} which 
If galaxies consisted solely of observable matter then the rotation  
From the gravitational lensing caused by ~\cite{Einstein:1956zz,}
Given 
and the galaxy rotation curves 
The SM does not provide any 
Dark matter and energy

dark energy 

One of the most apparent observations that the SM 
shortcomings of the SM is the lack of an explanation for why we observe an asymmetry in the quantities of matter and anti-matter in the observable universe.
Andrei Sakharov 

The most ...  

One of the greatest inconsistencies observed 

Whilst the SM assumes that neutrinos are massless, the ``Homestake'' experiment's measurements showed that the fraction of electron neutrinos arriving from the Sun was at the most half (if not less) what was expected\cite{PhysRevLett.20.1205}. 
Neutrino flavour oscillation would explain the observed solar neutrino flux, but would require neutrinos to have a non-zero mass. 
In 2013, the T2K collaboration presented results which confirmed the existence of neutrino oscillation\cite{PhysRevD.88.032002}. 
Whilst there are upper bounds on the neutrino masses from cosmological constraints, no experiment to date has been sensitive enough to determine the masses\cite{1475-7516-2006-06-019}. 

Neutrinos have been observed to oscillate between different flavours, implying that they have non-zero masses in contrast to the SM~\cite{Fukuda:1998mi,Ahmad:2001an}.
%% T2K Collaboration - PhysReVD
%% Super-Kamiokande - 1998mi, SNO - Sudbury Neutrino Observatory

GUT
Hierarchy Problem
Fine tuning problem

In addition to the above, many physicists 

\item On top of the above inconsistencies, many scientists are uncomfortable with the fact that the SM contains a large number of finely tuned experimentally derived parameters and hope that these values would emerge naturally from a more ``complete'' description of reality~\cite{}.

%%% Solutions
Many theories have been proposed
