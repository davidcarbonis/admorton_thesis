\section{The Standard Model of Particle Physics}\label{sec:sm}

The Standard Model (SM) is the current model which describes the fundamental matter particles of nature, fermions, and their interactions with three of the fundamental forces of nature: the weak, strong, and electromagnetic forces through force carrying particles (the gauge bosons)\cite{LagrangiansSM}.
An important property which distinguishes between the fermions and gauge bosons is spin. 
Spin is an intrinsic property of particles, with each particle having a specific quantum value, and can be likened to, despite being different from, angular momentum from classical mechanics\cite{QM}. 
The spin quantum number, $s$,  takes half-integer values, with spin z-direction, $s_z$, having a sign denoting whether the spin is polarised either along the same direction as the z-axis (usually a ''positive'' sign) or the opposing direction of the z-axis (usually a ''negative'' sign)\cite{QM}. 
Fermions are half-integer spin particles (i.e. $s = \pm\[\frac{1}{2}\],\pm\[\frac{3}{2}\],\pm\[\frac{5}{2}\],…$) which have three so-called ''generations'', and belong to one of two families: quarks and leptons\cite{ElectroweakStrong}. 
Quarks experience all of the fundamental forces of nature, whilst leptons experience all but the strong force\cite{LagrangiansSM}. 
In each generation, for fermions and quarks alike, there are two different fundamental particle\cite{LagrangiansSM}. 
Each subsequent generation of particles are identical, except for their quantum number and mass. 
Quark particles in each generation either have an electrical charge of \[\frac{+2}{3}\] or \[\frac{-1}{3}\] and fermions have either electrical charge -1 or 0 (neutral)\cite{ElectroweakStrong}. 

Quarks are the fundamental particles of which hadrons, composite particles formed of quarks, are formed. 
Hadrons are either mesons which are formed of two quarks or baryons which are formed of three quarks. 
Exotic hadrons formed of larger groupings (four or more) of quarks have been hypothesised, but only one resonance, namely a tetraquark candidate whose quark content still has to be confirmed, has been observed\cite{PhysRevLett.112.222002}. 
The first generation of quarks comprises of the up and down quarks, which form the protons and neutrons that are found in conventional atomic matter. 
The second and third generations are each subsequently more massive than the first generation and comprise of the strange and charm quarks and top and bottom quarks respectively. 

Each charged lepton has an associated neutral, near massless, lepton known as a neutrino. 
As neutrinos have no associated electrical charge, their only interaction with other particles in the SM is through the weak force. 
As with the quarks, each subsequent generation’s particles are more massive than the last. 
Whilst the SM assumes that neutrinos are massless, the ''Homestake'' experiment’s measurements showed that the fraction of electron neutrinos arriving from the Sun was at the most half (if not less) what was expected\cite{PhysRevLett.20.1205}. 
Neutrino flavour oscillation would explain the observed solar neutrino flux, but would require neutrinos to have a non-zero mass. 
In 2013, the T2K collaboration presented results which confirmed the existence of neutrino oscillation\cite{PhysRevD.88.032002}. 
Whilst there are upper bounds on the neutrino masses from cosmological constraints, no experiment to date has been sensitive enough to determine the masses\cite{1475-7516-2006-06-019}. 

In the SM there are four gauge bosons, each of which is an integer spin particle (i.e. 0, \pm1, \pm2, …) that mediate the weak, strong and electromagnetic forces. 
The photon (\gamma), a massless particle, mediates the electromagnetic force, the charged W\pm and neutral $Z^0$ boson mediates the weak force and eight types of gluon mediate the strong force\cite{LagrangiansSM}. 

The mathematical formulation of the SM model is through renormalisable Quantum Field Theory (QFT). 
QFTs treat matter as the excitation of fermionic fields which permeate the Universe. 
The Lagrangian formalisation, used in QFTs to describe the dynamics of a system, has the Lagrangian ''$L$'' described as the difference between the kinetic and potential energy of the system. 
QFTs usually make use of the Lagrangian Density ''$\mathscr{L}$'', defined as:

equation

With the general form of the Lagrangian Density being defined as:

equation

Where ∂_μ φ_i  ≡  ∂φ/(∂x^μ ) is the four-gradient of the field ϕ and where the i’s are implicitly summed according to Einstein summation convention.

The Lagrangian acts upon a system, with all information pertaining to the system’s quantum state being described by a wave function. 
The amplitude of the wave function can be interpreted as the probability amplitude from which a measurement of an observable physical quantity can be obtained. 

An important feature of modern physical theories is that the laws of physics pertaining to a system do not vary under observation – they are “invariant”. 
Examples of such invariant or conserved quantities include electrical charge from the U(1) group’s symmetry in electromagnetism, energy-momentum from space-time symmetry and angular momentum from rotational symmetry. 
These equivalent descriptions of the same system are related by groups of transformations, which if invariant when applied to the wave function, relate to observable properties. 
If the transformations on the system have no space-time dependence, the transformation is said to be “global”, and if the transformations do have a space-time dependence then the transformation is said to be “local”. 
A Lagrangian which has continuous local symmetry is said to be gauge invariant. 

However, as defined above, the Lagrangian Density L is not gauge invariant due to its dependence on the derivative ∂_μ. To illustrate this, the Lagrangian which describes free-field fermions, 

equation

Which when undergoing a local phase transformation,

equation

Transforms as:

equation

This transformation is clearly not invariant. Invariance can be restored by introducing a gauge field A_μ (x), associated with the ψ(x) field, which transforms according to the gauge transformation. 
The minimal substitution in〖 L〗_0 which achieves this is the replacement of the derivative ∂_μ  with the so-called “covariant derivative”:

equation

Which transforms as in the same way as the ψ(x) field:

equation

Thus the Lagrangian which describes the system in the QFT, remains invariant. 
The interaction between the vector gauge field A_μ (x) and the ψ(x) can be interpreted as excitations in the vector field interacting with the particles described by ψ(x), such as photons interacting with electrons in Quantum Electrodynamics. 
The constant c is the coupling constant for the vector field, which differs between different gauge fields. 
In the case of Quantum Electrodynamics, c=q – where q is the charge of an electron.

The differences between bosonic and fermionic particles can now be considered in the context of how they are affected by considering the individual particles within a system and how they are ordered. 
As particles with integer-spin must be quantised according to Bose-Einstein statistics and half-integer spin particles by Fermi-Dirac statistics, their wave functions must be symmetrical and anti-symmetrical respectively:

equation
equation

As such, the Fermi Exclusion Principle, where two fermions are unable to exist in the same quantum state, can be formalised as:

equation

Quantum Electrodynamics (QED) is the theory which describes the Electromagnetic force between all electrically charged particles within with the SM. 
It has a single gauge boson, the neutrally charged photon. 
Because of the photon’s lack of mass, the Electromagnetic force has an infinite range. 
As mentioned above, the Electromagnetic interaction conserves electrical charge, which is described by the U(1) hypercharge group symmetry in QFT. 

The Weak force is responsible for weak isospin processes. 
It has three massive gauge bosons, the electrically charged W± and neutral Z0 bosons. 
The projection of the weak isospin along the z-axis is the conserved quantity of the Weak interaction, which in QFT is described by the SU(2)weak isospin group symmetry. 
The range and strength of the Weak force is considerably less than that of the Electromagnetic force due to the short lifespan of the massive gauge bosons. 

Both the Electromagnetic and Weak interactions can be described as a single interaction: the Electroweak interaction. 
The conserved quantity of this force, the weak hypercharge, is related to the conserved quantities of electrical charge and the z-projection of weak isospin, of its two constituent interactions. 
At sufficiently high energies, the two separate manifestations of the electroweak force unify into a single force. 
However, the SU(2) weak isospin x U(1) hypercharge symmetry of the electroweak interaction is not exact as whilst local invariance requires that the gauge boson fields be massless in order for the QFT to be renormalisable, the W± and Z0 bosons are relatively massive. 
In order to retain a renormalisable theory, an additional mechanism, which introduces the masses of the weak bosons, must be introduced. 
The Higgs mechanism is the simplest solution to the breaking of the symmetry of the electroweak interaction. 

The fundamental theory of strong interactions is Quantum Chromodynamics (QCD). 
The strong force is mediated by eight massless spin-1 gluons, which acts upon the conserved strong force charge: colour. 
The conservation of colour is described by the SU(3) colour group and the symmetry is exact. 
Colour charge is unrelated to the visual perception of colour, but stems from the fact that unlike the electroweak interaction which has positive and negative charges, there are three colour charges. 
An important difference between the gauge bosons of the electromagnetic and strong forces is that whilst the photon is chargeless, gluons are not. 
Gluons carry both a colour and anti-colour charge and only interact with coloured particles (quarks and other gluons). 

A phenomena unique to the strong interaction is that the effective strong interaction coupling constant alpha_s tends to zero as the energy scales increase, despite the gluon being massless. 
In other words, the constant α_(s) increases as the separation between colour charged particles increases. 
This “asymptotic freedom” is caused by the virtual quark-anti quark sea containing virtual gluons which increases the force between quarks for greater separations, in contrast to the screening effect of virtual electrically neutral photons in electromagnetic interactions. 
The increase in α_s with the separation between colour charged particles means all particles must be “colourless”, with the consequence that quarks are confined to exist in bound states. 
If enough energy is put into a bound state, with the intent of breaking “colour confinement”, the energy of the colour fields between the coloured particles will increase until it is more energetically favourable for quark pair production to occur than to increase the separation between the original two quarks. 

\subsection{Beyond The Standard Model}\label{bsm}

The SM is far from being a complete theory. 
One of the most pressing problems was that while the SU(3) colour symmetry is exact, the SU(2) isospin x U(1) hypercharge “electroweak” symmetry is said to be “broken”. 
QFTs require massless vector fields in order to be locally invariant but the W± and Z0 bosons are observed to be massive in contrast to the massless photon. 
The Higgs mechanism is the simplest solution to this paradox, with the scalar Higgs field being responsible for the massive bosons. 
Both the CMS and ATLAS experiments at CERN have independently confirmed the existence of an unknown boson at ≈ 125GeV, which was later confirmed to be consistent with the Higgs Boson, the smallest possible excitation of its associated namesake field. 
Searches to determine whether this is the SM Higgs or not (several theories including SUSY propose multiple Higgs) will take place after the phase-0 upgrades of the LHC. 

Aside from electroweak symmetry breaking, there are a number of other fundamental phenomena that the SM does not account for. 
These include, but are not limited to, the lack of explanation for gravity, the lack of candidates for dark matter, and observed matter/antimatter asymmetry in the universe. 
Gravity is left unexplained as the SM, a quantum mechanical theory, is incompatible with General Relativity, a classical theory. 

Supersymmetry (SUSY), a popular extension of the SM, makes progress with these questions by proposing that every fundamental fermion has a massive boson “superpartner” (and vice versa for the gauge bosons). 
The superpartners generalise space-time symmetries, allowing for bosons and fermions to be related, a straightforward unification of the strengths of the weak, electromagnetic and strong interactions at high energies and a more “natural” emergence of the Higgs potential. 
These supersymmetric particles (sparticles) spin differs from their SM counterparts by 1/2 (ie. bosons’ supersymmetric partners are fermions, and vice versa). 
As supersymmetry is ‘broken’, the expected sparticle masses are considerably greater than their equivalent partner masses. 

Additionally, some SUSY models also include candidates for a dark matter particle which is consistent with observations. 
However, while superpartners should be observed at the LHC if SUSY exists at the electroweak scale, no superpartner has been observed so far, with results being used to constrain the range of the possible superpartner masses. 
If SUSY, in any of its forms, exists, then higher energy runs of the LHC will be required to illuminate it. 


