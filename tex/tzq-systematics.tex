\chapter{Background Estimation}\label{chapter:bkg}
Monte Carlo simulated samples are used to model the signal and background processes expected to be observed in the proton-proton collision data used.
Despite the efforts made to ensure that these samples accurately represent all the effects 

all the effects observed in 
Typically, most of the discrepancies observed can be accounted for by applying various corrective scale factors.
Where a particular process is poorly described and/or lacking sufficient statistics in the signal region, data-driven estimates of the background are used to model .

\section{Data and Simulation Samples}\label{sec:samples}
Out of the 37.8\fbinv of the proton-proton collision data at $\sqrt{13}$ collected by CMS during 2016, 35.8\fbinv was certified by the collaboration as ``good'' to be used for physics analysis.

The difference between the certified value and the total data recorded is the result of various factors such as the unavailability of a detector.
Due to the prescaling of the electron triggers during the start of the most luminous runs, as discussed in Chapter~\ref{sec:triggerStrategy}, the ee channel uses a reduced dataset of 35.6\fbinv were none of the triggers considered were prescaled.
Events in the double lepton and single lepton datasets from across these ``good'' data runs are considered where the double and single lepton triggers respectively have fired, using the strategy described in Chapter~\ref{sec:triggerStrategy}.

The MC samples used to model signal and background processes that were considered are listed in Table~\ref{tab:mcList}, which includes information on the number of events generated, their cross sections and the order in perturbative accuracy in QCD to which the generators calculated the processes.

To determine the impact of a number of theoretical uncertainties for a number of processes, a several dedicated samples, listed in Table~\ref{tab:theorySampleList}, were used.
These systematic uncertainties are discussed further in Chapter~\ref{sec:theorySysts}.

For all the MC samples considered, the hadronisation of all the MC samples considered was undertaken using PYTHIA 8.
The NNPDF3.0 family of PDF sets was used as input for the generators of the MC samples, where the corresponding PDF sets were used depending on whether the sample was produced at LO or NLO and used either the four or five flavour scheme.

\begin{table}[htbp]
\topcaption {
The MC processes and their associated total number of events, cross sections and generators (and order in perturbative QCD accuracy they are calculated to), considered for the search for tZq in the dilepton final state. Both generators considered for the Z+jet background are also listed below.
}
\label{tab:mcList}
  \centering
  \resizebox{\textwidth}{!}{
% This right-aligns numbers in column, but centers them under column title.
 \begin{tabular}{cccc}
   \hline
   \textbf{MC process} & \textbf{Events} & \textbf{Cross section (pb)} & \textbf{Generator (Order)}   \\
   \hline
   tZq  & 14.5M & 0.0758  & aMC@NLO (NLO) \\
   \hline
   tHq  & 3.5M & 0.07462  & Madgraph (LO) \\
   \hline
   tWZ/tWll  & 50K & 0.01104  & Madgraph (LO) \\
   \hline
   t tW-channel & 7M & 35.85 & POWHEG (NLO) \\
   $\overline{\text{t}}$ tW-channel & 6.9M & 35.85 & POWHEG (NLO) \\
   \hline
   t s-channel & 2.9M & 10.32 & aMC@NLO (NLO) \\
   \hline
   t t-channel & 67.2M & 136.02 & POWHEG (NLO) \\
   $\overline{\text{t}}$ t-channel & 38.8M & 80.95 & POWHEG (NLO) \\
   \hline
   \ttbar & 77.1M & 831.76 & POWHEG (NLO) \\
   \hline
   \ttbarZ $\rightarrow$ ll$\nu\nu$ & 13.9M & 0.2529   & aMC@NLO (NLO) \\
   \ttbarZ $\rightarrow$ qq & 749K & 0.5297   & aMC@NLO (NLO) \\
   \hline
   \ttbarW $\rightarrow$ l$\nu$ & 5.3M & 0.2001   & aMC@NLO (NLO) \\
   \ttbarW $\rightarrow$ qq & 833K & 0.405  & aMC@NLO (NLO) \\
   \hline
   \ttbarH $\rightarrow$ bb & 3.8M & 0.2942 & POWHEG (NLO) \\
           $\rightarrow$ non bb & 4.0M & 0.2123 & POWHEG (NLO) \\
   \hline
   W+jets & 24.1M & 61526.7 & aMC@NLO (NLO) \\
   \hline
   Z+jets ($m_{Z} \geq 50\GeVcc $ & 146M & 5765.4 & Madgraph (LO) \\
   Z+jets ($10 \GeVcc \leq m_{Z} < 50\GeVcc$ & 35.3M & 18610.0 & Madgraph (LO) \\
   \hline
   Z+jets ($m_{Z} \geq 50\GeVcc $ & 151M & 5765.4 & aMC@NLO (NLO) \\
   Z+jets ($10 \GeVcc \leq m_{Z} < 50\GeVcc$ & 106M & 18610.0 & aMC@NLO (NLO) \\
   \hline
   WW $\rightarrow$ l$\nu$qq & 9.0M & 49.997  & POWHEG (NLO) \\
      $\rightarrow$ ll$nu\nu$ & 2.0M & 12.178 & POWHEG (NLO) \\
   \hline
   WZ $\rightarrow$ l$\nu$qq & 24.2M & 10.73 & aMC@NLO (NLO) \\
      $\rightarrow$ llqq & 26.5M & 5.606 & aMC@NLO (NLO) \\
      $\rightarrow$ lll$\nu$ 1.9M & 5.26 & aMC@NLO (NLO) \\
   \hline
   ZZ $\rightarrow$ ll$\nu\nu$ & 8.8M & 0.5644 & POWHEG (NLO) \\
      $\rightarrow$ llqq & 15.3M & 3.222 & aMC@NLO (NLO) \\
      $\rightarrow$ llll & 10.7M & 1.204 & aMC@NLO (NLO) \\
   \hline
   WWW & 240K & 0.2086 & aMC@NLO (NLO) \\
   \hline
   WWZ & 250K & 0.1651 & aMC@NLO (NLO) \\
   \hline
   WZZ & 247K & 0.05565 & aMC@NLO (NLO) \\
   \hline
   ZZZ & 249K & 0.01398 & aMC@NLO (NLO) \\
   \hline
   
 \end{tabular}}
\end{table}

\begin{table}[htbp]
\topcaption {
The dedicated MC samples used to determine the impact of theoretical uncertainties, including the associated total number of events, cross sections and generators (and order in perturbative QCD accuracy they are calculated to), considered for the search for tZq in the dilepton final state.
}
\label{tab:theorySampleList}
  \centering
 \resizebox{\textwidth}{!}{
 \begin{tabular}{cccc}
   \hline
   \textbf{MC process} & \textbf{Events} & \textbf{Cross section (pb)} & \textbf{Generator (Order)}   \\
   \hline
   tZq scale up & 6.9M & 0.0758  & aMC@NLO (NLO) \\
   tZq scale down & 7.0M & 0.0758  & aMC@NLO (NLO) \\
   \hline
   t tW-channel scale up & 998K & 35.85 & POWHEG (NLO) \\
   t tW-channel scale down & 994K & 35.85 & POWHEG (NLO) \\
   $\overline{\text{t}}$ tW-channel scale down & 1.0M & 35.85 & POWHEG (NLO) \\
   $\overline{\text{t}}$ tW-channel scale down & 999K & 35.85 & POWHEG (NLO) \\
   \hline
   t t-channel scale up & 5.7M & 136.02 & POWHEG (NLO) \\
   t t-channel scale down & 5.9M & 136.02 & POWHEG (NLO) \\
   t t-channel matching up & 6.0M & 136.02 & POWHEG (NLO) \\
   t t-channel matching down & 6.0M & 136.02 & POWHEG (NLO) \\
   $\overline{\text{t}}$ t-channel scale up & 4.0M & 80.95 & POWHEG (NLO) \\
   $\overline{\text{t}}$ t-channel scale down & 3.9M & 80.95 & POWHEG (NLO) \\
   $\overline{\text{t}}$ t-channel matching up & 4.0M & 80.95 & POWHEG (NLO) \\
   $\overline{\text{t}}$ t-channel matching down & 4.0M & 80.95 & POWHEG (NLO) \\
   \hline
   \ttbar ISR up & 156.5M & 831.76 & POWHEG (NLO) \\
   \ttbar ISR down & 149.8M & 831.76 & POWHEG (NLO) \\
   \ttbar FSR up & 152.6M & 831.76 & POWHEG (NLO) \\
   \ttbar FSR down & 156.0M & 831.76 & POWHEG (NLO) \\
   \ttbar matching up & 58.9M & 831.76 & POWHEG (NLO) \\
   \ttbar matching down & 58.2M & 831.76 & POWHEG (NLO) \\
   \hline   
 \end{tabular}}
\end{table}

\section{Simulation Corrections}\label{sec:simCorrections}
Simulation is unable to fully recreate all the effects observed in data, either because certain parameters are not precisely known or cannot be be calculated.
To account for these discrepancies, corrective scale factors are used to reweight MC on a per event basis.
Such scale factors are usually derived as a function of \pt and $\eta$ so as to account for the variation of the detector response in both.
These corrections are used to correct simulation for lepton identification, isolation and reconstruction efficiencies, b-tagging efficiencies, the poor modelling of pileup in simulation, and the detector resolutions observed in data.

\subsection{Miscalibrated Tracker APV}\label{subsec:hipEffect}
During the first half of data taking in 2016 the silicon strip detector suffered from instantaneous luminosity dependent  hit finding inefficiencies, particularly in high occupancy regions, due to saturation in the pre-amplifier in the front end electronics~\cite{Fiori:2016ebh}.
This issue was resolved by changing the configuration of the electronics.
While the affected part of the dataset has been reprocessed to mitigate the impact on the quality of the data taken, there is still a negative impact on the detector efficiency for objects that rely upon highly efficient tracking data.
This is accounted for by the weighting of events appropriately when the scale factors are produced, either centrally by CMS or those derived for the analysis (\ie the trigger scale factors), so that a single scale factor is applied to a simulated event.

%In most cases, centrally produced scale factors are derived for the whole of the 2016 dataset to account for this, but 
%for muons they were provided for both the affected and unaffected parts of the dataset separately.
%Consequently, the muon scale factors were weighted according to the luminosity they corresponded to before their application to simulation.
%Similarly, the muon trigger scale factors that were derived were also produced 

\subsection{Lepton Efficiency}\label{subsec:leptonRecoSFs}
The identification, isolation and reconstruction efficiencies of leptons are calculated using measurements of $Z \rightarrow l^{+} l ^{-}$ events with the \emph{tag-and-probe} method~\cite{CMS:2008rxa}.
Using events within a dilepton invariant mass window to ensure a high purity, from this large statistics lepton sample, the method ``tags'' and ``probes'' the leptons where one has passed a tight and the other a loose selection criteria.
For source of each efficiency and lepton flavour, the efficiency is given as the fraction of events where the probe leptons passed the relevant selection criteria.
This methodology is used to create corrective scale factors for each component and these are multiplicatively applied to each leptons' event weight as functions of their \pt, $\eta$, and flavour.

The trigger efficiency of electrons and muons is calculated using a method which considers events that pass the lepton selection criteria in the signal region~\ref{sec:signalRegion} and which are selected by triggers which are weakly correlated (also known as cross triggers) with the triggers used in the analysis~\cite{Khachatryan:2016kzg}.
From this collection of events, the number of events that pass and fail the analysis triggers are counted to produce the trigger efficiency:

\begin{equation}
\epsilon_{trigger} = \frac{N_{X triggers + lepton triggers}}{N_{X triggers}} \\
\end{equation}

where $N_{X trigger}$ is the number of events which have passed the lepton selection criteria and the cross triggers, and $N_{X triggers + lepton triggers}$ is $N_{X trigger}$ and the number of events which have also passed the lepton triggers.

As the triggers requirements are applied to both simulated and data events, a scale factor of the ratio of the trigger efficiency in data and in simulation is applied to the event weight in simulation.

%For the scale factors derived for the ee and e$\mu$ channels, a constant scale factor was found to be sufficient to account for the differences between data and siulation.
%In the case of the $\mu\mu$ channel however, the scale factor was produced as a function of both \pt and $\eta$ due to the trigger turn-on curve in data being impacted by the miscalibrated tracker APV (as discussed in Chapter~\ref{subsec:hipEffect}.

\subsection{Lepton Energy Corrections}\label{subsec:leptonEnergyCorrections}
\subsubsection{Electron Regression and Energy Scale and Smearing Corrections}
Two types of energy corrections which have been produced by the CMS EGM POG are applied to electrons and photons, energy regression and energy scale and smearing corrections.
These corrections are applied to both MC simulation and data and are used to improve the electron resolution obtained and to resolve the observed discrepancies between them.

Using simulation for tuning, energy regression obtains the best possible energy resolution by using the detector information to correct the reconstructed object energy.
The disagreement between data and MC is resolved by scaling the data energy to the MC energy scale and smearing the MC so that it has the same energy resolution as data. 

These corrections are pre-applied onto the PF electron collections used.

\subsubsection{Rochester Corrections}
The muon momentum scale and resolution correction methods developed by the University of Rochester~\cite{rochester}, known as \emph{Rochester Corrections}, are used to remove any muon momentum bias from any detector misalignment, reconstruction or uncertainties in the magnetic field for both MC and data.
These corrections are derived with high \pt ($> 20\GeVc$) muons from Z $ \rightarrow \mu\mu$ decays using a two step method, where the muons are binned in charge, $\eta$ and $\phi$.
The first step requires the mean inverse transverse momenta of the muons reconstructed from data and simulation to be the same as the corresponding values from a perfectly aligned detector.
These corrections are tuned in the second step by using the $M_{\mu^{+1}\mu^{-1}}$ peak for a perfectly aligned detector to calibrate the corrections.
This removes any sensitivity to detector efficiencies or physics modelling.

The Rochester Corrections are applied to each muon an event weight that is a function of the muon's charge, \pt, $\eta$ and $phi$.

\subsection{Jet Energy Corrections}\label{subsec:jesjer}
As described in Chapter~\ref{subsubsec:JECs}, the JECs are applied to account for the non-uniform response in \pT and $eta$ of the detector by comparing the differences between the generator level and detector level responses.

In addition to these corrections, as the Jet Energy Resolution (JER) observed in data is approximately 10\% poorer than that in observed simulation, the 4-vectors of simulated jets are smeared as functions of generator level and reconstructed \pt and $\eta$ to account for this~\cite{Khachatryan:2016kdb}.

\subsection{b-tagging Efficiency}\label{subsec:btagEff}
The B-Tag and Vertexing (BTV) Physics Object Group measures the b-tagging efficiency and misidentification rates for b and light flavoured jets in data and MC simulation (multijet and \ttbar) of the algorithms which they support~\cite{Sirunyan:2017ezt}.
From these measurements b-tagging efficiency scale factors are produced and provided for analysts to apply to simulated events to correct differences observed between data and simulation.
These scale factors, as functions of the jet flavour, \pT and $\eta$, to alter the weight of the selected MC events.
This methodology was chosen as it involves only changing the weight of the selected MC events which, unlike other methods, avoids events migrating into different b-tag multiplicity bins and having events with potentially undefined variables such as the top mass.

\subsection{\PU Modelling}\label{subsec:puSF}
It is challenging to model variations in the number of \PU interactions that result from the changing LHC conditions.
Therefore MC events are reweighted as a function of the number of primary vertices so that the PU interactions simulated are resembles what is observed in data.

The \PU SF is determined as a function of the number of primary vertices, $n_{PV}$, present by comparing $n_{PV}$ in minimum bias data over the running period considered to $n_{PV}$ for simulated events.

\subsection{Top quark \pt}
A scale factor is applied to \ttbar MC as a function of the top's and anti-top's generator level transverse momenta to account for the \pt spectra of top quarks in data being significantly softer than that predicted by LO and NLO precision MC simulation~\cite{Khachatryan:2015oqa}.

\section{Signal Region Simulated Backgrounds}\label{sec:simBackgrounds}
The impact of the application of the full event selection and simulation corrections in the signal region on the simulated samples is shown in Table~\ref{tab:signalYields}.
It is clear that despite the high purity selection criteria used that sufficient 

The 
\begin{table}[!htbp]
\centering
\begin{tabular}{|l|c|c|c|c|c|}
\hline
Channel &  $ee$ & $\mu\mu$ & Combined \\
\hline
Signal (SM tZq) & 30.242$\pm$5.584 &  54.508$\pm$10.420 & 84.750$\pm$16.004     \\
Backgrounds: & & & \\
tWZ\@: & 6.439$\pm$0.964 & 10.779$\pm$1.709 & 17.218$\pm$2.673    \\
tHq: & 0.173$\pm$0.042 & 0.372$\pm$0.089 & 0.545$\pm$0.131    \\
ttW\@: & 7.249$\pm$1.593 & 10.681$\pm$1.667 & 17.930$\pm$3.260    \\
ttZ\@: & 61.882$\pm$0.619 & 110.471$\pm$16.499 & 172.353$\pm$17.118    \\
ttH\@: & 4.916$\pm$0.472 & 9.554$\pm$0.945 & 14.470$\pm$1.417    \\
\ttbar: & 1653.457$\pm$311.455 & 3219.360$\pm$580.378 & 482.817$\pm$891.833    \\
tW\@: & 95.989$\pm$12.814 & 177.527$\pm$32.238 & 273.516$\pm$45.052    \\
s-channel: & 0.000$\pm$0.000 & 0.000$\pm$0.000 & 0.000$\pm$0.000    \\
t-channel: & 0.612$\pm$0.103 & 0.995$\pm$0.643 & 1.607$\pm$0.746    \\
WW\@: & 1.339$\pm$0.692 & 2.447$\pm$0.535 & 3.786$\pm$1.227    \\
WZ\@: & 73.466$\pm$11.836 & 126.656$\pm$21.511 & 200.122$\pm$33.437    \\
ZZ\@: & 51.827$\pm$7.912 & 92.980$\pm$15.931 & 144.807$\pm$23.843    \\
WWW\@: & 0.114$\pm$0.084 & 0.305$\pm$0.102 & 0.419$\pm$0.186    \\
WWZ\@: & 1.327$\pm$0.147 & 2.207$\pm$0.220 & 3.534$\pm$0.367    \\
WZZ\@: & 1.540$\pm$0.260 & 2.470$\pm$0.416 & 4.010$\pm$0.676    \\
ZZZ\@: & 0.661$\pm$0.074 & 1.087$\pm$0.139 & 1.748$\pm$0.213    \\
W + jets: & 0.000$\pm$0.000 & 0.000$\pm$0.000 & 0.000$\pm$0.000    \\
Z + jets: & 3293.634$\pm$666.112 & 6245.353$\pm$1257.550 & 9538.987$\pm$1923.662    \\
\hline
NPLs: & 37.787$\pm$11.336 & 30.889$\pm$9.267 & 68.676$\pm$20.603   \\
\hline
Data & 5284$\pm$132.100 & 9665$\pm$241.625 & 14949$\pm$373.725    \\
Total MC & 5284.867$\pm$1020.79 & 10067.752$\pm$2637.863 & 15352.619$\pm$3658.653    \\
Total MC + NPLs & 5322.654$\pm$1032.126 & 10098.641$\pm$2647.130 & 15361.377$\pm$3679.256    \\
\hline
\end{tabular}
\caption{MC and data yields after full event selection for the separate channels and the channels combined.}\label{tab:signalYields}
\end{table}

\section{Data-driven Background Estimation}\label{sec:dataDrivenBackground}

\subsection{Non-Prompt Leptons}\label{sec:NPLs}
Leptons which are produced from events where at least one jet is incorrectly reconstructed as a lepton (predominately electrons) or a lepton from the decay of heavy quarks (predominately muons) are known as \emph{non-prompt leptons} (NPLs).
Given the difficulty in accurately modelling QCD processes and the very low statistics of such processes passing the lepton selection and isolation criteria, the NPL contribution is estimated with data.

The data-driven background estimation of background NPL background uses the same methodology as top quark pair production~\cite{CMS:2016syx} and same-sign SUSY searches~\cite{CMS:2015vqc}.

Initially one 
The vast majority of the same-sign event yields found are the result of non-prompt lepton and charge misidentified leptons, with some contribution from prompt leptons.
As these backgrounds are independent of the charge of the lepton pairs, it is expected that the nominal (opposite-sign) sample would have a similar contribution \cite{CMS:2015vqc}.

Therefore using a same-sign control region which only differs from the nominal signal region region by requiring that the pair of selected leptons have the same charge, a data driven 

The contribution of opposite-sign NPLs in data is estimated using Equation~\ref{eq:NPL},

\begin{equation}
 N_{data}^{OS non-prompt} = (N_{data}^{SS} - N^{SS}_{real + mis-ID}).\frac{N_{MC}^{OS non-prompt}}{N_{MC}^{SS non-prompt}}
\end{equation}\label{eq:NPL}

where $N_{data}^{SS}$ is the total number of same sign events observed in data, $N^{SS}_{real + mis-ID}$ is the expected number of real same-sign events and events with charge misidentification and $N_{MC}^{OS non-prompt}$ and $N_{MC}^{SS non-prompt}$ the number of opposite-sign and same-sign NPLs observed in MC used to appropriately scale the estimate.

The method requires that the same-sign control region established uses the same selection criteria as the nominal signal region, albeit with same-sign lepton pairs instead of opposite-sign ones.
This control region is dominated by NPL events, but also contains contributions from prompt lepton events, charge misidentification and real same-sign pairs.

The ratio of MC opposite-sign over same-sign events is referred to as R, and is calculated using generator level information from reconstructed objects which have matched to a generator level particle. 
R is calculated from the W + jets, \ttZ and \ttW leptonic decaying, and single top MC samples where there are sufficient statistics given that these processes are expected to be the predominant source of non-prompt leptons for this analysis. 

\begin{table}[!htbp]
\centering
\begin{tabular}{| l |  c |  c |  c |  c |  c |}
\hline
Source &  $ee$ & $\mu\mu$ & Combined \\ 
\hline
\ttbar (SS): & a$\pm$b &  c $\pm$d & e$\pm$f    \\
Z + jets (SS): & a$\pm$b &  c$\pm$d & e$\pm$r    \\
Single Top (SS): & a$\pm$b & c$\pm$d & e$\pm$r    \\
VV (SS): & a$\pm$b & c$\pm$d & e$\pm$f    \\
ttV (SS): & a$\pm$b &  c$\pm$d & e$\pm$f    \\ 
\hline
Total background (SS): & a$\pm$b & c$\pm$d & e$\pm$f   \\ 
Data: & a$\pm$b & c$\pm$d & e$\pm$f    \\ 
\hline
SS data (bkg): & a$\pm$b & c$\pm$d & e$\pm$f \\
\hline
Non-prompt (SS): & a$\pm$b & c$\pm$d & e$\pm$f \\
Non-prompt (OS): & a$\pm$b & c$\pm$d & e$\pm$f \\
R (OS/SS): & a$\pm$b & c$\pm$d & e$\pm$f \\
\hline
Non-prompt estimation: & a$\pm$b & c$\pm$d & e$\pm$f \\
\hline
\end{tabular}
\caption{Non-prompt lepton estimation following all selection cuts}
\label{tab:fakeLeptonYields}
\end{table}

Using the Equation

\subsection{Z+jets background}\label{subsec:zPlusJetsEstimation}

Madgraph - normalises well but poor jet multiplicity
aMC@NLO - bad normalisation, but good higher jet multiplicity description

\subsection{\ttbar background}\label{subsec:ttbarEstimation}
The \ttbar enriched control region defined in Chapter~\ref{subsec:ttbarCR} was designed to provide an orthogonal region which was topologically similar to the signal region in order to validate:
\begin{itemize}
\item whether or not the simulated \ttbar sample used accurately modelled the \ttbar process;
\item and if not, to be used to derive a data-driven estimate for \ttbar.
\end{itemize}

\begin{table}[htbp]
\topcaption {
The event yields after the selection criteria have been applied for the \ttbar control region.
}
\label{tab:ttbarCR}
  \centering
% This right-aligns numbers in column, but centers them under column title.
 \begin{tabular}{cc}
   \hline
   \textbf{MC process} & \textbf{$e\mu$}  \\
   \hline
   tZq & 1.709\\
   \ttbar & 11778.461 \\
   Z+jets & 80.9921\\
   tW & 488.632\\
   Other & 166.200\\
   \hline
   Data & 12509.0 \\
   Total MC & 12515.995 \\
   \hline
 \end{tabular}
\end{table}

\section{Multivariate Analysis Techniques}\label{sec:mvas}
Multivariate Analysis (MVA) techniques are commonly used to further discriminate between signal and background processes given the difficulty in identifying rare or background dominated processes through the sole use of individual cuts.

Therefore, given the small cross section and topology of the dilepton final state of tZq, a MVA method was used to enhance the separation between the signal and background following the application of the selection cuts described in Chapter~\ref{chapter:tzq-search}.
The \emph{Boosted Decision Tree}~(BDT) MVA technique was used as it a widely used and supported technique which those undertaking this analysis were familiar with.

\subsection{Boosted Decision Trees}\label{subsec:bdt}
A decision tree in its simplest form is a series of sequential binary decisions (nodes) on a single variable at a time in order classify an event as signal or background.

\emph{Boosting} extends the concept of a decision tree from a single tree to a forest of trees with the aim of both stabilising the decision trees' response and enhancing their performance.
This involves training multiple trees in succession on the same training sample which has been reweighted based on the past trees' performance.
At the end of the process all the trees are combined into a single classifier which is given by the weighted average of the trees, thus creating a strong learner out of an ensemble of weak learners.

\emph{Bagging} is a similar concept to boosting, but involves each tree being trained on a random subset of the training sample, where every element has an equal probability of being sampled, rather than the whole training sample.

Following the evaluation of a number of Boosting and Bagging algorithms for decision trees, it was determined that the Gradient Boost algorithm as implemented in the ``XGBoost Library'' provided the optimal performance for the search presented~\cite{xgboost}.

\emph{Gradient Boosting} models the shortcomings of the model 


The 


The \emph{Adaptive Boost} (AdaBoost) and \emph{Gradient Boosting} algorithms are two of the most popular boosting algorithms used.

Adaboost

This exponential loss approach however, is its degraded performance in noisier environments as it is not robust in 	

The number of nodes or \emph{depth} of the tree considered
As each node's criteria are dependent on those which have preceded it, the decision tree is capable of separating signal 

\emph{Overtraining}
A training sample, consisting of a subset of the data to be classified, is used as the input 





BDT features and hyperparameters chosen separately for ee and mumu channels
Features chosen using recursive feature elimination
Hyperparameters are selected by using a Gaussian process to optimise the classifier’s performance
hyperparameters = learning rate, n-estimators, max tree depth
feature = input variable

\subsection{BDT input variables}
Using the selected reconstructed physics objects, a large number of

\begin{table}[htbp]
\topcaption { The name and descriptions of the variables chosen by recursive feature elimination to be used as input to the BDT to discriminate between potential tZq signal events and the dominant.
}
\label{tab:bdtVariables}
  \centering
% This increases column spacing.
\resizebox{\textwidth}{!}{
% This right-aligns numbers in column, but centers them under column title.
\begin{tabular}{cccc}
   \hline
   \textbf{Variable} & \textbf{Description} & \textbf{$ee$} & \textbf{$\mu\mu$} \\
   \hline
    bTagDisc & b-tag discriminator of the leading b-tagged jet & $\checkmark$ & $\checkmark$ \\
    fourthJetPt & \pt of the fourth jet & $\checkmark$ & $\checkmark$ \\
    jetHt & Total \HT of every jet & $X$ & $\checkmark$ \\
    jetMass & Total mass of every jet & $\checkmark$ & $\checkmark$ \\
    jjDelR & $\Delta R$ between the leading jets & $\checkmark$ & $\checkmark$ \\
    leadJetEta & $\eta$ of the leading jet & $\checkmark$ & $\checkmark$ \\
    leadJetPt & \pt of the leading jet & $\checkmark$ & $\checkmark$ \\
    met & \met & $\checkmark$ & $\checkmark$ \\
    secJetPt & \pt of the second jet & $\checkmark$ & $\checkmark$ \\
    thirdJetPt & \pt of the third jet & $\checkmark$ & $\checkmark$ \\
    topMass & $m_{top}$ & $\checkmark$ & $\checkmark$ \\
    totHtOverPt & Total \HT divided by total \pt & $\checkmark$ & $\checkmark$ \\
    wPairMass & $m_{W}$ & $\checkmark$ & $\checkmark$ \\
    wQuark2Eta & $\eta$ of the second W boson candidate jet & $X$ & $\checkmark$ \\
    wwdelR & $\Delta R$ between the W boson candidate jets & $\checkmark$ & $X$ \\
    zEta & $\eta$ of the Z boson & $X$$ & $\checkmark$ \\
    zHt & \HT of the Z boson & $\checkmark$ & $\checkmark$ \\
    zMass & $m_{Z}$ & $\checkmark$ & $\checkmark$ \\
    zTopDelR & $\Delta R$ between the Z boson and top quark & $X$ & $\checkmark$ \\
    zjminR & Minimum $\Delta R$ between the Z boson and a jet & $\checkmark$ & $\checkmark$ \\
    zlb1DelR & $\Delta R$ between the Z boson and leading b-tagged jet & $\checkmark$ & $X$ \\
   \hline
 \end{tabular}}
\end{table}

\editComment{LOTS of PLOTS of the input variable distributions}


\subsection{BDT Training and Output}
Each sample of events for each process considered is split into a training and testing sample.


\section{Systematic Uncertainties}\label{chapter:systematics}
%%% Intro

\editComment{Plan to move to results chapter}

For any meaningful and robust measurement to be made in any physics analysis, it is vital that the sources of systematic uncertainties associated with it are both understood and controlled.
This is particularly important when searching for tZq dilepton final state given that the high statistics of the background processes result in their systematic uncertainties being of a comparable order to the statistical uncertainties of the signal process.
Therefore without additional data to reduce the statistical uncertainty, the sensitivity of the search will be limited by the systematic uncertainties.

%%% Sources
These sources of uncertainty either originate from experimental or theoretical uncertainties and typically influence the result in one of two ways:
\begin{itemize}
\item \textbf{Rate or normalisation uncertainties} impact the number of events present and thus influence the  normalisation of the distributions considered.
\item \textbf{Shape or scale factor uncertainties} impact the shape of the distributions as they involve the scaling of individual events as a function of their kinematics in order to correct inconsistencies between simulation and data.
\end{itemize}

These uncertainties, as well as the statistical uncertainties arising from the size of the simulated samples available, are treated as nuisance parameters in the statistical fit model which is discussed, along with their impact, in Chapter~\ref{sec:statisticalModel}.

\subsection{Experimental Uncertainties}
\subsubsection{Jet Energy Corrections}
The Jet Energy Corrections group also provides the uncertainties associated with the JES and JER they determine, discussed in Chapters~\ref{subsubsec:JECs} and~\ref{subsec:jesjer}, are determined by the Jet Energy Corrections group~\cite{Khachatryan:2016kdb}. 

The impact that the JES has on the jet kinematics is evaluated by varying the corrective JES up and down by a standard deviation.
The uncertainty associated with the JER smearing is accounted for by varying the smearing factor up and down by the associated statistical uncertainty.
Recently the uncertainties associated with the JER have been updated during the reprocessing of the 2016 dataset to include the systematic uncertainties in addition to the statistical uncertainties.
At the time of writing this thesis, these reprocessed samples and the impact of the revised total JER uncertainties has not been propagated through the analysis.

\subsubsection{\MET Uncertainties}
As \MET is calculated from the sum of the \pT of all the PF objects and the remaining unclustered energy deposits, the uncertainties associated from both have to be considered.

The impact of the uncertainties associated with both the JES and JER on the PF \MET are accounted for by propagating the JEC uncertainties through to the \MET and evaluating the impact they have.
As the unclustered energy remains uncorrected, the impact on the \MET uncertainty is evaluated by varying the difference between the \MET and total \pT of the PF objects up and down by 10\%, the default energy uncertainty.

During the time of writing this thesis, the method of determining the uncertainty associated with the unclustered energy was in the process of being replaced with a more precise method, where each particle is varied by its resolution.
Given the small impact of the \MET uncertainties, as discussed in Chapter~\ref{sec:uncertainitiesImpact}, it is anticipated that change will have a minimal impact on the final result.

\subsubsection{Pileup Reweighting}
The uncertainty associated with the primary vertex distributions used in the \PU reweighting is determined by varying the expected minimum bias cross section used in simulation $\pm X%$ in order to ascertain the impact of greater or lesser amounts of \PU on the analysis.

\subsubsection{Parton Density Functions}\label{subsec:pdfSysts}
%%Discussion of what PDFs are, is given in an earlier chapter 
The impact of the PDF uncertainties are evaluated according to the PDF4LHC recommendations~\cite{Butterworth:2015oua}, where they are estimated as the standard deviation of the weights of the nominal and the variations of the PDF set.

For almost all of the MC samples considered, this is achieved by considering the nonimal event weight and one hundred alternative PDF weights which are stored as per-event weights in the LHE 
event header for almost all of the MC samples considered.

The single top tW-channel samples are the exception to this as at the time of their generation it was not possible to generate per-event weights to account for the PDF variations for this process.
Therefore, the LHAPDF (Les Houches Accord Parton Distribution Function) library is used to access both the nominal PDF weight and 50 eigenvalues from the NNPDF3.0 set to provide one hundred alternative event weights to be evaluated.

\subsubsection{b-tagging Uncertainties}
The uncertainties associated with the b-tagging scale factors described in Chapter~\ref{subsec:btagEff} are obtained by varying their value by $\pm 1\sigma$, as calculated by the BTV POG.

\subsubsection{Non-prompt Lepton Contributions}
Tis data-driven estimate of the instrumental backgrounds should have no dependence on either the lepton flavour or selection cuts.
Therefore the variation of the ratio of opposite-sign over same-sign events as a function of the lepton flavour and the cut level was considered to be well accounted for by a 30\% rate uncertainty.

\subsubsubsection{Luminosity Uncertainties}
CMS uses the pixel detector, DTs, HF, the Fast Beam Conditions Monitor and Pixel Luminosity Telescope to monitor and measure the instantaneous and integrated luminosity.
During Run 2, the primary offline luminosity measurements made by the CMS Luminosity Group used the pixel detector using the Pixel Cluster Counting (PCC) method due its stability over time for up an average \PU of 150 and the high precision results obtained with it during Run I.
The PCC algorithm is able to achieve such a precision by measuring the instantaneous luminosity through the number of pixels present. 
This is possible as the probability of pixel hit belonging to multiple tracks is very small due to the very low occupancy of the detector, inferring that the number of pixel hits are linearly proportional to the number of interactions during a bunch crossing~\cite{CMS:2017_lumi}.

Using Van der Meer (VdM) scans during dedicated LHC runs to calibrate the absolute luminosity scale calibrations of the detectors~\cite{vanderMeer:1968zz}.
The overall uncertainty in the integrated luminosity collected by CMS in 2016 was estimated to be 2.5\%~\cite{CMS:2017_lumi}.
Given that each MC sample is appropriately normalised with respect to the 

%The MC events produced are weighted by a scale factor in order to correctly normalise them with respect to the data they are compared against.
%This normalisation scale factor is given by:
%\begin{equation}
%SF_{dataset} = \frac{\pazocal{L} \sigma}{N_{MC}^{Events}}
%\end{equation}
%where $\pazocal{L}$ is the amount of total integrated luminosity considered in the data used, $\sigma$ the cross section of the MC sample considered and $N_{MC}^{Events}$ is number of simulated events considered for the process.

\subsubsection{Lepton Efficiencies}
The uncertainties associated with the lepton identification, isolation and reconstruction efficiency scale factors discussed in Chapter~\ref{subsec:leptonRecoSFs} are varied +/- 1 sigma.

%The uncertainty associated with the lepton trigger efficiencies are calculated using 
%The conservative method of Clopper-Pearson intervals~\cite{Cousins:2009kz} was used to determine the uncertainities associated with the trigger scale factors used.

%The potential correlations between the cross triggers and lepton triggers used are one source of systematic uncertainty.
%If both triggers are independent, then the efficiency of fulfilling both trigger selections can be expressed as:
%\begin{equation}
%\epsilon_{X + lepton triggers} = \epsilon_{X triggers} \times \epsilon_{lepton triggers}
%\label{eq:triggerCorrelation}
%\end{equation}
%
%and if both trigger selections are uncorrelated, then the ratio of the left and right hand sides ($\alpha$) of Equation~\ref{eq:triggerCorrelation} would be 1.
%Table~\ref{tab:triggerAlpha} shows that the values of $\alpha$ determined for each channel only differ slightly from 1.
%\begin{table}[htbp]
%\topcaption {
%The values of $\alpha$, expressing the strength of correlation between the lepton and cross triggers used to determine the trigger scale factors, for each channel.
%}
%\label{tab:triggerCorrelation}
%  \centering
%  \resizebox{\textwidth}{!}{
%% This right-aligns numbers in column, but centers them under column title.
% \begin{tabular}{cc}
%   \hline
%   \textbf{Channel} & \textbf{$\alpha$}   \\
%   \hline   
%   ee & 1.0 \\
%   $\mumu$ & 1.0  \\
%   e$\mu$ & 1.0  \\
%   \hline
% \end{tabular}}
%\end{table}
%
%\editComment{Update values of alpha}

\subsection{Theoretical Uncertainties}\label{sec:theorySysts}

\subsubsection{Factorisation and renormalisation scales}
The factorisation and renormalisation scales ($\mu_{f}$,$\mu_{s}$) used at the Matrix Element and Parton Shower levels are parametrised as functions of $Q^{2}$.
In order to consider the impact of the uncertainty associated with the choice of scales used, $Q^{2}$ is varied up and down by factors of 2 and 0.5 respectively.

For the majority of the MC samples considered, the variations in $\mu_{f}$ and $\mu_{s}$ are stored in the LHE event header as per-event weights.
These weights are produced for where one scale is fixed as the other is varied or both are varied simultaneously.
The event weights for the simultaneously varied scales were used to reweighting each event in order to evaluate the impact of the $\mu_{f}$ and $\mu_{s}$ uncertainties.

In contrast to the ME level, the impact of the PS shower scale uncertainties was evaluated through the use of dedicated samples where the PS scale had been varied up and down.
These centrally produced samples are listed in Table~\ref{tab:theorySampleList} as the ``scale up'' and ``scale down'' samples.
In the case of \ttbar however, these samples are listed as ISR (initial-state radiation) and FSR (final-state radiation), as it includes the variations in the gluon emissions of the incoming and outgoing partons.

As mentioned above in Chapter~\ref{subsec:pdfSysts}, it was not possible for the single top tW-channel MC samples to be produced with per-event weights to account for the matrix element factorisation and renormalisation scales.
Dedicated samples for this process, listed in Table~\ref{tab:theorySampleList}, where the matrix element and parton shower scales are varied are used to evaluate these systematic uncertainties.

\subsubsection{Parton Shower Matching Thresholds}
As discussed in Chapter~\ref{subsec:eventGenerators}, all of the MC samples considered use model the hard scattering process through a dedicated Matrix Element generator, with PYTHIA 8 being used to perform the subsequent PS and hadronisation.

The uncertainty associated with the choice of the matching threshold used is evaluated by using the dedicated matching samples.
Such samples have been generated for the \ttbar and single top t-channel backgrounds, listed in Table~\ref{tab:theorySampleList}, where the model's matching threshold parameter \emph{hdamp} is varied up and down by one standard deviation~\cite{CMS:2016kle}.
