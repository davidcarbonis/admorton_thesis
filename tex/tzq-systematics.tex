\chapter{Background Estimation}\label{chapter:bkg}
Despite 

\section{Data-driven Background Esimation}\label{sec:dataDrivenBackground}
\subsection{Non-Prompt Leptons}\label{sec:NPLs}
Backgrounds which involve decays into lepton + jets and where at least one jet is incorrectly reconstructed as a lepton (predominately electrons) or a lepton from the decay of heavy quarks (predominately muons), which pass the lepton selection and isolation criteria, are estimated with data.

The estimation of this background uses the same methodology as when performing top quark pair production~\cite{CMS:2016syx} and same-sign SUSY searches~\cite{CMS:2015vqc}.
The vast majority of the same-sign event yields found are the result of non-lepton and charge misidentified leptons, with some contribution from prompt leptons.
As these backgrounds are independent of the charge of the lepton pairs, it is expected that the nominal (opposite-sign) sample would have a similar contribution \cite{CMS:2015vqc}.

To estimate this contribution of opposite-sign non-prompt leptons in data, the same-sign event yields with the expected prompt-lepton contribution subtracted, is multiplied by a ratio of opposite-sign over same-sign non-prompt lepton events taken from MC.

The method requires that the same-sign control region established uses the same selection criteria as the nominal signal region, albeit with same-sign lepton pairs instead of opposite-sign ones.
This control region is dominated by non-lepton lepton events, but also contains contributions from prompt lepton events, charge misidentification and real same-sign pairs.

This data driven estimate is obtained using the following equation:

\begin{equation}
 N_{data}^{OS non-prompt} = (N_{data}^{SS} - N^{SS}_{real + mis-ID}).\frac{N_{MC}^{OS non-prompt}}{N_{MC}^{SS non-prompt}}
\end{equation}

where $N_{data}^{SS}$ is the total number of same sign events observed in data, $N^{SS}_{real + mis-ID}$ is the expected number of real same-sign events and events with charge misidentification and $N_{MC}^{OS non-prompt}$ and $N_{MC}^{SS non-prompt}$ the number of opposite-sign and same-sign non-prompt leptons observed in MC used to appropriately scale the estimate.

This ratio of MC opposite-sign over same-sign events is referred to as R, and is calculated using generator level information from reconstructed objects which have matched to a generator level particle. R is calculated from the W + jets, \ttZ and \ttW leptonic decaying, and single top MC samples with sufficient statistics given that these processes are expected to be the predominant source of non-prompt leptons for this analysis. 

\begin{table}[!htbp]
\centering
\begin{tabular}{| l |  c |  c |  c |  c |  c |}
\hline
Source &  $ee$ & $\mu\mu$ & Combined \\ 
\hline
\ttbar (SS): & a$\pm$b &  c $\pm$d & e$\pm$f    \\
Z + jets (SS): & a$\pm$b &  c$\pm$d & e$\pm$r    \\
Single Top (SS): & a$\pm$b & c$\pm$d & e$\pm$r    \\
VV (SS): & a$\pm$b & c$\pm$d & e$\pm$f    \\
ttV (SS): & a$\pm$b &  c$\pm$d & e$\pm$f    \\ 
\hline
Total background (SS): & a$\pm$b & c$\pm$d & e$\pm$f   \\ 
Data: & a$\pm$b & c$\pm$d & e$\pm$f    \\ 
\hline
SS data (bkg): & a$\pm$b & c$\pm$d & e$\pm$f \\
\hline
Non-prompt (SS): & a$\pm$b & c$\pm$d & e$\pm$f \\
Non-prompt (OS): & a$\pm$b & c$\pm$d & e$\pm$f \\
R (OS/SS): & a$\pm$b & c$\pm$d & e$\pm$f \\
\hline
Non-prompt estimation: & a$\pm$b & c$\pm$d & e$\pm$f \\
\hline
\end{tabular}
\caption{Non-prompt lepton estimation following all selection cuts}
\label{tab:fakeLeptonYields}
\end{table}

\subsection{Z+jets background}\label{subsec:zPlusJetsEstimation}
Madgraph - normalises well but poor jet multiplicity
aMC@NLO - bad normalisation, but good higher jet multiplicity description

\section{Multivariate Analysis Techniques}\label{sec:mvas}
Single top 
\subsection{Boosted Decision Trees}\label{subsec:bdt}

BDT implemented in XGBoost Library
BDT features and hyperparameters chosen separately for ee and mumu channels
Features chosen using recursive feature elimination
Hyperparameters are selected by using a Gaussian process to optimise the classifier’s performance
hyperparameters = learning rate, n-estimators, max tree depth
feature = input variable

\subsubsection{BDT input variables}
Variables chosen

\begin{table}[htbp]
\topcaption { The name and descriptions of the variables chosen by recursive feature elimination to be used as input to the BDT to discriminate between potential tZq signal events and the dominant.
}
\label{tab:bdtVariables}
  \centering
% This increases column spacing.
\resizebox{\textwidth}{!}{
% This right-aligns numbers in column, but centers them under column title.
\begin{tabular}{cccc}
   \hline
   \textbf{Variable} & \textbf{Description} & \textbf{$ee$} & \textbf{$\mu\mu$} \\
   \hline
    bTagDisc & b-tag discriminator of the leading b-tagged jet & $\checkmark$ & $\checkmark$ \\
    fourthJetPt & \pt of the fourth jet & $\checkmark$ & $\checkmark$ \\
    jetHt & Total \HT of every jet & $X$ & $\checkmark$ \\
    jetMass & Total mass of every jet & $\checkmark$ & $\checkmark$ \\
    jjDelR & $\Delta R$ between the leading jets & $\checkmark$ & $\checkmark$ \\
    leadJetEta & $\eta$ of the leading jet & $\checkmark$ & $\checkmark$ \\
    leadJetPt & \pt of the leading jet & $\checkmark$ & $\checkmark$ \\
    met & \met & $\checkmark$ & $\checkmark$ \\
    secJetPt & \pt of the second jet & $\checkmark$ & $\checkmark$ \\
    thirdJetPt & \pt of the third jet & $\checkmark$ & $\checkmark$ \\
    topMass & $m_{top}$ & $\checkmark$ & $\checkmark$ \\
    totHtOverPt & Total \HT divided by total \pt & $\checkmark$ & $\checkmark$ \\
    wPairMass & $m_{W}$ & $\checkmark$ & $\checkmark$ \\
    wQuark2Eta & $\eta$ of the second W boson candidate jet & $X$ & $\checkmark$ \\
    wwdelR & $\Delta R$ between the W boson candidate jets & $\checkmark$ & $X$ \\
    zEta & $\eta$ of the Z boson & $X$$ & $\checkmark$ \\
    zHt & \HT of the Z boson & $\checkmark$ & $\checkmark$ \\
    zMass & $m_{Z}$ & $\checkmark$ & $\checkmark$ \\
    zTopDelR & $\Delta R$ between the Z boson and top quark & $X$ & $\checkmark$ \\
    zjminR & Minimum $\Delta R$ between the Z boson and a jet & $\checkmark$ & $\checkmark$ \\
    zlb1DelR & $\Delta R$ between the Z boson and leading b-tagged jet & $\checkmark$ & $X$ \\
   \hline
 \end{tabular}}
\end{table}

\editComment{LOTS of PLOTS of the input variable distributions}

\subsection{BDT Training and Output}
Each sample of events for each process considered is split into a training and testing sample.
\chapter{Systematic Uncertainties}\label{chapter:systematics}
%%% Intro
Understanding and minimising the impact of uncertainties in the resolution and efficiency of the detector and in the modelling of the simulation used to predict the signal and background processes is essential in order to make meaningful measurements.

%%% Sources
There are two 

These uncertainties, as well as the statistical uncertainties arising from the size of the simulated samples available, have been treated as nuisance parameters in the statistical fit model, as discussed in~\ref{chapter:results}.

\section{Experimental Uncertainties}
\subsection{Jet Energy Corrections}
As it has been observed that there are differences between simulated and jet energies , 
corrections are
https://arxiv.org/pdf/1607.03663.pdf
JINST 12 (2017) P02014

Smearing jets in MC simulation 

\subsection{Pileup Reweighting}
The \PU interactions included in the simulated samples used do not describe the number of primary interactions observed in data well, these interactions are estimated 
As such, they are reweighted to 

\subsection{b-tagging Uncertainties}
The uncertainties of the scale factors described in Chapter~\ref{subsec:btagEff} are obtained by varying their value by $\pm 1\sigma$, as calculated by the BTV POG. 
\editComment{Impact?}

\subsection{Parton Distribution Functions}
%%% ME_PS
\subsection{Non-prompt Lepton Contributions}
As this data-driven estimate of the instrumental backgrounds should have no dependence on either the lepton flavour or selection cuts, the variation of the ratio of opposite-sign over same-sign events as a function of the lepton flavour and the cut level should be well accounted for by a 30\% systematic uncertainty.

\subsection{Luminosity Uncertainties}
CMS uses five detectors, the pixel detector, DTs, HF, the Fast Beam Conditions Monitor and Pixel Luminosity Telescope to monitor and measure the instantaneous and integrated luminosity, with absolute calibrations of the detectors made through conducting Van der Meer (VdM) scans during dedicated LHC runs.
The luminosity value and its associated uncertainty used was determined by the CMS Luminosity Group using 
Pixel Cluster Counting (PCC) 

The overall uncertainty \cite{CMS:2017_lumi} %2016 lumi
\subsection{Lepton Efficiencies}
In order to account for the difference in performance for lepton trigger, identification, isolation and reconstruction efficiencies 

The lepton trigger efficiencies for both channels are measured independently, 

\section{Theoretical Uncertainties}

\subsection{Parton Density Functions}
%%Discussion of what PDFs are, is given in an earlier chapter
In order to calculate the uncertainties on the PDFs used in the generation of the MC samples considered, the 

A number of MC samples however, do not store the PDF weights as part of the event's 
\subsection{Factorisation and renormalisation scales}
The factorisation and renormalisation scales at the Matrix Element 
\subsection{Parton Shower Uncertainties}
\section{Impact of the Uncertainties}
The effect of each of the systematics considered on the event rate, in percentage, are shown in Table~\ref{tab:systImpact}.
These rates, whilst providing a useful insight into which of the systematics are the most important, do not show how the shape of each fitted variable and the MVA discriminant is influenced by each uncertainty.
\editComment{Make some comment on most important/impactful systematics and how better understanding them would improve the result}

\begin{table}[!htbp]
\begin{center}
\linespread{2}
\resizebox{\textwidth}{!}{\begin{tabular}{|l|c|c|c|c|}
\hline
Systematic      &  tZq                  & DY                   & \ttbar{}                  & Other         \\
($ee$ / $\mu\mu$) & (\%)  & (\%)  & (\%)  & (\%)  \\
\hline
Trigger             &  $_{-4.23\%}^{+4.24\%}$ /  $_{-0.21\%}^{+6.07\%}$   & $_{-4.72\%}^{+4.07\%}$ / $_{-0.32\%}^{+6.37\%}$  & $_{-5.08\%}^{+4.41\%}$ / $_{-0.55\%}^{+5.54\%}$ & $_{-4.72\%}^{+4.85\%}$ / $_{-4.47\%}^{+5.97\%}$  \\
JER             &  $_{-5.27\%}^{+6.02\%}$ /  $_{-6.11\%}^{+5.39\%}$   & $_{-11.81\%}^{+16.54\%}$ / $_{-14.18\%}^{+16.71\%}$  & $_{-7.98\%}^{+7.84\%}$ / $_{-6.13\%}^{+8.24\%}$  & $_{--1.96\%}^{+2.11\%}$ / $_{-1.62\%}^{+1.82\%}$  \\
JES             &  $_{-0.04\%}^{+0.19\%}$ /  $_{-0.13\%}^{+0.13\%}$   & $_{-0.55\%}^{+0.29\%}$ / $_{-0.17\%}^{+0.13\%}$  & $_{-1.30\%}^{+0.02\%}$ / $_{-0.20\%}^{+0.20\%}$  & $_{-0.0.01\%}^{+0.11\%}$ / $_{-0.14\%}^{+0.18\%}$  \\
Pileup             &  $_{-0.42\%}^{+0.43\%}$ /  $_{-0.17\%}^{+0.43\%}$   & $_{-2.35\%}^{+2.26\%}$ / $_{-2.57\%}^{+1.75\%}$  & $_{-1.52\%}^{+0.52\%}$ / $_{-0.09\%}^{+1.35\%}$  & $_{-0.86\%}^{+0.38\%}$ / $_{-0.15\%}^{+0.26\%}$  \\
bTag             &  $_{-2.78\%}^{+3.38\%}$ /  $_{-3.38\%}^{+2.99\%}$   & $_{-5.30\%}^{+5.11\%}$ / $_{-5.02\%}^{+5.12\%}$  & $_{-2.89\%}^{+3.02\%}$ / $_{-3.12\%}^{+3.77\%}$  & $_{-3.43\%}^{+3.25\%}$ / $_{-3.24\%}^{+3.00\%}$  \\    
PDF             &  $_{-9.98\%}^{+13.22\%}$ /  $_{-9.24\%}^{+11.94\%}$   & $_{-1.56\%}^{+1.73\%}$ / $_{-2.95\%}^{+2.16\%}$  & $_{-2.99\%}^{+1.85\%}$ / $_{-2.95\%}^{+2.16\%}$  & $_{-8.56\%}^{+9.95\%}$ / $_{-8.51\%}^{+9.40\%}$  \\
$Q^{2}$Scaling             &  $_{-2.82\%}^{+1.36\%}$ /  $_{-3.06\%}^{+1.33\%}$   & $_{-15.00\%}^{+2.92\%}$ / $_{-14.64\%}^{+2.05\%}$  & $_{-11.38\%}^{-1.38\%}$ / $_{-11.40\%}^{+0.0\%}$  & $_{-5.01\%}^{+1.37\%}$ / $_{-5.07\%}^{+1.8\%}$  \\
\hline
\end{tabular}
}
\caption{Rate impact of systematics on MC templates}\label{tab:systImpact}
\end{center}
\end{table}
