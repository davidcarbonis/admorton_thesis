\chapter{Introduction}\label{chapter:intro}

\emph{``If I have seen further it is by standing on the shoulders of giants''}
\emph{Letter to Robert Hooke FRS, February 15th 1676, by Sir Isaac Newton FRS (1643-1727)}

The idea that nature can be explained through rational explanations, such as the ancient philosophical concepts of \emph{Atomism} and the Ancient Greek's \emph{Classical Elements}, is one that stretches back into time immemorial.

Following the scientific revolution of the 17th century the scientific method replaced such philosophical reasoning as the basis for exploring the nature of reality.
By formulating hypotheses whose predictions can tested by empirical evidence, successive generations of scientists
have built upon and improved on the ideas of those before them.
By amending existing theories or proposing new theories supported by new and more precise measurements, unified descriptions of seemingly unrelated phenomena have emerged, such as James Clark Maxwell's theory of electromagnetism.
This process has taken us from John Dalton's atomic theory and Sir Isaac Newton's laws of motion to the Standard Model (SM) of Particle Physics in the present day, describing all known elementary particles and three of the four fundamental forces of nature.


The SM is one of the greatest and most powerful scientific theories, making remarkably accurate predictions that have withstood incredible experimental scrutiny.
Despite the completion of the SM with the discovery of the Higgs Boson in 2012~\cite{HiggsCMS,HiggsATLAS} at the Large Hadron Collider (LHC), it is clear that the Standard Model cannot be a complete description of reality for a number of reasons; including the following:
\begin{itemize}
\item Gravity is not accounted for within the SM and at high energy densities it is fundamentally irreconcilable with the classical theory of General Relativity~\cite{}.
\item There is strong experimental evidence that the observed rotation curves of galaxies and gravitational lensing cannot be accounted for by SM particles alone and that there must be a large weakly interacting \emph{Dark Matter} component of the Universe~\cite{Bertone:2004pz}.
\item The presence of so-called \emph{Dark Energy} has also been inferred from astronomical and cosmological observations to account for the observed rate of expansion of the Universe~\cite{Peebles:2002gy}.
\item Neutrinos have been observed to oscillate between different flavours, implying that they have non-zero masses in contrast to SM expectations~\cite{Fukuda:1998mi,Ahmad:2001an}.
\item There is currently no explanation that accounts for the clear abundance of matter over anti-matter in the observable universe.
\end{itemize}

In addition to these, many scientists are uncomfortable with the fact that the SM contains a large number of finely tuned experimentally derived parameters, preferring a theory from which these values would emerge naturally, resulting in a more ``complete'' description of reality~\cite{Burdman:2007ck}.

One of the approaches study these issues is to investigate increasingly higher energy scales to both make precise measurements and rare tests of our existing theories and to look for new physics beyond them.
The LHC at CERN is the most powerful and luminous particle accelerator and collider built to date and provides physicists the capability to study an unprecedented number of events involving the heaviest known fundamental particle, the top quark.
Many of the top quark's properties, stemming from a mass near the electroweak symmetry breaking scale and a lifetime shorter than the strong force’s characteristic time, have no equivalent for the other five quarks.
As such, study of the top quark allows for unique opportunities to probe the weak force and the nature of the individual quarks.

This thesis presents a search for a predicted but undiscovered singly produced top quark process and a number of the contributions towards a study considering a potential future particle detector upgrade.

The analysis presented looks for, and makes a cross section measurement of, a single top quark which is produced in association with a Z boson in the final state involving two leptons using proton-proton collision data at $\sqrt{13}$ collected by the Compact Muon Solenoid (CMS) at the LHC during 2016.
This process has been predicted by the Standard Model but has yet to be measured given both its rarity and similarity to more commonly produced background processes.
As the process involves the Z boson coupling between both the top quark and W boson, it is a particularly sensitive probe for any new physics in the electroweak sector.

The CMS detector at the High Luminosity LHC will require a track finding system to provide information to the trigger system in order to discriminate in favour of potentially interesting physics against increasingly large backgrounds.
During the development of one of the proposed track finding systems, studies were undertaken regarding various track fitting algorithms which would fit precise track helix parameters to the tracks found and the ability of the system to find tracks with $\pT > 3\GeVc$.
In this thesis the studies concerning the development of a Linearised $\chi^{2}$ fitter and the ability of the proposed system to find tracks with $\pT > 2GeVc$ are presented.


\editComment{add references for SM inconsisitencies}