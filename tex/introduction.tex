\chapter{Introduction}\label{chapter:intro}

\emph{``If I have seen further it is by standing on the shoulders of giants''}
\emph{Letter to Robert Hooke FRS, February 15th 1676, by Sir Isaac Newton FRS (1643-1727)}

The idea that nature can be explained through rational explanations, such as the ancient philosophical concepts of \emph{Atomism}and the Ancient Greek's \emph{Classical Elements}, is one that stretches back into time immemorial.

Following scientific revolution of the 17th century the scientific method replaced such philosophical reasoning as the basis of exploring the nature of reality.
By formulating hypotheses whose predictions can tested by empirical evidence, successive generations of scientists
have built upon and improved on the ideas of those before them.
By amending existing theories or proposing new theories supported by new and more precise measurements, unified descriptions of seemingly unrelated phenomena have emerged, such as James Clark Maxwell's theory of electromagnetism.

This process has taken us from John Dalton's atomic theory and Sir Isaac Newton's Laws of Motion to the present day with the Standard Model of Particle Physics which describes all known elementary particles and three of the four fundamental forces.
The Standard Model has been an incredibly successful theory, making predictions which have not only withstood incredible experimental scrutiny with very accurate results, but also have been found to have very few inconsistencies with reality.

Such inconsistencies and other questions that the Standard Model does not address, such as the observed oscillation of neturino flavours and gravity, have led to the construction of increasingly powerful and luminous particle accelerators.
These accelerators produce are used to create the high energy environment required in order to provide the conditions for particle detectors to make measurements of interesting rare and short-lived physical processes against vast backgrounds of uninteresting events.
These events not only allow for ever more precise measurements and rarer tests of the Standard Model, but also to search for new physics beyond it.

The Large Hadron Collider at CERN is currently the largest and most powerful particle accelerator and collider that has been built to date.
It is not only capable of accessing physics at higher energy scales than before, but producing an unprecedented amount of luminosity, enabling the 

This thesis presents both a search for a singly produced top quark process that is predicted by the Standard Model a number of the the contributions towards a study considering a potential future particle detector upgrade.


 that will require the ability to discriminate in favour of potentially interesting physics against increasingly large backgrounds is discussed.