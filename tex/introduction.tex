\chapter{Introduction}\label{chapter:intro}

\emph{``If I have seen further it is by standing on the shoulders of giants''}
\emph{Letter to Robert Hooke FRS, February 15th 1676, by Sir Isaac Newton FRS (1643-1727)}

The idea that nature can be explained through rational explanations, such as the ancient philosophical concepts of \emph{Atomism}and the Ancient Greek's \emph{Classical Elements}, is one that stretches back into time immemorial.

Following scientific revolution of the 17th century the scientific method replaced such philosophical reasoning as the basis of exploring the nature of reality.
By formulating hypotheses whose predictions can tested by empirical evidence, successive generations of scientists
have built upon and improved on the ideas of those before them.
By amending existing theories or proposing new theories supported by new and more precise measurements, unified descriptions of seemingly unrelated phenomena have emerged, such as James Clark Maxwell's theory of electromagnetism.
This process has taken us from John Dalton's atomic theory and Sir Isaac Newton's Laws of Motion to the Standard Model of Particle Physics in the present day, describing all known elementary particles and three of the four fundamental forces of nature.


The Standard Model has been one of the greatest and most powerful scientific theories, making predictions which have withstood incredible experimental scrutiny and been found to very accurate with very few inconsistencies with reality.
Despite these minor deviations however, there are many ... that the Standard Model cannot be a complete description of reality.
Perhaps the most glaring omission from the theory is gravity, with 

Despite the successes of General Relativity, it is irreconcilable with the Standard Model at the very high 
The strong evidence for a \emph{Dark Matter} component of the Universe to explain the 
very high energy densities where 
-Despite the successes of the Standard Model and General Relativity, they are irreconcilable at very high energy scales...
- observed oscillation of neturino flavours
- dark matter
- matter anti-matter symmetry


Increasingly powerful and luminous particle accelerators have been constructed to create the high energy environment required to probe the 

in order to provide the conditions for particle detectors to make measurements of interesting rare and short-lived physical processes against vast backgrounds of uninteresting events.

The Large Hadron Collider at CERN is 
capable of accessing physics at higher energy scales than previous colliders and producing an unprecedented amount of luminosity.


is currently the largest and most powerful particle accelerator and collider that has been built to date.
It is not only capable of accessing physics at higher energy scales than ever before, but is also capable of producing an unprecedented amount of luminosity, allowing physicists to make precise measurements and rare tests 
, but also to search for new physics beyond it.


This thesis presents a search for a predicted but undiscovered singly produced top quark process and a number of the contributions towards a study considering a potential future particle detector upgrade.

The analysis presented looks for, and makes a cross section measurement of, a single top quark which is produced in association with a Z boson in the final state involving two leptons using proton-proton collision data at $\sqrt{13}$ collected by the Compact Muon Solenoid at the Large Hadron Collider during 2016.
This process has been predicted by the Standard Model but has yet to be measured given both its rarity and similarity to more commonly produced background processes.
As the process involves the Z boson coupling between both the top quark and W boson, it is particularly sensitive to any new physics in the electroweak sector which would manifest as ..

The Compact Muon Solenoid at the High Luminosity Large Hadron Collider will require a track finding system to provide information to the trigger system in order to discriminate in favour of potentially interesting physics against increasingly large backgrounds.
During the development of one of the proposed track finding systems, studies were undertaken regarding various track fitting algorithms which would fit precise track helix parameters to the tracks found and the ability of the system to find tracks with $\pT > 3\GeVc$.
In this thesis the studies concerning the development of a Linearised $\chi^{2}$ fitter and the ability of the proposed system to find tracks with $\pT > 2GeVc$ are presented.
