\chapter{Introduction}\label{chapter:intro}

\emph{``If I have seen further it is by standing on the shoulders of giants''}
\emph{Letter to Robert Hooke FRS, February 15th 1676, by Sir Isaac Newton FRS (1643-1727)}

The idea that nature can be explained through rational explanations, such as the ancient philosophical concepts of \emph{Atomism} and the Ancient Greek's \emph{Classical Elements}, is one that stretches back into time immemorial.

Following the scientific revolution of the 17th century the scientific method replaced such philosophical reasoning as the basis for exploring the nature of reality.
By formulating hypotheses whose predictions can tested by empirical evidence, successive generations of scientists
have built upon and improved on the ideas of those before them.
By amending existing theories or proposing new theories supported by new and more precise measurements, unified descriptions of seemingly unrelated phenomena have emerged, such as James Clark Maxwell's theory of electromagnetism.
This process has taken us from John Dalton's atomic theory and Sir Isaac Newton's laws of motion to the Standard Model (SM) of Particle Physics in the present day, describing all known elementary particles and three of the four fundamental forces of nature.


The SM is one of the greatest and most powerful scientific theories, making remarkably accurate predictions that have withstood incredible experimental scrutiny.
Despite the completion of the SM with the discovery of the Higgs Boson in 2012~\cite{HiggsCMS,HiggsATLAS} at the Large Hadron Collider (LHC), it is clear that the SM cannot be a complete description of reality for a number of reasons; including the following:
\begin{itemize}
\item Gravity is not accounted for within the SM and at high energy densities it is fundamentally irreconcilable with the classical theory of General Relativity~\cite{Sola:2013gha}.
\item There is strong experimental evidence that the observed rotation curves of galaxies and gravitational lensing cannot be accounted for by SM particles alone and that there must therefore be a large weakly interacting \emph{Dark Matter} component to the Universe~\cite{Bertone:2004pz}.
\item The presence of so-called \emph{Dark Energy} has also been inferred from astronomical and cosmological observations to account for the observed rate of expansion of the Universe~\cite{Peebles:2002gy}.
\item Neutrinos have been observed to oscillate between different flavours, implying that they have non-zero masses in contrast to SM expectations~\cite{Fukuda:1998mi,Ahmad:2001an}.
\item There is currently no explanation that accounts for the clear abundance of matter over anti-matter in the observable universe.
\end{itemize}

In addition to these, many scientists are uncomfortable with the fact that the SM contains a large number of finely tuned experimentally derived parameters, preferring a theory from which these values would emerge naturally, resulting in a more ``complete'' description of reality~\cite{Burdman:2007ck}.

One of the approaches to study these issues is to investigate increasingly higher energy scales to test our existing theories and to look for new physics beyond them.
The LHC at CERN is the most powerful and luminous particle accelerator built to date and provides physicists the capability to study an unprecedented number of events.
In addition to discovering the Higgs boson, the unprecedented collision energies and number of events produced provide physicists the capability to probe the consistency of the SM through precision measurements and to search for new physics at the and above the \TeV level.

As the heaviest known fundamental particle, the top quark provides a unique means to probe multiple aspects of the SM.
The top quark's mass of $173.0 \pm 0.4 \GeV$~\cite{Tanabashi:2018oca} not only places it near the electroweak symmetry breaking scale, but has the consequence of the top quark having a lifetime shorter than the strong force's characteristic time.
Consequently, the top quark decays before it can be confined into a hadron, thus making measurements of its properties more accessible compared to the other quarks.
As such, studying the top quark provides unique opportunities to probe the electroweak force and the properties of individual quarks.

This thesis presents a search for an as yet unobserved SM process in which a single top quark is produced  in association with a Z boson, known as \emph{tZq}, in the file state containing two leptons based on proton-proton collision data at $\sqrt{s} = 13\TeV$ collected by the Compact Muon Solenoid (CMS) experiment at the LHC during 2016.
tZq is a process which is a particularly sensitive probe of the electroweak sector as not only is the top quark produced through electroweak interactions, but also the Z boson coupling to both the top quark and W boson.

This thesis also presents studies relating to the future upgrade of the CMS silicon tracker for the High Luminosity (HL-LHC).
The High Luminosity LHC will be capable of providing an instantaneous luminosity up to an order of magnitude greater than the LHC today.
Consequently, the CMS experiment's tracking detector will require a track finder to provide information to the online trigger in order to discriminate in favour of potentially interesting physics against increasingly large backgrounds.
During the development of one possible track finder, studies were undertaken regarding various track fitting algorithm would best find tracks down to transverse momenta of just 3\GeV and precisely fit track helix parameters to them.
The studies presented in this thesis detail the development of a linearised $\chi^{2}$ fitter and the impact on the proposed system of reducing the minimum track transverse momenta requirement from the baseline specification of 3\GeV to 2\GeV.
