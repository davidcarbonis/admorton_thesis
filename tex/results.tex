\chapter{Results}\label{chapter:results}
Following providing the multivariate analysis technique described in Section~\ref{sec:mvas} with the simulated samples, and their systematic variations, and data, the resultant set of BDT discriminator distributions can be used to perform a measurement.

The following chapter describes the statistical methodology used to analyse these distributions and produce the first measurement of the signal process' cross section along with its expected significance.
Following a discussion of the impact that the systematic uncertainties have on the fitted results, the result presented is compared with those from the already published searches for tZq production made using the trilepton final state.

\section{Statistical Methodology}\label{sec:statisticalModel}
%% Limit Setting text
%%The Higgs Analysis Combined Limit (\combine) tool~\cite{Combine}, a framework based on the RooStats package~\cite{Moneta:2010pm,Schott:2012zb}, was used to perform a binned Maximum Likelihood Fit (MLF) to analyse the measurement made using the $CL_{S}$ method~\cite{Read:2002hq,CMS-NOTE-2011-005,Khachatryan:2014jba,AsymptoticFormulae}.

The Higgs Analysis Combined Limit (\combine) tool~\cite{Combine}, a framework based on the RooStats package~\cite{Moneta:2010pm,Schott:2012zb}, was used to perform a binned Maximum Likelihood Fit (MLF) to determine the cross section of the signal process using the profile likelihood method~\cite{AsymptoticFormulae}.

\subsection{Likelihood Model}\label{subsec:likelihoodModel}
For the signal and background processes considered in the search, the expected event yield $\lambda$ in bin $i$ of the distribution considered (\ie the BDT discriminator) is given by Equation~(\ref{eq:yields1}):

\begin{equation}
\lambda_{i} = \mu s_{i} + \sum\limits_{j}^{n_{bkgs}} b_{j} \;
\label{eq:yields1}
\end{equation}

where $s$ and $b$ are the expected number of signal and background events, respectively, the index $j$ runs over the number of background sources, $n_{bkgs}$, and $\mu$ is the signal strength modifier.
The signal strength modifier is typically used instead of directly determining the expected (and observed) cross section of a process as it makes the comparison of different results (particularly from different experiments) more straightforward. 
The relationship between $\mu$ and the observed and expected cross sections, $\sigma_{obs}$ and $\sigma_{s}$, is given by Equation~(\ref{eq:signalModifier}):

\begin{equation}
\mu = \frac{\sigma_{obs}}{\sigma_{s}}  \;
\label{eq:signalModifier}
\end{equation}
 
The uncertainties associated with the simulated predictions for the signal and background processes are accounted for by the inclusion of a set of nuisance parameters $\theta$.
Therefore, as $s_{i}$ and $b_{i}$ are dependent on $\theta$, they become $s_{i} = s_{i} (\theta)$ and $b_{i} = b_{i} (\theta)$.

Assuming that the number of observed events, $n_{i}$, in any given bin of the distribution considered will be distributed according to Poisson statistics, the probability of observing $n_{i}$ is given by Equation~(\ref{eq:poissonProb}):

\begin{equation}
\mathcal{P} ( n_{i} | \lambda_{i} ) = \frac{\lambda_{i}}{n_{i}!} e^{- \lambda_{i}} = \frac{ \big( \mu s_{i}(\theta) + b_{i}(\theta) \big)^{n_{i}}}{n_{i} !} e^{- \mu s_{i}(\theta) - b_{i}(\theta)}  \;
\label{eq:poissonProb}
\end{equation}

A probability density function, $\rho ( \theta | \tilde{\theta} )$, is used to describe all the sources of uncertainty for the nuisance parameters, where $\tilde{\theta}$ is the set of nominal values for the the best estimate of the nuisances.
For the search presented in this thesis, it is assumed that each source of systematic uncertainty is either 100\% correlated or uncorrelated.
This allows each systematic uncertainty to be incorporated into the likelihood function in a clean factorised form.
Shape uncertainties are modelled by vertically morphing the nominal shape template up and down by one $\sigma$.
The normalisation/rate uncertainties are treated as log-normal distributed nuisance parameters~\cite{AsymptoticFormulae,Baak:2014fta}.
%%% log-uniform - for NPL to allow it to be simultaneously fitted with the signal normalisation

Thus, the likelihood for the entire dataset can be expressed as the product of the Poisson probabilities, $\mathcal{P}$, for all bins and the nuisance parameters' probability density function, as given by Equation~(\ref{eq:poissonLikelihood}).

\begin{equation}
\mathcal{L} ( n_{i} | \mu , \theta ) = 
\prod_{i=1}^{N} \mathcal{P} \big( n_{i} | \mu s_{i}(\theta) + b_{i}(\theta) \big) \rho ( \theta | \tilde{\theta} ) \;
\label{eq:poissonLikelihood}
\end{equation}

A test statistic, $q_{\mu} $, can be constructed to evaluate the compatibility of data with the \emph{signal plus background} (s+b) ($\mu = 1$) and \emph{background only} (b-only) ($\mu = 0$) hypotheses or between the different hypotheses.
The test statistic used by the ATLAS and CMS is defined as the log-likelihood ratio in Equation~(\ref{eq:testStatistic}):

\begin{equation}
q_{\mu} =  -2 \ln \frac{ \mathcal{L}(data | \mu , \hat{\theta}_{\mu})} { \mathcal{L}(data | \hat{\mu} , \hat{\theta})  } \textrm{ , where } 0 \leq \hat{\mu} \leq \mu \;
\label{eq:testStatistic}
\end{equation}

where $\hat{\theta}_{\mu}$ refers to the maximum likelihood estimators of $\theta$ for a given $\mu$, $\hat{\mu}$ and $\hat{\theta}$ correspond to the global maximum of the likelihood and can refer to either real observed data or \emph{pseudo-data}.
By definition $\hat{\mu}$ cannot take negative values as physics defines the signal rate as positive. 
The constraint $\hat{\mu} < \mu$ is applied to ensure a one-sided confidence interval.

\subsection{Signal Strength Modifier Calculation and Significance}\label{subsec:CLsMethod}
% Combine uses MultiDimFit
% Best fit uncerts are at 68% CL (one sigma)

The signal strength modifier for the signal process was calculated using the profile likelihood method, which maximises the likelihood function in Equation~(\ref{eq:poissonLikelihood}) by allowing $\mu$ and $\theta$ to float.
Using the global likelihood maximum values of $\mu$ and $\theta$, $\hat{\mu}$ and $\hat{\theta}$, the test statistic was used to determine the 68\% confidence limits for the measured signal strength modifier by allowing $\hat{\theta}_{\mu}$ to float in order to maximise the likelihood and varying $\mu$ until a value that represents 68\% agreement is obtained.

%Significance
Both the expected and observed significances for the signal strength modifier were calculated by evaluating the s+b hypothesis and data, respectively, against the b-only hypothesis using the the test statistic in Equation~(\ref{eq:testStatistic}).
Therefore the significances are the fractions of the events for the b-only hypothesis likelihood function whose likelihood values exceed that of the observed value for data or the median value for the s+b hypothesis.
% 68% CLs for significances? Not provided ...

%Asimov:
%As it is difficult to analytically solve Equation~(\ref{eq:testStatistic}), pseudo-data is generated to obtain the values for s+b and b hypotheses' likelihood functions.  
As producing pseudo-data using an ensemble of toy MC samples to obtain the values for s+b and b hypotheses' likelihood functions can be computationally intensive, the \emph{asymptotic} method is used when the number of expected events is sufficiently large.
The asymptotic method produces one representative dataset, known as the \emph{Asimov dataset}, that which is defined as being constructed such that all observable quantities are equal to their expectation values.
This method was used for the analysis presented as it removed the need to generate toy MC datasets.
A full description of the asymptotic approximation is given in~\cite{AsymptoticFormulae}.

%% Limit setting text
%As the analysis was initially conducted blind, \emph{pseudo-data} was generated for the \emph{expected} outcomes in order to construct probability distribution functions for the s+b and b-only hypotheses for a given signal strength.
%For the unblinding of the analysis, the \emph{observed} measurement was made by replacing the pseudo-data used for the s+b probability distribution functions with the actual data.
%From these probability distribution functions, the probability values for the s+b and b-only hypotheses can be defined as Equations~(\ref{eq:pmu})-~(\ref{eq:1-pb}):
%
%\begin{equation}
%p_{\mu} = P ( q_{\mu} \geq  q_{\mu}^{obs} | \mu s + b ) = \int^{\infty}_{q_{\mu}^{obs}} f ( q_{\mu} | \mu , \theta_{\mu}^{obs} ) dq_{\mu} \;
%\label{eq:pmu}
%\end{equation}
%
%\begin{equation}
%1 - p_{b} = P ( q_{\mu} \geq  q_{\mu}^{obs} | b ) = \int^{\infty}_{q_{0}^{obs}} f ( q_{\mu} | 0 , \theta_{0}^{obs} ) dq_{\mu} \;
%\label{eq:1-pb}
%\end{equation}
%
%The ratio of these probabilities is used to define $CL_{S} (\mu)$ in Equation~(\ref{eq:CLs}).
%
%\begin{equation}
%CL_{S} (\mu) = \frac{ p_{\mu} }{ 1 - p_{b} }\;
%\label{eq:CLs}
%\end{equation}
%
%If, for a given $\mu$, $CL_{S} < \alpha$, then it is said that the signal process in question is excluded with a $(1 - \alpha)$ Confidence Level (CL).
%Therefore, to obtain the 95\% CL upper limit,  $\mu$ would be altered until $CL_{S} = 0.05$.

\clearpage
\newpage

\section{Statistical Analysis Results}\label{sec:results}
The observed signal strength for tZq production was determined to be $6.213_{-2.695}^{+2.339}$ and $4.725_{-2.015}^{+1.916}$ for the $ee$ and $\mu\mu$ channels, respectively, corresponding to a significances of $2.722 \sigma$ and $2.501\sigma$, respectively. %$4.74^{+1.69}_{-1.51$}
Using the reference NLO cross section of $\sigma (tZq, Z \rightarrow l^{+} l^{-}$) = 94.2~fb~\cite{Campbell:2013yla}, these signal strengths corresponds to cross section of $194.8_{-84.7}^{+73.4}$~fb and $148.5_{-63.4}^{+60.3}$~fb for the $ee$ and $\mu\mu$ final states, respectively. %$94.2^{+95.3}_{-98.4475}$.
These results are consistent within two $\sigma$ of the SM predictions and the measured combined signal strengths of $0.75 \pm 0.28$ and $1.45^{0.48}_{-0.42}$ made using the trilepton final state at $\sqrt{s} = 13 \TeV$ by the ATLAS and CMS collaborations, respectively~\cite{Aaboud:2017ylb,Sirunyan:2017nbr}.

%%%$0.75 \pm 0.28$ ATLAS and $1.45^{0.48}_{-0.42}$ CMS

The results presented here are the initial results obtained following the unblinding of the analysis for this thesis.
Whilst CMS has given permission for this unblinding, the results have not been fully reviewed by the collaboration and therefore these results should not be considered to have been endorsed by CMS.
It is expected that further work will need to be done in order to achieve the required standard for journal publication on behalf of the CMS collaboration.
At their request, no combined result for the two final states has been presented in this thesis.

The observed signal strengths, cross sections, and expected and observed significances for the $ee$ and $\mu\mu$ channels are shown in Table~\ref{tab:shapetxs}.

\begin{table}[!h]
   \centering
   \caption{The expected signal strengths and corresponding cross sections for
   the $ee$ and $\mu\mu$ channels.}
   \begin{tabular}{cccc}
       \hline
       Channel & $ee$ & $\mu\mu$ & Combined \\
        \hline
       Signal Strength & $6.21_{-2.70}^{+2.34}$ & $4.73_{-2.02}^{+1.92}$ & - \\
       Cross section (fb) & $194.8_{-84.7}^{+73.4}$ & $148.5_{-63.4}^{+60.3}$ & - \\
       Significance (expected) ($\sigma$) & $0.46$ & $0.54$ & $0.70$\\
       Significance (observed) ($\sigma$) & $2.72$ & $2.50$ & - \\
    \end{tabular}
   \label{tab:shapetxs}
\end{table}
%
%\begin{table}[!h]
%   \centering
%   \caption{The expected signal strengths and corresponding cross sections for
%   the individual channels and the channels combined.}
%   \begin{tabular}{cccc}
%       \hline
%       Channel & $ee$ & $\mu\mu$ & \textbf{Combined} \\
%        \hline
%       Signal Strength & $6.213_{-2.695}^{+2.339}$ & $4.725_{-2.015}^{+1.916}$ & $5.366_{-1.638}^{+1.565}$ \\
%%       Cross section (fb) & $X_{-Z}^{+Y}$ & $X_{-Z}^{+Y}$ & $X_{-Z}^{+Y}$ \\
%       Significance (expected) ($\sigma$) & $0.460$ & $0.544$ & $0.705$ \\
%       Significance (observed) ($\sigma$) & $2.722$ & $2.501$ & $3.547$ \\
%    \end{tabular}
%   \label{tab:shapetxs}
%\end{table}


\subsection{Post-fit BDT Discriminant Distributions}
The BDT discriminant distributions following the MLF for data and simulation are shown in Figure~\ref{fig:postfitBDT}.
When compared to the pre-fit distributions in Figure~\ref{fig:postfitBDT}, it can be seen that the MLF has constrained the impact of the systematic uncertainties and increased the tZq yield to obtain the best possible answer.
While the tZq contribution is more evident in the BDT discriminant distributions following the MLF, it is clear from the $\hat{\mu}$ that the tZq yield is still consistent with the SM predictions.

\begin{figure}[!h]
\centering
\includegraphics[width=0.78\textwidth]{figs/results/postfit_ee.pdf}
\\
\includegraphics[width=0.78\textwidth]{figs/results/postfit_mumu.pdf}
\caption{
Post-fit distributions of the BDT discriminant for the $ee$ channel (top) and $\mu\mu$ channel (bottom) for simulation describing the s+b hypothesis and data.}
\label{fig:postfitBDT}
\end{figure}

\subsection{Post-fit Impact of the Systematic Uncertainties}\label{sec:uncertainitiesImpact}
Figure~\ref{fig:systematicsPull} illustrates the impact of each of the sources of systematic uncertainty on the signal strength modifier $\hat{\mu}$ for the $ee$ and $\mu\mu$ channels.
The left-hand side of this plot shows best fit value of the nuisance parameters where the asymmetric error bars are the pre-fit uncertainty divided by the post-fit uncertainty.
The right-hand side illustrates the impact of varying a nuisance parameter to its $\pm \sigma$ post-fit values on the $\hat{\mu}$.

All of the experimental and theoretical scale uncertainties were constrained by the MLF, with the ME and PS scale and PDF uncertainties having the greatest, and comparable, impact on the $\hat{\mu}$ for both of the signal process' final states.
Consequently, measurement's precision would be best improved by an improved theoretical understanding of tZq production and the dominant background processes and by a reduction in the uncertainty on the parton distributions used for generating MC samples. 
%
%Given the significant multi-jet component of the signal process' final state, it is not unexpected that jet related uncertainties
%Despite having a lesser impact on the $\hat{\mu}$ than the uncertainties associated with the PDFs, the JES and JER uncertainties have 
%Therefore, given the 
%

%% ee JES, 2nd most significant experiemental uncert; JER one of the lesser impactful uncerts
%% mumu JES better constrained, JER same order of magnitiude as other non-PDF largest uncerts

Whilst the cross section normalisation uncertainties associated with the NPLs and minor background contributions were not constrained by the MLF, they have a negligible impact on the $\hat{\mu}$.
In contrast, despite being constrained in the fit, both the \ttbar and Z+jets normalisation uncertainties had a significant impact on the $\hat{\mu}$, with the \ttbar cross section uncertainty having one of the greatest impacts on the $\hat{\mu}$ for the $\mu\mu$ final state.
Additionally, the \ttbar and Z+jets cross section uncertainties were offset from their pre-fit values.
Given that the reason behind these offsets was not understood at the time of the unblinding of this analysis, it is imperative to establish the cause to ensure that the two most important background processes for this search are properly understood for future measurements.

\begin{figure}[htbp]
\begin{center}
\includegraphics[width=0.82\textwidth]{figs/results/systematicsImpact_ee.pdf}
\\
\includegraphics[width=0.82\textwidth]{figs/results/systematicsImpact_mumu.pdf}
\caption{The best fit value and uncertainties of the nuisance parameters are shown on the left-hand side of the plot, where $\widehat{\theta}$ and $\theta_{0}$ are the post-fit and pre-fit values for a nuisance parameter and $\Delta \theta$ is the pre-fit uncertainty.
The right-hand side of the plot shows the impact that each systematic uncertainty has on the signal strength parameter $\hat{\mu}$ when varied by $\pm 1 \sigma$.
The top and bottom plots refer to the $ee$ and $\mu\mu$ channels, respectively.}
\label{fig:systematicsPull}
\end{center}
\end{figure}

\section{Discussion of other searches for tZq at the Large Hadron Collider}
The search for the decay of a single top quark produced in association with a Z boson presented in this thesis is the first one that has been made using the dilepton final state at the LHC.

Previously the production of a single top quark in association with a Z boson has been searched for using the trilepton final state at the LHC by the ATLAS and CMS collaborations.
The CMS Collaboration has performed analyses at both $\sqrt{s} = 8 \TeV$ and $\sqrt{s} = 13 \TeV$.
The search at $\sqrt{s} = 8 \TeV$ used the 2012 dataset of 19.7\fbinv and measured a signal strength of $1.22^{+0.98}_{-0.85}$, corresponding to an observed (expected) significance of $2.4 \sigma$ ($1.8\sigma$)~\cite{Sirunyan:2017kkr}.
The subsequent search by CMS observed tZq production at $\sqrt{s} = 13 \TeV$, using the 2016 dataset of 35.9\fbinv. 
This search measured a signal strength of $1.31^{+0.35}_{-0.33} (\textrm{stat}) ^{+0.31}_{-0.25}(\textrm{syst})$ with an observed (expected) significance of $3.7 \sigma$ ($3.1\sigma$)~\cite{Sirunyan:2017nbr}.
The first evidence for tZq production was found by the ATLAS collaboration at $\sqrt{s} = 13 \TeV$ using 36.1\fbinv of data collected during 2015-2016, measuring a signal strength of $0.75 \pm 0.28$ at an observed (expected) significance of $4.2\sigma$ ($5.4\sigma$)~\cite{Aaboud:2017ylb}.

%%% How do trilepton and dilepton differ?
The signal strength measured using the dilepton final state is consistent within two standard deviations of both the SM prediction and the measurements of tZq that have been made using the trilepton final state.
The difference between the expected and observed significances of the trilepton final state  measurements and the expected significance of the dilepton final state measurement presented is due to the differing backgrounds of these two final states.
As searches for the trilepton final state require the presence of three leptons, backgrounds with only two leptons are suppressed.
Therefore, the largest background processes for this final state are WZ+jets, \ttZ~and those with large production cross sections that contribute to the NPL background, such as Z+jets and \ttbar.
In contrast, the dilepton final state's requirement of two leptons that are compatible with a Z boson decay suppresses processes that do not produce leptons from a Z boson decaying and those that contribute to the NPL background.
The final state of two leptons and multiple jets however, is identical to those of a large number of background processes that have cross sections many orders of magnitude larger than those for the trilepton final state, such as Z+jets and \ttbar.
Consequently, searches for tZq production using the dilepton final state are statistically limited to a greater degree than those using the trilepton final state.