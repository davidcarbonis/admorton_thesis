\chapter{Results}\label{chapter:results}
The analysis using described in the preceding chapters are detailed in this chapter
Following the application of the event selection criteria and evaluation of the systematic uncertainties described in Chapter~\ref{chapter:tzq-search}, the result 

\section{Statistical Model}
Calculation of the observed signal strength, observed significance, and
expected significance was performed by using the Higgs Analysis Combined Limit
tool (\combine{})~\cite{Combine}. The signal strength was determined using a
maximum likelihood fit.  The significances were calculated using an asymptotic
approximation~\cite{AsymptoticFormulae}, using the Asimov dataset.

All systematic uncertainties were incorporated into the fit as nuisance
parameters, with luminosity and fake rate taken as rate uncertainties and the
remainder as shape uncertainties.
 
The fit was performed on the two channels simultaneously. Most systematics were
assumed to be 100\% correlated between channels.

\subsection{Cross section extraction}
 
By performing a simultaneous fit of the BDT discriminant distribution in the
background-enriched sample and the BDT discriminant in the signal sample, any
events in excess of the background-only hypothesis will be determined. This
excess can then be compared to the SM expectation for tZq production in order
to calculate the observed signal strength and measure the cross section. A
measured signal strength of 0.0 would correspond to an observation of the
background-only hypothesis alone, whilst 1.0 is the SM expectation for tZq
production.

\subsection{Signal strength significance}

The observed signal strengths, measured cross sections, and corresponding
significances for the individual channels and the channels combined in the
signal region using pseudo data, are shown in Table~\ref{tab:shapetxs}. These
are [IN AGREEMENT / NOT IN AGREEMENT] with the SM cross section of
$X^{+Y}_{-Z}$.
 
\begin{table}[!h]
   \centering
   \caption{The observed signal strengths and corresponding cross sections for
   the individual channels and the channels combined at the 95\% CL.}
   \begin{tabular}{cccc}
       \hline
       Channel & $ee$ & $\mu\mu$ & \textbf{combination} \\
        \hline
        % \multicolumn{4}{c}{\combine{}} \\
        % \hline
        Signal strength & $X_{-Z}^{+Y}$ & $X_{-Z}^{+Y}$ & $X_{-Z}^{+Y}$ \\
       Cross section (fb) & $X_{-Z}^{+Y}$ & $X_{-Z}^{+Y}$ & $X_{-Z}^{+Y}$ \\
       Significance (expected) & $X_{-Z}^{+Y}$ & $X_{-Z}^{+Y}$ & $X_{-Z}^{+Y}$ \\
       Significance (observed) & $X_{-Z}^{+Y}$ & $X_{-Z}^{+Y}$ & $X_{-Z}^{+Y}$ \\
        \hline
        % \multicolumn{4}{c}{\textsc{Theta}} \\
        % \hline
        % Signal strength & $X_{-Z}^{+Y}$ & $X_{-Z}^{+Y}$ & $X_{-Z}^{+Y}$ \\
        % Cross section (fb) & $X_{-Z}^{+Y}$ & $X_{-Z}^{+Y}$ & $X_{-Z}^{+Y}$ \\
        % Significance (expected) & $X_{-Z}^{+Y}$ & $X_{-Z}^{+Y}$ & $X_{-Z}^{+Y}$ \\
        % Significance (observed) & $X_{-Z}^{+Y}$ & $X_{-Z}^{+Y}$ & $X_{-Z}^{+Y}$ \\
        % \hline
    \end{tabular}
   \label{tab:shapetxs}
\end{table}

\section{Interpretation of the results}
\section{Other results from the Large Hadron Collider}
The search for a singly produced top in association with a Z boson in the dilepton final state presented is the first one made at the LHC and follows in the footsteps of the searches for the trilepton final state at $\sqrt{8}$ and $\sqrt{13}$ using data collected by the CMS experiment in 2012 and 2016 respectively.
Despite the dilepton final state having a larger cross section than the trilepton final state, the different final state topology makes it much more difficult to isolate the signal process.

ATLAS: 13 TeV trilepton Aaboud:2017ylb