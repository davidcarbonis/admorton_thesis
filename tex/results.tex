\chapter{Results}\label{chapter:results}
Following the application of the event selection criteria and evaluation of the systematic uncertainties described in Chapters~\ref{chapter:tzq-search} and, the result 

\section{Statistical Model}\label{sec:statisticalModel}
Calculation of the observed signal strength, observed significance, and expected significance was performed by using the Higgs Analysis Combined Limit tool (\combine{})~\cite{Combine}. 
The signal strength was determined using a maximum likelihood fit.  
The significances were calculated using an asymptotic approximation~\cite{AsymptoticFormulae}, using the Asimov dataset.

All systematic uncertainties were incorporated into the fit as nuisance parameters, with luminosity and fake rate taken as rate uncertainties and the remainder as shape uncertainties.
 
The fit was performed on the two channels simultaneously. 
Most systematics were assumed to be 100\% correlated between channels.

\subsection{Cross section extraction}
 
By performing a simultaneous fit of the BDT discriminant distribution in the background-enriched sample and the BDT discriminant in the signal sample, any events in excess of the background-only hypothesis will be determined.
This excess can then be compared to the SM expectation for tZq production in order to calculate the observed signal strength and measure the cross section.
A measured signal strength of 0.0 would correspond to an observation of the background-only hypothesis alone, whilst 1.0 is the SM expectation for tZq sproduction.

\subsection{Signal strength significance}

The observed signal strengths, measured cross sections, and corresponding significances for the individual channels and the channels combined in the signal region using pseudo data, are shown in Table~\ref{tab:shapetxs}. 
These are [IN AGREEMENT / NOT IN AGREEMENT] with the SM cross section of  $X^{+Y}_{-Z}$.
 
\begin{table}[!h]
   \centering
   \caption{The observed signal strengths and corresponding cross sections for
   the individual channels and the channels combined at the 95\% CL.}
   \begin{tabular}{cccc}
       \hline
       Channel & $ee$ & $\mu\mu$ & \textbf{combination} \\
        \hline
        % \multicolumn{4}{c}{\combine{}} \\
        % \hline
        Signal strength & $X_{-Z}^{+Y}$ & $X_{-Z}^{+Y}$ & $X_{-Z}^{+Y}$ \\
       Cross section (fb) & $X_{-Z}^{+Y}$ & $X_{-Z}^{+Y}$ & $X_{-Z}^{+Y}$ \\
       Significance (expected) & $X_{-Z}^{+Y}$ & $X_{-Z}^{+Y}$ & $X_{-Z}^{+Y}$ \\
       Significance (observed) & $X_{-Z}^{+Y}$ & $X_{-Z}^{+Y}$ & $X_{-Z}^{+Y}$ \\
        \hline
        % \multicolumn{4}{c}{\textsc{Theta}} \\
        % \hline
        % Signal strength & $X_{-Z}^{+Y}$ & $X_{-Z}^{+Y}$ & $X_{-Z}^{+Y}$ \\
        % Cross section (fb) & $X_{-Z}^{+Y}$ & $X_{-Z}^{+Y}$ & $X_{-Z}^{+Y}$ \\
        % Significance (expected) & $X_{-Z}^{+Y}$ & $X_{-Z}^{+Y}$ & $X_{-Z}^{+Y}$ \\
        % Significance (observed) & $X_{-Z}^{+Y}$ & $X_{-Z}^{+Y}$ & $X_{-Z}^{+Y}$ \\
        % \hline
    \end{tabular}
   \label{tab:shapetxs}
\end{table}

\subsection{Impact of the Systematic Uncertainties}\label{sec:uncertainitiesImpact}
The effect of each of the systematics considered on the event rate, in percentage, for the signal and control regions are shown in Tables~\ref{tab:systImpact}.
These rates, whilst providing a useful insight into which of the systematics are the most important, do not show how the shape of each fitted variable and the MVA discriminant is influenced by each uncertainty.
\editComment{Make some comment on most important/impactful systematics and how better understanding them would improve the result}

\begin{table}[!htbp]
\begin{center}
\linespread{2}
\resizebox{\textwidth}{!}{\begin{tabular}{|l|c|c|c|c|}
\hline
Systematic      &  tZq                  & DY                   & \ttbar{}                  & Other         \\
($ee$ / $\mu\mu$) & (\%)  & (\%)  & (\%)  & (\%)  \\
\hline
Trigger             &  $_{-4.23\%}^{+4.24\%}$ /  $_{-0.21\%}^{+6.07\%}$   & $_{-4.72\%}^{+4.07\%}$ / $_{-0.32\%}^{+6.37\%}$  & $_{-5.08\%}^{+4.41\%}$ / $_{-0.55\%}^{+5.54\%}$ & $_{-4.72\%}^{+4.85\%}$ / $_{-4.47\%}^{+5.97\%}$  \\
JER             &  $_{-5.27\%}^{+6.02\%}$ /  $_{-6.11\%}^{+5.39\%}$   & $_{-11.81\%}^{+16.54\%}$ / $_{-14.18\%}^{+16.71\%}$  & $_{-7.98\%}^{+7.84\%}$ / $_{-6.13\%}^{+8.24\%}$  & $_{--1.96\%}^{+2.11\%}$ / $_{-1.62\%}^{+1.82\%}$  \\
JES             &  $_{-0.04\%}^{+0.19\%}$ /  $_{-0.13\%}^{+0.13\%}$   & $_{-0.55\%}^{+0.29\%}$ / $_{-0.17\%}^{+0.13\%}$  & $_{-1.30\%}^{+0.02\%}$ / $_{-0.20\%}^{+0.20\%}$  & $_{-0.0.01\%}^{+0.11\%}$ / $_{-0.14\%}^{+0.18\%}$  \\
Pileup             &  $_{-0.42\%}^{+0.43\%}$ /  $_{-0.17\%}^{+0.43\%}$   & $_{-2.35\%}^{+2.26\%}$ / $_{-2.57\%}^{+1.75\%}$  & $_{-1.52\%}^{+0.52\%}$ / $_{-0.09\%}^{+1.35\%}$  & $_{-0.86\%}^{+0.38\%}$ / $_{-0.15\%}^{+0.26\%}$  \\
bTag             &  $_{-2.78\%}^{+3.38\%}$ /  $_{-3.38\%}^{+2.99\%}$   & $_{-5.30\%}^{+5.11\%}$ / $_{-5.02\%}^{+5.12\%}$  & $_{-2.89\%}^{+3.02\%}$ / $_{-3.12\%}^{+3.77\%}$  & $_{-3.43\%}^{+3.25\%}$ / $_{-3.24\%}^{+3.00\%}$  \\    
PDF             &  $_{-9.98\%}^{+13.22\%}$ /  $_{-9.24\%}^{+11.94\%}$   & $_{-1.56\%}^{+1.73\%}$ / $_{-2.95\%}^{+2.16\%}$  & $_{-2.99\%}^{+1.85\%}$ / $_{-2.95\%}^{+2.16\%}$  & $_{-8.56\%}^{+9.95\%}$ / $_{-8.51\%}^{+9.40\%}$  \\
$Q^{2}$Scaling             &  $_{-2.82\%}^{+1.36\%}$ /  $_{-3.06\%}^{+1.33\%}$   & $_{-15.00\%}^{+2.92\%}$ / $_{-14.64\%}^{+2.05\%}$  & $_{-11.38\%}^{-1.38\%}$ / $_{-11.40\%}^{+0.0\%}$  & $_{-5.01\%}^{+1.37\%}$ / $_{-5.07\%}^{+1.8\%}$  \\
\hline
\end{tabular}
}
\caption{Rate impact of systematics on MC templates}\label{tab:systImpact}
\end{center}
\end{table}

\section{Interpretation of the results}
\section{Other results from the Large Hadron Collider}
The search for a singly produced top in association with a Z boson in the dilepton final state presented is the first one made at the LHC and follows in the footsteps of the searches for the trilepton final state at $\sqrt{8}$ and $\sqrt{13}$ using data collected by the CMS experiment in 2012 and 2016 respectively.
Despite the dilepton final state having a larger cross section than the trilepton final state, the different final state topology makes it much more difficult to isolate the signal process.

ATLAS: 13 TeV trilepton Aaboud:2017ylb