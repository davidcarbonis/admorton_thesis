\chapter{Analysis Strategy }\label{chapter:tzq-search}
%Following from the ... next chapters of this thesis describe the search for a singly produced top quark in association with a Z boson using the reconstructed proton-proton collision data at $\sqrt{13}$ collected by the CMS detector during 2016.
%Using the particles identified and reconstructed, as discussed in Chapter~\ref{chapter:data-mc}, 

The first stage of a physics analysis is to determine an event selection based on the topology of the process in order to create a region which is enriched with signal events.
The main challenge in defining a signal enriched region for the dilepton final state of tZq is that its topology is identical to final states of other processes with much larger cross sections, resulting in a background dominated.

Whilst the 

The sysy


The main challenge for this search 


\section{Event Selection}\label{sec:signalRegion}
A number of selection criteria 


The tZq process at leading order consists of a single top quark, a radiated Z boson and a recoil quark.
Therefore, the final state contains two leptons from the Z boson decay and at least four jets, 
, a b-jet originating from 

\subsection{Trigger Strategy}
As the presence of the two leptons from the Z boson decay is an essential component of the event reconstruction, the trigger strategy consists of selecting events from datasets identified by the 

 whilst retaining as low a transverse momenta threshold as possible

the final state of the process searched for contains two leptons 

contains two leptons (\ie two electrons or two muons)

Events in the \ttbar control region 

\begin{table}[htbp]
\topcaption {
Triggers and datasets used for each decay channel.
}
\label{tab:triggersDatasets}
  \centering
% This increases column spacing.
  \addtolength{\tabcolsep}{1ex}
% This right-aligns numbers in column, but centers them under column title.
  \begin{tabular}{|l|l|1|}
   \hline
   \bf{Final State} & \bf{Dataset} & \bf{HLT Paths}  \\
   \hline
    ee & SingleElectron &  HLT_Ele32_eta2p1_WPTight_Gsf &  \\
    & DoubleElectron & HLT_Ele23_Ele12_CaloIdL_TrackIdL_IsoVL_DZ \\
   \hline
    $\mu\mu$ & SingleMuon &  HLT_Iso(Tk)Mu24  \\  
        & DoubleMuon  & HLT_Mu17_TrkIsoVVL_(Tk)Mu8_TrkIsoVVL_DZ \\  

   \hline
   e$\mu$ & MuonEG &  HLT_Mu12_TrkIsoVVL_Ele23_CaloIdL_TrackIdL_IsoVL(_DZ) &  \\  
          &        &  HLT_Mu8_TrkIsoVVL_Ele23_CaloIdL_TrackIdL_IsoVL(_DZ) & \\
   \hline
   
 \end{tabular}
 \addtolength{\tabcolsep}{-1ex}
\end{table}


\subsection{Event Cleaning}
Following the trigger requirements, a number of filters are applied in order to so that events with beam or detector anomalies which result in anomalous \MET are not considered for use in the analysis.

\begin{itemize}
\item \textbf{Primary Vertex Filter} - ensures that the primary vertex is well reconstructed by requiring it to be within $|z| \leq 24\cm$ of the interaction point and within $d_{0} < 2\cm$ of the beam line.
\item \textbf{Beam Halo Filter} - beam halos are machine induced particles (\eg beam-gas, beam-pipe interactions) which circulate with the beam at radii up to 5m. The filter removes events with calorimeter and muon chamber energy deposits consistent with either halo particles or particle showers caused by halos interacting with the collimator blocks that used to clean halos from the beam.
\item \textbf{HBHE Noise and Isolation Filters} - removes events where anomalous noise is present in the HCAL's hybrid photodiodes or readout boxes, which registers as large isolated energy deposits which would infer the presence of large \MET, by considering the channel multiplicities, pulse shape of the readout and the neighbouring activity in the calorimeters and tracker.
\item \textbf{ECAL Trigger Primitive Filter} - the L1 trigger primitive readout can be used to estimate the energy deposited in approximately 70\% of the channels which lack regular data links and are masked out for reconstruction. As trigger the primitives have a narrower energy acceptance range than the read-out, when the energy is near their saturation energy the measured energy is likely to be underestimated, resulting in high anomalous \MET. 
%\item \textbf{ECAL Endcap SC Filters} - NOT recommended for 2016
\item \textbf{Bad Charged Hadron Filter} - removes events where a muon is not defined as a PF muon due to its low quality, but makes its way into the PF MET calculation as a charged hadron candidate.
\item \textbf{Bad Muon Filter} - removes events where a muon is defined as a PF muon, but is still has too low a quality and large \pT to be considered.
\end{itemize}

\subsection{Lepton Selection}
The final state 
\subsubsection{Electrons}
Electrons
\subsubsection{Muons}
Muons


\cite{Chatrchyan:2012xi}


\subsubsection{Lepton Isolation}
In order to differentiate between leptons promptly produced at the primary vertex from those that result from hadronic decays, a relative isolation variable $I^{rel}$ is defined.
This isolation variable is defined as the summed energy of all PF particles within a cone around the PF lepton, with the estimated neutral charged pileup contamination, $\rho$, removed, divided by the lepton \pT.

Two different pileup contamination estimation methods are used in the work presented, one for electrons and the other for muons.

For electrons, the $\rho$ * effective area pileup alleviation technique is used.
This involves the estimating the expected energy within the isolation cone from the median energy density per area of pileup contamination ($\rho$), estimated for each event, with the effective area given as a function of the electron's pseudorapidity, which is expressed as: 

\begin{equation}
I^{rel}_{rho * A_{eff}} = \sum p_T(CH) + max (0.0, \sum E_{\rm T}(NH) + \sum E_{\rm T}(Photon) -rho*A_{\rm eff} )/p_T \\
\end{equation}\label{eq:rhoEffA}

Muons employ the $\Delta\beta$ pileup alleviation technique, where the average pileup energy inside the cone is estimated as half of that originating from charged hadrons.
This is expressed as:

\begin{equation}
I^{rel}_{\Delta\beta} = \sum p_T(CH) + max (0.0, \sum E_{\rm T}(NH) + \sum E_{\rm T}(Photon) - 0.5 * \sum E_{\rm T}(PU))/p_T \\
\end{equation}\label{eq:deltaBeta}

Where CH is the charged hadrons from the Primary Vertex (PV), NH is the neutral hadrons and PU the charged hadrons from pileup.

\subsubsection{Z Boson Candidate Invariant Mass Requirements}
The presence of a Z boson in the final state requires that two leptons selected must be consistent with a Z boson decay, thus requiring them to have the same flavour and opposite charge and an invariant mass within 20\GeVcc of the known Z mass.
This mass window was determined on the basis of including sufficient signal events 

\subsection{Jet, b-tagging and W Boson Candidate Requirements}
\subsubsection{Jet Requirements}
Jets are considered from the PF jet collection which reconstructs jets using the \emph{anti-\kt} algorithm with R = 0.4 with charged hadrons originating from \PU vertices excluded from clustering.
Following identification, the jet energy corrections are applied as described in Chapter~\ref{subsubsec:JECs}.

Jets are considered in the analysis if they have a $\pT > 30\GeVc$, are within $|\eta| < 4.7$ and meet the \emph{loose} working point jet identification criteria developed by CMS.
Additionally, selected leptons (electron or muon) which lie within a cone of $\Delta R = 0.4$ of a selected jet are not considered to be a prompt leptons and instead part of the jet in question.

The loose jet ID was designed to reject the majority of the fake tracks produced from detector and/or electronics noise whilst maintaining a high selection efficiency for real jets by requiring all jets to have part of their energy deposited in both the ECAL and HCAL and be composed of more than one particle.

\editComment{Necessary to spell this out? Or comment out ...}
The loose jet identification criteria are as follows:

for jets with $\eta \leq 2.70$ the loose ID criteria are:
\begin{itemize}
\item the fraction of the jet energy from btoh neutral electromagnetic particles in the ECAL and neutral hadronic particles in the HCAL is less than $0.99$.
\item at least two constituent particles are present.
\end{itemize}

with these applying in addition for $\eta \leq 2.40$:
\begin{itemize}
\item the fraction of the jet energy from charged electromagnetic particles in the ECAL is less than $0.99$ and greater than 0.0 for charged hadronic particles in the HCAL.
\item at least one charged particle is present.
\end{itemize}


for jets with $ 2.70 \leq \eta \leq 3.0$ the loose ID criteria are:
\begin{itemize}
\item the fraction of the jet energy from neutral electromagnetic particles in the ECAL is greater than than $0.01$ and less than $0.98$ for neutral hadronic particles in the HCAL.
\item at least three neutral particles are present.
\end{itemize}

and for jets with $\eta > 3.0$ the loose ID criteria are:
\begin{itemize}
\item the fraction of the jet energy in the ECAL that is from neutral electromagnetic particles is less than $0.90$.
\item at least eleven neutral particles are present.
\end{itemize}

\subsubsection{b-tagging Requirements}
For each jet, the CSVv2 tagging algorithm described in Chapter~\ref{subsec:objReco-bJets}, is evaluated to medium WP

Either one or two of the jets present is required to be b-tagged given that the top quark predominantly decays into a bottom quark and the W boson and recoil quark ... 

\subsubsection{W Boson Candidate Invariant Mass Requirements}
In contrast to the previous tZq searches where the W boson decays into a lepton and its associated antineutrino, the W boson in the dilepton final state decays hadronically, allowing for the top quark to be fully reconstructed.
Each possible pair of jets is considered, with the pair with a dijet invariant mass closest to the W boson mass of 80.4\GeVcc being chosen as the W candidate.
The leading b-jet however, is not excluded from consideration as the hardest b-jet is assumed to be produced from the decay of the top quark. 

\section{Data and MC Simulation Samples}
The analysis uses proton-proton collision data at $\sqrt{13}$ in 2016 ,using 

\section{Simulation Corrections}\label{sec:simCorrections}
As 
\subsection{\PU modelling}
\subsection{•}

\section{Background Estimation}\label{sec:bkgEst}
In any high energy particle physics analysis, accurate modelling of the background processes is essential in order to 
\subsection{Z+jets Background Control Region}\label{subsec:zPlusJetsCR}
Z+jets constitute the largest component of the 

\subsection{\ttbar Background Control Region}\label{subsec:ttbarCR}
The selection criteria for the \ttbar control region differs from the signal region definition defined in Chapter~\ref{sec:signalRegion}, by requiring that the two leptons selected to have different flavours.
... JETS ...
As the branching ratio for 

\subsection{Non-Prompt Leptons}\label{subsec:NPLs}
Backgrounds which involve decays into lepton + jets and where at least one jet is incorrectly reconstructed as a lepton (predominately electrons) or a lepton from the decay of heavy quarks (predominately muons), which pass the lepton selection and isolation criteria, are estimated with data.

The estimation of this background uses the same methodology as when performing top quark pair production~\cite{CMS:2016syx} and same-sign SUSY searches~\cite{CMS:2015vqc}.
The vast majority of the same-sign event yields found are the result of non-lepton and charge misidentified leptons, with some contribution from prompt leptons.
As these backgrounds are independent of the charge of the lepton pairs, it is expected that the nominal (opposite-sign) sample would have a similar contribution \cite{CMS:2015vqc}.

To estimate this contribution of opposite-sign non-prompt leptons in data, the same-sign event yields with the expected prompt-lepton contribution subtracted, is multiplied by a ratio of opposite-sign over same-sign non-prompt lepton events taken from MC.

The method requires that the same-sign control region established uses the same selection criteria as the nominal signal region, albeit with same-sign lepton pairs instead of opposite-sign ones.
This control region is dominated by non-lepton lepton events, but also contains contributions from prompt lepton events, charge misidentification and real same-sign pairs.

This data driven estimate is obtained using the following equation:

\begin{equation}
 N_{data}^{OS non-prompt} = (N_{data}^{SS} - N^{SS}_{real + mis-ID}).\frac{N_{MC}^{OS non-prompt}}{N_{MC}^{SS non-prompt}}
\end{equation}

where $N_{data}^{SS}$ is the total number of same sign events observed in data, $N^{SS}_{real + mis-ID}$ is the expected number of real same-sign events and events with charge misidentification and $N_{MC}^{OS non-prompt}$ and $N_{MC}^{SS non-prompt}$ the number of opposite-sign and same-sign non-prompt leptons observed in MC used to appropriately scale the estimate.

This ratio of MC opposite-sign over same-sign events is referred to as R, and is calculated using generator level information from reconstructed objects which have matched to a generator level particle. R is calculated from the W + jets, \ttZ and \ttW leptonic decaying, and single top MC samples with sufficient statistics given that these processes are expected to be the predominant source of non-prompt leptons for this analysis. 

\begin{table}[!htbp]
\centering
\begin{tabular}{| l |  c |  c |  c |  c |  c |}
\hline
Source &  $ee$ & $\mu\mu$ & Combined \\ 
\hline
\ttbar (SS): & a$\pm$b &  c $\pm$d & e$\pm$f    \\
Z + jets (SS): & a$\pm$b &  c$\pm$d & e$\pm$r    \\
Single Top (SS): & a$\pm$b & c$\pm$d & e$\pm$r    \\
VV (SS): & a$\pm$b & c$\pm$d & e$\pm$f    \\
ttV (SS): & a$\pm$b &  c$\pm$d & e$\pm$f    \\ 
\hline
Total background (SS): & a$\pm$b & c$\pm$d & e$\pm$f   \\ 
Data: & a$\pm$b & c$\pm$d & e$\pm$f    \\ 
\hline
SS data (bkg): & a$\pm$b & c$\pm$d & e$\pm$f \\
\hline
Non-prompt (SS): & a$\pm$b & c$\pm$d & e$\pm$f \\
Non-prompt (OS): & a$\pm$b & c$\pm$d & e$\pm$f \\
R (OS/SS): & a$\pm$b & c$\pm$d & e$\pm$f \\
\hline
Non-prompt estimation: & a$\pm$b & c$\pm$d & e$\pm$f \\
\hline
\end{tabular}
\caption{Non-prompt lepton estimation following all selection cuts}
\label{tab:fakeLeptonYields}
\end{table}
