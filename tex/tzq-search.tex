\chapter{Analysis Strategy and Event Selection}\label{chapter:tzq-search}
The next chapters of this thesis describe the search for the dilepton final state of a singly produced top quark in association with a Z boson (tZq) using the reconstructed proton-proton collision data at $\sqrt{13}$ collected by the CMS detector during 2016.
The main challenge for searching for the dilepton final state is that topology is identical to the final states of other processes with much larger cross sections.
In contrast to previous analyses for tZq, which have searched for the trilepton final state~\cite{Sirunyan:2017kkr,Sirunyan:2017nbr}, where backgrounds



which have focussed on the , where both the Z and W bosons decay leptonically, only the Z boson decays leptonically, with the W boson decaying hadronically.


The presence of only two leptons from a Z boson presents the the main challenge for this analysis - 


As 

The first stage of a	 physics analysis is to determine an event selection based on the topology of the process in order to create a region which is enriched with signal events.
This is 


At leading order, tZq consists single top quark, a radiated Z boson and a recoil quark.

Therefore, the final state contains two leptons from the Z boson decay and at least four jets, 
, a b-jet originating from 


%%% Discuss signal/bkg processes
The single largest background process 


\section{Trigger Strategy}\label{sec:triggerStrategy}
As the search for the dilepton final state for tZq relies on the identification of the two leptons from the Z boson decay, the trigger strategy consists of selecting events from datasets identified by the presence of leptons.
Given that the signal process being searched for is dominated by background processes and will likely be limited by statistics, it is essential to reconstruct and select as many signal events as possible.
To this end both single and double lepton triggers with the lowest possible transverse momenta thresholds are considered 
to ensure that the maximum possible statistics can be obtained over the largest possible phase space.

In addition to the ee and $\mu\mu$ final state for tZq, the muon + electron (MuonEG) dataset, and single electron and muon datasets, are considered for a \ttbar enriched control region discussed in Chapter~\ref{subsec:ttbarCR}. 
This control region uses the e$\mu$ final state to validate the modelling of the \ttbar MC sample used in a region topologically similar to the signal region.

%Whilst the Z boson can decay into a tau anti-tau pair, despite these leptons being simulated and reconstructed in the MC, they are not considered in the event selection for this analysis due to the difficulty simulating them due to their high mass.

\begin{table}[htbp]
\topcaption {
Triggers and datasets used for each decay channel.
}
\label{tab:triggersDatasets}
  \centering
% This increases column spacing.
  \addtolength{\tabcolsep}{1ex}
% This right-aligns numbers in column, but centers them under column title.
  \begin{tabular}{ccr@{\hspace{4ex}}r@{\hspace{4ex}}r@{\hspace{4ex}}}
   \hline
   \bf{Final State} & \bf{Dataset} & \bf{HLT Paths}  \\
   \hline
    ee & DoubleElectron & HLT\_Ele23\_Ele12\_CaloIdL\_TrackIdL\_IsoVL\_DZ \\
    & SingleElectron &  HLT\_Ele32\_eta2p1\_WPTight\_Gsf   \\
   \hline
    $\mu\mu$ & DoubleMuon  & HLT\_Mu17\_TrkIsoVVL\_(Tk)Mu8\_TrkIsoVVL\_DZ \\  
    & SingleMuon &  HLT\_Iso(Tk)Mu24  \\  
   \hline
   e$\mu$ & MuonEG &  HLT\_Mu12\_TrkIsoVVL\_Ele23\_CaloIdL\_TrackIdL\_IsoVL(\_DZ)   \\  
          &        &  HLT\_Mu8\_TrkIsoVVL\_Ele23\_CaloIdL\_TrackIdL\_IsoVL(\_DZ)  \\
    & SingleElectron &  HLT\_Ele32\_eta2p1\_WPTight\_Gsf   \\
    & SingleMuon &  HLT\_Iso(Tk)Mu24  \\  
   \hline
   
 \end{tabular}
 \addtolength{\tabcolsep}{-1ex}
\end{table}

\section{Event Cleaning}\label{sec:metFilters}
Following the trigger requirements, a number of filters are applied in order to so that events with beam or detector anomalies which result in anomalous \MET are not considered for use in the analysis.

\begin{itemize}
\item \textbf{Primary Vertex Filter} - ensures that the primary vertex is well reconstructed by requiring it to be within $|z| \leq 24\cm$ of the interaction point and within $d_{0} < 2\cm$ of the beam line.
\item \textbf{Beam Halo Filter} - beam halos are machine induced particles (\eg beam-gas, beam-pipe interactions) which circulate with the beam at radii up to 5m. The filter removes events with calorimeter and muon chamber energy deposits consistent with either halo particles or particle showers caused by halos interacting with the collimator blocks that used to clean halos from the beam.
\item \textbf{HBHE Noise and Isolation Filters} - removes events where anomalous noise is present in the HCAL's hybrid photodiodes or readout boxes, which registers as large isolated energy deposits which would infer the presence of large \MET, by considering the channel multiplicities, pulse shape of the readout and the neighbouring activity in the calorimeters and tracker.
\item \textbf{ECAL Trigger Primitive Filter} - the L1 trigger primitive readout can be used to estimate the energy deposited in approximately 70\% of the channels which lack regular data links and are masked out for reconstruction. As trigger the primitives have a narrower energy acceptance range than the read-out, when the energy is near their saturation energy the measured energy is likely to be underestimated, resulting in high anomalous \MET. 
%\item \textbf{ECAL Endcap SC Filters} - NOT recommended for 2016
\item \textbf{Bad Charged Hadron Filter} - removes events where a muon is not defined as a PF muon due to its low quality, but makes its way into the PF MET calculation as a charged hadron candidate.
\item \textbf{Bad Muon Filter} - removes events where a muon is defined as a PF muon, but is still has too low a quality and large \pT to be considered.
\end{itemize}

\section{Physics Objects}\label{sec:eventSelection}
The physics objects used in the analysis have been identified and reconstructed using the particle flow algorithm
described in Chapter~\ref{chapter:data-mc}.




\subsection{Lepton Selection}
As the tZq dilepton final state is distinguished by the presence of exactly two leptons from the Z boson decay, being able to identify isolated leptons with a low misidentification rate is essential.


\subsubsection{Electrons}
Given the need to identify isolated electrons produced from the signal process rather than 

Tight ~ 70\% efficiency
Veto ~ 95\% efficiency


All electrons: \pT > 35/15 \GeVc, $\eta < 2.50$, exclude electrons in the EB/EE gap $1.4442 \leq \eta_{SC} \leq 1.566$, EB $d_{0} < 0.05\cm$ and $d_{z} < 0.10\cm$ and EE $d_{0} < 0.10\cm$ and $d_{z} < 0.20\cm$

\subsubsection{Muons}
PF muons 

Tight  
suppress hadronic punch-through into the muon system and non-prompt muons.
\begin{itemize}
\item is a PF Muon and is also both a tracker and global muon.
\item the $\chi^{2}/ndf$ of the global muon track fit is less than ten. 
\item at least one muon chamber is included in the global track fit.
\item that muon segments are found in at least two muon stations.
\item $d_{0} < 0.2\cm$ and $d_{z} < 0.5\cm$.
\item the muon must have at least one hit in the pixel detector.
\item hits must be present in at least six tracker layers in order to achieve a good \pT.
\end{itemize}

Loose muons are defined 
\begin{itemize}
\item is both a PF muon and is also either a tracker or global muon
\end{itemize}

\cite{Chatrchyan:2012xi}


\subsubsection{Lepton Isolation}
\editComment{more detail}
In order to differentiate between leptons promptly produced at the primary vertex from those that result from hadronic decay and to enable a precise momentum measurement by ensuring they are sufficiently separated from hadrons and photons, a relative isolation variable $I^{rel}$ is defined.
The isolation variable is defined as the summed energy of all PF particles within a cone around the PF lepton, with the estimated neutral charged pileup contamination, $\rho$, removed, divided by the lepton \pT.

Two different pileup contamination estimation methods are used in the analysis presented, one for electrons and the other for muons.

For electrons, the $\rho$ * effective area ($rho * A_{eff}$) pileup alleviation technique is used.
This method involves the estimating the expected energy within the isolation cone from the median energy density per area of pileup contamination, estimated for each event, with the effective area given as a function of the electron's pseudorapidity.

This is given as: 

\begin{equation}
I^{rel}_{rho * A_{eff}} = \sum p_T(CH) + max (0.0, \sum E_{\rm T}(NH) + \sum E_{\rm T}(Photon) -rho*A_{\rm eff} )/p_T \\
\end{equation}\label{eq:rhoEffA}

Muons employ the $\Delta\beta$ pileup alleviation technique, where the average pileup energy inside the cone is estimated as half of that originating from charged hadrons.
This is expressed as:

\begin{equation}
I^{rel}_{\Delta\beta} = \sum p_T(CH) + max (0.0, \sum E_{\rm T}(NH) + \sum E_{\rm T}(Photon) - 0.5 * \sum E_{\rm T}(PU))/p_T \\
\end{equation}\label{eq:deltaBeta}

Where CH is the charged hadrons from the Primary Vertex (PV), NH is the neutral hadrons and PU the charged hadrons from pileup.

\subsubsection{Z Boson Candidate Invariant Mass Requirements}
The presence of a Z boson in the final state requires that two leptons selected must be consistent with a Z boson decay, thus requiring them to have the same flavour and opposite charge and an invariant mass within 20\GeVcc of the known Z mass.
This mass window was determined on the basis of including sufficient signal events 

\subsection{Jet, b-tagging and W Boson Candidate Requirements}
\subsubsection{Jet Requirements}
Jets are considered from the PF jet collection which reconstructs jets using the \emph{anti-\kt} algorithm with R = 0.4 with charged hadrons originating from \PU vertices excluded from clustering.
Following identification, the jet energy corrections are applied as described in Chapter~\ref{subsubsec:JECs}.

Jets are considered in the analysis if they have a $\pT > 30\GeVc$, are within $|\eta| < 4.7$ and meet the \emph{loose} working point jet identification criteria developed by CMS.
Additionally, selected leptons (electron or muon) which lie within a cone of $\Delta R = 0.4$ of a selected jet are not considered to be a prompt leptons and instead part of the jet in question.

The loose jet ID was designed to reject the majority of the fake tracks produced from detector and/or electronics noise whilst maintaining a high selection efficiency for real jets by requiring all jets to have part of their energy deposited in both the ECAL and HCAL and be composed of more than one particle.

\editComment{Necessary to spell this out? Or comment out ...}
The loose jet identification criteria are as follows:

for jets with $\eta \leq 2.70$ the loose ID criteria are:
\begin{itemize}
\item the fraction of the jet energy from both neutral electromagnetic particles in the ECAL and neutral hadronic particles in the HCAL is less than $0.99$.
\item at least two constituent particles are present.
\end{itemize}

with these applying in addition for $\eta \leq 2.40$:
\begin{itemize}
\item the fraction of the jet energy from charged electromagnetic particles in the ECAL is less than $0.99$ and greater than 0.0 for charged hadronic particles in the HCAL.
\item at least one charged particle is present.
\end{itemize}


for jets with $ 2.70 \leq \eta \leq 3.0$ the loose ID criteria are:
\begin{itemize}
\item the fraction of the jet energy from neutral electromagnetic particles in the ECAL is greater than than $0.01$ and less than $0.98$ for neutral hadronic particles in the HCAL.
\item at least three neutral particles are present.
\end{itemize}

and for jets with $\eta > 3.0$ the loose ID criteria are:
\begin{itemize}
\item the fraction of the jet energy in the ECAL that is from neutral electromagnetic particles is less than $0.90$.
\item at least eleven neutral particles are present.
\end{itemize}

\subsubsection{b-tagging}
The CSVv2 tagging algorithm described in Chapter~\ref{subsec:objReco-bJets} is used to tag jets, with a working point (WP) cut applied to the b-tag discriminator.
If the value of a jet's discriminator exceeds that of the Medium WP and has $|\eta| < 2.40$, the jet is considered to be a b-jet.
From the \emph{Loose}, \emph{Medium} and \emph{Tight} WPs defined by CMS~\ref{Sirunyan:2017ezt}, as given in Table~\ref{tab:bTagWPs} in Chapter~\ref{subsec:objReco-bJets}, the Medium WP was chosen as it provided the optimum performance in terms of providing as large statistics as possible for the signal process without too great a compromise on the purity of the selection.

\subsubsection{W Boson Candidate Invariant Mass Requirements}
In contrast to the previous tZq searches where the W boson decays into a lepton and its associated antineutrino, the W boson in the dilepton final state decays hadronically, allowing for the top quark to be fully reconstructed.
The W boson candidate is constructed by considering each possible pair of jets, with the pair with a dijet invariant mass closest to the known W boson mass of 80.4\GeVcc being chosen as the W candidate.
The leading b-jet however, is excluded from consideration as the hardest b-jet is assumed to be produced from the decay of the top quark. 

\subsection{\MET}\label{subsec:met}
Whilst the signal region does not explicitly cut on \MET, it is used in one of the Z+jets control regions, described in Chatper~\ref{subsec:zPlusJetsCR}, and in the boosted decision tree, discussed in Chapter~\ref{subsec:bdt}, in order to discriminate against backgrounds such as \ttbar which do feature significant amounts of \MET.

\section{Data and MC Simulation Samples}\label{sec:samples}
The analysis uses proton-proton collision data at $\sqrt{13}$ in 2016, considering events in the double lepton and single lepton datasets where the double and single lepton triggers respectively have fired, using the strategy described in Chapter~\ref{sec:triggerStrategy}.

Table~\ref{tab:mcList} lists the MC samples used to simulate the signal and background processes that were considered.
 with the number of events generated, their cross sections and 
 
 the generator used and the order in perturbative accuracy in QCD was used to produce them.

Additional simulated samples 
The hadronisation of all the MC samples considered was done using PYTHIA 8.


\begin{table}[htbp]
\topcaption {
The MC processes and their associated total number of events, cross sections and generators, considered for the search for tZq in the dilepton final state. Both generators considered for the Z+jet background are also listed below.
}
\label{tab:mcList}
  \centering
  \resizebox{\textwidth}{!}{
% This increases column spacing.
  \addtolength{\tabcolsep}{1ex}
% This right-aligns numbers in column, but centers them under column title.
  \begin{tabular}{ccr@{\hspace{4ex}}r@{\hspace{4ex}}r@{\hspace{4ex}}r@{\hspace{4ex}}}
   \hline
   \bf{MC process} & \bf{Events} & \bf{Cross section (pb)} & \bf{Generator (Order)}   \\
   \hline
   tZq  & 14.5M & 0.0758  & aMC@NLO (NLO) \\
   \hline
   tHq  & 3.5M & 0.07462  & Madgraph (LO) \\
   \hline
   tWZ/tWll  & 50K & 0.01104  & Madgraph (LO) \\
   \hline
   t tW-channel & 7M & 35.85 & POWHEG (NLO) \\
   $\overline{\text{t}}$ tW-channel & 6.9M & 35.85 & POWHEG (NLO) \\
   \hline
   t s-channel & 2.9M & 10.32 & aMC@NLO (NLO) \\
   \hline
   t t-channel & 67.2M & 136.02 & POWHEG (NLO) \\
   $\overline{\text{t}}$ t-channel & 38.8M & 80.95 & POWHEG (NLO) \\
   \hline
   \ttbar & 77.1M & 831.76 & POWHEG (NLO) \\
   \hline
   \ttbarZ $\rightarrow$ ll$\nu\nu$ & 13.9M & 0.2529   & aMC@NLO (NLO) \\
   \ttbarZ $\rightarrow$ qq & 749K & 0.5297   & aMC@NLO (NLO) \\
   \hline
   \ttbarW $\rightarrow$ l$\nu$ & 5.3M & 0.2001   & aMC@NLO (NLO) \\
   \ttbarW $\rightarrow$ qq & 833K & 0.405  & aMC@NLO (NLO) \\
   \hline
   \ttbarH $\rightarrow$ bb & 3.8M & 0.2942 & POWHEG (NLO) \\
           $\rightarrow$ non bb & 4.0M & 0.2123 & POWHEG (NLO) \\
   \hline
   W+jets & 24.1M & 61526.7 & aMC@NLO (NLO) \\
   \hline
   Z+jets ($m_{Z} \geq 50\GeVcc $ & 146M & 5765.4 & Madgraph (LO) \\
   Z+jets ($10 \GeVcc \leq m_{Z} < 50\GeVcc$ & 35.3M & 18610.0 & Madgraph (LO) \\
   \hline
   Z+jets ($m_{Z} \geq 50\GeVcc $ & 151M & 5765.4 & aMC@NLO (NLO) \\
   Z+jets ($10 \GeVcc \leq m_{Z} < 50\GeVcc$ & 106M & 18610.0 & aMC@NLO (NLO) \\
   \hline
   WW $\rightarrow$ l$\nu$qq & 9.0M & 49.997  & POWHEG (NLO) \\
      $\rightarrow$ ll$nu\nu$ & 2.0M & 12.178 & POWHEG (NLO) \\
   \hline
   WZ $\rightarrow$ l$\nu$qq & 24.2M & 10.73 & aMC@NLO (NLO) \\
      $\rightarrow$ llqq & 26.5M & 5.606 & aMC@NLO (NLO) \\
      $\rightarrow$ lll$\nu$ 1.9M & 5.26 & aMC@NLO (NLO) \\
   \hline
   ZZ $\rightarrow$ ll$\nu\nu$ & 8.8M & 0.5644 & POWHEG (NLO) \\
      $\rightarrow$ llqq & 15.3M & 3.222 & aMC@NLO (NLO) \\
      $\rightarrow$ llll & 10.7M & 1.204 & aMC@NLO (NLO) \\
   \hline
   WWW & 240K & 0.2086 & aMC@NLO (NLO) \\
   \hline
   WWZ & 250K & 0.1651 & aMC@NLO (NLO) \\
   \hline
   WZZ & 247K & 0.05565 & aMC@NLO (NLO) \\
   \hline
   ZZZ & 249K & 0.01398 & aMC@NLO (NLO) \\
   \hline
   
 \end{tabular}}
 \addtolength{\tabcolsep}{-1ex}
\end{table}


\begin{table}[htbp]
\topcaption {
The MC processes and their associated total number of events, cross sections and generators, considered for the search for tZq in the dilepton final state.
}
\label{tab:theorySampleList}
  \centering
% This increases column spacing.
  \addtolength{\tabcolsep}{1ex}
% This right-aligns numbers in column, but centers them under column title.
  \begin{tabular}{ccr@{\hspace{4ex}}r@{\hspace{4ex}}r@{\hspace{4ex}}@{\hspace{4ex}}}
   \hline
   \bf{MC process} & \bf{Events} & \bf{Cross section (pb)} & \bf{Generator (Order)}   \\
   \hline
   tZq scale up & 6.9M & 0.0758  & aMC@NLO (NLO) \\
   tZq scale down & 7.0M & 0.0758  & aMC@NLO (NLO) \\
   \hline
   t tW-channel scale up & 998K & 35.85 & POWHEG (NLO) \\
   t tW-channel scale down & 994K & 35.85 & POWHEG (NLO) \\
   $\overline{\text{t}}$ tW-channel scale down & 1.0M & 35.85 & POWHEG (NLO) \\
   $\overline{\text{t}}$ tW-channel scale down & 999K & 35.85 & POWHEG (NLO) \\
   \hline
   t t-channel scale up & 5.7M & 136.02 & POWHEG (NLO) \\
   t t-channel scale down & 5.9M & 136.02 & POWHEG (NLO) \\
   t t-channel matching up & 6.0M & 136.02 & POWHEG (NLO) \\
   t t-channel matching down & 6.0M & 136.02 & POWHEG (NLO) \\
   $\overline{\text{t}}$ t-channel scale up & 4.0M & 80.95 & POWHEG (NLO) \\
   $\overline{\text{t}}$ t-channel scale down & 3.9M & 80.95 & POWHEG (NLO) \\
   $\overline{\text{t}}$ t-channel matching up & 4.0M & 80.95 & POWHEG (NLO) \\
   $\overline{\text{t}}$ t-channel matching down & 4.0M & 80.95 & POWHEG (NLO) \\
   \hline
   \ttbar ISR up & 156.5M & 831.76 & POWHEG (NLO) \\
   \ttbar ISR down & 149.8M & 831.76 & POWHEG (NLO) \\
   \ttbar FSR up & 152.6M & 831.76 & POWHEG (NLO) \\
   \ttbar FSR down & 156.0M & 831.76 & POWHEG (NLO) \\
   \ttbar matching up & 58.9M & 831.76 & POWHEG (NLO) \\
   \ttbar matching down & 58.2M & 831.76 & POWHEG (NLO) \\
   \hline   

 \end{tabular}
 \addtolength{\tabcolsep}{-1ex}
\end{table}

\section{Simulation Corrections}\label{sec:simCorrections}
As 

\subsection{Miscalibrated Tracker APV}
During 2016, 
\subsection{Lepton Reconstruction}
\subsection{•}
\subsection{\PU modelling}

\section{Signal Region}\label{signalRegion}
The signal enriched region event selection is chosen to take advantage of being able to fully reconstruct the W boson, and thus the top quark, in order to 
To reduce 

As shown in Figure~\ref{}, at leading order tZq consists of a top quark, a recoil quark and a radiated Z boson.

\section{Control Regions}\label{sec:controlRegions}
In any high energy particle physics analysis, accurate modelling of the background processes is essential in order to be able to make a measurement.
As the main challenge in searching for the dilepton final state of tZq is that the signal region is dominated by background processes, it particularly important to ensure that the background processes are accurately modelled in simulation.
In order to confirm whether or not a data-driven estimations are required, background enriched control regions which are topologically similar and orthogonal to the signal region are used for the two largest background processes are defined.
The same trigger, event cleaning, number of oppositely charged leptons, number of jets required and W and Z boson mass selection criteria are applied to the control regions so that they occupy a topologically similar phase space to the signal region.
Where required, these control regions are extrapolated to provide a data-driven estimation of the background in the signal region as discussed in Chapter~\ref{sec:dataDrivenBackground}.

\subsection{Z+jets Background Control Region}\label{subsec:zPlusJetsCR}
Despite the majority of such events being rejected by the signal region criteria, given the large cross section for Z+jets, the remaining events from this process dominate the signal region to form the single largest background.
Given the size of this background and the difficulties in accurately modelling higher order contributions from QCD processes, it is essential to ensure that both the normalisation and modelling of the jets for this background are well described.

Two different control regions were explored 
Initially a large statistics 

0-bjets

Later a control region which used the same b-jet selection as the signal region, 

inv w mass cut and met cut

\editComment{Write about which one is used in the fit}

\subsection{\ttbar Background Control Region}\label{subsec:ttbarCR}
The selection criteria for the \ttbar control region differs from the signal region definition defined in Chapter~\ref{sec:eventSelection}, by requiring that the two oppositely leptons selected to have different flavours, \ie one electron and one muon.
As the branching ratio for a W boson (produced by the top quark and anti-top quark decays) to decay into either an electron or muon is the same, this produces an enriched background control region for \ttbar which is topologically similar to the signal region. 

\begin{table}[htbp]
\topcaption {
The event yields after the selection criteria have been applied for the \ttbar control region.
}
\label{tab:ttbarCR}
  \centering
% This increases column spacing.
  \addtolength{\tabcolsep}{1ex}
% This right-aligns numbers in column, but centers them under column title.
  \begin{tabular}{ccr@{\hspace{4ex}}r@{\hspace{4ex}}}
   \hline
   \bf{MC process} & \bf{$e\mu$}  \\
   \hline
   tZq & 1.709\\
   \ttbar & 11778.461 \\
   Z+jets & 80.9921\\
   tW & 488.632\\
   Other & 166.200\\
   \hline   
   Signal
   Background
   \hline
   Data & 12509.0 \\
   Total MC & 12515.995 \\
   \hline
 \end{tabular}
 \addtolength{\tabcolsep}{-1ex}
\end{table}
