\chapter{Event Selection and Background Estimation for Single Top Physics Searches}\label{chapter:tzq-search}
Using the reconstructed high level objects produced
\section{Event Selection}
\subsection{Trigger}
\subsection{Event Cleaning}
Following the trigger requirements, a number of filters are applied to  
\subsection{Lepton Selection}
The final state 
\subsubsection{Electrons}
Electrons
\subsubsection{Muons}
Muons
\subsection{Lepton Isolation}
In order to differentiate between leptons promptly produced at the primary vertex from those that result from hadronic decays, a relative isolation variable $I^{rel}$ is defined.
This isolation variable is defined as the summed energy of all PF particles within a cone around the PF lepton, with the estimated neutral charged pileup contamination, $\rho$, removed, divided by the lepton \pT.

Two different pileup contamination estimation methods are used in the work presented, one for electrons and the other for muons.
For electrons the $\rho$ * effective area pileup alleviation technique is employed.
This involves the estimating the expected energy within the isolation cone from the median energy density per area of pileup contamination ($\rho$), estimated for each event, with the effective area given as a function of the electron's pseudorapidity, which is expressed as: 

\begin{equation}
\begin{align}
I^{rel}_{rho * A_{eff}} = \sum p_T(CH) + max (0.0, \sum E_{\rm T}(NH) + \sum E_{\rm T}(Photon) -rho*A_{\rm eff} )/p_T \\
\end{align}
\end{equation}\label{eq:rhoEffA}

Muons use the $\Delta\beta$ pileup alleviation technique, where the average pileup energy inside the cone is estimated as half of that originating from charged hadrons.
This is expressed as:

\begin{equation}
\begin{align}
I^{rel}_{\Delta\beta} = \sum p_T(CH) + max (0.0, \sum E_{\rm T}(NH) + \sum E_{\rm T}(Photon) - 0.5 * \sum E_{\rm T}(PU))/p_T \\
\end{align}
\end{equation}\label{eq:deltaBeta}

Where CH is the charged hadrons from the Primary Vertex (PV), NH is the neutral hadrons and PU the charged hadrons from pileup.

\subsubsection{Invariant Mass Requirements}
The presence of a Z boson in the final state requires  
Therefore, the two leptons selected must have the same flavour and opposite charge
\subsection{Jet Selection and b-tagging requirements}

\section{Background Estimation}\label{sec:bkgEst}
\subsection{Z+jets Background Estimation}\label{subsec:zPlusJets}
\subsection{\ttbar Background Estimation}\label{subsec:ttbar}
\subsection{Non-Prompt Leptons}\label{subsec:NPLs}
Backgrounds which involve decays into lepton + jets and where at least one jet is incorrectly reconstructed as a lepton (predominately electrons) or a lepton from the decay of heavy quarks (predominately muons), which pass the lepton selection and isolation criteria, are estimated with data.

The estimation of this background uses the same methodology as when performing top quark pair production~\cite{CMS:2016syx} and same-sign SUSY searches~\cite{CMS:2015vqc}.
The vast majority of the same-sign event yields found are the result of non-lepton and charge misidentified leptons, with some contribution from prompt leptons.
As these backgrounds are independent of the charge of the lepton pairs, it is expected that the nominal (opposite-sign) sample would have a similar contribution \cite{CMS:2015vqc}.

To estimate this contribution of opposite-sign non-prompt leptons in data, the same-sign event yields with the expected prompt-lepton contribution subtracted, is multiplied by a ratio of opposite-sign over same-sign non-prompt lepton events taken from MC.

The method requires that the same-sign control region established uses the same selection criteria as the nominal signal region, albeit with same-sign lepton pairs instead of opposite-sign ones.
This control region is dominated by non-lepton lepton events, but also contains contributions from prompt lepton events, charge misidentification and real same-sign pairs.

This data driven estimate is obtained using the following equation:

\begin{equation}
\begin{align}
 N_{data}^{OS non-prompt} = (N_{data}^{SS} - N^{SS}_{real + mis-ID}).\frac{N_{MC}^{OS non-prompt}}{N_{MC}^{SS non-prompt}}
\end{align}
\end{equation}

where $N_{data}^{SS}$ is the total number of same sign events observed in data, $N^{SS}_{real + mis-ID}$ is the expected number of real same-sign events and events with charge misidentification and $N_{MC}^{OS non-prompt}$ and $N_{MC}^{SS non-prompt}$ the number of opposite-sign and same-sign non-prompt leptons observed in MC used to appropriately scale the estimate.

This ratio of MC opposite-sign over same-sign events is referred to as R, and is calculated using generator level information from reconstructed objects which have matched to a generator level particle. R is calculated from the W + jets, \ttZ and \ttW leptonic decaying, and single top MC samples with sufficient statistics given that these processes are expected to be the predominant source of non-prompt leptons for this analysis. 

%\begin{table}[!htbp]
\centering
\begin{tabular}{| l |  c |  c |  c |  c |  c |}
\hline
Source &  $ee$ & $\mu\mu$ & Combined \\ 
\hline
\ttbar (SS): & a$\pm$b &  c $\pm$d & e$\pm$f    \\
Z + jets (SS): & a$\pm$b &  c$\pm$d & e$\pm$r    \\
Single Top (SS): & a$\pm$b & c$\pm$d & e$\pm$r    \\
VV (SS): & a$\pm$b & c$\pm$d & e$\pm$f    \\
ttV (SS): & a$\pm$b &  c$\pm$d & e$\pm$f    \\ 
\hline
Total background (SS): & a$\pm$b & c$\pm$d & e$\pm$f   \\ 
Data: & a$\pm$b & c$\pm$d & e$\pm$f    \\ 
\hline
SS data (bkg): & a$\pm$b & c$\pm$d & e$\pm$f \\
\hline
Non-prompt (SS): & a$\pm$b & c$\pm$d & e$\pm$f \\
Non-prompt (OS): & a$\pm$b & c$\pm$d & e$\pm$f \\
R (OS/SS): & a$\pm$b & c$\pm$d & e$\pm$f \\
\hline
Non-prompt estimation: & a$\pm$b & c$\pm$d & e$\pm$f \\
\hline
\end{tabular}
\caption{Non-prompt lepton estimation following all selection cuts}
\label{tab:fakeLeptonYields}
\end{table}
