\chapter{Analysis Strategy and Event Selection}\label{chapter:tzq-search}
The next chapters of this thesis describe the search for the dilepton final state of a singly produced top quark in association with a Z boson (tZq) using the reconstructed proton-proton collision data at $\sqrt{13}$ collected by the CMS detector during 2016.

In contrast to the previous searches for tZq which focussed on the trilepton final state~\cite{Sirunyan:2017kkr,Sirunyan:2017nbr}, the dilepton final state consists purely of leptons and jets with no \MET.
While this topology allows for the full reconstruction of all the particles involved, including the top quark, it also presents the main challenge for this analysis.

The presence of two leptons and multiple jets is identical to the topologies of the large number of background processes with a multi-jet component with cross sections many orders of magnitude larger than tZq.
Consequently the signal region will inevitably be dominated by such backgrounds regardless of the selection requirements applied to the leptons and jets present.
Therefore, the analysis is designed to ensure that the backgrounds are understood and constrained as far as possible and to have the highest possible acceptance of the signal in order to maximise the ease in measuring the expected tZq contribution.
To further enhance the separation of the signal against the background processes, a multivariate analysis is performed and is described in Chapter~\ref{sec:mvas}.

\section{Background Processes}\label{sec:backgroundProcesses}
\subsection{Vector Boson + Jets backgrounds}
QCD multijet events in association with a vector boson (V+jets) have the largest cross sections of any process which produces prompt leptons at the LHC, as illustrated in Figure~\ref{fig:crossSections}.
As only one promptly produced lepton from the W boson decay is expected for W+jets, the presence of any additional leptons would have to be the result of a heavy quark decay or incorrectly reconstructed jet. 
Requiring any selected lepton to be sufficiently isolated from jets and heavy leptons will suppress the presence of these leptons.

In contrast, while the Z+jets cross section is an order of magnitude smaller than W+jets cross section, the two prompt leptons from the Z boson decay makes it much more difficult to distinguish between this background and the signal.
While the requirement of the presence of at least four jets, with one being a b-jet, will reject the majority of the Z+jets events, the size of the Z+jets cross section ensures that the process will dominate any signal region defined.

\subsection{Top physics backgrounds}
Following the V+jets, \ttbar has the next largest production cross section.
While lepton isolation and jet cleaning will significantly suppress the fully hadronic and semi-leptonic \ttbar final states, the contributions from the dilepton final state where the two leptons pass the signal region criteria will be one of the largest backgrounds.
As the two W boson decay leptonically, the resultant significant quantities of \MET should be able to be used to constrain this final state.

In contrast to the final states of singly produced top quarks in the s- and t-channels, singly produced top quarks in the tW-channel are expected to provide a significant contribution to the background.
Both the leading and next-to-leading order diagrams for the dilepton final state 


While the \ttbar in association with a vector boson processes have cross sections many orders of magnitude smaller than \ttbar, the fully hadronic \ttbar decay
A suitable mass cut on the reconstructed Z boson mass should reject the majority of the dilepton pairs produced

Despite their cross sections being of a similar order of magnitude or smaller than that of the signal process, the similar topology of a number of 

The rare signal top processes of tHq and tWZ have cross sections
\ttZ

\subsection{Multi-boson backgrounds}
Multiple vector bosons produced through electroweak production form the remainder of the background processes.

Where at least one of the bosons is a Z boson which decays leptonically,

As many of the diboson and triboson processes will include the presence of a Z boson decaying leptonically, events which involve the remaining vector bosons decaying hadronically 
Given the small production cross sections of the tribosonic backgrounds, despite the 

\section{Trigger Strategy}\label{sec:triggerStrategy}
As the search for the tZq dilepton final state relies on the identification of the two leptons from the Z boson decay, the trigger strategy consists of selecting events from datasets identified by the presence of leptons.


Given that the signal process being searched for is dominated by background processes and will likely be limited by statistics, it is essential to reconstruct and select as many signal events as possible.
Ideally the single and double lepton triggers with the lowest possible transverse momenta thresholds would be considered to ensure that the maximum possible statistics can be obtained over the largest possible phase space.
The high instantaneous luminosity at the start of the most luminous data taking periods however, required a number of the L1 triggers to be prescaled to prevent the trigger bandwidth constraints being exceeded.
As only the triggers considered for the ee channel were impacted by this, these triggers were chosen on the basis of largest amount data using unprescaled triggers with the lowest possible transverse momenta thresholds.

Table~\ref{tab:triggersDatasets} lists the triggers applied to data and MC events for each channel, including the e$mu$ final state which is considered for a \ttbar enriched control region discussed in Chapter~\ref{subsec:ttbarCR}.
%Whilst the Z boson can decay into a tau anti-tau pair, despite these leptons being simulated and reconstructed in the MC, they are not considered in the event selection for this analysis due to the difficulty simulating them due to their high mass.

\begin{table}[htbp]
\topcaption {
Triggers and datasets used for each decay channel.
}
\label{tab:triggersDatasets}
  \centering
   \resizebox{\textwidth}{!}{
   \begin{tabular}{ccc}
   \hline
   \textbf{Final State} & \textbf{Dataset} & \textbf{HLT Paths}  \\
   \hline
    ee & DoubleElectron & HLT\_Ele23\_Ele12\_CaloIdL\_TrackIdL\_IsoVL\_DZ \\
    & SingleElectron &  HLT\_Ele32\_eta2p1\_WPTight\_Gsf   \\
   \hline
    $\mu\mu$ & DoubleMuon  & HLT\_Mu17\_TrkIsoVVL\_(Tk)Mu8\_TrkIsoVVL\_DZ \\  
    & SingleMuon &  HLT\_Iso(Tk)Mu24  \\  
   \hline
   e$\mu$ & MuonEG &  HLT\_Mu12\_TrkIsoVVL\_Ele23\_CaloIdL\_TrackIdL\_IsoVL(\_DZ)   \\  
          &        &  HLT\_Mu8\_TrkIsoVVL\_Ele23\_CaloIdL\_TrackIdL\_IsoVL(\_DZ)  \\
          &        &  HLT\_Mu23\_TrkIsoVVL\_Ele12\_CaloIdL\_TrackIdL\_IsoVL(\_DZ) \\
    & SingleElectron &  HLT\_Ele32\_eta2p1\_WPTight\_Gsf   \\
    & SingleMuon &  HLT\_Iso(Tk)Mu24  \\  
   \hline
 \end{tabular}}
\end{table}

\section{Event Cleaning}\label{sec:metFilters}
Following the trigger requirements, a number of filters are applied so that events with beam or detector anomalies which result in anomalous \MET are not considered for use in the analysis:

\begin{itemize}
\item \textbf{Primary Vertex Filter} - ensures that the primary vertex is well reconstructed by requiring it to be within $|z| \leq 24\cm$ of the interaction point and within $d_{0} < 2\cm$ of the beam line.
\item \textbf{Beam Halo Filter} - beam halos are machine induced particles (\eg beam-gas, beam-pipe interactions) which circulate with the beam at radii up to 5m. The filter removes events with calorimeter and muon chamber energy deposits consistent with either halo particles or particle showers caused by halos interacting with the collimator blocks that used to clean halos from the beam.
\item \textbf{HBHE Noise and Isolation Filters} - removes events where anomalous noise is present in the HCAL's hybrid photodiodes or readout boxes, which registers as large isolated energy deposits which would infer the presence of large \MET, by considering the channel multiplicities, pulse shape of the readout and the neighbouring activity in the calorimeters and tracker.
\item \textbf{ECAL Trigger Primitive Filter} - the L1 trigger primitive readout can be used to estimate the energy deposited in approximately 70\% of the channels which lack regular data links and are masked out for reconstruction. As trigger the primitives have a narrower energy acceptance range than the read-out, when the energy is near their saturation energy the measured energy is likely to be underestimated, resulting in high anomalous \MET. 
%\item \textbf{ECAL Endcap SC Filters} - NOT recommended for 2016
\item \textbf{Bad Charged Hadron Filter} - removes events where a muon is not defined as a PF muon due to its low quality, but makes its way into the PF MET calculation as a charged hadron candidate.
\item \textbf{Bad Muon Filter} - removes events where a muon is defined as a PF muon, but is still has too low a quality and large \pT to be considered.
\end{itemize}

\section{Physics Objects}\label{sec:physicsObjects}
In order to meet the analysis strategy's criteria for a signal enriched region and background enriched control regions, a number of event selection criteria are applied to the physics objects that have been identified and reconstructed using the particle flow algorithm described in Chapter~\ref{chapter:data-mc}.
The following section describes the selection criteria used in this analysis and how the final state products are used to reconstruct their mother particles.

\subsection{Lepton Selection}
All PF electrons and muons considered must pass a set of kinematic requirements and a set identification criteria defined by CMS.
The kinematic requirements are applied to ensure that the leptons lie within detector acceptance and that their transverse momenta lies in the region where the trigger's turn-on is well described.
The identification criteria have been designed to be efficient at selecting isolated prompt leptons and rejecting leptons which have been produced non-promptly from decays from within jets or taus or from incorrectly reconstructed tracks.

Different working points are used by identification criteria defined by CMS, where the lepton selection efficiency is traded off against the purity of the leptons selected.
For both lepton flavours the ``tightest'' working point is used to select high purity collections of leptons, with the ``loosest'' working point used to veto events with any additional leptons.

\subsubsection{Electrons}\label{subsubsec:electronSelection}
For any PF electron candidate to be considered it must meet the following kinematic requirements:

\begin{itemize}
\item the \pt of the leading and subleading electrons considered must be greater than the 35\GeVc and 15\GeVc respectively.
\item electrons must have $|\eta| \leq 2.50$ to ensure that the electrons are within the ECAL acceptance.
\item as accurate reconstruction cannot be undertaken in the transition region between the ECAL barrel and endcap, electrons with $1.4442 \leq \eta \leq 1.566$ are not considered.
\item the longitudinal impact parameter, $d_{z}$, of the electron must be less than 0.10\cm in the barrel and 0.20\cm in the endcap disks.
\item the transverse impact parameter, $d_{0}$, of the electron must be less than 0.05\cm in the barrel and 0.10\cm in the endcap disks.
\end{itemize}

The \emph{tight} and \emph{veto} working points (WPs) of the \emph{cut based} identification criteria which are approximately  70\% and 95\% efficient respectively, are used to select electrons and to veto any additional electrons.
The cut based identification uses a mixture of manually set and multivariate analysis (MVA) tuned variables for btoh the barrel and endcap disks.

The manually set variables are:
\begin{itemize}
\item \textbf{$N^{missing}_{inner hits}$} - as photons which subsequently convert do not leave hits in the innermost layers of the tracker, electrons are rejected if the expected number of missing hits is exceeded.
\item \textbf{a conversion veto} - is applied for all working points, where any electron which fails the electron conversion veto is rejected.
\end{itemize}

The MVA tuned variables include:
\begin{itemize}
\item \textbf{Full $5 \times 5 \sigma_{i\eta i\eta}$} - the shower shape variable, which describes the shape of the shower in $\eta$.
\item \textbf{$\Delta \eta_{seed}$ and $\Delta \phi_{seed}$} - the distances in $\eta$ and $\phi$ between the ECAL supercluster and where the track has been extrapolated to from the primary vertex.
\item \textbf{$\frac{h}{E}$} - the ratio of hadronic to electromagnetic energy deposited in the supercluster around the crystal with the largest energy deposit.
\item \textbf{$I^{rel}_{EA}$} the relative isolation of the electron with effective area pileup alleviation for a cone size of 0.3, which is described further in Chapter~\ref{subsubsec:relIso}.
\item \textbf{$1/E - 1/p$} - the difference in the inverse energy of the ECAL supercluster and inverse track momentum, which is used to describe the energy loss 
\end{itemize}

The cuts used for the tight and veto WPs for these variables are given in Table~\ref{tab:electronCuts}.

\begin{table}[htbp]
\topcaption {
The cuts used for the tight and veto working points of the cut based identification criteria for electrons for the barrel and endcap disks.
}
\label{tab:electronCuts}
  \centering
  \resizebox{\textwidth}{!}{
% This right-aligns numbers in column, but centers them under column title.
 \begin{tabular}{ccccc}
   \hline
   \textbf{Variable} & \multicolumn{2}{c}{\textbf{Tight WP}} & \multicolumn{2}{c}{\textbf{Veto WP}}   \\
    & Barrel & Endcap & Barrel & Endcap \\
    \hline   
    Full $5\times5 \sigma_{i\eta i\eta}$ & $< 0.00998$ & $< 0.0292$ & $< 0.0115$ & $< 0.037$ \\
    $\Delta \eta_{seed}$ & $<0.00308$ & $<0.00605$ & $<0.00749$ & $<0.00895$ \\
    $\Delta \phi_{seed}$ & $<0.0816$ & $<0.0394$ & $<0.228$ & $<0.213$ \\
    $\frac{h}{E}$ & $<0.0414$ & $<0.0641$ & $<0.356$ & $<0.211$	\\
    $I^{rel}_{EA}$ & $<0.0588$ & $<0.0571$ & $<0.175$ & $<0.159$ \\
    $1/E - 1/p$ & $<0.0129$ & $<0.0129$ & $<0.299$ & $<0.15$ \\
    $N^{missing}_{inner hits}$ & $\leq 1$ & $\leq 1$ & $\leq 2$ & $\leq 3$ \\
    pass conversion veto & $\checkmark$ & $\checkmark$ & $\checkmark$ & $\checkmark$ \\
    \hline
 \end{tabular}}
\end{table}

\subsubsection{Muons}\label{subsubsec:muonSelection}
Similar to the electrons, PF muons are required to meet a set of kinematic requirements and identification and isolation criteria.

In the case of the kinematic requirements, PF muons candidates are required to:
\begin{itemize}
\item the \pt of the leading and subleading electrons considered must be greater than the 26\GeVc and 20\GeVc respectively.
\item muons must have $|\eta| \leq 2.50$ to ensure that the muon are within the acceptance of the muon systems.
\end{itemize}

The \emph{tight} and \emph{loose} identification and isolation criteria~\cite{Chatrchyan:2012xi} are used to select muons and veto any additional muons.

The tight muon criteria suppresses hadronic punch-through into the muon system and non-prompt muons, creating a high purity collection of particle flow muons.

These criteria are:
\begin{itemize}
\item muon a PF Muon and is also both a tracker and global muon.
\item $\chi^{2}/ndf$ of the global muon track fit is less than ten. 
\item at least one muon chamber is included in the global track fit.
\item that muon segments are found in at least two muon stations.
\item $d_{0} < 0.2\cm$ and $d_{z} < 0.5\cm$.
\item the muon must have at least one hit in the pixel detector.
\item hits must be present in at least six tracker layers in order to achieve a good \pT measurement.
\end{itemize}

The tight isolation cut applied to the resultant collection of tight muons is 95\% efficient, and rejects muons that have a relative isolation, with $\Delta\beta$ pileup corrections, greater than 0.15 for a cone size of 0.4.
This pileup correction for the relative isolation is described further in Chapter~\ref{subsubsec:relIso}.

Given that the loose cuts require the muon to be a particle flow muon and either a global muon or tracker muon, by definition all PF muons considered pass the loose identification cut.
The loose isolation cut is 98\% efficient and rejects muons with a relative isolation which is greater than 0.25.

\subsubsection{Lepton Isolation}\label{subsubsec:relIso}
A relative isolation variable $I^{rel}$ is used in order to:
\begin{itemize}
\item differentiate between leptons promptly produced at the primary vertex from those resulting from heavy jet or lepton decays.
\item to ensure that leptons are sufficiently separated from hadrons and photons to enable a precise momentum measurement of the lepton 
\end{itemize}

$I^{rel}$e is defined as the summed energy of all PF particles within a fixed radius cone of $\Delta R$ around the PF lepton, with the estimated neutral charged pileup contamination, $\rho$, removed, divided by the lepton \pT.

As only charged hadrons ($CH$) have associated tracks which can be used to determine if they are consistent with the primary vertex, the pileup contamination contribution from neutral hadrons ($NH$) and photons is typically estimated with one of two methods.

In the analysis presented, electrons use the $\rho$ * effective area ($rho * A_{\rm eff}$) technique using a $\Delta R$ of 0.3.
This method estimates the neutral pileup contributions by subtracting the median energy density per area of pileup contamination, $\rho$, which has been multiplied by the effective area of the electron, $A_{\rm eff}$, which is characterised as a function of the supercluster's $\eta$:

\begin{equation}
I^{rel}_{rho * A_{eff}} = \sum p_T(CH) + max (0.0, \sum E_{\rm T}(NH) + \sum E_{\rm T}(Photon) -rho*A_{\rm eff} )/p_T \\
\end{equation}\label{eq:rhoEffA}

The $\Delta\beta$ pileup mitigation method is used for muons using a $\Delta R$ of 0.4 in the analysis presented.
Using the fact that the ratio of neutral to charged hadron production in the hadronisation of pileup interactions is approximately 0.5, half of the transverse energy of charged hadrons from pileup is subtracted from the neutral hadron and photon transverse energies~\cite{Chatrchyan:2012vp}:

\begin{equation}
I^{rel}_{\Delta\beta} = \sum p_T(CH) + max (0.0, \sum E_{\rm T}(NH) + \sum E_{\rm T}(Photon) - 0.5 * \sum E_{\rm T}(PU))/p_T \\
\end{equation}\label{eq:deltaBeta}

\subsubsection{Z Boson Candidate}
The presence of a Z boson in the final state requires that two leptons selected must be consistent with a Z boson decay.
Therefore, the leptons must have the same flavour and opposite charge and an invariant mass which is consistent with the known Z boson mass.
%It was determined that the invariant mass of the two leptons had to be within $\20\GeVcc$ of the known Z boson mass on the basis of ensuring sufficient signal events and sufficient backgrounds for the MVA (discussed in 
%This mass window was determined on the basis of 
%including sufficient signal events 

\subsection{Jet, b-tagging, W Boson and Top Quark Candidate Requirements}
\subsubsection{Jet Requirements}
Jets are considered from the PF jet collection which reconstructs jets using the \emph{anti-\kt} algorithm with R = 0.4 with charged hadrons originating from \PU vertices excluded from clustering.
Following identification, the jet energy corrections are applied as described in Chapter~\ref{subsubsec:JECs}.

Jets are considered in the analysis if they have a $\pT > 30\GeVc$, are within $|\eta| < 4.7$ and meet the \emph{loose} working point jet identification criteria developed by CMS.
In addition, selected leptons (electron or muon) which lie within a cone of $\Delta R = 0.4$ of a selected jet are not considered to be a prompt leptons and instead part of the jet in question.

The loose jet ID was designed to reject the majority of the fake tracks produced from detector and/or electronics noise while maintaining a high selection efficiency for real jets by requiring all jets to have part of their energy deposited in both the ECAL and HCAL and be composed of more than one particle.

The loose jet identification criteria are as follows:

for jets with $\eta \leq 2.70$ the loose ID criteria are:
\begin{itemize}
\item the fraction of the jet energy from both neutral electromagnetic particles in the ECAL and neutral hadronic particles in the HCAL is less than $0.99$.
\item at least two constituent particles are present.
\end{itemize}

With these additional criteria applying for jets for $\eta \leq 2.40$:
\begin{itemize}
\item the fraction of the jet energy from charged electromagnetic particles in the ECAL is less than $0.99$ and greater than 0.0 for charged hadronic particles in the HCAL.
\item at least one charged particle is present.
\end{itemize}

For jets with $ 2.70 \leq \eta \leq 3.0$ the loose ID criteria are:
\begin{itemize}
\item the fraction of the jet energy from neutral electromagnetic particles in the ECAL is greater than than $0.01$ and less than $0.98$ for neutral hadronic particles in the HCAL.
\item at least three neutral particles are present.
\end{itemize}

And for jets with $\eta > 3.0$ the loose ID criteria are:
\begin{itemize}
\item the fraction of the jet energy in the ECAL that is from neutral electromagnetic particles is less than $0.90$.
\item at least eleven neutral particles are present.
\end{itemize}

\subsubsection{b-tagging Requirements}
The CSVv2 tagging algorithm described in Chapter~\ref{subsec:objReco-bJets} is used to tag jets, with a working point (WP) cut applied to the b-tag discriminator.
If the value of a jet's discriminator exceeds that of the Medium WP and has $|\eta| < 2.40$, the jet is considered to be a b-jet.
From the \emph{Loose}, \emph{Medium} and \emph{Tight} WPs defined by CMS~\cite{Sirunyan:2017ezt}, as given in Table~\ref{tab:bTagWPs} in Chapter~\ref{subsec:objReco-bJets}, the Medium WP was chosen as it provided the optimum performance in terms of providing as large statistics as possible for the signal process without too great a compromise on the purity of the selection.

\subsubsection{W Boson Candidate}
In contrast to the previous tZq searches where the W boson decays into a lepton and its associated antineutrino, the W boson in the dilepton final state decays hadronically, allowing for the top quark to be fully reconstructed.
The W boson candidate is constructed by considering each possible pair of jets, with the pair with a dijet invariant mass closest to the known W boson mass of 80.4\GeVcc being chosen as the W candidate.
%Additionally, the invariant mass of the two selected jets was required to be within $20\GeVcc$ of the known W boson mass.

The leading b-jet however, is not considered to have been produced by the W boson decay as the hardest b-jet is assumed to be produced from the decay of the top quark.

\subsubsection{Top Quark Candidate}
As there is no \MET present in the dilepton final state it is possible to fully reconstruct the top quark from the leading b-jet and the W boson candidate.
While no additional event selection cuts are applied to the top quark candidate, the various quantities that could be constructed from it were used by the multivariate analysis discussed in Chapter~\ref{sec:mvas}.

\subsection{\MET}\label{subsec:met}
Whilst the signal region does not explicitly cut on \MET, it is used in one of the Z+jets control regions, described in Chapter~\ref{subsec:zPlusJetsCR}, and by the multivariate analysis discussed in Chapter~\ref{sec:mvas}, in order to discriminate against backgrounds such as \ttbar which do feature significant amounts of \MET.

\section{Signal Region}\label{sec:signalRegion}
At leading order tZq consists of a top quark, a recoil quark and a radiated Z boson, as illustrated in Figure~\ref{fig:feyn_tZq}.
Therefore the event selection for the final state must consist of two leptons compatible with a Z boson decay and at least four jets, one from the top quark decay, two from the W boson, and the recoil jet.

For any lepton to be considered for selection, it must be produced within the ECAL's acceptance of $|\eta| < 2.5$ and have a transverse momenta in a region where the lepton triggers are well described in simulation.
Electrons are required to have a transverse momenta greater than $35\GeVc$ and $15\GeVc$ for the leading and sub-leading electrons respectively, while transverse momenta greater than $26\GeVc$ and $20\GeVc$ are required for the leading and sub-leading muons.
Exactly two leptons passing the tight identification and isolation criteria with no additional veto leptons present in the event ensures low lepton misidentification and non-prompt lepton acceptance rates and a high rejection efficiency of events containing a differing number of leptons of either flavour.
In order to be consistent with being produced by a Z boson, both of the selected leptons must be of the same flavour, have opposite charges and and have a compatible invariant mass.
It was found that requiring the leptons' invariant mass to be within $20\GeV$ of the known Z boson mass sufficiently accounted for detector resolution effects and included sufficient background events that could be used in the later multivariate analysis.

At least four jets will be present, one from the recoil quark, one from the top quark's decay and two from the decay of the W boson produced by the decaying top quark.
Additional jets can be produced by gluon splitting from by Initial State Radiation (ISR) or Final State Radiation (FSR).
With the decreasing probability of these higher order contributions occurring and the increasing difficulty in accurately simulating these higher multiplicity events however, the maximum number of jets considered is limited to six.
Therefore four to six jets are required to be present and each must have a $\pT > 30\GeVc$ and $|\eta| < 4.7$ and pass the loose jet ID to ensure a high selection efficiency while rejecting the majority of the fake jets.

Given the top quark having a near 100\% probability of decaying into a bottom quark, the event selection requires at least one of the selected jets to be b-tagged by the CSVv2 tagging algorithm at the medium WP.
Additional b-jets may also be present in the final state from either the W boson decay or from the bottom quark in the initial state that produces the top quark if it originates from gluon splitting.
Considering the increased difficulties in modelling multiple b-jets and decreasing event yields as their multiplicity increases, either one or two of the jets found are required to be b-tagged.

With all the jets identified, the W boson candidate constructed from the two jets with the closest invariant mass to the known W boson mass is considered.
To be consistent with a W boson decay, the invariant mass of the two jets is required to be within $20\GeVcc$ of the known W boson mass.
As additional b-jets will be softer than the one produced from the top quark decay, the leading b-jet is not considered to have been produced from the W boson decay.


Therefore the complete event selection for the signal region was chosen to be:

\begin{itemize}
\item Exactly two same flavour and opposite sign electrons or muons which pass the tight identification and isolation cuts. The leading and sub-leading electrons must have a $\pT > 35\GeVc (26\GeVc)$ and be within $|\eta| < 2.50$. The leading and sub-leading muons $\pT > 15\GeVc (20\GeVc)$ respectively and be within $|\eta| < 2.0$.
\item No additional electrons or muons which pass the same kinematic cuts and the veto or loose identification and isolation cuts respectively. 
\item The invariant mass of the two selected leptons must be within $20\GeVcc$ of the known Z boson mass.
\item Four to six jets which pass the loose jet ID requirements and have a $\pt > 30\GeVc$ and are within $|\eta| < 4.7$.
\item One or two of the selected jets are considered to be b-tagged by the CSVv2 tagging algorithm at the medium WP.
\item The invariant mass of the two jets which are closest to the known W boson mass must be within $20\GeVcc$. The leading b-jet is not considered to have originated from the W boson decay.
\end{itemize}

No \MET cut was applied in the signal region event selection despite there being no \MET directly produced by the signal process.
This decision was taken as \MET was anticipated to be a useful variable the multivariate analysis (discussed in Chapter~\ref{sec:mvas}) that could be used to discriminate against background processes such as \ttbar.

\section{Control Regions}\label{sec:controlRegions}
In any high energy particle physics analysis, accurate modelling of the background processes is essential in order to be able to make a measurement.
Given that the main challenge in the search for the dilepton final state of tZq is that the signal region will be dominated by background processes, it is especially important to ensure that the background processes are accurately modelled in simulation if any meaningful measurement is to be made.

In order to confirm whether or not simulation adequately describes the data and thus if data-driven estimations are required instead, background enriched control regions which are topologically similar and orthogonal to the signal region are used for the two largest background processes are defined.
The same trigger, event cleaning, and baseline event selection of two oppositely charged leptons, number of jets required and W and Z boson mass selection criteria are applied to the control regions so that they occupy a topologically similar phase space to the signal region.
Where required, these control regions would be extrapolated to provide a data-driven estimation of the background in the signal region as discussed in Chapter~\ref{sec:dataDrivenBackground}.

\subsubsection{Z+jets Background Control Regions}\label{subsec:zPlusJetsCR}
Despite the majority of the Z+jet contribution being rejected by the signal region's jet and b-jet criteria, due to the large cross section of the process the surviving events dominate the signal region to form the single largest background.
Given the scale of this background and the difficulties in accurately modelling higher order contributions from QCD processes, it is essential to ensure that both the normalisation and modelling of the jets for this background are well described.

Initially a high statistics Z+jets enriched control region was defined by requiring that none of the jets present are b-tagged, in contrast to the one to two required for the signal region.
Given the large cross section and that the vast majority of the jets produced by the Z+jets process are light jets, whilst the top quarks in \ttbar, the second largest background, predominately decay into b quarks, this produces a high purity region with large statistics.

Despite the good description of the jet multiplicity and jet \pT shapes in this control region, discussed in depth in Chapter~\ref{subsec:zPlusJetsEstimation}, as the kinematics of b-tagged jets differs from light jets, this control region's topology may not be similar 

Therefore an alternative Z+jets enriched control region was defined by requiring same b-jet selection (one to two b-jets) as the signal region, but requiring an inverted W boson mass cut and a \MET cut of 50\GeV.
The inverted W boson mass cut rejects events containing a hadronically decaying W boson and the \MET cut suppresses \ttbar which contains significant quantities of \MET from the leptonically decaying W bosons.

\subsubsection{\ttbar Background Control Region}\label{subsec:ttbarCR}
\ttbar events form the second largest background process, where events with two leptons produced from W bosons decaying having an invariant mass which is compatible with the Z boson mass window in the being selected in the signal region.
The selection criteria for the \ttbar control region differs from the signal region definition defined in Chapter~\ref{sec:signalRegion}, by requiring that:
\begin{itemize}
\item the two oppositely charged leptons selected to have different flavours (\ie one is an electron and the other a muon).
\item a \pt cut of x and y is used for the electron and muon respectively, as the leading lepton \pt thresholds of the HLT paths considered for the e$\mu$ final state differ from those for the same flavour final states, as as given in Chapter~\ref{sec:triggerStrategy}. 
\end{itemize} 

As the branching ratio for a W boson (produced by the top quark and anti-top quark decays) to decay into either an electron or muon is the same, this produces a \ttbar enriched background control region which is topologically similar to the \ttbar contributions in the signal region. 

\section{Experimental Blinding}\label{sec:blinding}
Despite even the best intentions, there is the potential for an experiment to be inadvertently optimised based on previous observations~\cite{Roodman:2003rw}.
In order to prevent any such unintentional biasing of the result, experiments are ``blinded'' so that the result is not known until optimisation based off simulation or data outside of the signal region has been completed.

To this end a $\chi^{2}$-like variable in the two dimensional space of the reconstructed top quark and W boson masses was constructed, as defined in Equation~\ref{eq:blindingChi2}:

\begin{equation}\label{eq:blindingChi2}
   \chi^{2} = {\left(\frac{m_{\mathrm{jj+b}}-m_{\mathrm{t}}}{\sigma_{\mathrm{t}}}\right)}^{2} + {\left(\frac{m_{\rm jj}-m_{\rm W}}{\sigma_{\mathrm{W}}}\right)}^{2}
\end{equation}

where $m_{\rm jj}$ is the W boson mass reconstructed from the two candidate jets, $m_{\rm W}$ is the known W boson mass, $m_{\mathrm{jj+b}}$ is the top quark mass reconstructed from the leading b-jet and two jets associated with the W boson decay, $m_{\mathrm{t}}$ is the known top mass, and $\sigma_{\mathrm{t}}$ and $\sigma_{\mathrm{W}})$ are the reconstruction resolution terms for the top quark and W boson masses.
Both $\sigma_{\mathrm{t}}$ and $\sigma_{\mathrm{W}}$ were determined \editComment{HOW?}

Using the simulated samples, cut on this $\chi^{2}$-like variable were chosen to define a blinded signal region which balanced containing the majority of the signal process while minimising background contamination and a \emph{side band} region which contains signal-like events and rejects any extreme outliers.

These cuts were determined to be ... \editComment{add new/current cuts}
