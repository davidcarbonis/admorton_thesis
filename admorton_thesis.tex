\documentclass[11pt,a4paper]{report}
\usepackage{thesis}

\pdfoutput=1

\setcounter{secnumdepth}{3}

% Put here some packages required or/and some personnal commands
\usepackage{ptdr-definitions}

\usepackage{calc}
\usepackage{textcomp}
\usepackage{amsmath}
\usepackage{amssymb}
\usepackage{graphicx}
\usepackage{enumerate}
\usepackage{xspace}
\usepackage{topcapt}
\usepackage{lineno}
\usepackage{subfig}
\usepackage{mathtools}
\usepackage[usenames,dvipsnames,svgnames,table]{xcolor}
\usepackage{longtable}
\usepackage{appendix}
\usepackage{multirow}
\usepackage{pifont}% http://ctan.org/pkg/pifont
\usepackage{slashed}
\usepackage{bookmark}
\usepackage{bm}

\newcommand{\cmark}{\ding{51}}%
\newcommand{\xmark}{\ding{55}}%

% Local definitions.
\newcommand{\editComment}[1]{{\color{Red} #1}} % Indicates who is organising each section.

\newcommand{\ttbar}{\ensuremath{t\overline{t}}\xspace} % t-tbar
\newcommand{\ttZ}{$\text{t}\overline{\text{t}}Z$}
\newcommand{\ttbarZ}{$\text{t}\overline{\text{t}}Z$}
\newcommand{\ttW}{$\text{t}\overline{\text{t}}W$}
\newcommand{\ttbarW}{$\text{t}\overline{\text{t}}W$}
\newcommand{\ttH}{$\text{t}\overline{\text{t}}H$}
\newcommand{\ttbarH}{$\text{t}\overline{\text{t}}H$}
\newcommand{\ttbarGamma}{$\text{t}\overline{\text{t}}\gamma$}
\newcommand{\ttbarY}{$\text{t}\overline{\text{t}}\gamma$}
\newcommand{\ttY}{$\text{t}\overline{\text{t}}\gamma$}
\newcommand{\ttV}{$\text{t}\overline{\text{t}}V$}
\newcommand{\ttbarV}{$\text{t}\overline{\text{t}}V$}
 
% units
\newcommand{\T}{\unit{T}}
\newcommand{\ms}{\unit{ms}}
\newcommand{\ns}{\unit{ns}}
\newcommand{\mrad}{\unit{mrad}}
\newcommand{\Gbps}{\unit{Gb\hspace{-0.16em}/\hspace{-0.08em}s}}
\newcommand{\Tbps}{\unit{Tb\hspace{-0.16em}/\hspace{-0.08em}s}}
\newcommand{\Hz}{\unit{Hz}}
\newcommand{\kHz}{\unit{kHz}}
\newcommand{\MHz}{\unit{MHz}}
\newcommand{\GHz}{\unit{GHz}}
 
% Don't use ``c'' in GeV units, to be consistent with TDR.
\renewcommand{\GeVc}{\GeV}
\renewcommand{\GeVcc}{\GeV}

% Command definitions.
\newcommand{\pT}{\ensuremath{p_{\mathrm{T}}}\xspace}
\newcommand{\sector}{sector\xspace}
\newcommand{\segment}{\sector}
\newcommand{\qpt}{\ensuremath{q\hspace{-0.08em}/\hspace{-0.08em}\pt}\xspace}
\newcommand{\rphi}{$r$-$\varphi$\xspace}
\newcommand{\rz}{$r$-$z$\xspace}
%\newcommand{\PU}{pile-up\xspace}
\newcommand{\PU}{PU\xspace} % PU is now defined in Section 1
\newcommand{\OT}{outer tracker\xspace}
\renewcommand{\HT}{Hough Transform\xspace} % redefinition of CMS command for total transverse energy ...
\newcommand{\KF}{Kalman Filter\xspace}
\newcommand{\LR}{Linear Regression\xspace}
\renewcommand{\DR}{Duplicate Removal\xspace}
\newcommand{\MS}{multiple scattering\xspace}
\newcommand{\mat}[1]{\mathbf{#1}}
\newcommand{\combine}{\texttt{combine}}

\linespread{1.25} %% 1.5 line spacing as nominal is 1.2, ergo 1.2*1.25=1.5
\linenumbers


%%%%%%%%%%%%%%%  Title page %%%%%%%%%%%%%%%%%%%%%%%% 

% >> Title: please make sure that the non-TeX equivalent is in PDFTitle below
\begin{document}
\title{
Search for the Production of a Single Top Quark in association with a Z Boson at the LHC 
}

\author{Alexander D. J. Morton}

\maketitle

\abstract{
A search for the production of a single top quark in association with a Z boson and an additional jet using data from proton-proton collisions at $\sqrt{s} = 13\TeV$ collected by the Compact Muon Solenoid (CMS) experiment at the Large Hadron Collider is presented.
This is a rare process that is predicted by the Standard Model 
This search focussed on identifying the final state containing two leptons from the Z boson decay, two jets from the decay of the W boson produced by the top quark decay, a b-jet from the top quark decay and a recoil jet.
The signal was dominated by backgrounds involving a real Z boson or two promptly produced leptons consist with a Z boson decay, primarily Z+jet and top quark pair production.
As such, a Boosted Decision Tree was used to enhance the separation between the signal and background processes.
Using a dataset corresponding to 35.8\fbinv, a limit on the production cross section has been set at $X_{Y}^{Z}$ fb.
This corresponds to an OBSERVED/EXPECTED signal significance of $999 \sigma$ when compared to the background only hypothesis, which is consistent with the Standard Model prediction and the limits set by searches for the final state where the Z boson and W boson decay leptonically.

The CMS experiment's new tracking detector at the High-Luminosity Large Hadron Collider will require the ability to reconstruct all charged tracks with transverse momentum greater than 2-3\GeV within 4\mus so that they can be used in the Level-1 trigger decision.
One of the proposed track finders is an FPGA-based based solution using a fully time-multiplexed architecture, where track candidates are reconstructed using a projective binning algorithm based on the Hough Transform.
Studies into the suitability of a linearised $\chi^{2}$ algorithm for fitting track parameters were undertaken and it was found that its performance was inferior compared to that of a combinatorial \KF fitter.
The impact of reducing the minimum track transverse momentum from 3\GeV to 2\GeV on the proposed system was also evaluated.
The resulting degradation of of performance was found to be recoverable by improving the handling of multiple scattering in the track finding and fitting algorithms.
}

\chapter*{Declaration of Authorship} \label{sec:declaration}
The work described in this thesis was conducted solely by the author, except where collaboration with others occurred as stated within the text, during their time as a candidate for a research degree at this University.
No work contained within this thesis has not been submitted to this or any other university as part of the requirement for any qualification.
When the published work of others has been consulted, it has been clearly attributed within the text.

The work described in Chapter~\ref{chapter:tk-upgrade} formed part of the author's contributions to the development of the proposed track finding system that is described within that chapter.
To produce the results presented in this chapter, the author worked in collaboration with other members of the 

he work described in Chapter~\ref{chapter:results} was 

Figures from CMS publications are labelled ``CMS'' or``CMS Simulatio'', when the data are only taken from simulation.

\clearpage
\newpage

\chapter*{Acknowledgements} \label{sec:acknowledgments}
This thesis would not have been possible without the time, help, thoughts, and advice of many people.

First and foremost, I would like to thank my supervisor \textit{Joanne Cole} for her continual support, guidance and advice over the course of the last four years.
I am not sure how I can ever express my gratitude for her ability to always find the time for me and for the uncountable number of corrections she made to this thesis.

I would also like to thank \textit{Catherine Mackay} and \textit{Corin Hoad} in particular for their work on the statistical analysis of the tZq analysis presented in this thesis.
Without both of their efforts I doubt I would have survived 
My thanks go to \textit{Ivan Reid} whose 


I would like to thank all those I've worked with throughout the UK CMS Collaboration and during my time at CERN.

Specifically, for 
\textit{Andrew Rose}
\textit{Mark Pesaresi}
\textit{Geoffrey Hall}
\textit{Seth Zenz} recieves

I would like to thank all my friends and family for their support through the years.
To \textit{Darije \v{C}ustovi{c}}, \textit{Diana Lucaci}, \textit{Reuben Hill}, \textit{Thore Bucking}, and \textit{Annabel Shaw}; for not only occasionally providing me with a place to stay in London, but your 

Finally, but most importantly, I would like to thank \textit{Peter Hobson} and the Science and Technologies Facilities Council.
Without them, I would not have had the privilege of being able to embark on this work. 

\tableofcontents
\listoftables
\listoffigures

\chapter{Introduction}\label{chapter:intro}

\emph{``If I have seen further it is by standing on the shoulders of giants''}
\emph{Letter to Robert Hooke FRS, February 15th 1676, by Sir Isaac Newton FRS (1643-1727)}

The idea that nature can be explained through rational explanations, such as the ancient philosophical concepts of \emph{Atomism}and the Ancient Greek's \emph{Classical Elements}, is one that stretches back into time immemorial.

Following scientific revolution of the 17th century the scientific method replaced such philosophical reasoning as the basis of exploring the nature of reality.
By formulating hypotheses whose predictions can tested by empirical evidence, successive generations of scientists
have built upon and improved on the ideas of those before them.
By amending existing theories or proposing new theories supported by new and more precise measurements, unified descriptions of seemingly unrelated phenomena have emerged, such as James Clark Maxwell's theory of electromagnetism.
This process has taken us from John Dalton's atomic theory and Sir Isaac Newton's Laws of Motion to the Standard Model (SM) of Particle Physics in the present day, describing all known elementary particles and three of the four fundamental forces of nature.


The SM has been one of the greatest and most powerful scientific theories, making remarkably accurate predictions which have withstood incredible experimental scrutiny.
Despite the completion of the SM with the discovery of the Higgs Boson in 2012~\cite{HiggsCMS,HiggsATLAS} at the Large Hadron Collider (LHC), it is clear that the Standard Model cannot be a complete description of reality:
\begin{itemize}
\item Gravity is not accounted for by the quantum field theory of the SM and at high energy densities it is fundamentally irreconcilable with the classical theory of General Relativity~\cite{}.
\item There is strong experimental evidence that the observed galaxy rotation curves and gravitational lensing cannot be accounted for by SM particles and that there must be a large weakly interacting \emph{Dark Matter} component of the Universe~\cite{Bertone:2004pz}.
\item The presence of a \emph{Dark Energy} energy has also been inferred from astronomical and cosmological observations to account for the observed rate of expansion of the Universe~\cite{Peebles:2002gy}.
\item Neutrinos have been observed to oscillate between different flavours, implying that they have non-zero masses in contrast to the SM~\cite{Fukuda:1998mi,Ahmad:2001an}.
\item There is currently no explanation that accounts for the clear abundance of matter over anti-matter in the observable universe.
\item On top of the abovce inconsistencies, many scientists are uncomfortable with the fact that the SM contains a large number of finely tuned experimentally derived parameters and hope that these values would emerge naturally from a more ``complete'' description of reality~\cite{•}.
\end{itemize}

One of the approaches to attempt to answer these questions is to investigate increasingly higher energy scales to both make precise measurements and rare tests of our existing theories and to look for new physics beyond them.
The LHC at CERN is the most powerful and luminous particle accelerator and collider built to date and provides physicists the capability to study an unprecedented number of events involving the heaviest known fundamental particle, the top quark.
Many of the top quark's properties, stemming from a mass near the electroweak symmetry breaking scale and a lifetime shorter than the strong force’s characteristic time, have no equivalent for the other five quarks.
As such, study of the top quark allows for unique opportunities to probe the weak force and the nature of the individual quarks.

This thesis presents a search for a predicted but undiscovered singly produced top quark process and a number of the contributions towards a study considering a potential future particle detector upgrade.

The analysis presented looks for, and makes a cross section measurement of, a single top quark which is produced in association with a Z boson in the final state involving two leptons using proton-proton collision data at $\sqrt{13}$ collected by the Compact Muon Solenoid (CMS) at the LHC during 2016.
This process has been predicted by the Standard Model but has yet to be measured given both its rarity and similarity to more commonly produced background processes.
As the process involves the Z boson coupling between both the top quark and W boson, it is a particularly sensitive probe for any new physics in the electroweak sector.

The CMS detector at the High Luminosity LHC will require a track finding system to provide information to the trigger system in order to discriminate in favour of potentially interesting physics against increasingly large backgrounds.
During the development of one of the proposed track finding systems, studies were undertaken regarding various track fitting algorithms which would fit precise track helix parameters to the tracks found and the ability of the system to find tracks with $\pT > 3\GeVc$.
In this thesis the studies concerning the development of a Linearised $\chi^{2}$ fitter and the ability of the proposed system to find tracks with $\pT > 2GeVc$ are presented.


\editComment{add references for SM inconsisitencies}

\chapter{An Introduction to the Standard Model and Top Quark Physics}\label{chapter:theory}
\section{The Standard Model}\label{sec:sm}
The Standard Model (SM) of particle physics describes all the known elementary matter particles and their interactions with the weak, strong, and electromagnetic forces using renormalisable Quantum Field Theory (QFT).
QFT describes particles as excitations of quantum fields, whose dynamics are typically described using the Lagrangian formalism~\cite{LagrangiansSM}.

This chapter introduces and briefly describes the theoretical framework of the SM, the shortcomings of the SM and the physics of the top quark.
The second section of the chapter discusses the motivations and context of the search for a single top quark produced in association with a Z boson presented in this thesis.

Throughout this thesis \emph{natural units}, where the fundamental constants $c$, $\hbar$ and $k_{B}$ (Boltzmann constant) are set to unity, and Einstein's summation convention are used.

\subsection{Fundamental Particles}\label{subsec:particles}
The SM describes all matter as being made up of spin-$\frac{1}{2}$ particles known as fermions that interact through the fundamental forces, which are mediated by spin-$1$ gauge bosons.
The spin-$0$ Higgs boson arises as a consequence of electroweak symmetry breaking as a means to imbue the fermions and weak force gauge bosons with mass.

Matter consists of six quarks, fundamental particles that interact through the strong, electromagnetic and weak forces, and six leptons, fundamental particles that experience the electromagnetic and weak forces~\cite{LagrangiansSM}.
Each fermion has an associated anti-matter equivalent, which has identical mass but opposite charge.
Both types of fermion are subdivided into three ``generations'' of particles where each subsequent generation of particles is identical, except for their quantum number and mass~\cite{ElectroweakStrong}.
Table~\ref{tab:fermions} lists the charges, weak isospins and masses of the quarks and leptons for each of the three generations.

The ``up-type'' and ``down-type'' quarks have an electrical charges of $+\frac{2}{3}$ and $-\frac{1}{3}$,  respectively, and \emph{colour charges} (or anti-colour charges) of red, blue or green.
As the phenomena of \emph{colour confinement} (described in Section~\ref{subsec:QCD}) only allows for colourless states, quarks form composite particles collectively called hadrons.
Typically hadrons are composed of a quark anti-quark pair, known as mesons, or of groups of three quarks, referred to as baryons.
Exotic hadrons formed of larger groupings of quarks can be also formed, with both tetraquark and pentaquark states having been observed by the LHCb detector~\cite{Aaij:2014jqa,Aaij:2015tga} and elsewhere~\cite{Tanabashi:2018oca}.

Each generation of leptons consists of a charged lepton that interacts through the electromagnetic and weak forces, and a corresponding neutral near massless lepton, known as a neutrino, that interacts solely through the weak force.
As with the quarks, the charged lepton of each subsequent generation is more massive than the last.
Initially, it was assumed that neutrinos were massless, but the discovery of neutrino flavour oscillation implies that they must have non-zero masses. 
The hierarchy of the neutrino mass eigenstates is currently unknown~\cite{Nath:2018rqn}.

\begin{table}[htbp]
\topcaption {
The Standard Model fermions and their properties~\cite{Tanabashi:2018oca}.
}
\label{tab:fermions}
  \centering
  \resizebox{\textwidth}{!}{
% This right-aligns numbers in column, but centers them under column title.
 \begin{tabular}{llllccc}
   \hline
   & \textbf{Generation} & \textbf{Particle} & \textbf{Mass [\MeV]} & \textbf{Electric Charge} & \textbf{Weak Isospin}\\
   \hline
   \multirow{3}{*}{Quarks}  & \multirow{2}{*}{I} & up (\textit{$u$})  & $2.2^{+0.5}_{-0.4}$ & $+ \frac{2}{3}$ & $+ \frac{1}{2}$ \\
   & & down (\textit{$d$}) & $4.8^{+0.5}_{-0.3}$ & $- \frac{1}{3}$ & $- \frac{1}{2}$ \\
   & \multirow{2}{*}{II} & charm (\textit{$c$})  & $1.275^{+0.025}_{-0.035} \times 10^{3}$ & $+ \frac{2}{3}$ & $+ \frac{1}{2}$ \\
   & & strange (\textit{$s$})  & $95^{+9}_{-3}$	 & $- \frac{1}{3}$ & $- \frac{1}{2}$ \\
   & \multirow{2}{*}{II} & top (\textit{$t$})  & $(173.1 \pm 0.9) \times 10^{3}$ & $+ \frac{2}{3}$ & $+ \frac{1}{2}$ \\
   & & bottom (\textit{$b$})  & $(4.18^{+0.04}_{0.03}) \times 10^{3}$ & $- \frac{1}{3}$ & $- \frac{1}{2}$ \\
   \hline
   \multirow{3}{*}{Leptons}  & \multirow{2}{*}{I} & electron (\textit{$e$})  & $0.511$ & $-1$ & $- \frac{1}{2}$ \\
   & & electron neutrino (\textit{$\nu_{e}$})  & $< 2 \times 10^{-6}$ & $0$ & $+ \frac{1}{2}$ \\
   & \multirow{2}{*}{II} & muon (\textit{$\mu$})  & $106$ & $-1$ &  $-\frac{1}{2}$ \\
   & & muon neutrino (\textit{$\nu_{\mu}$})  & $< 0.19$ & $0$ & $+ \frac{1}{2}$ \\
   & \multirow{2}{*}{II} & tau (\textit{$\tau$})  & $1777$ & $0$ & $- \frac{1}{2}$ \\
   & & tau neutrino (\textit{$\nu_{\tau}$})  & $<18.2$ & $0$ & $+ \frac{1}{2}$ \\   
   \hline   
 \end{tabular}}
\end{table}

The SM contains five integer spin gauge bosons, shown in table~\ref{tab:bosons}, along with their corresponding masses, charges, and weak isospins.
The four spin-$1$ vector bosons mediate the electromagnetic, weak and strong forces.
The massless photon, $\gamma$, mediates the electromagnetic force, while the massive neutral $Z^0$ and charged  $W^\pm$ bosons mediate the weak force.
Massless gluons mediate the strong force and have one of eight colour states~\cite{LagrangiansSM}. 
The spin-$0$ Higgs boson accounts for fundamental particles acquiring mass.

\begin{table}[htbp]
\topcaption {
The fundamental forces of nature and the SM bosons which mediate them~\cite{Tanabashi:2018oca}.
}
\label{tab:bosons}
  \centering
  \resizebox{\textwidth}{!}{
% This right-aligns numbers in column, but centers them under column title.
 \begin{tabular}{lcccc}
   \hline
   Bosons & Mass [\GeV] & Electrical Charge & Colour Charge & Weak Isospin \\
   \hline
   Photon ($\gamma$) & $0$ & $0$ & $0$ & $0$ \\
   \hline
   W ($\text{W}^{\pm}$) & $80.385 \pm 0.015$ & $\pm 1$ & $0$ & $\pm 1$ \\
   Z ($\text{Z}^0$) & $91.1876 \pm 0.0021$ & $0$ & $0$ & $0$ \\
   Higgs ($\text{h}^{0}$) & $125 \pm 0.24$ & $0$ & $0$ & $- \frac{1}{2}$ \\
   \hline
   \multirow{2}{*}{Gluon ($g$)} & \multirow{2}{*}{$0$} & \multirow{2}{*}{$0$} & \multirow{2}{*}{$0$} & $r \overline{g}$, $r \overline{b}, g \overline{r}, g \overline{b}, b \overline{r}, b \overline{g} $ \\
   & & & &  $\frac{1}{\sqrt{2}}(r \overline{r} - g \overline{g}), \frac{1}{\sqrt{6}}(r \overline{r} + g \overline{g} - 2 b \overline{b})$ \\
   \hline   
 \end{tabular}}
\end{table}	

\subsection{Gauge Symmetries}\label{subsec:gaugeSymmetries}
The idea that the laws of physics are consistent for all observers, even if the measurements differ between observers, is a fundamental component of all modern physical theories~\cite{Haywood}.
Systems that are unchanged or \emph{invariant} under a given transformation are considered to possess a corresponding \emph{symmetry}.

As shown by Noether's theorem, the generator(s) of any such symmetry conserve a corresponding quantity~\cite{Noether:1918zz}.
Examples of such quantities include the conservation of energy-momentum from space-time symmetry or electrical charge from the $U(1)$ symmetry in electromagnetism.
If a symmetry transformation has no space-time dependence it is said to have a \emph{global symmetry} and conversely, if it has a space-time dependence it is said to have a \emph{local} or \emph{gauge symmetry}~\cite{Cheng:1985bj}.

These concepts can be demonstrated by considering applying the $U(1)$ gauge symmetry of Quantum Electrodyanmics, the theory of electromagnetism, to the Lagrangian of a relativistic spin-$\frac{1}{2}$ free-fermion field (\eg electrons) with a wavefunction $\psi(x)$ and mass $m$~\cite{QFT}:

\begin{equation}
\mathcal{L} = \bar{\psi}(x) (i {\gamma}^{\mu} \partial_{\mu} - m) \psi(x) \;
\label{eq:diracLagrangian}
\end{equation}

where $\partial_{\mu}$ is the partial derivative operator ${\gamma}^{\mu}$ are the Dirac matrices, defined in Appendix~\ref{app:maths}.

If we consider this Lagrangian to have a global $U(1)$ symmetry, then $\psi(x)$ transforms as:

\begin{equation}
\psi(x) \rightarrow \psi'(x) = e^{-i q \alpha} \psi(x) \\
\label{eq:globalTransformation}
\end{equation}

which leaves the Lagrangian in Equation~(\ref{eq:diracLagrangian}) unchanged as $q$ is a constant and $\alpha$ is an arbitrary phase.

If Equation~(\ref{eq:diracLagrangian}) has a local $U(1)$ symmetry, then $\psi(x)$ transforms according to:

\begin{equation}
\psi(x) \rightarrow \psi'(x) = \psi(x) e^{-i q \alpha (x) } \;
\label{eq:localTransformation}
\end{equation}

As such a local transformation involves $\alpha$ being dependent on $x$, the derivative term in Equation~(\ref{eq:diracLagrangian}) now transforms as:

\begin{equation}
\begin{split}
\bar{\psi}(x) \partial_{\mu} \psi(x) \rightarrow \bar{\psi}'(x) \partial_{\mu} \psi'(x) &= \bar{\psi} (x) e^{i q \alpha (x) } \partial_{\mu} \big(e^{-i q \alpha (x) } \psi(x) \big) \\
&= \bar{\psi} (x) \partial_{\mu} \psi(x) - i \bar{\psi} (x) \partial_{\mu} \alpha (x) \psi(x) \\
\end{split}
\label{eq:derivativeLocalTransformation}
\end{equation}

which consequently results in the Lagrangian no longer being invariant:

\begin{equation}
\mathcal{L} \rightarrow \mathcal{L}' = \mathcal{L} + \bar{\psi}(x) \big( i {\gamma}^{\mu} \partial_{\mu} \alpha(x) \big) \psi(x) \;
\label{eq:localLagrangian}
\end{equation}

For the Lagrangian to remain invariant under local transformations, a vector or \emph{gauge} field, $A_{\mu}(x)$, associated with the $\psi(x)$ field, can be introduced, which transforms as follows:

\begin{equation}
A_{\mu}(x) \rightarrow A_{\mu}(x)' + \frac{1}{q} \partial_{\mu} \alpha(x) \;
\label{eq:vectorField}
\end{equation}

This field can be simply introduced by replacing the derivative $\partial_{\mu}$ with the \emph{gauge covariant derivative}~\cite{QFT}, which is defined as $D_{\mu} = \partial_{\mu} - i A_{\mu}(x)$.
As $D_{\mu}$ transforms as:

\begin{equation}
D_{\mu}\psi(x) \rightarrow e^{-i q \alpha (x) } D_{\mu} \psi(x) \;
\label{eq:Dtransform}
\end{equation}

the non-invariant term in Equation~(\ref{eq:localLagrangian}) cancels out and ensures that the Lagrangian remains invariant under the local $U(1)$ gauge transformations.

The presence of this gauge field allows for the inclusion of a gauge invariant term containing a field strength tensor $F_{\mu \nu}$, that describes the geometry of $A_{\mu}(x)$, in the Lagrangian.
The general form of $F_{\mu \nu}$ is given by:

\begin{equation}
F^{a}_{\mu \nu} = \partial_{\mu} A^{a}_{\nu} - \partial_{\nu} A^{a}_{\mu} + g f^{abc} A^{b}_{\mu} A^{c}_{\nu} \;
\label{eq:fieldStrengthTensor}
\end{equation}

where $g$ is the self-coupling constant and $f^{abc}$ are the structure constants of the symmetry group.

For the case of QED, as $U(1)$ has only one generator, which self-commutes, $g$ is zero.

Therefore, with the addition of the simplest gauge invariant term for incorporating $F_{\mu \nu}$, the QED Lagrangian is given by:

\begin{equation}
\mathcal{L}_{QED} = \bar{\psi} (i {\gamma}^{\mu} D_{\mu} - m) \psi - \frac{1}{4} F_{\mu \nu} F^{\mu \nu} \;
\label{eq:qedLagrangian}
\end{equation}

where excitations of the gauge field $A_{\mu}$ correspond to the massless photon and $q$ represents the electric charge of the electron. 

Similarly, by requiring the SM Lagrangian to be gauge invariant under the $SU(3)_{C}$ gauge symmetry of the strong force and under the $SU(2)_{L} \times U(1)_{Y}$ gauge symmetry of the electroweak force, the gauge fields and their associated gauge bosons for the electromagnetic, weak and strong forces naturally emerge.

The resultant SM Lagrangian is constructed of four terms:

\begin{equation}
\mathcal{L}_{SM} = \mathcal{L}_{Gauge} + \mathcal{L}_{Fermion} + \mathcal{L}_{Higgs} + \mathcal{L}_{Yukawa} \;
\end{equation}

where $\mathcal{L}_{Gauge}$ describes the spin-$1$ gauge boson fields that arise from requiring that the Lagrangian is invariant under local transformations of the symmetry group and $\mathcal{L}_{Fermion}$ describes the fermion fields.
The $\mathcal{L}_{Higgs}$ and $\mathcal{L}_{Yukawa}$ terms arise as a consequence of the breaking of electroweak symmetry and describe the scalar spin-0 Higgs field and the interactions between the Higgs field and fermions and gauge bosons, respectively.

\subsection{Electroweak Theory}\label{subsec:QED}
\subsubsection{Quantum Electrodynamics}\label{subsec:QED}
Quantum Electrodynamics (QED) is the Abelian gauge theory that describes how the electromagnetic force interacts with electrically charged particles.
QED is based on the $U(1)_{EM}$ gauge group, which describes the conservation of electrical charge, \emph{q}, and the mediation of the force by the massless and chargeless photon.
The massless nature of the photon results in the electromagnetic force having an infinite range. 

In contrast to the featureless void of the classical vacuum, in QFT the vacuum is the ground state of the quantum field.
Given that neither the position nor the momentum of the photon field can be precisely known as a consequence of Heisenberg's uncertainty principle, the field experiences random fluctuations.
These fluctuations are interpreted as virtual electron-anti-electron pairs that are continually materalising out of the vacuum before annihilating~\cite{coughlan2006ideas}.
During their brief existence, these virtual electrons and anti-electrons interact with the electromagnetic fields of real particles - being attracted to oppositely signed and repelled by same signed particles.
This results in the vacuum acting as a dielectric medium which partially screens the strength of a charged particle's field.
At shorter distances however, the effective strength of a charged particle's field increases as the impact of screening is reduced.

\subsubsection{Weak Interactions}\label{subsec:weakForce}
The weak force acts upon \emph{weak isospin}, $T$, and is mediated by the massive electrically charged W$^{\pm}$ and electrically neutral Z$^{0}$ gauge bosons~\cite{ElectroweakStrong}.
The weak force conserves weak isospin along the z-axis, $T_{3}$.

Given that the chirality of a fermion determines the value of $T_{3}$, W$^{\pm}$ bosons, which have $T_{3} = \pm 1$, can only interact with left-handed fermions, which have $T_{3} = \pm \frac{1}{2}$~\cite{Cheng:1985bj}.
This property makes charged weak interactions unique in being the only interactions during which fermion flavour can change and violate parity (P)~\cite{Lee:1956qn,Wu:1957my} and charge-parity (CP) symmetries~\cite{Christenson:1964fg}.
The violation of CP symmetry results in weak interactions involving matter and anti-matter occurring at different rates.
Such processes have a bias towards matter production, which partially accounts for the observed matter-anti-matter asymmetry in the universe.
As Z$^{0}$ bosons have $T_{3} = 0$, they interact with both left and right handed fermions and conserve fermion flavour and CP symmetry.

\subsubsection{Electroweak Unification}\label{subsec:electroweak}
%%% Intro
%The electromagnetic and weak forces are described in the SM as a unified \emph{electroweak force}.
The theory of \emph{electroweak} interactions, formulated by Glashow, Salam and Weinberg~\cite{Glashow:1961tr,Salam:1964ry,Weinberg:1967tq}, is described by the $SU(2)_{L} \times U(1)_{Y}$ gauge group and hypothesises that its two seemingly disparate constituent forces - weak and electromagnetic - are described as a single unified electroweak force above some threshold energy.


The $U(1)_{Y}$ component of the theory has a single generator and an associated gauge field B$_{\mu}$ with coupling constant $g'$.
This field acts on, and conserves, weak hypercharge, $Y_{W}$, which is related to electrical charge, $Q$, and the z-projection of weak isospin, $T_{3}$, by $Q = T_{3} + \frac{1}{2} Y_{W}$.

The $SU(2)_{L}$ component of the theory has three generators, ${T_{i}} = \frac{{\sigma_{i}}}{2}$, which manifest as the gauge fields W$_{\mu}^{i}$ with coupling constant $g$, where $i = 1, 2, 3$ and $\bm{\sigma}$ are the Pauli spin matrices (defined in Appendix~\ref{app:maths}).
As $SU(2)$ transformations are non-Abelian, W$_{\mu}^{i}$ are able to interact with themselves.


The B$_{\mu}$ and W$_{\mu}^{i}$ gauge fields are related to the four physically observed gauge bosons as follows:

\begin{equation}
\begin{split}
\text{A}_{\mu} = sin(\theta_{W}) W^{3}_{\mu} + cos(\theta_{W}) B_{\mu} \\ 
\text{Z}_{\mu} = cos(\theta_{W}) W^{3}_{\mu} - sin(\theta_{W}) B_{\mu} \\ 
\text{W}^{\pm}_{\mu} = \frac{1}{\sqrt{2}} ( W^{1}_{\mu} \mp i W^{2}_{\mu} ) \;
\end{split}
\label{eq:ewBosons}
\end{equation}

where $\theta_{W}$ is the weak mixing or \emph{Weinberg} angle, which is defined as:

\begin{equation}
\theta_{W} = \frac{g}{g^{2}+g'^{2}} \;
\label{eq:weakMixingAngle}
\end{equation}

The W$^{\pm}$ gauge bosons only interact with the left-handed components of the fermion field, $\psi_{L}$.
The left- and right-handed components of the fermion field $\psi$ are obtained using the projection operators, $P_{L/R}$, as follows:

\begin{equation}
\psi_{L/R} = P_{L/R} \psi \;
\end{equation}

where $P_{L/R} = \frac{1}{2}(1 \pm \gamma^{5})$ and $\gamma^{5} =i  \gamma^{0} \gamma^{1} \gamma^{2} \gamma^{3}$. 


Under the $SU(2)_{L}$ group of transformations, $\psi_{L}$ transforms as doublets, and $\psi_{R}$ as a singlet.
The $\psi_{L}$ doublet consists of either a left-handed pair of up-type and down-type quarks of the same generation or a charged lepton and its associated neutrino.
As no right-handed neutrinos have been observed, the $\psi_{R}$ singlet consists of a right-handed up- or down-type quark or a charged lepton.

%The left- and right-handed components of the fields are as follows:

%\begin{equation}
%\psi_{L} = 
%\begin{pmatrix} 
%u_{L} \\
%d_{L} 
%\end{pmatrix}
%,
%\begin{pmatrix} 
%l_{L} \\
%\nu_{L} 
%\end{pmatrix}
%;
%\psi_{R} = u_{R}, d_{R}, l_{R} \;
%\label{eq:weakDoublet}
%\end{equation}

As the weak flavour eigenstates of the down-type quarks do not coincide with their mass eigenstates, charged weak interactions allow for flavour changing interactions.
The Cabibbo-Kobayashi-Maskawa (CKM) matrix, in Equation~(\ref{eq:ckm}), is a unitary matrix that describes the proportion of the mass eigenstates $d$, $s$, and $b$ that are present in the weak flavour eigenstates $d'$, $s'$, and $b'$~\cite{Tanabashi:2018oca}.
The individual elements of the CKM matrix describe the strength of the couplings for charged weak interactions.

\begin{equation}
\begin{pmatrix} 
d' \\
s' \\
b'
\end{pmatrix}
=
\begin{pmatrix} 
V_{ud} & V_{us} & V_{ub} \\
V_{cd} & V_{cs} & V_{cb} \\
V_{td} & V_{ts} & V_{tb}
\end{pmatrix}
\begin{pmatrix} 
d \\
s \\
b
\end{pmatrix}
\label{eq:ckm}
\end{equation}

The current best estimates of the elements of the CKM matrix, which have been determined by a global fit of the measurements various experiments have performed, are~\cite{Tanabashi:2018oca}.

\begin{equation}
V_{CKM} = 
\begin{pmatrix} 
0.97420 \pm 0.00021 & 0.2243 \pm 0.0005 & 0.00394 \pm 0.00036 \\
0.218 \pm 0.004 &  0.997 \pm 0.017 & 0.0422 \pm 0.0008 \\
0.0081 \pm 0.0005 & 0.0394 \pm 0.0023 & 1.019 \pm 0.025
\end{pmatrix}
\label{eq:ckmElements}
\end{equation}

%%%Spontaneous symmetry breaking
\subsubsection{Spontaneous symmetry breaking}\label{subsec:higgs}
Originally, the SM lacked a mechanism to include massive gauge fields in its Lagrangian, without breaking the gauge invariance of weak isospin rotations~\cite{Griffiths}.
The inclusion of gauge invariant mass terms in the electroweak Lagrangian was achieved through the spontaneous symmetry breaking \emph{Higgs mechanism} proposed by Brout, Engler, Higgs, Guralnik, Hagen and Kibble~\cite{Englert:1964et,Higgs:1964pj,Guralnik:1964eu}.

The Higgs mechanism introduces a complex scalar field with four degrees of freedom, $\phi$.
As the potential of $\phi$, $V(\phi)$, has a non-zero vacuum expectation value (\textit{VEV}), it has an infinite number of ground states.
Figure~\ref{fig:higgsPotential} shows that while $V(\phi)$ is symmetrical, the rotational symmetry of the field is spontaneously broken when a single ground state for the vacuum is chosen.

\begin{figure}[htbp]
\begin{center}
\includegraphics[width=0.97\textwidth]{figs/sm/higgspotential.png}
\caption{The potential of the Higgs field as a function of its real and imaginary components~\cite{Ellis:2013jnq}. The potential's infinite minima form a circle in phase space.}
\label{fig:higgsPotential}
\end{center}
\end{figure}

Through the spontaneous symmetry breaking of the Higgs potential, three of the four degrees of freedom of the Higgs field couple to and provide mass terms for the weak gauge bosons.
The remaining degree of freedom manifests as a single massive scalar field excitatation known as the \emph{Higgs boson}~\cite{Cheng:1985bj}.
Both the CMS and ATLAS experiments at CERN have independently confirmed the existence of a Higgs boson with a mass of $125 \pm 0.24 \GeV$~\cite{HiggsCMS,HiggsATLAS}. 

While the introduction of a Higgs field was motivated to explain the broken electroweak symmetry, it has allowed for of gauge invariant \emph{Yukawa} mass terms for fermions to be added to the SM Lagrangian.
In these terms, the strength of a fermion's Yukawa coupling to the Higgs results in the fermions gaining a non-zero mass~\cite{Cheng:1985bj}.
The experimental evidence for the Higgs coupling to fermions include the recent observations of \ttH~production~\cite{Sirunyan:2018hoz} and of the Higgs boson decaying a $\tau \overline{\tau}$ pair~\cite{CERN-EP-2018-221} and $b \overline{b}$ pairs~\cite{Sirunyan:2017guj}.

\subsection{Quantum Chromodynamics}\label{subsec:QCD}
The strong force and its interactions is described by Quantum Chromodynamics (QCD).
QCD is based on the non-Abelian $SU(3)_{colour}$ gauge group, which describes strong interactions through eight massless spin-$1$ gauge bosons called \emph{gluons} that act upon the \emph{colour} charge, \emph{C}, carried by quarks~\cite{ElectroweakStrong}.
Quarks carry either a red, green or blue colour charge, with anti-quarks possessing equivalent anti-colour charges.
Given the non-Abelian nature of QCD, gluons can self-couple as they themselves carry both a colour and and anti-colour charge, unlike the photon, for example, which is electrically neutral.

The self-coupling nature of gluons results in the phenomenon known as \emph{asymptotic freedom}~\cite{ElectroweakStrong,coughlan2006ideas,devenish2004deep}, whereby the strength of the strong coupling constant, $\alpha_{s}$, decreases with distance (increasing momenta).
This occurs as, like the QED vacuum of a sea of virtual $e^{+}e^{-}$ pairs, QCD considers the vacuum to be occupied by a virtual \emph{sea} of gluons and $q\overline{q}$ pairs.
In contrast to photons in QED however, as gluons self-couple, the virtual gluons have an attractive effect greater than the screening effect of virtual $q\overline{q}$ pairs.
Therefore, while $\alpha_{s}$ is sufficiently small inside a hadron for partons to behave as free particles, increasingly large amounts of energy are required to pull a hadron apart.
This results in the \emph{colour confinement} of partons~\cite{ElectroweakStrong,Griffiths,devenish2004deep}.
This behaviour of $\alpha_{s}$ means that when partons are liberated from hadrons, such as in the high energy hadron collisions of the LHC, the resultant shower of partons form new hadrons in a process known as hadronisation~\cite{Andersson:1983ia}.

In QED, the contribution to the calculation of the Matrix Element for a process decreases with increasing order of the diagram considered due to the electromagnetic coupling constant being considerably smaller than one.
In contrast however, higher order contributions in QCD become increasingly important as $\alpha_{s}$ increases, making higher order QCD calculations more and more difficult to perform.
It has been demonstrated that QCD calculations can be temporally split (factorised) into components that describe the long and short distance behaviours.
This allows the short distance components to be described using perturbation theory, such as the hard scattering of hadrons, while the long distance components are described using non-perturbative phenomenological models, such as Parton Distribution Functions (PDFs).
For a given hadron, PDFs describe the number density of each parton flavour as a function of the fraction of the hadron's momentum (Bjorken $x$) at a given energy scale.
PDFs are constrained by fits made to measurements made by a variety of different experiments~\cite{Ball:2014uwa,devenish2004deep}.
Figure~\ref{fig:pdf} shows the results of one the fit known as NNPDF3.0 which was used for the generation of the simulation samples considered in this thesis~\cite{Ball:2014uwa}.

\begin{figure}[htb]
\begin{center}
\includegraphics[width=0.97\textwidth]{figs/sm/pdf.png}
\caption{The proton parton distribution functions $xf(x)$ as a function of the momentum fraction determined by the NNPDF3.0 fit for factorisation scales of $\mu_{F} = 10\GeV^{2}$ (left) and $\mu_{F} = 10^{4}\GeV^{2}$~\cite{Ball:2014uwa}.
}
\label{fig:pdf}
\end{center}
\end{figure}

\section{Top Physics}\label{sec:top-physics}
The existence of a third generation of quarks was first hypothesised in 1973 by Makoto Kobayashi and Toshihide Maskawa as the CP violation observed in kaon decays was not possible with only two generations of quarks~\cite{Kobayashi:1973fv}.
This hypothesis was reinforced with the discovery of a third generation (tau) lepton in 1975 and a third generation down-type (bottom) quark in 1977~\cite{Herb:1977ek}, which strongly implied the existence of a weak isospin partner to the bottom quark.
As the top quark was more massive than initially assumed, it would remain unobserved until a sufficiently powerful collider was built.
Finally in 1995 the top quark was observed at the Tevatron at the Fermi National Accelerator Laboratory by the CDF and D\O\xspace experiments~\cite{Abe:1995hr,D0:1995jca}.

The top quark's mass, $m_{top}$, of $173.0 \pm 0.4 \GeV$~\cite{Tanabashi:2018oca} makes it the most massive known fundamental particle and is responsible for imbuing it with properties that have no equivalent for the other five quarks~\cite{Tanabashi:2018oca}.
Unlike the other five quarks, the top quark is massive enough to decay into an on-shell W boson, giving it a much shorter lifetime than the other quarks.
This lifespan of $5 \times 10^{-25}$ seconds is several orders of magnitude smaller than the characteristic timescale of the strong interaction~\cite{Quadt}.
Consequently, the top quark is the only quark that decays before it can hadronise, making it a unique probe into the nature of a ``bare'' quark, such as its spin and polarisation, through studying the angular distributions of its decay products~\cite{Khachatryan:2015dzz}.
This also makes it possible to determine the helicity of the W boson involved in the decay.
Measurements of the Wtb vertex allow for the $\abs{V_{tb}}$ element of the Cabibbo-Kobayashi-Maskawa (CKM) matrix to be directly measured and thus test whether the CKM matrix is unitary, as presumed, or otherwise~\cite{Shibata:2008sy}.

The top quark predominantly decays into a bottom quark and a W boson, as shown in Figure~\ref{fig:topDecay}.
Currently, the most precise measurement of the branching ratio for this decay mode has been measured to be $1.014 \pm 0.003 \textrm{(stat)} \pm 0.032 \textrm{(syst)}$ by the CMS Collaboration~\cite{Khachatryan:2014nda}.

\begin{figure}[htbp]
\begin{center}
\includegraphics[width=0.57\textwidth]{figs/top-physics/topDecay.png}
\caption{The main decay mode of the top quark into a b-quark and W boson, where the W boson decays either leptonically or hadronically~\cite{topDiagrams}.}
\label{fig:topDecay}
\end{center}
\end{figure}

Given all these properties, the top quark makes an excellent probe of the Wtb vertex and is sensitive to any anomalous couplings that would impact it.
Additionally, with the top mass being greater than that of any other fundamental particle, it has the strongest Yukawa coupling to the Higgs field.
Consequently, many believe that the top quark has a special role to play in electroweak symmetry breaking and Beyond the Standard Model (BSM) Physics~\cite{Giammanco:2017xyn}
 
%The top quark has not been studied to the same extent as the other quarks due to its later discovery and the relatively low top quark production rate at the Tevatron limiting statistics.
%The greater operational energy and integrated luminosity provided by the LHC however, has produced greater statistics will be available to probe the nature of the top quark~\cite{Shibata:2008sy}. 

\subsection{Top quark pair production}\label{subsec:ttbarTheory}
At hadron colliders, top quarks are predominantly produced by pair production (\ttbar) through strong interactions.
As illustrated in the Feynman diagrams in figure~\ref{fig:feyn_ttbar}, at Leading Order (LO) \ttbar events are produced by either gluon fusion or quark-anti-quark annihilation. 
While approximately 85\% of \ttbar events produced at the Tevatron occured via quark fusion, 80-90\% of \ttbar events at the LHC are produced by gluon fusion for $\sqrt{s} = 8-14\TeV$~\cite{Tanabashi:2018oca,Deliot:2011np}.
These differences in production rates occur for two reasons:
\begin{itemize}
\item Higher centre-of-mass energies results in smaller Bjorken $x$, resulting in a much larger fraction of the proton's energy being carried by gluons.
\item The Tevatron was a proton-anti-proton collider, both quarks involved in quark fusion could be valance quarks, unlike the LHC where one would have to be a sea quark. 
\end{itemize} 

\begin{figure}[htbp]
\begin{center}
\includegraphics[width=0.97\textwidth]{figs/top-physics/ttbar_feyn.jpg}
\caption{The three Leading Order Feynman diagrams for top quark pair production at hadron colliders. Quark-anti-quark annihilation is illustrated on the top row and gluon fusion on the bottom.}
\label{fig:feyn_ttbar}
\end{center}
\end{figure}

As the top quark predominately decays into a W boson and a b-quark, the three different decay modes of pair produced top quarks are characterised by the manner in which the two W bosons decay:
\begin{itemize}
\item \textbf{hadronic} decays occur when both W bosons decay into a quark and anti-quark.
\item \textbf{lepton + jets} decays occur when one W boson decays into a lepton and its associated anti-neutrino, while the other W boson decays hadronically.
\item \textbf{dilepton} decays occur when both W bosons decay into a lepton and its associated anti-neutrino.
\end{itemize}
 

Top quark pair production can also occur in association with a vector boson (\ttV), albeit at relatively small cross sections compared to both \ttbar and single top production (see Section~\ref{subsec:tZqTheory}).
%Despite the relatively small production cross sections for \ttV processes, it is important that they are well understood as they form some of the irreducible background processes for other rare processes, such as tHq and tZq production~\cite{Khachatryan:2014ewa}.

\subsection{Single top quark production}\label{subsec:singleTopTheory}
Top quarks can also be produced singly through weak interactions, albeit with smaller cross sections than that for \ttbar production given the relative weakness of the electroweak coupling compared to the strong coupling.

There are three main SM single top production mechanisms, which are categorised by the virtuality of the W boson involved in the interaction.

\begin{figure}[htbp]
\centering
\includegraphics[width=1.00\textwidth]{figs/top-physics/singletop_feyn.jpg}
\caption{The leading order diagrams for each of the three single top production mechanisms: (a) s-channel, (b) t-channel and (c) single top production in association with a W boson (tW production).}
\label{fig:singleTopDiagrams}
\end{figure}

Figure~\ref{fig:singleTopDiagrams}(a) shows the first of these mechanisms, which is known as s-channel production. 
This is quark-anti-quark annihilation producing an off-shell W boson that decays into a top and anti-b quark.
This process has the lowest production cross section of the three at the LHC due to the charge-asymmetric initial state.
Given its low cross section and a final state topology similar to larger background processes, the s-channel has yet to be observed at the LHC~\cite{Khachatryan:2016ewo}.

The t-channel production mechanism, as shown in Figure~\ref{fig:singleTopDiagrams}(b), is the dominant single top prodution mechanism at the LHC.
The process involves the scattering of a W boson off a sea b quark or produced a b quark produced by gluon splitting.
Initially observed at the Tevatron~\cite{Aaltonen:2009jj,Abazov:2009ii}, the t-channel process has since been studied at higher energies at the LHC, with all results to date remaining consistent with the SM~\cite{Berta:2017ghf,Morton:2018wkb}.	

The tW production mechanism, as shown in Figure~\ref{fig:singleTopDiagrams}(c), is the process in which  a top quark is produced in association with an on-shell W boson.
In contrast to being negligible at the Tevatron, tW production is observable at the LHC and was discovered in 2014~\cite{Chatrchyan:2014tua}.

Single top production processes are a powerful probe of the electroweak interactions of the top quark.
In contrast to \ttbar, these processes allow for the Wtb vertex involved in top quark production to be probed in addition to providing complimentary measurements of the Wtb vertex in top quark decays.

Understanding single top quark production processes is also important from an experimental viewpoint as:
\begin{itemize}
\item These processes form backgrounds for not only SM processes such as \ttbar, but also for Higgs and BSM physics searches, such those which introduce new electroweak couplings.
\item Precision measurements of these processes can be used to compliment measurements of \ttbar processes in constraining Parton Distribution Functions~\cite{Guffanti:2010yu}.
\end{itemize}


\subsection{Single top production in association with a Z boson}\label{subsec:tZqTheory}
The analysis presented in this thesis is the search for the production of a single top quark in association with a Z boson with an additional jet, known as \emph{tZq} production, using the dilepton final state.

The high centre-of-mass energies and integrated luminosities available at the LHC have made it possible to not only perform precision studies of \ttbar and single top quark process, but also to make measurements of processes involving the tZ vertex.
Such measurements provide not only the ability to perform precision tests of SM predictions, but are also sensitive to new physics such as the existence of new electroweak bosons, new fermions or Flavour Changing Neutral Currents (FCNC).

\begin{figure}[htbp]
\centering
\includegraphics[width=\textwidth]{figs/top-physics/CMS-TOP-17-005_Figure_001.pdf}
\caption{Leading order \ttW~(left) and \ttZ~(right) production diagrams~\cite{Sirunyan:2017uzs}. Unlike \ttZ~and \ttH~production, the gauge boson in \ttW~is not radiated from the top quark, but from the initial state quarks.}
\label{fig:feyn_ttV}
\end{figure}

It may initially assumed that given the larger production cross section for \ttbar compared to single top processes that \ttbar processes would provide the best conditions to probe the electroweak interactions with the top quark.
The tW coupling however, can only be probed through the single top tW process as the W boson couples to the initial state quarks for \ttW~processes, as illustrated in figure~\ref{fig:feyn_ttV}.
tH has yet to be observed~\cite{CMS:2018jsz} as it is much more difficult to access than \ttH~due to the destructive interference between the tH and HW vertices~\cite{Maltoni:2001hu}.

%CMS has made measurements of \ttH, \ttW, and production, all of which are consistent with the corresponding SM predictions~\cite{Sirunyan:2017uzs,Sirunyan:2018hoz}.

In contrast,~\ttZ~has a lower production cross section than the combined tZ and $\overline{\text{t}}$Z production cross sections~\cite{Campbell:2013yla} as tZq contains fewer particles in the final state and thus is easier to produce.
CMS has made measurements of \ttH, \ttW, and \ttZ, all with signifiances in excess of five standard deviations and consistent with their SM predictions~\cite{Sirunyan:2018hoz,Sirunyan:2017uzs}.

tZq production is a rare SM process where a single top quark is produced in association with a Z boson with an additional jet.
Unlike \ttZ~where the Z boson is radiated from one of the top quarks, tZq involves the Z boson being radiated off one of the quark legs, as shown in the top two rows of Figure~\ref{fig:feyn_tZq}, or from the exchanged W boson, as shown shown in the bottom left diagram in Figure~\ref{fig:feyn_tZq}.
As tZq production is sensitive the WWZ coupling, unlike \ttZ~production, and is expected to be as sensitive to this coupling as WZ production, this process provides a unique precision probe of electroweak interactions with the top quark~\cite{Campbell:2013yla}.

In addition, tZq production needs to be well understood as it forms one of the irreducible backgrounds for other rare SM processes, such as tH production, as well as BSM processes such as Flavour Changing Neutral Currents (FCNC)~\cite{AguilarSaavedra:2004wm}.

\begin{figure}[h]
\centering
\includegraphics[width=0.37\textwidth]{figs/top-physics/tZq_feyn1.jpg}
\includegraphics[width=0.37\textwidth]{figs/top-physics/tZq_feyn2.jpg}
\includegraphics[width=0.37\textwidth]{figs/top-physics/tZq_feyn3.jpg}
\includegraphics[width=0.37\textwidth]{figs/top-physics/tZq_feyn4.jpg}
\includegraphics[width=0.37\textwidth]{figs/top-physics/tZq_feyn5.jpg}
\includegraphics[width=0.37\textwidth]{figs/top-physics/tZq_feyn6.jpg}
\caption{Leading order tZq production diagrams, where the Z boson is radiated off one of the quark lines in the diagrams in the top two rows, where the Z boson is radiated off the exchanged W in the lower left diagram and from the the non-resonant contribution to the tZq process in the bottom right diagram.}
\label{fig:feyn_tZq}
\end{figure}

As the top quark predominately decays into a W boson and a b-quark, the four possible final states are characterised by the decay mode of the Z boson and W boson:
\begin{itemize}
\item \textbf{trilepton:} when the W boson decays into a lepton and neutrino and the Z boson decays into a lepton and anti-lepton.
\item \textbf{dilepton:} when the Z boson decays into a pair of leptons and the W boson into a quark and anti-quark. 
\item \textbf{single lepton:} where the W boson decays into a lepton and neutrino and the Z boson decays into a quark and anti-quark.
\item \textbf{hadronic:} both the W boson and Z boson decay into a quark and anti-quark.
\end{itemize}

The physics analysis presented in this thesis is the first search at CMS for tZq using the dilepton final state.
The initial searches for tZq however, used the trilepton final state, as while it has a smaller production cross section than either of the dilepton or hadronic final states, it is the easiest to distinguish against background processes.

The first search for tZq however, was unable to observe the process, making a measurement with an observed significance of 2.9 $\sigma$~\cite{Sirunyan:2017kkr}.
Both ATLAS and CMS have since been able to observe the trilepton final state for tZq at $\sqrt{s} = 13\TeV$ as a result of the tZ and $\overline{\text{t}}$Z cross sections increasing with the centre-of-mass energy at a similar rate to \ttZ~and the large integrated luminosity delivered by the LHC at $\sqrt{s} = 13\TeV$~\cite{Aaboud:2017ylb,Sirunyan:2017nbr}.
Such an increase in statistics has also made it possible to search for the other tZq final states, including the dilepton final state, allowing for complimentary measurements of this process to be made.

The observed results presented in this work and the previous CMS searches for tZq using the trilepton final state at $\sqrt{s} = 13\TeV$ use the reference next-to-leading order production cross section for tZq where the Z boson decays leptonically, for $m_{ll} > 30 \GeV$~\cite{Sirunyan:2017nbr}:

\begin{equation}
\sigma ( \textrm{tZq}, Z \rightarrow l^{+} l^{-}) = 94.2^{+1.9}_{-1.8}\textrm{scale}\pm{2.5}\textrm{ (PDF) fb} \;
\label{tZqCrossSection}
\end{equation}

The analysis strategy and full event selection requirements used in the analysis of this process is discussed in detail in Chapter~\ref{chapter:tzq-search}, the modelling of the backgrounds in Chapter~\ref{chapter:bkg} and the statistical methodology used to perform the measurement of this process in Chapter~\ref{chapter:results}.

\section{Beyond the Standard Model Physics}\label{sec:bsm}
The SM has been incredibly successful at accurately predicting the majority of the properties of the known fundamental particles up to the electroweak scale.
However, given the inability of the SM to incorporate gravity and to fully address a number of experimental observations, such as massive neutrinos, it is apparent that there must be new physics beyond the Standard Model.

\subsection{Shortcomings of the Standard Model Physics}\label{subsec:shortcomings}
One of the major and most apparent shortcomings of the SM is its inability to explain why there is an asymmetry between matter and anti-matter in the universe.
While CP symmetry violation does occur within the SM, it is insufficient to account for the amount of matter observed in the universe.

Gravity is currently described by the extremely successful classical theory of General Relativity (GR).
GR however, is fundamentally incompatible with the SM and  has produced contradictory results, such as their predictions for the cosmological constant differing by 120 orders of magnitude~\cite{Adler:1995vd}.
While attempts have been made to reconcile the two theories, no successful quantum theory of gravity has yet been produced~\cite{Sola:2013gha}.	

One of the other serious theoretical issues with the SM is the \emph{hierarchy problem} concerning the lack of explanation for the vast differences observed between the electroweak scale and the Grand Unified Theory and Plank scales where gravity becomes strong~\cite{Burdman:2007ck}.
The mass of the Higgs boson presents a related hierarchy problem.
As the vacuum expectation value of the Higgs field determines the mass of the weak bosons, for the observed masses of these bosons, one would expect a vacuum expectation value of approximately 246\GeV.
Given that the radiative corrections to the observable mass of the Higgs boson are proportional to the energy scale of any new physics, this would imply that the Higgs vacuum expectation value would be either zero or of the order Plank's constant.
In order to obtain the observed Higgs mass of 125\GeV, the cancellations required from the radiative corrections must be extremely ``fine tuned''.
While there is nothing fundamentally wrong with this, many scientists find such fine tuning to be \emph{unnatural}.

%Other astronomical and cosmological inconsistencies include the presence of \emph{dark matter} (DM) and \emph{dark energy} in the universe.
%The observations of the rotation of galaxies, gravitational lensing, structure of the universe and the Cosmic Microwave background, indicates that there must be a form of ``dark'' matter present~\cite{Aghanim:2018eyx}.
%The accelerating expansion of the universe is also unaccounted for and implies the existence of a ``dark energy'' to drive this~\cite{Peebles:2002gy,Aghanim:2018eyx}.

Perhaps the greatest inconsistency experimentally observed with the SM is the fact that neutrinos are not massless.
The first indication of massive neutrinos was made by the Homestake experiment, which found that the fraction of electron neutrinos arriving from the Sun was at most half what was expected~\cite{PhysRevLett.20.1205}.
While this observation could be explained by neutrinos experiencing flavour oscillations, this would require neutrinos to have mass in contrast to the expectations of the SM in order for their flavour eigenstates to mix with their mass eigenstates.
Further experiments have confirmed however, that neutrinos do undergo flavour oscillations and thus must have mass~\cite{Fukuda:1998mi,Ahmad:2001an,PhysRevD.88.032002}.

%\subsection{Flavour Changing Neutral Currents}\label{sec:fcncs}
%Given the limited experimental evidence of BSM physics, a large number of BSM physics models, driven by theoretical and ascetic arguments, have been proposed to account for the shortcomings of the SM.
%While the analysis presented in this thesis concerns the search for a SM process, the tZq cross section is sensitive to modifications of the tZ coupling posited by a number of BSM theories.
%
%As eluded to in Section~\ref{subsec:weakForce}, any flavour changing process involving a neutral weak current in the SM cannot occur at the tree level and requires a loop processes involving a virtual W exchange.
%A number of BSM theories however, introduce top quark FCNC decay contributions at the tree level, such as Supersymmetry (SUSY) models and those proposing additional Higgs doublets and/or quark singlets~\cite{AguilarSaavedra:2004wm}.
%The presence of such new tZ couplings would enhance the production rate of both \ttZ and tZq by several orders of magnitude and should be observable at the LHC. 
%As of to date however, no evidence for BSM FCNCs have been observed for the tZ coupling for both single top and \ttbar processes~\cite{Sirunyan:2017kkr}.


\section{The LHC accelerator and the CMS experiment}\label{sec:lhc-cms}
\subsection{The Large Hadron Collider}\label{subsec:lhc}

The Large Hadron Collider (LHC) at the European Organisation for Nuclear Research (CERN), in Geneva, Switzerland is the highest-energy particle accelerator constructed to date. 
It is designed to operate at a centre of mass (CoM) energy of 14\TeV, through two 7\TeV proton beams travelling in 2808 bunches of up to $1.15 \times 10^{11}$ protons at a collision rate of 25\nsm which corresponds to a design luminosity of $10^{34}\percms$. 
The LHC can also operate in a heavy-ion mode, where lead ions are collided at 2.76\TeV per nucleon usually for one month a year~\cite{Bayatian:2006zz}.

The beams collide at four interaction points around the LHC, with one of the four major experiments being based at each of them. 
The experiments are: A Toroidal LHC Apparatus (ATLAS) and the Compact Muon Solenoid (CMS) detectors, which are the two multi-purpose experiments; the Large Hadron Collider beauty (LHCb) is an experiment which specialises in b-physics and; A Large Ion Collider Experiment (ALICE), as the name suggests, specialises in heavy ion physics~\cite{Bruning:782076}.

Three smaller experiments are situated close to one of the four main experiments and use the same collision points.
Both the TOTal Elastic and diffractive cross section Measurement (TOTEM) and LHC-forward (LHCf) experiments study diffractive physics in the very-forward regions of collisions at the CMS and ATLAS experiments' collision points respectively.
Monopole and Exotics Detector At the LHC (MoEDAL) shares the LHCb experiment's cavern and performs direct searches for magnetic monopoles and highly ionising stable and pseudo-stable massive particles.

\subsubsection{Motivation}
The core motivations behind the LHC are to shed light on the nature of the electroweak symmetry breaking, which the Higgs was presumed and found to be responsible, and to probe the consistency of the SM above the \TeV level through precision measurements of SM parameters and the Higgs mechanism.
Alternative theories to the SM, such as SUSY theories, additional dimensions or new fundamental forces and particles are expected to emerge at and above the TeV level, giving the potential to ascertain whether these theories have any basis beyond mere conjecture.

In order to explore and permit the discovery of physics at the \TeV level, the total centre of mass energy has to be greater than the energy region being explored as, due to the composite nature of the proton, only a fraction of the collision centre of mass energy is available.
Access to physics beyond the \TeV level is not excluded	as some signals would be ``unmissable'', but the majority of physics would be limited by statistics.
Compared to the total inelastic cross section, the production cross section of the Higgs boson and hypothesised SUSY particles, if they have \TeV masses (and exist), are predicted to be many orders of magnitude smaller.
Measurements of such processes, as well as precision measurements of SM parameters, require a high interaction rate, and consequently the LHC has a high beam luminosity so that there sufficient statistics available.
Protons are delivered in 2808 bunches per beam, as opposed to a continuous beam, which at design luminosity will separated by 25ns, resulting in an event rate of 40\MHz and an average of 25 inelastic proton-proton interactions, named pile-up (\PU) for each bunch crossing. 
%% CITE LHC TDE page 31 for 2808 bunches
%% CITE CMS VOL2, page 33 for pileup
This high event rate presents the experiments with the data acquisition and readout challenges, whilst retaining excellent signal to background resolution and sufficient radiation hardness in order to withstand the expected fluence.

The primary motivation behind operating the LHC in a heavy-ion mode is to search for evidence of the plasma of quarks and gluons, which is made possible through the resultant production of QCD matter under extreme temperature, density and low momentum fractions of partons~\cite{Baur:687318}.

\subsubsection{Accelerator Complex}
When operating in proton-proton mode, the preparation of the LHC beams starts at Linear accelerator 2 (Linac2). 
Protons from a hydrogen gas source are accelerated to 50\MeV and are injected into the Proton Synchrotron Booster which accelerates the protons to 1.4\GeV before injection into the Proton Synchrotron (PS). 
In the PS, the protons are accelerated to 26\GeV and are injected into the Super Proton Synchrotron (SPS) where they are accelerated to 450\GeV before finally entering the LHC~\ref{fig:cern-accelerator-complex}. 
When operating with lead ions, Linear accelerator 3 (Linac3) is used to initially accelerate the ions before injecting them into the Proton Synchrotron Booster, before the ions use the same accelerators as the protons do to prepare them for use in the LHC\~cite{Bruning:782076}. 

Sixteen Radio Frequency (RF) cavities (eight per beam), each operating at frequency of 400\MHz, at a temperature of 4.5K, and delivering a maximum of 2 MV, are used to accelerate the two beams up to their designed operational energies of 7\TeV over the course of circa twenty minutes.
Each of the two beams are accelerated in separate beam pipes, circulating in opposite directions,and requires 1232 dipole magnets to bend them along their circular path and 392 quadrupole magnets to focus them, with each magnet producing a 8.3T field whilst operating at 1.9K.
A more detailed description of the LHC accelerator chain at CERN can be found in~\cite{Schindl:397574}. 

\begin{figure}[htbp]
\begin{center}
\includegraphics[width=0.97\textwidth]{figs/lhc/Cern-Accelerator-Complex.jpg}
\caption{CERN complex, including the various linear accelerators, synchrotrons, LHC, LHC detectors and other aspects of the complex.}
\label{fig:cern-accelerator-complex}
\end{center}
\end{figure}

\subsection{The Compact Muon Solenoid}\label{subsec:cms}

\subsubsection{Overview}
The Compact Muon Solenoid (CMS) is a large, general purpose, hermetic particle detector and the smaller of the two multi-purpose experiments operating at the LHC at CERN.
The experiment is divided into a central cylindrical barrel section and two endcap disk sections at each end of the barrel.
A superconducting solenoid encompasses, moving from the interaction point at the centre of the detector outwards, an all silicon tracking detector, a homogeneous lead tungstate ($PbWO_{4}$) electromagnetic calorimeter (ECAL)and hadronic calorimeter (HCAL) comprised of plastic scintillating tiles interspaced with brass absorbers.
Beyond the solenoid there is an outer hadronic calorimeter (HO) and interspaced between the iron return yoke are three different types of Muon Detectors.
There is also a pair of very-forward calorimeters (HF) in the extended rapidity region\cite{oldcms}.

These detectors were designed in order to investigate the wide range of physics phenomena in the LHC's physics program, resulting in the accurate and precise identification and measurement of electrons, photons, jets and muons over both a large energy and momenta range.
Full detector resolution is achieved across $|\eta| < 3.0$, with the hadronic calorimetry having an extended coverage up to $|\eta| < 5.0$ in order to ensure good dijet mass and \MET resolutions.
Sufficient radiation hardness for the expected high fluence and data acquisition and trigger systems required to handle to event rate of the LHC environment had to be considered in the design of the various detectors.

The coordinate system adopted by the CMS experiment has the origin at the nominal interaction point at the centre of the detector. 
The z-axis is parallel to the anti-clockwise proton beam (i.e. towards the Jura mountains from the detector), the x-axis points towards the centre of the LHC, and the y-axis points vertically upwards.
The azimuthal angle, \phi, is the angle measured from the x-axis in the x-y plane and the polar angle, \theta, is the angle measured from the z-axis.
Pseudorapidity, defined as $\eta \equiv -ln\tan(\theta/2)$, is usually used in lieu of \theta as when the mass considered is negligible \eta converges towards rapidity, defined as $y \equiv	1/2 ln(E+p_{z}/E-p_{Z})$, which is Lorentz invariant along the z-axis.
As such, variables transverse to the z-axis (i.e. the beam line), such as the transverse energy (\ET), momentum (\pT), and missing energy (\MET), depend only on their x and y components.


\begin{figure}[htbp]
\begin{center}
\includegraphics[width=0.97\textwidth]{figs/cms/cms_120918_03.png}
\caption{Cutaway diagram of CMS’s layers, illustrating its onion-like nature and the location of the detecting technologies within.}
\label{fig:cern-accelerator-complex}
\end{center}
\end{figure}

%%%

\subsubsection{Tracker}
The tracker, surrounding the interaction point, is designed to provide efficient precision trajectory measurements of charged particles emerging from collisions and precise reconstruction of vertices over $\eta < 2.5$, whilst operating in a harsh radiation environment (max flux $\approx 10^{7}/s$) and minimising the number charged particles interacting with the tracker (i.e. scattering, producing Bremsstrahlung).
Silicon fulfills these requirements, but as well as the material budget, a financial budget had to be also considered.
As the fluence is sufficiently 

The tracker is composed of silicon, in order to limit charged particles interacting with the tracker (i.e. scattering, producing Bremsstrahlung), whilst providing the desired accuracy in track reconstruction\cite{oldcms}.

Measuring 5.8m 

%%Phase 1
During the End of Year Technical Stop that took place between data taking in 2015 and 2016, the silicon tracker was completely replaced. 

\subsubsection{Electromagnetic Calorimeter}
Beyond the tracker, the ECAL, a homogeneous calorimeter, measures the energies of electrons and photons using lead tungstate scintillating crystals. 
The choice of using lead tungstate crystals was based on the needs of both having a compact detector which could fit with the HCAL inside the solenoid and containing the showers' energy within this calorimeter, which were met with its short radiation length (0.89\cm) and small Molier\'{e} radius (2.2\cm).

61,200 crystals are mounted in the barrel, with 7,324 crystals for each endcap.
As the crystals only emit a small amount of scintillation light, the photodetectors used to amplify the light must be able to operate within the solenoid's field and and withstand the high radiation environment.
Avalanche photosdiodes and the more radiation hard vacuum photodiodes are used in the barrel and endcap regions respectively.
The emitted light emitted by these crystals is short, well defined and fast, with 80\% collected within one 25\ns bunch crossing, with the signals being digitised on-detector and buffered until a Level-1 Trigger decision has been made.
Information to the Level-1 Calorimeter Trigger

The barrel and endcap cystal systems are supplemented by a Preshower (ES) device, located in front of the ECAL endcaps, for discrimination between neutral pions and photons within the fidicial region $1.653 < |\eta| < 2.6$.
For each detector, two lead radiators initiate the electromagnetic showers and two silicon strip sensors, orthogonal to one another to provide fine resolution, are placed after the radiators measure the energy deposited.
The thickness of the radiators was chosen to ensure ~95\% of incident photons shower before reaching the second silicon strip sensor, namely two and one radiation lengths for the first and second lead radiators respectively.

%Following the excellent performance during Run I of the LHC, with an energy resolution of 0.3\% exceeding the design value of 0.5\%, the ECAL has continued to perform admirably, with .... \cite{TeixeiradeLima:2017tmj} %%ECAL performance Run II

\subsubsection{Hadronic Calorimeter}
Hadronic particles penetrate through the ECAL into the HCAL, where hadronic jets have their energies measured and are contained for determination of the missing transverse energy and protection of the muon detectors~\cite{HCAL:tdr}.
As such, the HCAL was designed to have as much absorber material within the solenoid coil as practical. 
The barrel (HB) and and endcap (HE) both use plastic scintillator tiles interspersed between brass and steel absorber plates, with the latter being used for the external plates for structural strengthening.
Wavelength shifting fibres embedding in the tiles converts the scintillated light and channels it to hybrid photodiodes.
The HB covers the rapidity range $|\eta| < 1.4$, with the HE overlapping it and providing coverage over the range $1.3 < |\eta| < 3.0$.

Due to space constraints within the solenoid, the HB cannot fully contain hadronic showers and as such is supplemented by an additional calorimeter in the barrel region outside the coil (HO) \footnote{Given the outer hadronic calorimeter's limited size and function, it will not be discussed further here. A thorough description of the HO can be found in~\cite{HO}}.

The forward calorimeters (HF) overlap with the HE and cover the $2.9 < \eta < 5.0$ rapidity region \cite{HF}.
As the forward region experiences the most severe radiation environment, the technology used must be able to withstand such large radiation doses($~10^{9}$ rad). 
Interspaced between steel absorbers, quartz fibres are used produce Cherenkov light due to their radiation hardness, fast response time, production of Cherenkov radiation above a certain energy threshold (thus ignoring low energy particles), and ability to give directional information due to the light being strongly correlated with the showers' trajectories.
The Cherenkov light produced is transmitted down the fibres to individually shielded photomultiplier tubes contained in readout boxes.

\subsubsection{The Superconducting Solenoid}
One of the defining features of the CMS detector is the superconducting solenoid which encompasses the silicon tracker and calorimetry.
The 220T cylindrical coil measures 13m long, has a 5.9m inner diameter, is situated inside a vacuum tank where it is cooled to its operation temperature by liquid helium to 4.5K, and operates at magnetic field of 3.8 Tesla\footnote{Whilst the solenoid was designed to operate at 4T, the CMS collaboration chose to operate it at 3.8T in order maximise the lifetime of the apparatus}.
The large bending power within the solenoid not only provides excellent momentum resolution for charged particles within the tracking detector, but it also prevents low transverse momentum charged particles from reaching the calorimetry and negatively impacting on energy resolution and isolation efficiency.
An iron return yoke guides and contains the return magnet field, which is sufficiently strong (~1.7T in the barrel and outermost endcap disk) enough to enable accurate momentum resolution for tracking and charge identification of high momentum (~1\TeVc) muons.

\subsubsection{Muon detectors}
Detecting muons is incredibly important for CMS (as implied by the experiment’s name), given many of the signatures of interesting events involve them, including those from SUSY models and the so called “gold-plated” SM Higgs decay into a pair of $Z^{0}$ bosons, which in turn decay into four muons . 
Being Minimum Ionising Particles (MIPs), muons pass through the detector and past the magnet with minimal interaction.
Consequently, the muon systems are placed outside the solenoid and provide a strong clean signals which can be triggered upon.

Given that the magnetic field outside the solenoid is non-uniform and the radiation environment varies, differing detector technologies are used in order to provide a high performance system which delivers the fast identification and momentum resolution required. 
Interspaced between the iron return yoke rings and disks are three gaseous detector technologies: Drift Tubes (DTs), Cathode Strip Chambers (CSCs) and Resistive Plate Chambers (RPCs).

The DTs are gaoperate in the barrel region across $|\eta| < 1.2$.
These

Interspaced between the iron return yoke rings and disks used are Drift Tubes (DTs) in the barrel covering , Cathode Strip Chambers (CSCs) in the endcaps covering $0.9 < \eta < 2.4$ and Resistive Plate Chambers (RPCs) in both barrel and endcaps across $|\eta| < 1.6$.


The barrel region experiences 

RPCs provide complimentary coverage to the DTs and CSCs, and while having coarser position resolution than the DTs and CPCs, they have fast response times and excellent time resolution.

Combining trigger candidates from the three systems gives an improved momentum resolution and efficiency than the stand-alone information from each of the individual systems\cite{oldcms}.

\subsubsection{Trigger and Data Acquisition Systems}
At design luminosity, the LHC bunch crossing (BX) rate of 40\MHz leads to an event rate of $\approx~10^{9}$ inelastic events per second.

Given the impossibility of storing such a volume of data, let alone reading all of it off the detector, the trigger system has to drastically reduce the data rate by selecting ``interesting'' events for storage for later analysis.



\subsubsection{Level-1 Trigger}
The first step is the Level-1 (L1) Trigger, consisting of custom-designed programmable hardware (FPGA technology where possible). 
Initially both the calorimeter and muon triggers search over a small local area for the signature of an interesting event, forwarding these onto the regional triggers which sorts the candidates in order of importance, before the global calorimeter and muon triggers determine the highest ranked objects across the entire experiment. 
These are sent to the Global Trigger, which either rejects an event or accepts it for further evaluation by the second step, the High-Level-Trigger (HLT), a software system (see Fig.~\ref{fig:trigger} for a more detailed breakdown). 
Events accepted by the HLT are then transferred to mass storage for offline storage and analysis\cite{oldcms}. 

The L-1 trigger analyses every bunch crossing (BX). 
As such, there are strict time limitations on how long it takes for the data can be collected and read out. 
As the selection cannot be done before the subsequent BX, the current L-1 trigger uses a pipelined approach, providing a latency of $\approx$ 3.5\mus . 
Within the latency constraint for reducing the data rate, the L-1 trigger has to deal with the effects of the pileup of events, both in-time (within the same BX) and out-of-time (events from different BXs), in each BX . 
Additionally, the constraints on bandwidth limits the volume of data a single board can receive and determining whether events being read in. 
In light of this, the current approach of having large amounts of data over small regions being brought to fewer boards so that objects of interest can be considered, has to discard data at each stage a larger region is considered . 
While this approach creates candidates within these constraints, by definition only the candidates from this coarser data set can be considered. 
Any candidates in the discarded data are lost\cite{oldcms}.

\begin{figure}[htbp]
\begin{center}
\includegraphics[width=0.97\textwidth]{figs/cms/trigger.png}
\caption{L-1 Trigger Architecture. Both muon and calorimeter triggers search for candidates locally before creating coarser datasets by passing on the most promising candidates to a higher level, and so on until the Global Trigger makes a final decision. However, unlike the Calorimeter Trigger, which looks at all the calorimeter systems concurrently, the Muon Trigger locally triggers on each of its detector technologies separately before submitting them to the Global Muon Trigger (with input from the Regional Calorimeter Trigger) before passing on fitted candidates to the Global Trigger.}
\label{fig:trigger}
\end{center}
\end{figure}

%ECAL and trigger primitives generated and , with the former being buffered until the Level-1 Calorimeter makes a decision based off the latter.



\chapter{The CMS Tracker Upgrade}\label{chapter:tk-upgrade}
 
\section{The High-Luminosity Large Hadron Collider} \label{sec:hl-lhc}
The High-Luminosity Large Hadron Collider (HL-HLC) upgrade is expected to be installed during Long Shutdown 3 (2023-2025), with the instantaneous luminosity of the LHC increasing up to $5-7.5 \times {10}^{34}$\percms, corresponding to an average number of proton-proton interactions per 40\MHz bunch crossing of 140 to 200, and a total integrated luminosity of 3000\fbinv to the ATLAS and CMS experiments.

Increasing the LHC's instantaneous luminosity is motivated by the need to replace the inner triplet quadrupole magnets which focus the beams at the ATLAS and CMS collision regions, that are expected to be near life expired due to radiation exposure by 2023~\cite{hl-lhc-prelim-design-report,CMSCollaboration:2015zni}.
This increase in instantaneous luminosity will provide the experiments the ability to overcome the diminishing statistical gains that occur the longer an experiment is operated for at constant luminosity, and so enable greater precision SM and Higgs measurements, searches for rare processes and their potential deviations from the SM, and the discovery reach for multi-\TeV massive particles.

The instantaneous luminosity of the machine and the beam parameters are related by: the number of bunches $n_{b}$, the number of protons per bunch $N^{2}_{p}$, the beam beta value (focal length) at the collision point $\beta^{*}$, and a crossing angle dependent luminosity geometrical reduction factor $R$,

\begin{equation}
L \propto \frac{n_{b}N^{2}_{p}}{\beta^{*}} R  \;.
\label{eq:machineLumi}
\end{equation}

As it is not practical to increase the number of proton bunches due to the resultant heat loads induced by electron clouds, the increase in the machine's luminosity will be achieved by increasing the number of protons per bunch and by  reducing $\beta^{*}$.
Replacing Linac2 with the new Linear accelerator 4 (Linac4) during the Long Shutdown 2 (2019-2020) will allow for the number of protons per bunch to be increased by a factor of two compared to the nominal LHC design (and to increase the injection energy by a factor of three)~\cite{linac4}.
The new more radiation tolerant quadrupole magnets to be installed during LS3 will provide the larger magnetic field strength and aperture required to provide the lower $\beta^{*}$ required for increasing the instantaneous luminosity. 

\section{The Phase-II Outer Tracker Upgrade}\label{sec:tk-upgrade}
To meet the significant challenges of, and exploit, the increased instantaneous luminosity environment of the HL-LHC,
the ``Phase-II Upgrade'' of the CMS detector has been proposed.
This upgrade will take place during the LS3 and will deliver the required improved radiation hardness for the increase in radiation and to manage the high \PU HL-LHC environment with greater detector granularity to reduce occupancy, and enhanced bandwidth and triggering capabilities to avoid compromising physics potential~\cite{CMSCollaboration:2015zni,P2TrackerTDR}.

The Phase-II upgrade will see both the entire silicon tracking detector being replaced with one comprised of a pixel Inner Tracker and pixel and strip Outer Tracker which have:
\begin{itemize}
\item \textbf{improved radiation hardness} - being able to withstand the increased fluence of the HL-LHC (up to $2.3\times10^{16} n_{eq}/cm^{2}$ for the innermost layers)and operate efficiently up to the targeted luminosity of 3000\fbinv, with a margin of $\approx50\%$ to accommodate the target being exceeded and the uncertainties in the anticipated radiation exposure.
\item \textbf{increased sensor granularity} - so that the channel occupancy is kept at or below the per cent (per mille) level for the Outer (Inner) Tracker, allowing for a high track reconstruction efficiency and a low misidentification rate under the increased \PU conditions. This will also enable improved track separation in dense environments, such as high \pT jets, compared to the current pixel detector and fully exploit the vast volume of data produced.
\item \textbf{reduced material in the tracking volume} - the current tracker's performance is significantly impacted by the amount of material present, as are the calorimeters and overall performance of CMS.
Significantly reducing the tracker's material budget will greatly enhance CMS' performance at the HL-LHC.
\item \textbf{robust pattern recognition} - enabling fast and efficient track finding, which is especially important for the HLT, in the high \PU environment.
\item \textbf{level-1 trigger contributions} - it has been shown that the L1 trigger performance will deteriorate in the high luminosity environment from both the rate increase and the reduced efficiencies of the L1 selection algorithms.
As raising the upgraded calorimeters' and muon chambers' trigger thresholds would have minimal impact on the rate, and would negatively impact sensitivity to low mass searches and measurements, the L1 bandwidth and latency will be increased (from 100\kHz to 750\kHz and from $3.2\mus$ to $12.5\mus$ respectively) and tracking information will be included in the L1 decision process to preserve and improve trigger performance.
\item \textbf{extended tracking acceptance} - the overall physics capabilities of the CMS experiment would greatly benefit from extended coverage of the tracker and calorimeters up to $|\eta| = 4$ in the forward region.
\end{itemize}

With the above requirements in mind, the pixel Inner Tracker is designed to cover up to $|\eta| = 4$ with $100-150\mum$ thick planar silicon pixel sensors, measuring either $25\times100\mum^{2}$ or $50\times50\mum^{2}$\footnote{This is a reduction of a factor of $\approx 6$ compared to the Phase-0 and Phase-I pixel detectors}, which provide the low (per mille) occupancy and track separation with the negligible inefficiencies required in the harsh radiation environment.
Akin to the previous pixel detectors, the Phase-II pixel is also designed for easy installation and removal to facilitate the replacement of degraded parts.
Further discussion of the Inner Tracker is detailed in the Phase-II Technical Design Report~\cite{P2TrackerTDR}.

As tracking information is required to make L1 decisions at the HL-LHC, the design of the Outer Tracker has been driven by the need to provide tracking information to the L1 trigger.
Given the implications for reading out every hit for the L1 trigger at the LHC bunch crossing rate of 40\MHz, a novel design of a pair of closely spaced silicon sensors, capable of rejecting hits generated by low \pT particles, has been proposed, where the ``\pT modules''~\cite{jjonespixel,markthesis} discriminate on charged particle \pT based on the local bend of the track within the magnetic field, as shown in figure.~\ref{fig:stubs}.
Pairs of clusters which are consistent with a track \pT above a configurable threshold (typically 2-3\GeV) will be correlated on-detector, and the resultant \emph{stubs} transferred to the L1 trigger, providing an effective data rate reduction of approximately a factor of 10~\cite{mpessimperf,2dptmoduleconcept}.

\begin{figure}[!t]
\centering
\includegraphics[width=5in]{figs/tk-upgrade/pTsketches.png}
% where an .eps filename suffix will be assumed under latex,
% and a .pdf suffix will be assumed for pdflatex; or what has been declared
% via \DeclareGraphicsExtensions.
\caption{Cluster matching in $p_\mathrm{T}$-modules~\cite{P2TrackerTDR}. (a) Correlating closely spaced clusters between two sensor layers, separated by a few mm, allows discrimination of transverse momentum based on the particle bend in the CMS magnetic field, assuming that the particle originated at the beam-line. (b) The same transverse momentum corresponds to a larger distance between signals for a given sensor spacing. (c) A larger spacing is needed in the endcap disks to achieve the same discrimination. Only tracks with \pT $> 2-3$\GeVc are transferred off-detector.
}
\label{fig:stubs}
\end{figure}

Two \pT modules are being developed for the Outer Tracker upgrade: 2S \emph{strip-strip} modules and PS \emph{pixel-strip} modules, both shown in figure~\ref{fig:2Spsmodules}.
The 2S~modules, are designed to be used at radii $r>60$\cm from the beam line, where the hit occupancies are lower and each sensor has an active area of 0.05\cm~$\times$~9.14\cm.
Both 2S~module strip layers have a pitch of 90\mum in the transverse plane, $r$-$\varphi$, and a strip length of 5.03\cm along the direction of the beam axis, $z$.
Each PS~module sensor layer has an active area of 4.69\cm~$\times$~9.60\cm, will be used at radii $20<r<60$\cm where the occupancies are highest.
The upper PS~module layers consist of an upper silicon strip sensor, and the lower a silicon pixel sensor, both with a pitch of 100\mum in $r$-$\varphi$, and a strip length in $z$ of 2.35\cm for the strips and 1.47\mm for the pixels.
The finer granularity provided by the pixel layer affords better resolution along the $z$ axis, which is crucial for vertex identification in the high \PU environment of the HL-LHC.
Further details on the two \pT modules can be found in~\cite{CMS_Upgrade_TP,P2TrackerTDR}.
 
\begin{figure}[tp]
\centering
\includegraphics[width=0.55\textwidth,trim={0truecm 0truecm 0truecm 1truecm},clip]{figs/tk-upgrade/2S_assembled.png}
\hfill
\includegraphics[width=0.44\textwidth,trim={0truecm 0truecm 0truecm 1truecm},clip]{figs/tk-upgrade/PS_assembled.png}
% where an .eps filename suffix will be assumed under latex,
% and a .pdf suffix will be assumed for pdflatex; or what has been declared
% via \DeclareGraphicsExtensions.
\caption{The 2S module (left) and PS module (right), described in the text~\cite{P2TrackerTDR}.}
\label{fig:2Spsmodules}
\end{figure}

The current proposed layout of the Phase-II Outer Tracker, referred to as the \emph{tilted barrel} geometry, is depicted in the upper diagram in figure~\ref{fig:trackerlayout}, and a previous proposal, referred to as the \emph{flat barrel} geometry, is shown in the lower diagram~\cite{CMS_Upgrade_TP}.
Both plots illustrate the PS and 2S module positions in the six barrel layers and the five endcap disks either side of the barrel, with only modules located at $|\eta| < 2.4$ being configured to send stub data off-detector.
Both geometries' names were inspired by whether or not the modules in the three innermost barrel layers being tilted so that their normals point towards the interaction region.
The advantages of the tilted geometry over the original flat barrel are that it not only improves stub-finding efficiency for tracks with large incident angles but also reduces the overall cost of the system~\cite{P2TrackerTDR}.
Due to the maturity of the preparations for the review between the three competing proposed track finding systems, discussed in Chapter~\ref{subsec:TrackFinderReview}, at the time the tilted barrel geometry was adopted for the Phase-II Outer Tracker TDR, it was decided to use the flat barrel geometry for results produced for the review.

\begin{figure}[tbp]
\centering
\includegraphics[width=0.8\textwidth,trim={1.1truecm 0truecm 1truecm 12truecm},clip]{figs/tk-upgrade/tiltedbarrelmap.pdf}
\includegraphics[width=0.8\textwidth,trim={0.7truecm 0truecm 1truecm 0truecm},clip]{figs/tk-upgrade/mersilayout.pdf}
\caption{One quadrant of the Phase-II Outer Tracker layout, showing the placement of the the PS (blue) and 2S (red) modules. The upper diagram shows the currently proposed \emph{tilted barrel} geometry~\cite{tiltedGeometry, P2TrackerTDR}, and the lower diagram shows an older proposal for the layout, known as the \emph{flat barrel} geometry \cite{CMS_Upgrade_TP}.}
\label{fig:trackerlayout}
\end{figure}

Out of the total L1 latency of 12.5\mus, $\approx 1\mus$ is required for generation, packaging and transmission of stubs from the tracker front-end electronics to the Data, Trigger and Control (DTC) system and $\approx 4\mus$ is available for the reconstruction of tracks from data arriving at the DTC, as shown in figure~\ref{fig:dataFlow}.
The rest of the available latency is allocated for the correlation of tracks with trigger primitives from the calorimeters and muon systems ($\approx 3.5\mus$), the propagation of the L1 decision to the front-end buffers ($1\mus$) and a safety margin ($3\mus$)~\cite{CMS_Upgrade_TP}.
Any Track Finder, which will take the stubs as input and output fully reconstructed tracks for the L1, proposed will be constrained by both being able to reconstruct tracks within the $4\mus$ latency constraint and how the detector is cabled to the DTC system.

\begin{figure}[tb]
\centering
\includegraphics[width=\textwidth]{figs/tk-upgrade/dataflow.pdf}
% where an .eps filename suffix will be assumed under latex,
% and a .pdf suffix will be assumed for pdflatex; or what has been declared
% via \DeclareGraphicsExtensions.
\caption{Illustration of data-flow and latency requirements from \pt-modules through to the off-detector electronics dedicated to forming the L1 trigger decision.}
\label{fig:dataFlow}
\end{figure}

\subsection{Level-1 Track Finding Proposals}\label{subsec:TrackFinderReview}

Three different L1 track finders have been explored by the CMS Collaboration.
One uses Associative Memory (\emph{AM}) ASICs for track finding and FPGAs for track fitting, and the other two all-FPGA approaches, one using a fully Time-Multiplexed approaching using the Hough Transform (\emph{TMTT}) and the other a ``road search'' (\emph{tracklet}) algorithm to reconstruct tracks respectively.

Hardware demonstrators for each of the three proposed L1 track finder projects were constructed to prove the feasibility of each approach, which were reviewed in December 2016.
As all of the work discussed in this chapter was on the FPGA-based \HT approach, more detailed descriptions and results of both the AM and tracklet projects' approaches are not discussed here, but are given in~\cite{AM,P2TrackerTDR} and~\cite{tracklet,P2TrackerTDR} respectively.

As mentioned above, at the time of the review the flat barrel geometry was used for all the studies undertaken, as depicted in the lower diagram in figure~\ref{fig:trackerlayout}.
Unless stated otherwise, the results discussed below use the flat barrel geometry instead of the tilted barrel geometry layout.

\section{An FPGA Based Track Finding Architecture and Processor}\label{sec:TMTT}
\subsection{The Track Finding Architecture}\label{subsec:TFA}
The proposed FPGA-based Hough Transform Track Finder is a scalable, flexible and redundant design based on a fully time-multiplexed architecture, as previously demonstrated by the Phase-I Calorimeter Trigger Upgrade~\ref{paragraph:L1}, for implementation on commercially available FPGAs.
A time-multiplexed design has a number of advantages, as discussed in~\ref{paragraph:L1}, including that only a single Track Finding Processor (TFP) is required to demonstrate the full system as each processor is identical in every respect.

Unlike the Phase-I Calorimeter Trigger, it is infeasible to process the entire output of the Phase-II Outer Tracker in a single processor for a given time slice because of the limits imposed by the input and total bandwidth a single FPGA-based processor could handle.
Therefore, as it was assumed at the time of the December 2016 review that the DTC system would be arranged such that it forms octants~\footnote{These detector octants are not uniform as the geometry of the tracker does not have an exact eight-fold symmetry} (\ie 45 degree $\varphi$-sectors, referred to as \emph {detector octants}) in the tracker, the baseline system proposed was divided into \emph{processor octants} that were offset from the detector octants by $\approx 22.5$ degrees in $\phi$, in order to handle data duplication across hardware boundaries.
This baseline system is illustrated in figure~\ref{fig:tmttarch}.

\begin{figure}[t]
\centering
\includegraphics[width=1.00\textwidth]{figs/tk-upgrade/tmttarch.pdf}
\caption{The baseline system architecture uses two neighbouring DTCs to time-multiplex and duplicate stub data across processing octant boundaries, before each DTC transmits 50\% of its data to one TFP and 50\% to the neighbouring TFP Based off current electronics and high speed links available, the data requires 18 TFPs per processing octant (one for each time slice, resulting in a full system requiring 144 TFPs).}
\label{fig:tmttarch}
\end{figure}

A hardware demonstrator of the baseline system consisting of five Imperial Master Processor Virtex-7 (MP7) cards~\cite{mp7ref}, capable of processing one phi-octant of the tracker with a time-multiplexing factor of 36, was used to validate the feasibility of the proposed full system using currently available hardware for the December 2016 review.
All of the results achieved, and a complete description of the system, are given in~\cite{TMTT_JINST}.

\subsection{The Track Finding Processor}\label{subsec:TFP}
The Track Finding Processor (figure~\ref{fig:TFP}) consists of four self-contained components:
\begin{itemize}
\item {\bf Geometric Processor (GP)} - responsible for pre-processing the stubs from the DTC.
\item {\bf Hough Transform (HT)} - a highly panellised initial coarse track finding.
\item {\bf Kalman Filter (KF)} - cleans tracks, precisely fits helix parameters and removes fake tracks.
\item {\bf Duplicate Removal (DR)} - a final pass filter that uses the precise fit information to remove duplicate tracks generated by the \HT.
\end{itemize}

\begin{figure}[!h]
\centering
\includegraphics[width=0.78\textwidth]{figs/tk-upgrade/demoslice1.pdf}
% where an .eps filename suffix will be assumed under latex,
% and a .pdf suffix will be assumed for pdflatex; or what has been declared
% via \DeclareGraphicsExtensions.
\caption{The four self-contained logical components of the Track Finding Processor, where each block represents a single FPGA. The two FPGAs for the two detector octant sources and the sink FPGA and the optical links between all components are also shown.}
\label{fig:TFP}
\end{figure}

\subsubsection{Geometric Processor}\label{subsubsec:GP}
Each GP performs two tasks, firstly the conversion of the 48-bit DTC stubs into a 64-bit format extended format that is used to reduce the HT processing load and secondly the assignment of stubs to thirty six sub-sectors, two sub-sectors in $\phi$ and eighteen in $\eta$ (where $\eta$ is the pseudo-rapidity). 
This division of the processing octants simplifies the task of the downstream logic required, allowing the track finding to be carried out independently and in parallel within each sub-sector. 
The relatively fine $\eta$ binning ensures that any track found by the \rphi HT is consistent in the \rz plane. Stubs compatible with more than one sub-sector, usually due to track curvature in $\phi$ are duplicated. 
The routing of stubs to sub-sectors occurs in three stages: a rough $\eta$ sorting into six bins, a fine $\eta$ sorting into three bins and a $\phi$ sorting into two bins. 
Each block in this router is highly reconfigurable and can easily be adapted to any alternative sub-sector definition.

\subsubsection{Hough Transform}
The Hough Transform algorithm is a widely used means of detecting geometric features in digital image processing \cite{HT} and is used by the TFP to find charged particles with \pT > 3\GeV in the \rphi plane. 

Within the tracking volume, permeated by a homogeneous 3.8T magnetic field ($B$), a radius of curvature ($R$) can be described as a function of its\pT and charge $q$:

\begin{equation}
R = \frac{\pt}{0.003\,qB} \;.
\label{eq:R}
\end{equation}

Assuming, to first order, that $R$ is constant, by neglecting energy losses such as through multiple scattering, and that only primary tracks from or near the primary interaction point are considered (other such tracks are not typically relevant to the L1 trigger), a stub with coordinates ($r$,$\varphi$) is related to $R$ by:

\begin{equation}
\frac r{2\,R} = \sin\left(\varphi-\phi\right) \;.
\label{eq:stub_R}
\end{equation}

where $\phi$ is the angle of the track in the transverse plane at the origin \cite{markthesis}. 
For large \pT (> 3\GeV) and thus large $R$, the small angle approximation can be used. Combining Eq.~\ref{eq:R} and Eq.~\ref{eq:stub_R}, one produces the key formula showing the transformation from stub positions to straight lines in the track parameter plane (Hough-space):

\begin{equation}
\phi = \varphi - \frac{0.0015\,qB}{\pt}\cdot r \;.
\label{eq:localHT}
\end{equation}

The point of intersection of these lines in Hough-space would therefore correspond to a circle in the \rphi plane which is consistent with the primary interaction point and all stubs involved.
As the line gradients in Hough-space is given by the radius of the stubs, they will always be positive, the stub radius is transformed to $r_{58} = r - 58cm$ in order to utilise a larger phase space, which leads to fewer \textit{fake} (in that the found track does not match to a simulated particle) and duplicated tracks.

Given that $R$ for the lowest \pT track (3\GeV) to be considered is greater than the outer radius of the tracking detector ($r$ = 1.2m), all relevant particles are expected to traverse through at least six barrel layers or endcap disks. 
The threshold for the identification of a track candidate however, is set at a minimum of five detector layers or disks in order to allow for detector or readout inefficiencies. 
This threshold can be further reduced to four layers to account for the reduced geometric coverage between $0.89 < \eta < 1.16$ or for dead detector layers or disks.

A more detailed description of the firmware implementation of the \HT for the demonstrator system is discussed in~\cite{IEEE} and~\cite{TMTT_JINST}.

\subsubsection{Kalman Filter}\label{subsubsec:KF}
\editComment{More detail - including on the covariance matrix ...}
Coarse \rphi helix parameters out of the \HT are used as the initial variables for track finding, with the segment assignment also providing a good seed value.
Given that in simulation over half the track candidates from by the HT are considered to be \textit{fake} or contain at least one stub associated with another particle, a Kalman Filter is used to both remove these incorrect stubs and reject fake tracks. 

In addition to the advantages of the Kalman filter for track reconstruction discussed by Fr{\"u}hwirth in \cite{Fruhwirth:1987fm}, the algorithm has several aspects making it suitable for FPGA implementation compared to global track fitting methods, namely the matrices:

\begin{itemize}
\item {are small.}
\item {are size independent of the number of measurements.}
\item {only involve the inversion of a small matrix.}
\end{itemize}

The initial estimates, or \textit{state}, of the track parameters and their uncertainties, $\chi^2$ value and other status information are updated by the KF iteratively applying stubs to update the state following the Kalman formalism, decreasing the uncertainty in the state. 
Each update of the state can be filtered on number of configurable criteria, including \pT, $\chi^2$, and the minimum number of stubs from PS modules, and can take into account and skip missing missing layers due to missing or incorrect stubs.
In the event multiple stubs are found on the same layer, each can be propagated with up to the four best states being kept and presented to a final state selector, with preference given to states with the fewest missing layers and the smallest $\chi^2$.
The final fit is always performed after a fixed period of time, so consequently there is no truncation in the traditional sense as all candidates will be read out, although events such as dense jets with many candidates and stubs per candidate will only be partially filtered.

A greater in-depth discussion of the mathematics and implementation of online track reconstruction using Kalman Filters on FPGAs in \cite{SSummers}.

\subsubsection{Duplicate Removal}
At the input to the DR, over half of the track candidates are unwanted duplicate tracks created by the HT.
By understanding how the HT produces these duplicate tracks, a more elegant and subtle DR algorithm can be used instead of having to compare pairs of tracks to see if they are the same as each other.
This approach is illustrated in figure~\ref{fig:DR}, where five stubs (blue lines in Hough Space) produce three candidates (green and yellow cells).
As all three candidates contain the same stubs, they will be fitted with identical helix parameters in the same cell (the yellow cell) regardless of the original HT cell.
The algorithm accepts only tracks whose fitted parameters are consistent to those that the HT found them in initially. There is however, a small subtlety, given that the algorithm eliminates unique tracks whose fitted parameters were not consistent, which results in the loss of a few percent of efficiency. 
By performing a second pass through the rejected tracks and rescuing those which are unique the lost efficiency can be recovered.

\begin{figure}[!h]
\centering
\includegraphics[width=0.80\textwidth]{figs/tk-upgrade/A50_algo.pdf}
% where an .eps filename suffix will be assumed under latex,
% and a .pdf suffix will be assumed for pdflatex; or what has been declared
% via \DeclareGraphicsExtensions.
\caption{Illustration of how duplicates are formed by the \rphi \HT.}
\label{fig:DR}
\end{figure}

A more detailed description of the firmware implementation of the \DR for the demonstrator system is discussed in~\cite{TMTT_JINST}.

\section{Simulation Studies}\label{sec:TmttSimStudies}
During the development of the \emph{TMTT} demonstrator system both before and following the December 2016 review, the author was involved in a number of simulation studies, the more substantive of which are presented below.

For these results, the reconstruction efficiency is measured relative to all generated charged particles from the primary interaction that produce stubs in at least four layers of the tracker and lies within $\pT > 3\GeV$, $|\eta| < 2.4$, $|z_{0}| < 30\cm$ and $d_{xy} < 1\cm$.
A track is a defined as being correctly reconstructed or \emph{matched} if the reconstructed track has stubs associated to the particle in at least four tracker layers.
Those track which fail this criteria are known either as \emph{unmatched} or \emph{fake} tracks.
If all a reconstructed track's stubs originated from the same particle, the track is defined as being \emph{perfectly} reconstructed.
This stricter latter definition is only used in quoting results from the entire chain (\ie all four components of the TFP discussed in Chapter~\ref{subsec:TFP}), as the presence of stubs incorrectly associated to a track is to be expected if only part of the TFP chain has been run.
If the reconstruction of a charged particle produces more than one track, these additional tracks are considered to be \emph{duplicates}.

\subsection{Linearised $\chi^{2}$ Track Fitter}\label{subsec:chi2}
Whilst the Kalman Filter was used as a track fitter for the December 2016 hardware demonstrator, a Linear Regression (LR) and a Linearised $\chi^{2}$ fitting algorithms were explored.
Both track fitting algorithms require a proceeding \emph{Seed Filter} (SF) stage to remove stubs in a HT cell which are inconsistent with a straight line in the \emph{r-z} plane to filter out both fake tracks and stubs incorrectly assigned to tracks.
The LR fitter~\cite{TMTT_FLP} was developed as an alternative to the KF and exploits the fact that sufficiently high \pT tracks should form a straight line in the \emph{\rphi} and \emph{r-z} planes to perform independent fits in each planes.
The linearised $\chi^{2}$ fit makes use of residuals between the stubs and the seeded track that minimise $\chi^{2}$ of the fit in order to obtain improved helix parameters for the track candidate.

As the \emph{TMTT} project was formed a significantly after the other two L1 track finder projects, there were naturally greater time and resource pressures in developing the multiple components of a complete system, including the ability to perform precision helix parameter fits on the \HT output with an algorithm that could be  implemented in hardware. 
Following discussions with members of the \emph{tracklet} and \emph{AM} projects, it was decided that the use of a linearised $\chi^{2}$ fit based off the one proposed by the \emph{tracklet} project would be investigated.
The mathematics forming the basis for the \emph{TMTT}'s implementation of the algorithm is detailed in~\cite{CMS_DN-14-043}. 
\editComment{Need to check re. there being a publicly available copy? Emailed Louise}

\subsubsection{General Form of a $\chi^{2}$ Fit}\label{subsubsec:chi2maths}
The parameters of each stub $s_{i}$, based on the track’s helix parameters $h$, are initially linearly expanded around the estimate of the helix parameters $h_{0}$:

\begin{equation}
s_{i}(h) = s_{i}(\overline{h_{0}} + \delta h) = s_{i}(\overline{h_{0}}) + \delta h \frac{\delta s_{i}}{\delta h} + \mathcal{O}(\delta h^{2}) \;.
\label{eq:chi1}
\end{equation}

With $\chi^{2}$ being defined as:

\begin{equation}
\chi^{2} = \frac{\sum_{i}\(s_{i}(h) - s_{track}(\overline{h_{0}}))\^{2}}{\sigma^{2}_{s}}  \;.
          = \frac{\sum_{i}\(s_{i}(\overline{h_{0}} - s_{track}(\overline{h_{0}}) + \delta h \frac{\delta s_{i}}{\delta h}))\^{2}}{\sigma^{2}_{s}} \;.
\label{eq:chi2}
\end{equation}

Which, in order to obtain $\delta h$, is minimised:

\begin{equation}
0 = \frac{\partial \chi^{2}}{\partial \delta h} = 2 \sum_{i}\frac{\partial s_{i}}{\delta h}(s_{i}(\overline{h_{0}}) - s_{track}(\overline{h_{0}}) + \delta h \frac{\partial s_{i}}{\partial h}) + 2 \sum_{i}\frac{\partial s_{i}}{\delta h}(\delta s_{i}^{T} + \delta h \frac{\partial s_{i}}{\partial h}) \;.
\label{eq:chi3}
\end{equation}

Where $\delta s_{i}^{T} = s_{i}(\overline{h_{0}}) - s_{track}(\overline{h_{0}})$ are the residuals between the hits and the seeded track.

By defining the matrices $D_{ij} = \frac{1}{\sigma_{i}} \frac{\delta s_{i}}{\delta h_{j}}$ and $M = D^{T} D$, Eqn.~\ref{eq:chi3} can be rewritten and solved for $\delta h$ as:

\begin{equation}
0 = D^{T} \delta s^{T} + M \delta h \Rightarrow \delta h = - M^{-1} D^{T} \delta s^{T} \;.
\label{eq:chi4}
\end{equation}

Eqn.~\ref{eq:chi4} gives a linear form which can be solved for $\delta h$ through updating the residuals with respect to the seeded track candidate.

Calculation of the derivatives for the barrel layer and endcap disk hits from this general form, including a correction factor for $\phi$ in the outer disks to account for the fact that these modules do not point directly towards the interaction point, are given in~\ref{app:chi2}.

Similarly the $chi^{2}$ of the fit can also be expressed in a linear form:

\begin{equation}
chi^{2} = (\delta s^{T} + D \delta h)^{T}(\delta s^{T} + D \delta h)\;.
        = (\delta s^{T} - DM^{-1}D^{T}\delta s^{t})^{T} (\delta s^{T} - DM^{-1}D^{T}\delta s^{t})\;.
        + \delta s^{T} (1 - DM^{-1}D^{T}) (1 - DM^{-1}D^{T}) \delta s
\label{eq:chi5}
\end{equation}

\subsubsection{Software Results and Firmware Feasibility}\label{subsubsec:chi2software}
Initially a floating point software implementation of the algorithm, using the derivatives derived from the general form of Eqn.~\ref{eq:chi4} and the $chi^{2}$ of the fit from  Eqn.~\ref{eq:chi5}, was used to validate the algorithm and optimise its performance.

Given that a significant proportion of tracks reconstructed by the \HT contain at least one incorrect stub, up to 40\% for \ttbar events at \PU of 200, and each additional iteration of the fitting algorithm yields diminishing improvements, the residuals associated to each stub were considered following each fitting iteration.
%%% Fake rate?!
\editComment{Fake rate impact?}
Stubs which are incorrectly associated with a track are expected have larger residuals than those which are correctly associated.
Therefore once the residuals of each stub were known following a fit, if the stub with the largest residual exceeded a predefined cut it was removed from the track and the track was refitted with its reduced collection of hits. 
During the optimisation of the cut, it was found that this approach had the potential to leave a track with fewer stubs than the minimum of four required required and lead to a matched track being discarded.
To avoid discarding matched tracks whilst retaining the benefits of improving the purity of the tracks and their helix parameter resolutions and removing fake tracks, a looser residual cut was applied for tracks only comprised of four stubs.

Figs.~\ref{} 
%%%100 events - 15 iterations (default)
%=========================================================================
%               TRACK-FINDING SUMMARY (before track fit)
%Number of track candidates found per event = 349.9800 +- 10.4617
%                     with mean stubs/track = 7.3075
%Fraction of track cands that are fake = 0.4415
%Fraction of track cands that are genuine, but extra duplicates = 0.3674
%Algorithmic tracking efficiency = 1640/1686 = 0.9727 +- 0.0040
%Perfect algorithmic tracking efficiency = 690/1686 = 0.4093 +- 0.0120 (no incorrect hits)
%=========================================================================
%                    GENERAL FITTING SUMMARY FOR: ChiSquared4ParamsApprox
%Number of fitted track candidates found per event = 87.6600 +- 1.8487
%                            with mean stubs/track = 5.5485
%Fraction of fitted tracks that are fake = 0.1189
%Fraction of fitted tracks that are genuine, but extra duplicates (post-cut) = 0.2008
%Algorithmic fitting efficiency (post-cut) = 1532/1686 = 0.9087 +- 0.0070
%Perfect algorithmic fitting efficiency(post-cut) = 1450/1686 = 0.8600 +- 0.0084 (no incorrect hits)
%=========================================================================
%                    GENERAL FITTING SUMMARY FOR: ChiSquared4ParamsTrackletStyle
%Number of fitted track candidates found per event = 86.8200 +- 1.8216
%                            with mean stubs/track = 5.5682
%Fraction of fitted tracks that are fake = 0.1186
%Fraction of fitted tracks that are genuine, but extra duplicates (post-cut) = 0.1864
%Algorithmic fitting efficiency (post-cut) = 1530/1686 = 0.9075 +- 0.0071
%Perfect algorithmic fitting efficiency(post-cut) = 1458/1686 = 0.8648 +- 0.0083 (no incorrect hits)
%=========================================================================
%
% 10 iterations
%=========================================================================
%                    GENERAL FITTING SUMMARY FOR: ChiSquared4ParamsApprox
%Number of fitted track candidates found per event = 87.7600 +- 1.8526
%                            with mean stubs/track = 5.5546
%Fraction of fitted tracks that are fake = 0.1195
%Fraction of fitted tracks that are genuine, but extra duplicates (post-cut) = 0.2007
%Algorithmic fitting efficiency (post-cut) = 1532/1686 = 0.9087 +- 0.0070
%Perfect algorithmic fitting efficiency(post-cut) = 1445/1686 = 0.8571 +- 0.0085 (no incorrect hits)
%=========================================================================
%                    GENERAL FITTING SUMMARY FOR: ChiSquared4ParamsTrackletStyle
%Number of fitted track candidates found per event = 86.9400 +- 1.8229
%                            with mean stubs/track = 5.5749
%Fraction of fitted tracks that are fake = 0.1190
%Fraction of fitted tracks that are genuine, but extra duplicates (post-cut) = 0.1865
%Algorithmic fitting efficiency (post-cut) = 1532/1686 = 0.9087 +- 0.0070
%Perfect algorithmic fitting efficiency(post-cut) = 1455/1686 = 0.8630 +- 0.0084 (no incorrect hits)
%=========================================================================
%
% 5 iterations
%=========================================================================
%                    GENERAL FITTING SUMMARY FOR: ChiSquared4ParamsApprox
%Number of fitted track candidates found per event = 95.1200 +- 2.1088
%                            with mean stubs/track = 5.9711
%Fraction of fitted tracks that are fake = 0.1605
%Fraction of fitted tracks that are genuine, but extra duplicates (post-cut) = 0.2055
%Algorithmic fitting efficiency (post-cut) = 1549/1686 = 0.9187 +- 0.0067
%Perfect algorithmic fitting efficiency(post-cut) = 1356/1686 = 0.8043 +- 0.0097 (no incorrect hits)
%=========================================================================
%                    GENERAL FITTING SUMMARY FOR: ChiSquared4ParamsTrackletStyle
%Number of fitted track candidates found per event = 94.5600 +- 2.1009
%                            with mean stubs/track = 5.9852
%Fraction of fitted tracks that are fake = 0.1592
%Fraction of fitted tracks that are genuine, but extra duplicates (post-cut) = 0.1952
%Algorithmic fitting efficiency (post-cut) = 1554/1686 = 0.9217 +- 0.0065
%Perfect algorithmic fitting efficiency(post-cut) = 1368/1686 = 0.8114 +- 0.0095 (no incorrect hits)
%=========================================================================
%
% 3 iterations
%=========================================================================
%                    GENERAL FITTING SUMMARY FOR: ChiSquared4ParamsApprox
%Number of fitted track candidates found per event = 123.7500 +- 2.9630
%                            with mean stubs/track = 6.6752
%Fraction of fitted tracks that are fake = 0.2790
%Fraction of fitted tracks that are genuine, but extra duplicates (post-cut) = 0.2188
%Algorithmic fitting efficiency (post-cut) = 1586/1686 = 0.9407 +- 0.0058
%Perfect algorithmic fitting efficiency(post-cut) = 1133/1686 = 0.6720 +- 0.0114 (no incorrect hits)
%=========================================================================
%                    GENERAL FITTING SUMMARY FOR: ChiSquared4ParamsTrackletStyle
%Number of fitted track candidates found per event = 123.6400 +- 3.0606
%                            with mean stubs/track = 6.6933
%Fraction of fitted tracks that are fake = 0.2815
%Fraction of fitted tracks that are genuine, but extra duplicates (post-cut) = 0.2124
%Algorithmic fitting efficiency (post-cut) = 1593/1686 = 0.9448 +- 0.0056
%Perfect algorithmic fitting efficiency(post-cut) = 1140/1686 = 0.6762 +- 0.0114 (no incorrect hits)
%=========================================================================
% 2 iterations
%=========================================================================
%                    GENERAL FITTING SUMMARY FOR: ChiSquared4ParamsApprox
%Number of fitted track candidates found per event = 166.3600 +- 4.1332
%                            with mean stubs/track = 7.0210
%Fraction of fitted tracks that are fake = 0.3949
%Fraction of fitted tracks that are genuine, but extra duplicates (post-cut) = 0.2199
%Algorithmic fitting efficiency (post-cut) = 1612/1686 = 0.9561 +- 0.0050
%Perfect algorithmic fitting efficiency(post-cut) = 886/1686 = 0.5255 +- 0.0122 (no incorrect hits)
%=========================================================================
%                    GENERAL FITTING SUMMARY FOR: ChiSquared4ParamsTrackletStyle
%Number of fitted track candidates found per event = 165.7800 +- 4.1926
%                            with mean stubs/track = 7.0332
%Fraction of fitted tracks that are fake = 0.3932
%Fraction of fitted tracks that are genuine, but extra duplicates (post-cut) = 0.2194
%Algorithmic fitting efficiency (post-cut) = 1617/1686 = 0.9591 +- 0.0048
%Perfect algorithmic fitting efficiency(post-cut) = 885/1686 = 0.5249 +- 0.0122 (no incorrect hits)
%=========================================================================
%
% 1 iterations
%=========================================================================
%                    GENERAL FITTING SUMMARY FOR: ChiSquared4ParamsApprox
%Number of fitted track candidates found per event = 243.3100 +- 6.1111
%                            with mean stubs/track = 7.2477
%Fraction of fitted tracks that are fake = 0.5061
%Fraction of fitted tracks that are genuine, but extra duplicates (post-cut) = 0.2212
%Algorithmic fitting efficiency (post-cut) = 1633/1686 = 0.9686 +- 0.0042
%Perfect algorithmic fitting efficiency(post-cut) = 544/1686 = 0.3227 +- 0.0114 (no incorrect hits)
%=========================================================================
%                    GENERAL FITTING SUMMARY FOR: ChiSquared4ParamsTrackletStyle
%Number of fitted track candidates found per event = 248.2600 +- 6.2051
%                            with mean stubs/track = 7.2671
%Fraction of fitted tracks that are fake = 0.5021
%Fraction of fitted tracks that are genuine, but extra duplicates (post-cut) = 0.2289
%Algorithmic fitting efficiency (post-cut) = 1637/1686 = 0.9709 +- 0.0041
%Perfect algorithmic fitting efficiency(post-cut) = 544/1686 = 0.3227 +- 0.0114 (no incorrect hits)
%=========================================================================

%% KF4
%=========================================================================
%                    GENERAL FITTING SUMMARY FOR: KF4ParamsComb
%Number of fitted track candidates found per event = 81.8900 +- 1.6735
%                            with mean stubs/track = 4.0000
%Fraction of fitted tracks that are fake = 0.2088
%Fraction of fitted tracks that are genuine, but extra duplicates (post-cut) = 0.0474
%Algorithmic fitting efficiency (post-cut) = 1590/1686 = 0.9431 +- 0.0056
%Perfect algorithmic fitting efficiency(post-cut) = 1590/1686 = 0.9431 +- 0.0056 (no incorrect hits)
%=========================================================================
%
%%% discuss results
\editComment{Discuss results!}
\editComment{Comment on purity, resolution, fake rate, and duplicate rate}
\editComment{Impact on killing genuine stubs assoc. with track?}

%%% Discuss making approximations and reducing precision to that of digital input
Following the validation the floating point implementation of the linear $\chi^{2}$ track fitter algorithm's performance in software, a version using approximated calculations of the track derivatives and fixed precision parameters was developed in order to help determine the feasibility of implementing the algorithm in hardware .

The resource constraints facing a L1 track finder rules out the use of floating-point calculations in firmware, given their resource intensive nature, necessitating the use of fixed-point arithmetic.
As the precision of the input track parameters will be limited by the number of bits the \HT uses to store each parameter, the precision used by was the natural starting point.

When considering the minimum parameter resolution which would not considerably impact on the track reconstruction efficiency or helix parameter resolutions, .... \editComment{Find out how many are required, what precision?}

\editComment{Results in this sec done with 100 ttbar+pu200 events. Larger stats (~5000) runs will be done prior to final draft.}

%% Matrix sizes

Whilst FPGAs can easily perform the addition and multiplication operations required to compute the matrices discussed in Chatper~\ref{subsubsec:chi2maths}, the calculation of the derivatives that form their elements would not trivial for them given the presence of divisions and trigonometric functions.
The use of lookup tables (LUTs) would therefore be essential in order to ensure latency budgets are not exceeded.
Using approximations for the track derivatives' calculations 

As the first order approximation of the derivatives for the barrel hits only depend on the radial position of the hit (\ie one of the six possible barrel layer radial positions), their implementation would be trivial with only a small number of LUTs (\editComment{insert number and explain}) being required.

The endcap regions however, are more complicated, with the derivatives having additional dependencies on $\tan$, $\lambda$, the radius of curvature of the track, and for the outer modules, a correction factor to obtain a measurement of $\phi$, 


As shown above, the size of the LUTs required for the endcap regions considerably exceeds the resources of those present on currently available FPGAs, such as the $433 \times 10^{3}$ LUTs available on the Virtex 7 690 FPGAs used by the MP7s in the demonstrator.
Whilst a future FPGA could implement this track fitting algorithm, it would not be feasible to implement on a current generation chip.
Given that one of the motivating factors behind this TFP design was to be able to demonstrate a complete system with currently available technology, and at the time of development the \KF had demonstrated superior performance and been shown to be feasible to implement in firmware, it was determined that work on the linearised $chi^{2}$ fitter would not be continued for the immediate future.


%Each CLB has 2 slices, 8 luts and 16 flip flops (8 storage element per slice). So multiply the number of slices in the data sheet by four and you get the number of luts

\subsubsection{Outlook}\label{subsubsec:chi2outlook}
%%%Future work
For the linearised $chi^{2}$ track fitting algorithm to be considered in the future, the following would to be addressed:
\begin{itemize}
\item An improved track fitting efficiency which obtains a high, if not 100\%, tracks purity, in order to be competitive with both the \KF and \LR which are currently able achieve 100\% purity.
\item Can the algorithm resources be reduced and/or are there sufficient resources on the board be implement the algorithm with.
\item Whether or not the inclusion of the fifth helix parameter, the vertex impact parameter in the x-y plane, $d_{0}$, would provide comparable or improved performance to other fitting algorithms which produce it.
\item The cause of the increased number of duplicate tracks compared to the \KF, and whether or not they can be reduced to a similar level.
\end{itemize}

\subsection{2 GeV Tracking}\label{subsec:Tmtt2GeV}
The flexibility to reconstruct tracks down to a lower \pT threshold of 2\GeV may be desirable if the trigger requirements demand it and the impact of this potential requirement on the proposed proposed track-finder system was studied.
These studies were initially undertaken as part of the the robustness studies required for the December 2016 demonstrator review, focussing on recovering tracking efficiency below 3\GeV with the \HT, and were subsequently built upon with modifications to the \KF algorithm. 
The results relating to \HT modifications were produced prior to the conclusion of the demonstrator review and were produced using the flat barrel geometry, and the results for the \KF improvements produced with the tilted barrel geometry.

\subsubsection{Hough Transform Optimisation}
Lowering the \HT \pT threshold from 3\GeV to 2\GeV required modifying the GP and HT configuration parameters to ensure adequate duplication in $\phi$ and increasing the number of the \qpt columns by 50\% to take into account the increased \pt range whilst maintaining the same precision, respectively.
The increased number of \qpt columns has the impact of increasing the required FPGA resources by 50\% and the output data rate from the \HT by a factor of 2.2.

Without any further modifications, there is a considerable degradation in the track reconstruction efficiency by the \HT in the range $2 < \pt < 2.7$\GeVc, due to these low momentum tracks being dominated by multiple scattering, resulting in stubs not always intersecting within a single \HT cell and thus failing to exceed the threshold criteria and generate track candidates.
To mitigate against such track reconstruction efficiency losses, the precision of the \HT cells along \qpt and $\phi_{T}$ for the range $2 < \pt < 2.7$\GeVc was reduced by a factor of two (\ie $2 \times 2$ cells were merged).
In addition to this, \KF state $\chi^2$ cuts for this low \pT range were optimised to reflect the decreased precision of the hits in these \HT cells and thus better reduce the number of duplicate and fake tracks as far as possible without impacting on the \HT track reconstruction efficiency.
This variable precision \HT, which has been separately implemented in firmware, together with the optimised \KF state cuts, has been shown in simulation to recover some of these losses, as shown in figure~\ref{fig:2GeVFlatEff}.

\begin{figure}[tbp]
\centering
\includegraphics[width=0.47\textwidth]{figs/tk-upgrade/results-lowPtTracking/htTrackingEffVsInvPtFlatGeometry_5000.pdf}
\includegraphics[width=0.47\textwidth]{figs/tk-upgrade/results-lowPtTracking/kfTrackingEffVsInvPtFlatGeometry_5000.pdf}
\caption{Plots post-\HT (left) and post-\KF (right) showing tracking efficiency in low pT range with only the number of \qpt columns increased (red), with the increased number of \qpt columns and \HT cell merging (black), and with the increased number of columns, HT cell merging and \KF state cuts optimisation(blue) for \ttbar events at \PU of 200. \editComment{Larger stats > 1000 + neater plots + legend to be added later.}}
\label{fig:2GeVFlatEff}	
\end{figure}

\editComment{Add plots to illustrate impact on resolutions? Here or in appendices? Worth including a table instead of plots? Table will be good for showing how different parts of the chain compare}

%\begin{table}[htbp]
%\topcaption {Track finding performance on simulated \ttbar events at a \PU of 200, after the \HT and the full chain (\HT + \KF + \DR) for the configurations of only increasing the number of \qpt columns (\emph{Default}), additionally using \HT cell merging (\emph{Merge}) and including \KF state cuts optimisation(\emph{Optimised}). The track finding efficiencies, the mean numbers of reconstructed tracks per event in the entire tracker, and the number of those tracks which are fakes or duplicates, are given using the efficiency definitions described in Chapter~\ref{TmttSimStudies}.}
%\label{tab:trackFindingPerformance2GeVHT}
%  \centering
%% This increases column spacing.
%  \addtolength{\tabcolsep}{1ex}
%% This right-aligns numbers in column, but centers them under column title.
%  \begin{tabular}{ccr@{\hspace{4ex}}r@{\hspace{4ex}}r@{\hspace{6ex}}}
%   \hline
%   \bf{Configuration} & \bf{Stage} & \bf{Efficiency [\%]} & \multicolumn{1}{r}{\bf{Total \# of tracks}} & \multicolumn{1}{r}{\bf{\# of fakes}} & \multicolumn{1}{r}{\bf{\# of duplicates}}  \\
%        \hline
%    Default & \bf{HT}     &  97.1 &  331  &  139 &   126 \\  
%    & \bf{Full chain}     &  94.4 &  79   &  16  &     3 \\      
%   \hline
%        \hline
%    Merge & \bf{HT}     &  97.1 &  331  &  139 &   126 \\  
%    & \bf{Full chain}     &  94.4 &  79   &  16  &     3 \\      
%   \hline
%        \hline
%    Optimised & \bf{HT}     &  97.1 &  331  &  139 &   126 \\  
%    & \bf{Full chain}     &  94.4 &  79   &  16  &     3 \\      
%   \hline
%   
% \end{tabular}
% \addtolength{\tabcolsep}{-1ex}
%\end{table}

%%% Results - 100 events
%% no merge
%=========================================================================
%               TRACK-FINDING SUMMARY (before track fit)
%Number of track candidates found per event = 718.6900 +- 15.9578
%                     with mean stubs/track = 7.1339
%Fraction of track cands that are fake = 0.3432
%Fraction of track cands that are genuine, but extra duplicates = 0.4418
%Algorithmic tracking efficiency = 2284/2439 = 0.9364 +- 0.0049
%Perfect algorithmic tracking efficiency = 1069/2439 = 0.4383 +- 0.0100 (no incorrect hits)
%=========================================================================
%                    GENERAL FITTING SUMMARY FOR: KF4ParamsComb
%Number of fitted track candidates found per event = 195.9300 +- 2.7182
%                            with mean stubs/track = 4.0000
%Fraction of fitted tracks that are fake = 0.2122
%Fraction of fitted tracks that are genuine, but extra duplicates (post-cut) = 0.0513
%Algorithmic fitting efficiency (post-cut) = 2176/2439 = 0.8922 +- 0.0063
%Perfect algorithmic fitting efficiency(post-cut) = 2176/2439 = 0.8922 +- 0.0063 (no incorrect hits)
%=========================================================================
%%  merge + no KF opt

%% optimised KF


Whilst the decreased precision HT cells at low \pT recovers a proportion of the lost tracks, figure~\ref{fig:2GeVFlatEff}  also shows that the \KF's performance is significantly degraded for $\pT < 3 \GeV$ compared to $\pT > 3 \GeV$.
These additional low \pT track losses are the result of the \KF not modelling the scattering of the expected hit positions.
This is illustrated by the distributions of $\chi^{2} \div \text{number of degrees of freedom}$ ($\frac{\chi^{2}}{ndf}$) as a function of $\frac{1}{\pT}$ in figure~\ref{fig:2GeVFlatChi2Ndf} for genuine tracks produced by the \KF, where $\frac{\chi^{2}}{ndf}$ is almost flat for a value of order unity for high \pT (low $\frac{1}{\pT}$) but begins to dramatically increase above approximately 3\GeV, in contrast to the ideal value of {$\frac{\chi^{2}}{ndf} = 1$ if all uncertainties were properly modelled.

\begin{figure}[tbp]
\centering
\includegraphics[width=\textwidth]{figs/tk-upgrade/results-lowPtTracking/kfChi2NdfVsInvPtFlatGeometry_5000.pdf}
\caption{Plot of $\frac{\chi^{2}}{ndf}$ as a function of $\frac{1}{\pT}$ for genuine tracks produced by the \KF.}
\label{fig:2GeVFlatChi2Ndf}
\end{figure}

\subsubsection{Kalman Filter Optimisation}
Incorporation of \MS into the \KF involved including a \emph{process noise} term, namely the variance of the multiple scattering angles, to the \emph{measurement noise} (\ie measurement error) term already present in the \KF covariance matrix.
In this updated form, the \KF now can consider stubs which are compatible with those that have undergone \MS, allowing for the reconstruction of tracks whose stubs were previously discarded and more accurate $\chi^{2}$ values which can be used to better discriminate against states of fake tracks.

For small deflection angles and relativistic particles, $\sigma_{\theta}$ for a each layer is given by~\cite{Lynch:1990sq}~:

\begin{equation}
\sigma_{\theta} = \frac{13.6\MeV}{\beta c p} q \sqrt{\frac{x}{X_{0}}} [1 + 0.088 \log_{10}{\frac{x}{X_{0}}}]  \;.
\label{eq:scatter1}
\end{equation}

where, the momentum, velocity, electrical charge of the incident particle and thickness of the scattering medium in radiation lengths are given by $p$, $\beta c$, $q$ and $\frac{x}/{X_{0}}$ respectively and the result being good to better than 11\%.

With the particles involved having relativistic velocities (\ie $\beta c \cong 1$) and scattering in the r-z plane ignored as the impact of multiple scattering is considerably smaller hit position resolution in r-z, the multiple scattering contribution in the \rphi plane can be expressed as:

\begin{equation}
\sigma_{\theta} = \frac{k}{\pT}
\label{eq:scatter2}
\end{equation}

where $k$ is a coefficient.

From the simplified form of Eq.~\ref{eq:scatter2}, two alternative forms of the coefficient $k$, which should require minimal resources and latency, were investigated:

\begin{itemize}
\item \textbf{constant coefficient - } a constant coefficient of the order of the average anticipated scattering angle is used as the anticipated typical scattering angle for $2-3\GeV$ tracks is of the order of a milliradian.
\item \textbf{layer dependent coefficient -} the coefficient used is a function of the layer ID (\ie which layer the stub is found) in order to take into account the impact of repeated scattering from passing through multiple layers increasing the uncertainty associated of the hit position.
\end{itemize}

The initial layer dependent coefficients were obtained through experimentally determining in simulation the \MS contribution to the observed variance in $\phi$.
Both these initial layer dependent coefficients and the initial constant coefficient of a milliradian were subsequently further optimised in order to recover as much tracking efficiency as  possible.
Similarly, the \KF state $chi^{2}$ cuts for both approaches were also tuned in order to reject the optimal number of fake and duplicate tracks without compromising on tracking efficiency.

During the comparative studies of these two approaches, it was observed in the results following the \DR stage that in contrast to the trend of the fraction of duplicates decreasing as \pT decreases, the fraction of duplications present increases above circa $3\GeV$, as shown in Fig~\ref{fig:2GeVfracDups}.
The presence of this trend following in tracks produced by the \HT prior to tracking fitting and \DR, implies that the \HT produces more duplicates at low \pT is the cause of this rather than a shortcoming of the \DR algorithm.
As the fraction of duplicate tracks produced where decreased precision \HT cells are used is well controlled, the \pT threshold for the $2 \times 2$ merging of \HT cells was increased from $2.7\GeV$ to $3.5\GeV$.
Whilst an increase in the duplicate rate is observed between $3.2\GeV$ and $5\GeV$ in figure~\ref{fig:2GeVfracDups}, a decreased number of duplicates were produced below $3.5\GeV$ which lead to an overall reduction in the number of duplicates produced of circa 2.8\%.
The increased \HT cell merging threshold also had the additional benefit of recovering an additional 0.2\% of the tracks which were previously lost to \MS.
Consequently, this change was adopted by the project and all the results presented below for the two \MS coefficient forms were produced using this increased threshold. 

\begin{figure}[tbp]
\centering
\includegraphics[width=0.47\textwidth]{figs/tk-upgrade/results-lowPtTracking/htFracDuplicatesVsInvPtTiltedGeometry_5000.pdf}
\includegraphics[width=0.47\textwidth]{figs/tk-upgrade/results-lowPtTracking/kfFracDuplicatesVsInvPtTiltedGeometry_5000.pdf}
\caption{The fraction of genuine tracks with duplicates as a function of $\frac{1}{\pT}$ following reconstruction by the \HT (left) and fitting and filtering by the \KF and \DR (right) for where the \HT cell merging \pT threshold is set to 2.7\GeV (red) and 3.5\GeV (blue). 
The constant coefficient for the \MS contribution approach was used for these \KF results.
\editComment{Add legend to plots and polish plots}
}
\label{fig:2GeVfracDups}
\end{figure}

In comparison to just the \HT optimisations alone, both approaches at incorporating the effects of MS in the \KF are capable of rejecting an additional 3\% of the incorrectly reconstructed tracks, increasing the tracking efficiency by circa 0.6\%, and improving upon the track parameter resolution for low \pT tracks in \ttbar at at \PU of 200.
The similar performance between the two coefficients though is due the amount of material traversed by a track not being constant for a single layer as 


%The amount of material traversed previously traversed between measurements will also vary, with some tracks featuring layers which have no stubs, and material contributions from the Inner Tracker, between the Inner and Outer Trackers and services needing consideration.

%%% Results

%% default vs Ian vs Alex
%% track efficiency vs pT; fake rate; pT res, eta res, phi0 res, z0 res

%%% No merge
%=========================================================================
%               TRACK-FINDING SUMMARY AFTER HOUGH TRANSFORM
%Number of track candidates found per event = 646.8546 +- 1.7561
%                     with mean stubs/track = 6.8726
%Fraction of track cands that are fake = 0.1913
%Fraction of track cands that are genuine, but extra duplicates = 0.5491
%Algorithmic tracking efficiency = 111949/117377 = 0.9538 +- 0.0006
%Perfect algorithmic tracking efficiency = 65155/117377 = 0.5551 +- 0.0015 (no incorrect hits)
%=========================================================================
%                    TRACK FIT SUMMARY FOR: KF4ParamsComb
%Number of fitted track candidates found per event = 193.7248 +- 0.4772
%                     with mean stubs/track = 4.0000
%Fraction of fitted tracks that are fake = 0.1083
%Fraction of fitted tracks that are genuine, but extra duplicates = 0.0648
%Algorithmic fitting efficiency = 109022/117377 = 0.9288 +- 0.0008
%Perfect algorithmic fitting efficiency = 109022/117377 = 0.9288 +- 0.0008 (no incorrect hits)
%=========================================================================
%%% Default MS - producing for 5000 events
%=========================================================================
%               TRACK-FINDING SUMMARY AFTER HOUGH TRANSFORM
%Number of track candidates found per event = 751.9842 +- 2.1109
%                     with mean stubs/track = 7.3469
%Fraction of track cands that are fake = 0.2820
%Fraction of track cands that are genuine, but extra duplicates = 0.4730
%Algorithmic tracking efficiency = 112949/117377 = 0.9623 +- 0.0006
%Perfect algorithmic tracking efficiency = 55340/117377 = 0.4715 +- 0.0015 (no incorrect hits)
%=========================================================================
%                    TRACK FIT SUMMARY FOR: KF4ParamsComb
%Number of fitted track candidates found per event = 215.9748 +- 0.5381
%                     with mean stubs/track = 4.0000
%Fraction of fitted tracks that are fake = 0.1332
%Fraction of fitted tracks that are genuine, but extra duplicates = 0.0943
%Algorithmic fitting efficiency = 109841/117377 = 0.9358 +- 0.0007
%Perfect algorithmic fitting efficiency = 109841/117377 = 0.9358 +- 0.0007 (no incorrect hits)
%=========================================================================
%%% Ian MS
%=========================================================================
%               TRACK-FINDING SUMMARY AFTER HOUGH TRANSFORM
%Number of track candidates found per event = 751.9842 +- 2.1109
%                     with mean stubs/track = 7.3469
%Fraction of track cands that are fake = 0.2820
%Fraction of track cands that are genuine, but extra duplicates = 0.4730
%Algorithmic tracking efficiency = 112949/117377 = 0.9623 +- 0.0006
%Perfect algorithmic tracking efficiency = 55340/117377 = 0.4715 +- 0.0015 (no incorrect hits)
%=========================================================================
%                    TRACK FIT SUMMARY FOR: KF4ParamsComb
%Number of fitted track candidates found per event = 216.3250 +- 0.5199
%                     with mean stubs/track = 4.0000
%Fraction of fitted tracks that are fake = 0.1033
%Fraction of fitted tracks that are genuine, but extra duplicates = 0.1124
%Algorithmic fitting efficiency = 110519/117377 = 0.9416 +- 0.0007
%Perfect algorithmic fitting efficiency = 110519/117377 = 0.9416 +- 0.0007 (no incorrect hits)
%=========================================================================
%%% Ian MS - 2p7
%=========================================================================
%               TRACK-FINDING SUMMARY AFTER HOUGH TRANSFORM
%Number of track candidates found per event = 712.2834 +- 1.9684
%                     with mean stubs/track = 7.2053
%Fraction of track cands that are fake = 0.2498
%Fraction of track cands that are genuine, but extra duplicates = 0.4922
%Algorithmic tracking efficiency = 112721/117377 = 0.9603 +- 0.0006
%Perfect algorithmic tracking efficiency = 59544/117377 = 0.5073 +- 0.0015 (no incorrect hits)
%=========================================================================
%                    TRACK FIT SUMMARY FOR: KF4ParamsComb
%Number of fitted track candidates found per event = 222.3146 +- 0.5343
%                     with mean stubs/track = 4.0000
%Fraction of fitted tracks that are fake = 0.0978
%Fraction of fitted tracks that are genuine, but extra duplicates = 0.1405
%Algorithmic fitting efficiency = 110347/117377 = 0.9401 +- 0.0007
%Perfect algorithmic fitting efficiency = 110347/117377 = 0.9401 +- 0.0007 (no incorrect hits)
%=========================================================================
%%% Alex MS
%=========================================================================
%               TRACK-FINDING SUMMARY AFTER HOUGH TRANSFORM
%Number of track candidates found per event = 751.9842 +- 2.1109
%                     with mean stubs/track = 7.3469
%Fraction of track cands that are fake = 0.2820
%Fraction of track cands that are genuine, but extra duplicates = 0.4730
%Algorithmic tracking efficiency = 112949/117377 = 0.9623 +- 0.0006
%Perfect algorithmic tracking efficiency = 55340/117377 = 0.4715 +- 0.0015 (no incorrect hits)
%=========================================================================
%                    TRACK FIT SUMMARY FOR: KF4ParamsComb
%Number of fitted track candidates found per event = 222.0674 +- 0.5324
%                     with mean stubs/track = 4.0000
%Fraction of fitted tracks that are fake = 0.1078
%Fraction of fitted tracks that are genuine, but extra duplicates = 0.1233
%Algorithmic fitting efficiency = 110553/117377 = 0.9419 +- 0.0007
%Perfect algorithmic fitting efficiency = 110553/117377 = 0.9419 +- 0.0007 (no incorrect hits)
%=========================================================================


%%% Improvements
There are a number of potential improvements for tracking down to 2\GeV which merit further investigation which include:
\begin{itemize}
\item determining suitable coefficients as functions of both \pT and $\eta$ experimentally through simulation in order to more accurately account for the amount of material traversed and thus a more accurate description of the uncertainty in the hit position caused by \MS.
\item whether separate \KF $\chi^{2}$ cuts for the \rphi and r-z planes could enhance performance, given that the dominant uncertainty contribution for the former varies depending on $\pT$.
\item \pt dependent threshold criteria for the \HT.
\item further optimisation of the \KF in relation to any such changes.
\end{itemize}


%%% OLD
%Despite the increase in track finding efficiency and the reduction in the number of fake tracks and duplicate tracks this rudimentary approximation can achieve (figure~\ref{fig:MS1}), the are only minor improvements in the track parameters resolution.
%This is not unexpected, given that a constant inflation in the \rphi plane hit resolution error is applied for each stub applied, whilst in reality the inflation would vary as a particle traverses material.
%The amount of material traversed previously traversed between measurements will also vary, with same tracks featuring layers which have no stubs, and material contributions from the Inner Tracker, between the Inner and Outer Trackers and services needing consideration.
%%%


\chapter{Event Simluation/Data Acquisition and Object Reconstruction}\label{chapter:data-mc}

\section{Event Simulation}

\section{Data Acquisition}

\section{Object Reconstruction}
\subsection{Initial Object Reconstruction}
\subsection{Particle-Flow}
\cite{CMS-PRF-14-001}
\subsection{High-Level Object Reconstruction}
\subsubsection{Jets}
\subsubsection{b-Jets}
\subsubsection{Missing Transverse Energy}


\chapter{Event Selection and Background Estimation for Single Top Physics Searches}\label{chapter:tzq-search}
Using the reconstructed high level objects produced
\section{Event Selection}
\subsection{Trigger}
\subsection{Event Cleaning}
\subsection{Lepton Selection}
\subsubsection{Electrons}
\subsubsection{Muons}
\subsubsection{Invariant Mass Requirements}
\subsection{Jet Selection and b-tagging requirements}

\section{Background Estimation}


\chapter{Background Estimation}\label{chapter:bkg}
Despite 

\section{Data-driven Background Esimation}\label{sec:dataDrivenBackground}
\subsection{Non-Prompt Leptons}\label{sec:NPLs}
Backgrounds which involve decays into lepton + jets and where at least one jet is incorrectly reconstructed as a lepton (predominately electrons) or a lepton from the decay of heavy quarks (predominately muons), which pass the lepton selection and isolation criteria, are estimated with data.

The estimation of this background uses the same methodology as when performing top quark pair production~\cite{CMS:2016syx} and same-sign SUSY searches~\cite{CMS:2015vqc}.
The vast majority of the same-sign event yields found are the result of non-lepton and charge misidentified leptons, with some contribution from prompt leptons.
As these backgrounds are independent of the charge of the lepton pairs, it is expected that the nominal (opposite-sign) sample would have a similar contribution \cite{CMS:2015vqc}.

To estimate this contribution of opposite-sign non-prompt leptons in data, the same-sign event yields with the expected prompt-lepton contribution subtracted, is multiplied by a ratio of opposite-sign over same-sign non-prompt lepton events taken from MC.

The method requires that the same-sign control region established uses the same selection criteria as the nominal signal region, albeit with same-sign lepton pairs instead of opposite-sign ones.
This control region is dominated by non-lepton lepton events, but also contains contributions from prompt lepton events, charge misidentification and real same-sign pairs.

This data driven estimate is obtained using the following equation:

\begin{equation}
 N_{data}^{OS non-prompt} = (N_{data}^{SS} - N^{SS}_{real + mis-ID}).\frac{N_{MC}^{OS non-prompt}}{N_{MC}^{SS non-prompt}}
\end{equation}

where $N_{data}^{SS}$ is the total number of same sign events observed in data, $N^{SS}_{real + mis-ID}$ is the expected number of real same-sign events and events with charge misidentification and $N_{MC}^{OS non-prompt}$ and $N_{MC}^{SS non-prompt}$ the number of opposite-sign and same-sign non-prompt leptons observed in MC used to appropriately scale the estimate.

This ratio of MC opposite-sign over same-sign events is referred to as R, and is calculated using generator level information from reconstructed objects which have matched to a generator level particle. R is calculated from the W + jets, \ttZ and \ttW leptonic decaying, and single top MC samples with sufficient statistics given that these processes are expected to be the predominant source of non-prompt leptons for this analysis. 

\begin{table}[!htbp]
\centering
\begin{tabular}{| l |  c |  c |  c |  c |  c |}
\hline
Source &  $ee$ & $\mu\mu$ & Combined \\ 
\hline
\ttbar (SS): & a$\pm$b &  c $\pm$d & e$\pm$f    \\
Z + jets (SS): & a$\pm$b &  c$\pm$d & e$\pm$r    \\
Single Top (SS): & a$\pm$b & c$\pm$d & e$\pm$r    \\
VV (SS): & a$\pm$b & c$\pm$d & e$\pm$f    \\
ttV (SS): & a$\pm$b &  c$\pm$d & e$\pm$f    \\ 
\hline
Total background (SS): & a$\pm$b & c$\pm$d & e$\pm$f   \\ 
Data: & a$\pm$b & c$\pm$d & e$\pm$f    \\ 
\hline
SS data (bkg): & a$\pm$b & c$\pm$d & e$\pm$f \\
\hline
Non-prompt (SS): & a$\pm$b & c$\pm$d & e$\pm$f \\
Non-prompt (OS): & a$\pm$b & c$\pm$d & e$\pm$f \\
R (OS/SS): & a$\pm$b & c$\pm$d & e$\pm$f \\
\hline
Non-prompt estimation: & a$\pm$b & c$\pm$d & e$\pm$f \\
\hline
\end{tabular}
\caption{Non-prompt lepton estimation following all selection cuts}
\label{tab:fakeLeptonYields}
\end{table}

\subsection{Z+jets background}\label{subsec:zPlusJetsEstimation}
Madgraph - normalises well but poor jet multiplicity
aMC@NLO - bad normalisation, but good higher jet multiplicity description

\section{Multivariate Analysis Techniques}\label{sec:mvas}
Single top 
\subsection{Boosted Decision Trees}\label{subsec:bdt}

BDT implemented in XGBoost Library
BDT features and hyperparameters chosen separately for ee and mumu channels
Features chosen using recursive feature elimination
Hyperparameters are selected by using a Gaussian process to optimise the classifier’s performance
hyperparameters = learning rate, n-estimators, max tree depth
feature = input variable

\subsubsection{BDT input variables}
Variables chosen

\begin{table}[htbp]
\topcaption { The name and descriptions of the variables chosen by recursive feature elimination to be used as input to the BDT to discriminate between potential tZq signal events and the dominant.
}
\label{tab:bdtVariables}
  \centering
% This increases column spacing.
\resizebox{\textwidth}{!}{
% This right-aligns numbers in column, but centers them under column title.
\begin{tabular}{cccc}
   \hline
   \bf{Variable} & \bf{Description} & \bf{$ee$} & \bf{$\mu\mu$} \\
   \hline
    bTagDisc & b-tag discriminator of the leading b-tagged jet & $\checkmark$ & $\checkmark$ \\
    fourthJetPt & \pt of the fourth jet & $\checkmark$ & $\checkmark$ \\
    jetHt & Total \HT of every jet & $X$ & $\checkmark$ \\
    jetMass & Total mass of every jet & $\checkmark$ & $\checkmark$ \\
    jjDelR & $\Delta R$ between the leading jets & $\checkmark$ & $\checkmark$ \\
    leadJetEta & $\eta$ of the leading jet & $\checkmark$ & $\checkmark$ \\
    leadJetPt & \pt of the leading jet & $\checkmark$ & $\checkmark$ \\
    met & \met & $\checkmark$ & $\checkmark$ \\
    secJetPt & \pt of the second jet & $\checkmark$ & $\checkmark$ \\
    thirdJetPt & \pt of the third jet & $\checkmark$ & $\checkmark$ \\
    topMass & $m_{top}$ & $\checkmark$ & $\checkmark$ \\
    totHtOverPt & Total \HT divided by total \pt & $\checkmark$ & $\checkmark$ \\
    wPairMass & $m_{W}$ & $\checkmark$ & $\checkmark$ \\
    wQuark2Eta & $\eta$ of the second W boson candidate jet & $X$ & $\checkmark$ \\
    wwdelR & $\Delta R$ between the W boson candidate jets & $\checkmark$ & $X$ \\
    zEta & $\eta$ of the Z boson & $X$$ & $\checkmark$ \\
    zHt & \HT of the Z boson & $\checkmark$ & $\checkmark$ \\
    zMass & $m_{Z}$ & $\checkmark$ & $\checkmark$ \\
    zTopDelR & $\Delta R$ between the Z boson and top quark & $X$ & $\checkmark$ \\
    zjminR & Minimum $\Delta R$ between the Z boson and a jet & $\checkmark$ & $\checkmark$ \\
    zlb1DelR & $\Delta R$ between the Z boson and leading b-tagged jet & $\checkmark$ & $X$ \\
   \hline
 \end{tabular}}
\end{table}

\editComment{LOTS of PLOTS of the input variable distributions}

\subsection{BDT Training and Output}
Each sample of events for each process considered is split into a training and testing sample.
\chapter{Systematic Uncertainties}\label{chapter:systematics}
%%% Intro
Understanding and minimising the impact of uncertainties in the resolution and efficiency of the detector and in the modelling of the simulation used to predict the signal and background processes is essential in order to make meaningful measurements.

%%% Sources
There are two 

These uncertainties, as well as the statistical uncertainties arising from the size of the simulated samples available, have been treated as nuisance parameters in the statistical fit model, as discussed in~\ref{chapter:results}.

\section{Experimental Uncertainties}
\subsection{Jet Energy Corrections}
As it has been observed that there are differences between simulated and jet energies , 
corrections are
https://arxiv.org/pdf/1607.03663.pdf
JINST 12 (2017) P02014

Smearing jets in MC simulation 

\subsection{Pileup Reweighting}
The \PU interactions included in the simulated samples used do not describe the number of primary interactions observed in data well, these interactions are estimated 
As such, they are reweighted to 

\subsection{b-tagging Uncertainties}
The uncertainties of the scale factors described in Chapter~\ref{subsec:btagEff} are obtained by varying their value by $\pm 1\sigma$, as calculated by the BTV POG. 
\editComment{Impact?}

\subsection{Parton Distribution Functions}
%%% ME_PS
\subsection{Non-prompt Lepton Contributions}
As this data-driven estimate of the instrumental backgrounds should have no dependence on either the lepton flavour or selection cuts, the variation of the ratio of opposite-sign over same-sign events as a function of the lepton flavour and the cut level should be well accounted for by a 30\% systematic uncertainty.

\subsection{Luminosity Uncertainties}
CMS uses five detectors, the pixel detector, DTs, HF, the Fast Beam Conditions Monitor and Pixel Luminosity Telescope to monitor and measure the instantaneous and integrated luminosity, with absolute calibrations of the detectors made through conducting Van der Meer (VdM) scans during dedicated LHC runs.
The luminosity value and its associated uncertainty used was determined by the CMS Luminosity Group using 
Pixel Cluster Counting (PCC) 

The overall uncertainty \cite{CMS:2017_lumi} %2016 lumi
\subsection{Lepton Efficiencies}
In order to account for the difference in performance for lepton trigger, identification, isolation and reconstruction efficiencies 

The lepton trigger efficiencies for both channels are measured independently, 

\section{Theoretical Uncertainties}

\subsection{Parton Density Functions}
%%Discussion of what PDFs are, is given in an earlier chapter
In order to calculate the uncertainties on the PDFs used in the generation of the MC samples considered, the 

A number of MC samples however, do not store the PDF weights as part of the event's 
\subsection{Factorisation and renormalisation scales}
The factorisation and renormalisation scales at the Matrix Element 
\subsection{Parton Shower Uncertainties}
\section{Impact of the Uncertainties}
The effect of each of the systematics considered on the event rate, in percentage, are shown in Table~\ref{tab:systImpact}.
These rates, whilst providing a useful insight into which of the systematics are the most important, do not show how the shape of each fitted variable and the MVA discriminant is influenced by each uncertainty.
\editComment{Make some comment on most important/impactful systematics and how better understanding them would improve the result}

\begin{table}[!htbp]
\begin{center}
\linespread{2}
\resizebox{\textwidth}{!}{\begin{tabular}{|l|c|c|c|c|}
\hline
Systematic      &  tZq                  & DY                   & \ttbar{}                  & Other         \\
($ee$ / $\mu\mu$) & (\%)  & (\%)  & (\%)  & (\%)  \\
\hline
Trigger             &  $_{-4.23\%}^{+4.24\%}$ /  $_{-0.21\%}^{+6.07\%}$   & $_{-4.72\%}^{+4.07\%}$ / $_{-0.32\%}^{+6.37\%}$  & $_{-5.08\%}^{+4.41\%}$ / $_{-0.55\%}^{+5.54\%}$ & $_{-4.72\%}^{+4.85\%}$ / $_{-4.47\%}^{+5.97\%}$  \\
JER             &  $_{-5.27\%}^{+6.02\%}$ /  $_{-6.11\%}^{+5.39\%}$   & $_{-11.81\%}^{+16.54\%}$ / $_{-14.18\%}^{+16.71\%}$  & $_{-7.98\%}^{+7.84\%}$ / $_{-6.13\%}^{+8.24\%}$  & $_{--1.96\%}^{+2.11\%}$ / $_{-1.62\%}^{+1.82\%}$  \\
JES             &  $_{-0.04\%}^{+0.19\%}$ /  $_{-0.13\%}^{+0.13\%}$   & $_{-0.55\%}^{+0.29\%}$ / $_{-0.17\%}^{+0.13\%}$  & $_{-1.30\%}^{+0.02\%}$ / $_{-0.20\%}^{+0.20\%}$  & $_{-0.0.01\%}^{+0.11\%}$ / $_{-0.14\%}^{+0.18\%}$  \\
Pileup             &  $_{-0.42\%}^{+0.43\%}$ /  $_{-0.17\%}^{+0.43\%}$   & $_{-2.35\%}^{+2.26\%}$ / $_{-2.57\%}^{+1.75\%}$  & $_{-1.52\%}^{+0.52\%}$ / $_{-0.09\%}^{+1.35\%}$  & $_{-0.86\%}^{+0.38\%}$ / $_{-0.15\%}^{+0.26\%}$  \\
bTag             &  $_{-2.78\%}^{+3.38\%}$ /  $_{-3.38\%}^{+2.99\%}$   & $_{-5.30\%}^{+5.11\%}$ / $_{-5.02\%}^{+5.12\%}$  & $_{-2.89\%}^{+3.02\%}$ / $_{-3.12\%}^{+3.77\%}$  & $_{-3.43\%}^{+3.25\%}$ / $_{-3.24\%}^{+3.00\%}$  \\    
PDF             &  $_{-9.98\%}^{+13.22\%}$ /  $_{-9.24\%}^{+11.94\%}$   & $_{-1.56\%}^{+1.73\%}$ / $_{-2.95\%}^{+2.16\%}$  & $_{-2.99\%}^{+1.85\%}$ / $_{-2.95\%}^{+2.16\%}$  & $_{-8.56\%}^{+9.95\%}$ / $_{-8.51\%}^{+9.40\%}$  \\
$Q^{2}$Scaling             &  $_{-2.82\%}^{+1.36\%}$ /  $_{-3.06\%}^{+1.33\%}$   & $_{-15.00\%}^{+2.92\%}$ / $_{-14.64\%}^{+2.05\%}$  & $_{-11.38\%}^{-1.38\%}$ / $_{-11.40\%}^{+0.0\%}$  & $_{-5.01\%}^{+1.37\%}$ / $_{-5.07\%}^{+1.8\%}$  \\
\hline
\end{tabular}
}
\caption{Rate impact of systematics on MC templates}\label{tab:systImpact}
\end{center}
\end{table}


\chapter{Results}\label{chapter:results}

\section{Statistical Methodology}\label{sec:statisticalModel}
The calculation of cross section and statistical significances were made using Higgs Analysis Combined Limit (\combine) tool~\cite{Combine}, which is based on the RooStats package~\cite{Moneta:2010pm,Schott:2012zb}.
The \combine tool is used to determine the signal strength using a binned Maximum Likelihood Fit (MLF) and the significances using using an asymptotic approximation~\cite{AsymptoticFormulae}, using the Asimov dataset.

The likelihood function is the product of the Pois


\begin{equation}
L = \prod\limits_{i=1}^{\N} _{i} \frac{•}{•} \;
\label{eq:poissonLikelihood}
\end{equation}

\begin{equation}
\mathcal{L} = - L = \sum\limits_{i=1}^{\N} _{i} \frac{\mu_{i}}{•} \;
\label{eq:minLogLikelihood}
\end{equation}

In the following section the signal and background yields are referred to as $s$ and $b$ respectively, where both represent event counts in the probability distribution function bins.
The uncertainties for the simulated 

All the systematic uncertainties were incorporated into the fit as nuisance parameters.
The normalisation uncertainties are incorporated into the fit as log-normal nuisance parameters

Morphing for shapes

Significia

 
The fit was performed on the two channels simultaneously. 
Most systematics were assumed to be 100\% correlated between channels.

\subsection{Confidence Levels Method}\label{subsec:CLs}
The 

confidence levels are determined on modified classical frequentist methods 

As there were sufficient statistics , the \emph{Asymptotic} CL$_{s}$ method.
This method uses one representive dataset, known as the \emph{Asimov dataset}, in lieu of an ensemble of toy MC samples.
The Asimov dataset is constructed such that ...
A full description of this methodology is given in~\cite{Cowan:2010js}.

\subsection{Data-driven background normalisation}\label{subsec:combineNormalisation}
floating in the fit



\section{Impact of the Systematic Uncertainties}\label{sec:uncertainitiesImpact}
The effect of each of the sources of systematic uncertainty considered in terms of the pull ($\frac{ \hat{\theta} - \theta_{0} }{\Delta \theta}$) and the postfit impact of varying the sources of uncertainty by $\pm 1 \sigma$ are shown in figure~\ref{fig:systematicsPull}.

\editComment{Remark on the lumi, jer are the largest uncerts and how the rest of the experimental uncerts are considerably smaller}

\begin{figure}[htbp]
\begin{center}
\includegraphics[width=0.97\textwidth]{figs/results/systematicsImpact.pdf}
\caption{The impact of each of the systematic uncertainties considered on the measurement made.}
\label{fig:systematicsPull}
\end{center}
\end{figure}

\section{Cross section extraction}
The cross section is 


By performing a simultaneous fit of the BDT discriminant distribution for the background-enriched sample and the BDT discriminant in the signal sample, any events in excess of the background-only hypothesis will be determined.

This excess can then be compared to the SM expectation for tZq production in order to calculate the observed signal strength and measure the cross section.

A measured signal strength of 0.0 would correspond to an observation of the background-only hypothesis alone, whilst 1.0 is the SM expectation for tZq sproduction.

\section{Signal strength significance}
The observed signal strengths, measured cross sections, and corresponding significances for the individual channels and the channels combined in the signal region using pseudo data, are shown in Table~\ref{tab:shapetxs}. 
These are [IN AGREEMENT / NOT IN AGREEMENT] with the SM cross section of  $X^{+Y}_{-Z}$.
 
\begin{table}[!h]
   \centering
   \caption{The observed signal strengths and corresponding cross sections for
   the individual channels and the channels combined at the 95\% CL.}
   \begin{tabular}{cccc}
       \hline
       Channel & $ee$ & $\mu\mu$ & \textbf{combination} \\
        \hline
        % \multicolumn{4}{c}{\combine{}} \\
        % \hline
        Signal strength & $X_{-Z}^{+Y}$ & $X_{-Z}^{+Y}$ & $X_{-Z}^{+Y}$ \\
       Cross section (fb) & $X_{-Z}^{+Y}$ & $X_{-Z}^{+Y}$ & $X_{-Z}^{+Y}$ \\
       Significance (expected) & $X_{-Z}^{+Y}$ & $X_{-Z}^{+Y}$ & $X_{-Z}^{+Y}$ \\
       Significance (observed) & $X_{-Z}^{+Y}$ & $X_{-Z}^{+Y}$ & $X_{-Z}^{+Y}$ \\
        \hline
        % \multicolumn{4}{c}{\textsc{Theta}} \\
        % \hline
        % Signal strength & $X_{-Z}^{+Y}$ & $X_{-Z}^{+Y}$ & $X_{-Z}^{+Y}$ \\
        % Cross section (fb) & $X_{-Z}^{+Y}$ & $X_{-Z}^{+Y}$ & $X_{-Z}^{+Y}$ \\
        % Significance (expected) & $X_{-Z}^{+Y}$ & $X_{-Z}^{+Y}$ & $X_{-Z}^{+Y}$ \\
        % Significance (observed) & $X_{-Z}^{+Y}$ & $X_{-Z}^{+Y}$ & $X_{-Z}^{+Y}$ \\
        % \hline
    \end{tabular}
   \label{tab:shapetxs}
\end{table}


\section{Interpretation of the results}

\section{Other results from the Large Hadron Collider}
The search for a singly produced top in association with a Z boson in the dilepton final state presented is the first one made at the LHC and follows in the footsteps of the searches for the trilepton final state at $\sqrt{8}$ and $\sqrt{13}$ using data collected by the CMS experiment in 2012 and 2016 respectively.

Despite the dilepton final state having a larger cross section than the trilepton final state, the different final state topology makes it much more difficult to isolate the signal process.
Consequently, the \editComment{expected/observed} significance of $A$ is smaller than the significances observed for the trilepton final state by CMS~\cite{Sirunyan:2017nbr} and ATLAS~\cite{Aaboud:2017ylb} at $\sqrt{13}$.


\chapter{Conclusion}\label{chapter:conclusion}

In this thesis, a search for the production of a singly produced top quark in association with a Z boson using proton-proton collision data recorded by the CMS detector at the LHC at $\sqrt{13}$ during 2016 and a number of software studies for a proposed track finder for a future CMS tracker upgrade were presented.

\section{Summary of Results}
Using the complete 2016 dataset of 35.9\fbinv, a shape based analysis was performed to search for tZq in the dilepton final state, using \editComment{a MVA technique} to aid in separating the signal process from the backgrounds, especially the dominant Z+jets background.
A cross section of $xxx^{yyy}_{zzz}$ fb was measured for the signal process, with an observed significance of $aaa \pm bbb \sigma$ compared to the expected significance from simulation of $ccc \pm ddd$. 
\editComment{Some comment about the significance of this result.}


Building on the successful development and demonstration of a track finding architecture for the CMS detector at the HL-LHC on currently available technology for both track finding and fitting charged particles with transverse momentum greater than 3\GeV at 40\MHz the latency constraints and operational conditions of the HL-LHC, 


where track candidates were identified using time-multiplexed Hough Transforms in the r-$\phi$ plane before being cleaned and fitted by a Kalman Filter and duplicates removed by exploiting how duplicates form in the Hough Transform. 
Whilst the demonstrator system had 

\section{Future CMS tZq measurements}
The tZq analysis

\section{Future development of an FPGA Based Track Finder for the CMS Tracker Upgrade}

%%% Future Improvements?
\subsubsection{Chi2}\label{subsubsec:chi2outlook}
%%%Future work
For the linearised $chi^{2}$ track fitting algorithm to be considered in the future, the following would to be addressed:
\begin{itemize}
\item An improved track fitting efficiency which obtains a high, if not 100\%, tracks purity, in order to be competitive with both the \KF and \LR which are currently able achieve 100\% purity.
\item Can the algorithm resources be reduced and/or are there sufficient resources on the board be implement the algorithm with.
\item Whether or not the inclusion of the fifth helix parameter, the vertex impact parameter in the x-y plane, $d_{0}$, would provide comparable or improved performance to other fitting algorithms which produce it.
\item The cause of the increased number of duplicate tracks compared to the \KF, and whether or not they can be reduced to a similar level.
\end{itemize}

\subsubsection{2 GeV}\label{subsubsec:2GeVoutlook}
There are a number of potential improvements for tracking down to 2\GeV which merit further investigation which include:
\begin{itemize}
\item understanding why the duplicate rate increases near the boundary between normal and reduced precision \HT cells increase when \MS is accounted for.
\item determining suitable coefficients as functions of both \pT and $\eta$ experimentally through simulation in order to more accurately account for the amount of material traversed and thus a more accurate description of the uncertainty in the hit position caused by \MS.
\item whether separate \KF $\chi^{2}$ cuts for the \rphi and r-z planes could enhance performance, given that the dominant uncertainty contribution for the former varies depending on $\pT$.
\item further studies into optimising the \pT threshold used for reducing the precision of \HT cells.
\item \pt dependent threshold criteria for the \HT.
\item further optimisation of the \KF in relation to any such changes.
\end{itemize}


\begin{appendices}
%\chapter{Calculation of the track derivatives used in Linearised $\chi^{2}$ Track Fitter}\label{app:chi2}
The derivation of the track derivatives that form the matrix elements of \emph{D} and \emph{M} in equations~\ref{eq:chi1}~--~\ref{eq:chi4} for the linearised $\chi^{2}$ track fit described in Chapter~\ref{tk-upgrade} was originally described in an internal CMS Detector Note~\cite{CMS_DN-14-043}.

In this appendix, it is shown how these track derivatives were obtained.
%
%By considering the $\phi$ and $z$ coordinates of a track's trajectory, as a function of r, 
%
%\begin{equation}
%\phi_{i} =  \phi_{0} - \arcsin ( \frac{r_{i} \rho^{-1}}{2} ) \;.
%z_{i} = z_{i} = z_{0} + 2 \rho t \arcsin (\frac{r_{i}}{2 \rho}) \;.
%\label{eq:projections}
%\end{equation}
%
%the residuals for both the barrel layer and endcap disk hits can be arrived at and approximated to first order.

\section{Barrel Layer Hits}
Considering that  

Wtuh 

\begin{equation}
f_{i}(\rho^{-1},\phi_{0}) = r_{i} \phi_{i} = r_{i} \phi_{0} - r_{i} \arcsin ( \frac{r_{i} \rho^{-1}}{2} ) \;
f_{i}(t,z_{0}) = z_{i} = z_{0} + 2 \rho t \arcsin (\frac{r_{i}}{2 \rho}) \;
\end{equation}

\begin{equation}
\frac{\partial s_{i}}{\partial \rho^{-1}} = - \frac{r_{i}^{2}}{2 \sqrt{1 - \frac{r_{i}^{2} \rho^{-2} }{4}}  \;.
\label{eq:barrel1}
\end{equation}

\begin{equation}
\frac{\partial s_{i}}{\partial \phi_{0}} = r_{i} \;.
\label{eq:barrel2}
\end{equation}

\begin{equation}
\frac{\partial s_{i}}{\partial t} = 2 \rho \arcsin (\frac{r_{i} \rho^{-1}}{2}) \approx r_{i} \;.
\label{eq:barrel3}
\end{equation}

\begin{equation}
\frac{\partial s_{i}}{\partial z_{0}} = 1 \;.
\label{eq:barrel4}
\end{equation}

As these first order approximation of these residuals depend solely on the position of barrel layers, making the task of tabulating the possible values straightforward.

\section{Endcap Disk Hits}
The hits in the endcap disks are not 

\begin{equation}
\frac{\partial r}{\partial \rho^{-1}} = - 2 \rho^{2} sin \Big( \frac{\rho^{-1}}{2t} (z_{disk} - z_{0}) \Big) + \frac{z_{disk} - z_{0}}{\rho^{-1} t} cos \Big( \frac{\rho^{-1}}{2t} (z_{disk} - z_{0}) \Big) \;.
\label{eq:endcap1}
\end{equation}

\begin{equation}
\frac{\partial r}{\partial \phi_{0}} = 0 \;.
\label{eq:endcap2}
\end{equation}

\begin{equation}
\frac{\partial r}{\partial t} = - \frac{z_{disk} - z_{0}}{t^{2}} cos \Big( \frac{\rho^{-1}}{2t} (z_{disk} - z_{0}) \Big) \;.
\label{eq:endcap3}
\end{equation}

\begin{equation}
\frac{\partial r}}{\partial z_{0}} = - \frac{1}{t} cos \Big( \frac{\rho^{-1}}{2t} (z_{disk} - z_{0}) \Big) \;.
\label{eq:endcap4}
\end{equation}

\begin{equation}
\frac{\partial \phi}{\partial \rho^{-1}} = - \frac{z_{disk} - z_{0}}{2t} \;.
\label{eq:endcap5}
\end{equation}

\begin{equation}
\frac{\partial \phi}{\partial \phi_{0}} = 1 \;.
\label{eq:endcap6}
\end{equation}

\begin{equation}
\frac{\partial \phi}{\partial t} = \frac{\rho^{-1}}{2t^{2}} (z_{disk} - z_{0} \;.
\label{eq:endcap7}
\end{equation}

\begin{equation}
\frac{\partial \phi}{\partial z_{0}} = \frac{\rho^{-1}}{2t} \;.
\label{eq:endcap8}
\end{equation}

\subsection{Outer Disks}
As the outer disks do not provide a direct measurement of $\phi$, $\phi_{hit}$, as the 2S modules' strips do not point directly towards the interaction point, a corrective factor is used to determine this using the 

\begin{equation}
\phi_{hit} = \phi_{centre} + \alpha ( r - r_{centre} )
\label{OuterHits1}
\end{equation}

\begin{equation}
r(h)  = r(h + \delta h) = r(h) + \delta h \frac{\partial r}{\partial h} + \mathcal{0} (\delta h^{2})
\phi(h)  = \phi(h + \delta h) = \phi(h) + \delta h \frac{\partial \phi}{\partial h} + \mathcal{0} (\delta h^{2})
\label{OuterHits2}
\end{equation}

\begin{equation}
\delta_{\phi} = \phi (h) - \phi_{hit} = \phi (h) - \phi_{centre} - \alpha ( r - r_{centre} )
\label{OuterHits3}
\end{equation}

\begin{equation}
\frac{\partial \delta_{phi}}{\partial h} = \frac{\partial \phi}{\partial h} - \alpha \frac{\partial r}{\partial h}
\label{OuterHits4}
\end{equation}

\chapter{Further information regarding Boosted Decision Trees}\label{app:bdt}

\section{BDT Input Features}\label{appsec:bdtFeatures}
The variables listed in table~\ref{allBdtVariables}

\begin{table}[htbp]
\topcaption {The name and descriptions of the variables chosen by recursive feature elimination to be used as input to the BDT to discriminate between potential tZq signal events and the dominant backgrounds.
}
\label{tab:allBdtVariables}
  \centering
% This increases column spacing.
\resizebox{\textwidth}{!}{
% This right-aligns numbers in column, but centers them under column title.
\begin{tabular}{cccc}
   \hline
   \textbf{Variable} & \textbf{Description} \\
   \hline
    wQuark1Pt & Leading W boson candidate jet \pt \\
    wQuark1Eta & Leading W boson candidate jet $\eta$ \\
    wQuark1Phi &  Leading W boson candidate jet $\phi$ \\
    wQuark2Pt & Subleading W boson candidate jet \pt \\
    wQuark2Eta & Subleading W boson candidate jet $\eta$ \\
    wQuark2Phi & Subleading W boson candidate jet $\phi$  \\
    wPairMass & W boson mass  \\
    wPairPt & W boson \pt  \\
    wPairEta & W boson $\eta$  \\
    wPairPhi & W boson $\phi$  \\
    mTW & W boson $m_{T}$  \\
    met & \met  \\
    nJets & Number of jets  \\
    leadJetPt & Leading jet \pt $\checkmark$ \\
    leadJetPhi & Leading jet $\phi$  \\
    leadJetEta & Leading jet $\eta$  \\
    leadJetbTag & Leading jet b-tag discriminator  \\
    secJetPt & Subleading jet \pt \\
    secJetPhi & Subleading jet $\phi$  \\
    secJetEta & Subleading jet $\eta$ \\
    secJetbTag & Subleading jet b-tag discriminator  \\
    thirdJetPt & Third jet \pt \\
    thirdJetPhi & Third jet $\phi$  \\
    thirdJetEta & Third jet $\eta$ \\
    thirdJetbTag & Third jet b-tag discriminator  \\
    fourthJetPt & Fourth jet \pt \\
    fourthJetPhi & Fourth jet $\phi$  \\
    fourthJetEta & Fourth jet $\eta$ \\
    fourthJetbTag & Fourth jet b-tag discriminator  \\
    nBjets & Number of b-tagged jets \\
    bTagDisc & Leading b-tagged jet b-tag discriminator \\
    lep1Pt & Leading lepton \pt \\
    lep1Eta & Leading lepton $\eta$ \\
    lep1Phi & Leading lepton $\phi$ \\
    lep1RelIso & Leading lepton $I^{rel}$ \\
    lep1D0 & Leading lepton $d_{0}$ \\
    lep2Pt & Subleading lepton \pt \\
    lep2Eta & Subleading lepton $\eta$ \\
    lep2Phi & Subleading lepton $\phi$ \\
    lep2RelIso & Subleading lepton $I^{rel}$ \\
    lep2D0 & Subleading lepton $d_{0}$ \\
    zMass & Z boson mass  \\
    zPt & Z boson \pt \\
    zEta & Z boson $\eta$ \\
    zPhi & Z boson $\phi$ \\
    topMass & Top quark mass \\
    topPt & Top quark \pt \\
    topEta & Top quark $\eta$ \\
    topPhi & Top quark $\phi$ \\
    j1j2delR & $\Delta R$ between the leading and subleading jets \\
    j1j2delPhi & $\Delta \phi$ between the leading and subleading jets \\
    w1w2delR & $\Delta R$ between the W boson jets \\
    w1w2delPhi & $\Delta \phi$ between the W boson jets \\
    zLepdelR & $\Delta R$ between the Z boson leptons \\
    zLepdelPhi & $\Delta \phi$ between the Z boson leptons \\
    zl1Quark1DelR & $\Delta R$ between the leading lepton and leading W boson jet \\
    zl1Quark1DelPhi & $\Delta \phi$ between the leading lepton and leading W boson jet \\
    zl1Quark2DelR & $\Delta R$ between the leading lepton and subleading W boson jet \\
    zl1Quark2DelPhi & $\Delta \phi$ between the leading lepton and subleading W boson jet \\
    zl2Quark1DelR & $\Delta R$ between the subleading lepton and leading W boson jet \\
    zl2Quark1DelPhi & $\Delta \phi$ between the subleading lepton and leading W boson jet \\
    zl2Quark2DelR & $\Delta R$ between the subleading lepton and subleading W boson jet \\
    zl2Quark2DelPhi & $\Delta \phi$ between the subleading lepton and leading W boson jet \\
    zlb1DelR & $\Delta R$ between the Z boson and leading b-tagged jet \\
    zlb1DelPhi & $\Delta \phi$ between the Z boson and leading b-tagged jet \\
    zlb2DelR & $\Delta R$ between the Z boson and subleading b-tagged jet\\
    zlb2DelPhi & $\Delta \phi$ between the Z boson and subleading b-tagged jet \\
    lepHt & ${\ensuremath{H_{\mathrm{T}}}$ of the Z boson leptons \\
    wQuarkHt & ${\ensuremath{H_{\mathrm{T}}}$ of the W boson quarks \\
    totPtVec & \pt of the system \\
    totEta & $\eta$ of the system \\
    totPhi & $\phi$ of the system \\
    totVecM & Invariant mass of the system \\
    totPt2Jet & Square of the sum of the two leading jets' \pT \\
    totJetPt & Sum of all the jets' \pT  \\
    wZdelR & $\Delta R$ between the W and Z bosons \\
    wZdelPhi & $\Delta \phi$ between the W and Z bosons \\
    zQuark1DelR & $\Delta R$ between the Z boson and leading W boson jet \\
    zQuark1DelPhi & $\Delta \phi$ between the Z boson and leading W boson jet \\
    zQuark2DelR & $\Delta R$ between the Z boson and subleading W boson jet \\
    zQuark2DelPhi & $\Delta \phi$ between the Z boson and subleading W boson jet \\
    zTopDelR & $\Delta R$ between the Z boson and top quark \\
    zTopDelPhi & $\Delta \phi$ between the Z boson and top quark\\
    zl1TopDelR & $\Delta R$ between the leading lepton and top quark \\
    zl1TopDelPhi & $\Delta \phi$ between the leading lepton and top quark \\
    zl2TopDelR & $\Delta R$ between the subleading lepton and top quark \\
    zl2TopDelPhi & $\Delta \phi$ between the subleading lepton and top quark \\
    wTopDelR & $\Delta R$ between the W boson and top quark \\
    wTopDelPhi & $\Delta \phi$ between the W boson and top quark \\
    w1TopDelR & $\Delta R$ between the leading W boson jet and top quark \\
    w1TopDelPhi & $\Delta \phi$ between the leading W boson jet and top quark \\
    w2TopDelR & $\Delta R$ between the subleading W boson jet and top quark \\
    w2TopDelPhi & $\Delta \phi$ between the subleading W boson jet and top quark \\
    zjminR & Minimum $\Delta R$ between the Z boson and a jet  \\
    minZJetPhi & Minimum $\phi R$ between the Z boson and a jet \\
    totHt & Total ${\ensuremath{H_{\mathrm{T}}}$ of the system \\
    jetHt & ${\ensuremath{H_{\mathrm{T}}}$ of all the jets present \\
    jetMass & Invariant mass of all the jets present \\
    jetPt & \pT of all the jets present \\
    jetEta & $\eta$ of all the jets present \\
    jetPhi & $\phi$ of all the jets present \\
    jetMass3 & Invariant mass of the leading three jets\\
    totHtOverPt & Total ${\ensuremath{H_{\mathrm{T}}}$ divided by the system's \pT \\
   \hline
 \end{tabular}}
\end{table}

\end{appendices}

\bibliographystyle{JHEP}
\bibliography{admorton_thesis}
 
\end{document}

